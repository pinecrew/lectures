\chapter{Релятивистская природа магнитного поля}
\section{Постулаты СТО}
    \begin{enumerate}
        \item \textbf{Принцип относительности Энштейна}
            \begin{quote}
                Уравнения, выражающие законы природы одинаковы во всех
                инерциальных системах отсчета, то есть они инвариантны
                относительно преобразований координат и времени при переходе из
                одной системы в другую.
            \end{quote}

    \item \textbf {Принцип ограниченности скорости света}
        \begin{quote}
            Скорость света в пустоте одинакова во всех системах отсчета и не
            зависит от скорости источника и приемника:
            \[
                c = 3\cdot10^8 \text{м/с} = \mathbf{inv}.
            \]
        \end{quote}
    \end{enumerate}

    \begin{comment}
        Пояснения к формуле преобразования временного интервала
        \[
            \Delta t’ = \Delta t\sqrt{1 - \beta^2}.
        \]

        Здесь \( \Delta t’ \) -- время между двумя событиями, происходящими в
        системе \( O’ \) (например, на космическом корабле), измеряемое по часам
        системы \( O’ \) -- \textbf{собственное время}. \( \Delta t \) -- это
        время между двумя этими же событиями, измеряемое по часам системы
        \( O \) (например, на Земле).
    \end{comment}

    \begin{example} 
        Пусть скорость системы \( O’ \) равна \( v = 0.99c \). Тогда 
        \[
            \Delta t = \frac{\Delta t’}{\sqrt{1 - 0.99^2}} \approx
            \frac{\Delta t’}{0.14} \approx 9\Delta t’.
        \]
        Это означает, что если в ракете \( \Delta t’ = 1 \)год, то на Земле
        пройдет \( \Delta t = 7 \)лет.
    \end{example}

    \begin{example} 
        В космических лучах встречаются протоны с энергией
        \( m = 10^{13}\text{МэВ} = 10^{19}\text{эВ} \). Энергия покоя протона:
        \( m_0 = 938\text{МэВ} = 10^9\text{эВ} \). А так как
        \[
            m = \frac{m_0}{\sqrt{1 - \frac{v^2}{c^2}}},
        \]
        то  такие протоны (с отношением \( m/m_0 = 10^{10} \)) имеют скорость
        \( v \), совпадающую со скоростью \( c = 3\cdot10^8 \)м/с с точностью до
        20 (!) знака. Определить, за какое время \( \Delta t’ \) (в его системе
        отсчета) такой протон пересечет нашу галактику, диаметр которой
        \( \diameter = 10^5 \) св. лет.
    \end{example}
    \begin{solution}
        По нашим часам: \( \Delta t = \frac{\diameter}{c} = 10^5 \) лет.
        По собственному времени протона:
        \[
            \Delta t’ = \Delta t\sqrt{1 - \beta^2} = \Delta t\frac{m_0}{m} =
            \Delta t \times 10^{-10} = 10^{-5}\text{года} \approx
            5\text{ минут}.
        \]

        Время, за которое пересечет галактику и вернется в первоначальную точку,
        то есть пересечет галактику дважды:
        \begin{align*}
                \Delta t_\textit{общ}{’} = 10\text{ мин}, \\
                \Delta t_\textit{общ} = 200000\text{ лет}.
        \end{align*}
    \end{solution}
\section{Определение магнитного поля}

	Ранее (п.~\ref{sec8.1}), было дано такое определение магнитному полю:
	\begin{definition}
        Если на точечный заряд \( q \), движущийся в некоторой системе отсчета в
        области \( V \) с некоторой скоростью \( v \), действует магнитная сила,
        то говорят, что в этой области и \textit{в этой системе отсчета} есть
        \textbf{магнитное поле} \( \vec{B} \), величина и направление которого
        определяется формулой Лоренца:
        \begin{equation}
            \vec{F} = q(\vec{v} \times \vec{B}) \label{eq24:1}
        \end{equation}
	\end{definition}
    
	Но в таком определении нет ничего о природе магнитного поля. Чтобы дать
    более точное определение, надо рассмотреть свойство \textit{инвариантности
    заряда}.

\section{Инвариантность электрического заряда}

	\begin{definition}
        \textbf{Инвариантность}, или \textbf{релятивистская инвариантность}, --
        это независимость физической величины от системы отсчета (наблюдения).
	\end{definition}
	
	Опыт показывает, что электрический заряд \( q \) -- величина инвариантная,
    то есть из всех систем отсчета она выглядит одинаково.
	
	\begin{remark}
        Все величины, в неподвижной (лабораторной) системе \( Oxyz \) будем 
        обозначать без штрихов: \( q \), \( m \), \( l \), \( \Delta t \), а в
        движущейся системе \( O’x’y’z’ \) со штрихами: \( q’ \), \( m’ \),
        \( l’ \), \( \Delta t’ \).
	\end{remark}
	
    \begin{remark}
        Будем рассматривать только инерциальные системы отсчета, причем все они
        являются эквивалентными.
	\end{remark}
	
	Далеко не все величины являются инвариантными, например:
	\begin{enumerate}
        \item масса:
        \[
            m’ = \frac{m}{\sqrt{1 - \frac{v^2}{c^2}}} =
            \frac{m}{\sqrt{1 - \beta^2}},
        \]
        где \( \beta = \frac{v}{c} \), \( c = 3\cdot10^8 \)м/с;
        
        \item длина:
        \[
            l’ = l\sqrt{1 - \beta^2};
        \]
        
        \item временной интервал:
        \[
            \Delta t’ = \Delta t\sqrt{1 - \beta^2}.
        \]
	\end{enumerate}
	
	Формально, величина заряда \( q \) определяется как поток поля \( \vec{E} \)
    через поверхность \( S \):
	\[
        q = \Ezero\iint\limits_S \vec{E}\cdot\dd\vec{S},
    \]
	а \( q’ \) -- как поток поля \( \vec{E}{’} \) через \( S’ \):
	\[
        q’ = \Ezero\iint\limits_{S’} \vec{E}{’}\cdot\dd\vec{S}.
    \]
	Однако, \( q’ = q \). Следовательно:
	\[
        \iint\limits_S \vec{E}\cdot\dd\vec{S} =
        \iint\limits_{S’} \vec{E}{’}\cdot\dd\vec{S}.
    \]
	
	Примеры, подтверждающие инвариантность заряда:
	\begin{enumerate}
        \item молекула \( \mathrm{D}_2 \): \( 2p + 2n + 2e \) и атом
            \( \mathrm{He}_2^4 \): \( 2p + 2n + 2e \).
            
            Так как электроны движутся с разными скоростями, то масса атома
            гелия не равна массе молекулы дейтерия, однако их заряд одинаков и
            равен нулю;
            
        \item нагревание вещества.
            
            При нагревании скорости ядер и электронов меняются по-разному,
            поэтому, если бы заряд был не инвариантен, то у вещества появлялся
            бы не нулевой суммарный заряд, однако этого не происходит.
        
        \item система двух зарядов.
        
            Их суммарная масса покоя \( \sum m = 2m \), а их суммарный заряд
            \( \sum q = 2q \). Если их вращать со скоростью \( \omega \), то их
            суммарная масса становится больше суммарной массы покоя:
            \( \sum m’ > 2m \), однако суммарный заряд сохраняется:
            \( \sum q’ = \sum q = 2q \).
	\end{enumerate}

\section{Релятивистское преобразование плотности заряда}

	Пусть имеется прямой провод с током \( i \). Он создает вокруг себя поле
    \( \vec{B} = \frac{\mu_0i}{2\pi r} \ne 0 \). Следовательно, на заряд
    \( q \), движущийся со скоростью \( v \) параллельно проводу, действует сила
    Лоренца:
	\begin{equation}
		F_\perp = qv\frac{\mu_0i}{2\pi r}.
        \label{eq24:2}
	\end{equation}
	
	Перейдем в движущуюся систему заряда \( q \). В его системе скорость \( v \)
    равна нулю, следовательно, сила Лоренца должна быть также равна нулю. А
    поскольку все инерциальные системы отсчета эквивалентны, то появляется
    противоречие.
	
	Далее, второй пример противоречия: \textit{картинки с чуваком и вагонеткой}
	
	Эти противоречия неразрешимы в рамках классической электростатики. Для этого
    необходимо привлечь постулаты специальной теории относительности (СТО).
    Рассмотрим заряженный участок провода длиной \( l \).
	
	Пусть линейная (погонная) плотность заряда на проводе равна \( \gamma =
    \frac{\Delta q}{\Delta l} \). В системе \( O’ \) длина участка провода
    \( l’ \) уменьшилась:
	\[
        l’ = l\sqrt{1 - \beta^2}.
    \]
	Однако, в силу инвариантности заряда, в \( l’ \) зарядов осталось столько
    же, сколько и в \( l \): \( \gamma l = \gamma’ l’ \). Это означает, что
    увеличилась плотность зарядов:
	\begin{equation}
		\gamma’ = \frac{\gamma}{\sqrt{1 - \beta^2}}.
        \label{eq24:3}
	\end{equation}
	
	Применим результаты (\ref{eq24:3}) к проводу с током \( i \). Так как в
    \( O \) провод в целом нейтрален, то:
	\[
        \gamma_+ =|\gamma_-|,
    \]
	то есть
	\[
        \gamma_\textit{рез} = \gamma_+ + \gamma_- = \gamma_+ |\gamma_-| = 0.
    \]
	
	Перейдем в систему электронов \( O’ \), движущихся со скоростью \( v \)
    влево, следовательно, в \( O’ \) провод движется со скоростью \( v \)
    вправо. В соответствии с (\ref{eq24:3}), плотность ионов в проводе:
	\[
        \gamma_+{’} = \frac{\gamma_+}{\sqrt{1 - \beta^2}}.
    \]
	А так как в \( O’ \) электроны неподвижны, то их плотность  -- плотность
    покоя -- ниже, чем в \( O \):
	\[
        \gamma_-{’} = \gamma_-\sqrt{1 - \beta^2}.
    \]
	Тогда результирующая плотность заряда в \( O’ \):
	\[
        \gamma_\textit{рез}{’} = \gamma_+{’} - |\gamma_-{’}| =
        \frac{\gamma_+}{\sqrt{1 - \beta^2}} - |\gamma_-\sqrt{1-\beta^2}| =
        \gamma_+\left(\frac{1}{\sqrt{1 - \beta^2}} - \sqrt{q - \beta^2}\right).
    \]
	
	Таким образом:
	\begin{equation}
		\gamma_\textit{рез}{’} = \gamma_+\left(\frac{1}{\sqrt{1 - \beta^2}} -
        \sqrt{q - \beta^2}\right) = \frac{\gamma_+\beta^2}{\sqrt{1 - \beta^2}}.
        \label{eq24:4}
	\end{equation}

\section{Релятивистская природа магнитного поля}

	Итак, в системе \( O’ \) провод оказался заряженным, хотя в системе \( O \)
    он был нейтральным. Следовательно, в \( O’ \) вокруг него появится
    электрическое поле \( \vec{E}{’} \), которого не было в системе \( O \).
	Это поле будет действовать в \( O’ \) на неподвижный заряд \( q \).
	
	\begin{remark}
        Для простоты рассуждений за скорость заряда \( v \) была взята дрейфовая
        скорость \( v = v_\textit{др} \) электронов в проводе. Если скорость
        заряда не равна дрейфовой, то результат, в принципе, не изменится.
	\end{remark}
	
	Итак, в системе \( O’ \) на заряд \( q \) действует поперечная сила
    \( F’ = E’q \), где \( E’ \):
	\[
        E’ = \frac{\gamma_\textit{рез}{’}}{2\pi\Ezero r} =
        \frac{\gamma\beta^2}{\sqrt{1 - \beta^2}2\pi\Ezero r} =
        \frac{\gamma_+\frac{v^2}{c^2}}{\sqrt{1 - \beta^2}2\pi\Ezero r}.
    \]
	Таким образом, сила в системе \( O’ \):
	\[
        F_\perp{’} = qv\frac{\gamma_+v}{2\pi\Ezero c^2r\sqrt{1 - \beta^2}}.
    \]

	Теперь выразим погонную плотность заряда:
	\[
        \gamma_+ = \frac{q_+}{l} = \frac{\rho_+(lS)}{l} = \rho_+S,
    \]
	где \( S \) -- сечение провода, \( \rho_+ \) -- объемный заряд в \( O \).
	
	Так как \( \rho_+ = n_+e_+ \), где \( n_+ \) --  концентрация зарядов,
    \( e_+ \) -- элементарный заряд, то:
	\[
        F_\perp{’} = qv\frac{(n_+e_+v)S}{2\pi\Ezero rc^2\sqrt{1 - \beta^2}},
    \]
	а так как \( n_+e_+v = j \), а \( jS = i \), то:
	\[
        F_\perp{’} = qv\frac{i}{2\pi\Ezero rc^2\sqrt{1 - \beta^2}}.
    \]
	Так как \( \mu_0 = \frac{1}{\Ezero c^2} \), то:
	\[
        F_\perp{’} = qv\frac{\mu_0i}{2\pi r}\cdot\frac{1}{\sqrt{1 - \beta^2}}.
    \]
	Преобразование силы при переходе из одной инерциальной системы в другую:
	так как, по определению,
    \[
        F_\perp{’} = \frac{\Delta p_\perp{’}}{\Delta t’}\ (\text{в}\ O’\ ),
    \]
    а
    \[
        F_\perp = \frac{\Delta p_\perp}{\Delta t}\ (\text{в}\ O\ ),
    \]
    и \( \Delta p_\perp{’} = \Delta p_\perp \), то:
	\[
        F_\perp = F_\perp{’}\sqrt{1 - \beta^2}.
    \]
	
	Следовательно, в системе \( O \), на заряд \( q \), движущийся в ней со
    скоростью \( v \), действует перпендикулярная сила:
	\[
        F_\perp = qv\frac{\mu_0i}{2\pi r} \equiv (\ref{eq24:2}).
    \]
	
	Итак, в системе \( O \) взаимодействие тока и заряда воспринимается как
    магнитное, а в системе \( O’ \) -- как электрическое. Но их величины и
    направления одинаковы.
	
\section{Выводы}

	\begin{enumerate}
        \item \textit{Если} верны постулаты СТО, верен закон Кулона, заряд
            инвариантен относительно преобразований системы отсчета, то явления,
            называемые магнитными, \textit{обязаны существовать}. Они являются
            следствием взаимодействия движущихся зарядов.
        
        \item Магнетизм является \textit{чисто релятивистским эффектом}
            (следствием конечной скорости света в вакууме). Если бы
            \( c\to\infty \) в вакууме, то магнитных явлений бы не было, то
            есть: \( F_\perp \to 0 \) (\( E_\perp{’} \to 0 \)) и
            \( \mu_0 = \frac{1}{\Ezero c^2} \to 0 \). Поэтому релятивистская
            теория магнетизма \textit{не требует существования магнитного
            заряда}.
        
        \item Каждое поле в отдельности, электрическое или магнитное, \textit{не
            имеет абсолютного смысла}: в одних системах отсчета оно может быть,
            а в других может и не быть. Поэтому о них надо говорить лишь с
            указанием системы отсчета. Каждое из них является лишь удобной
            формой описания взаимодействия движущихся зарядов.
        
        \item Понятие магнитно поля формально можно вывести лишь на основе
            постулатов СТО, закона Кулона и принципа инвариантности заряда.
            Однако, вместо того, чтобы каждый раз переходить из одной системы в
            другую, и вводится удобное понятие магнитного поля, входящее в
            формулу Лоренца и закон Био-Савара.
	\end{enumerate}

\section{Формулы преобразования полей}

	СТО дает формулы преобразования полей \( \vec{B} \) и \( \vec{E} \) при
    переходе из одной инерциальной системы в другую.
	
	Наиболее простой вид эти формулы имеют, когда \( v \ll c \) (\( \sqrt{1 - \beta^2} \approx 1 \)):
	\[
        \left\{
        \begin{array}{l}
            \vec{E}{’} = \vec{E} + \vec{v}\times\vec{B}, \\
            \vec{B}{’} = \vec{B} - \frac{1}{c^2} \vec{v}\times\vec{E}.
        \end{array}
        \right.
    \]
	
	Первое уравнение можно получить из формулы Лоренца, второе -- из закона
    Био-Савара.
