\section{Автоколебания}
\subsection{Общие положения}

Автоколебания -- это режим диссипативной колебательной системы, в котором
установившиеся параметры колебаний не зависят от начальных условий.

Автоколебательными обычно являются нелинейные системы с трением. При этом
движение изображающей точки на фазовой плоскости может иметь и непериодический
характер. В таких случаях выделяют 2 типа атоколебаний:
\begin{itemize}
    \item квазипериодические -- их спектр содержит пару-тройку основных
        гармоник, частоты которых находятся в иррациональном соотношении; в
        результате они выглядят как периодические, но таковыми не являются;
    \item хаотические колебания -- спектр этих колебаний сплошной, при этом
        фазовая точка движется вблизи некоторой кривой (называемой аттрактором),
        неограниченно приближаясь к ней.
\end{itemize}

Рассмотрим теперь примеры автоколебательных систем.
\subsection{Примеры}

\subsubsection{Колебательный контур с нелинейной обратной связью}
\subsubsection{Брюсселятор Пригожина}
Автоколебания могут наблюдаться и в химических реакциях. Одними из первых такую
реакцию наблюдали Белоусов и Жаботинский. Сейчас под реакцией
Белоусова-Жаботинского понимают класс химических реакций, протекающих в
колебательном режиме, при котором некоторые параметры реакции (цвет,
концентрация компонентов, температура и др.) периодически изменяются.

Одной из моделей такой раекции является брюсселятор Пригожина. Он описывается
следующей последовательностью реакций:
\begin{align*}
    A      & \xrightarrow{k_1} X,\\
    X + B  & \xrightarrow{k_2} Y + D,\\
    2X + Y & \xrightarrow{k_3} 3X,\\
    X      & \xrightarrow{k_4} E.
\end{align*}
Здесь вещества A и B берутся в избытке, а колебания в концентрациях
наблюдаются для промежуточных веществ X и Y.

Для концентраций промежуточных веществ получаются дифференциальные уравнения:
\[
    \left\{
        \begin{array}{l}
            \der{[X]}{t} = k_1[A] - k_2[B][X] + k_3[X]^2[Y] - k_4[X],\\
            \der{[Y]}{t} = k_2[B][X] - k_3[X]^2[Y].\\
        \end{array}
    \right.
\]
Теперь попробуем перейти к более удобным переменным. Для начала введём
безразмерные величины для концентраций
\[
    [X] = [X]_0 x,\quad [Y] = [Y]_0 y,
\]
где \( [X]_0 \) и \( [Y]_0 \) -- равновесные концентрации, которые можно
определить, решив систему
\[
    \left\{
        \begin{array}{l}
            k_1[A] - k_2[B][X]_0 + k_3[X]_0^2[Y]_0 - k_4[X]_0 = 0,\\
            k_2[B][X]_0 - k_3[X]_0^2[Y]_0 = 0.\\
        \end{array}
    \right.
\]
Нетрудно видеть, что
\[
    [X]_0 = \frac{k_1 [A]}{k_4},\quad [Y]_0 = \frac{k_2[B]}{k_3[X]_0} =
        \frac{k_2k_4}{k_1k_3}\frac{[B]}{[A]}.
\]
Тогда уравнения можно переписать в виде
\[
    \left\{
        \begin{array}{l}
            \der{x}{t} = k_1\frac{[A]}{[X]_0} - (k_2[B] + k_4)x +
                k_3[X_0][Y_0]x^2y,\\
            \der{y}{t} = k_2\frac{[B][X]_0}{[Y]_0}x -
                k_3[X]_0^2x^2y.\\
        \end{array}
    \right.
\]
Теперь подставим выражения для \( [X]_0 \) и \( [Y]_0 \):
\[
    \left\{
        \begin{array}{l}
            \der{x}{t} = k_4 - (k_2[B] + k_4)x + k_2[B]x^2y,\\
            \der{y}{t} = \frac{k_3k_1^2}{k_4^2}[A]^2 (x - x^2y).\\
        \end{array}
    \right.
\]
Теперь стоит перейти к новой временной координате. Положим \( \tau = k_4t \) и
будем писать дифференцирование по \( \tau \) при помощи точки сверху:
\[
    \left\{
        \begin{array}{l}
            \dot{x} = 1 - (\frac{k_2[B]}{k_4} + 1)x + \frac{k_2[B]}{k_4}x^2y,\\
            \dot{y} = \frac{k_3k_1^2}{k_4^3}[A]^2 (x - x^2y).\\
        \end{array}
    \right.
\]
Введём обозначения
\[
    a = \frac{k_2[B]}{k_4},\quad b = \frac{k_3k_1^2}{k_4^3}[A]^2
\]
и перепишем систему в ещё более привлекательном виде
\[
    \left\{
        \begin{array}{l}
            \dot{x} = 1 - (a + 1)x + ax^2y,\\
            \dot{y} = b(x - x^2y).\\
        \end{array}
    \right.
\]

