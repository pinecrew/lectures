\section{Поток и дивергенция векторного поля}

\subsection{Поток}

	Пусть в векторном поле \( \vec{a}(x, y, z) \)  находится двухсторонняя гладкая поверхность \( S \).
	
	Выделим на \( S \) малую площадку \( \Delta S \), на ней векторное поле имеет значение \( \vec{a} \). \( \vec{n} \) -- единичная нормаль к \( \Delta S \) на условно положительной стороне.
	
	\begin{definition}
	Скалярная величина \( \Delta\Phi = a_n\Delta S = |\vec{a}|\cos\alpha\Delta S = (\vec{a}\cdot\vec{n})\Delta S \) называется \textbf{элементарным потоком} поля \( \vec{a}(x, y, z) \) через площадку \( \Delta S \).
	\end{definition}
	
	\begin{definition}
	Предельная сумма всех элементарных потоков через всю поверхность \( S \) называется \textbf{потоком} поля \( \vec{a}(x, y, z) \) через поверхность \( S \).
	\begin{equation}
		\Phi = \lim_{\Delta S_k \to 0 \\ N \to \infty} \sum\limits_{k=1}^N (\vec{a}\cdot\vec{n})\Delta S_k \equiv \iint\limits_S (\vec{a}\cdot\vec{n})dS \label{eq4:1}
	\end{equation}
	\end{definition}
	
	Далее удобно ввести элемент ориентированной площади \( d\vec{S} \equiv \vec{n}dS \), направленный вдоль нормали \( \vec{n} \) и по величине равный \( dS \).
	
	Тогда поток поля \( \vec{a} \) через поверхность \( S \):
	\begin{equation}
		\Phi = \iint\limits_S \vec{a}\cdot d\vec{S} \label{eq4:2}
	\end{equation}
	
	\begin{remark}
	Элементарный поток \( \Delta \Phi = |\vec{a}| |\vec{n}|\cos\alpha\Delta S \)  -- величина \textit{алгебраическая}, то есть он может быть и положительным, и отрицательным, и равным нулю, в зависимости о угла \( \alpha \):
	\[ \left\{
	\begin{array}{r l}
		\alpha > \frac{\pi}{2}, & \Delta\Phi < 0 \\
		\alpha = \frac{\pi}{2}, & \Delta\Phi = 0 \\
		\alpha < \frac{\pi}{2}, & \Delta\Phi > 0
	\end{array}
	\right . \]
	
	Следовательно, и весь поток \( \Phi = \iint\limits_S \vec{a}\cdot d\vec{S} \) является алгебраической величиной.
	\end{remark}
	
	\begin{remark}
	Если поверхность \( S \) является замкнутой, то условно положительной стороной является внешняя, а положительной нормалью -- внешняя нормаль.
	
	В этом случае (\ref{eq4:2}) имеет вид:
	\begin{equation}
		\Phi = \oiint\limits_S \vec{a}\cdot d\vec{S}
	\end{equation}
	\end{remark}
	
\subsection{Свойства потока}

	\begin{enumerate}
	\item При смене стороны поток меняет знак.
	
	\begin{proof}
	\[ \Phi_1 = \iint\limits_S (\vec{a}\cdot\vec{n}_1)dS \]
	\[ \Phi_2 = \iint\limits_S (\vec{a}\cdot\vec{n}_2)dS \]
	
	А так как \( \vec{n}_1 = -\vec{n}_2 \), то \( \Phi_1 = -\Phi_2 \).
	\end{proof}
	
	\item Линейность.
	Если поле \( \vec{a} \) является суммой \( \vec{a} = \vec{a}_1 + \vec{a}_2 \), то поток \( \Phi_{\vec{a}}^S = \Phi_{\vec{a}_1}^S + \Phi_{\vec{a}_2}^S \), то есть
	\[ \Phi_{\vec{a}}^S = \iint\limits_S \vec{a}_1\cdot d\vec{S} + \iint\limits_S \vec{a}_2\cdot d\vec{S} = \iint\limits_S \vec{a}\cdot d\vec{S} \]
	
	\item Аддитивность.
	Если \( S \) -- это сумма гладких фрагментов \( S_k \), то
	\[ \iint\limits_S \vec{a}\cdot d\vec{S} = \sum\limits_1^N \iint\limits_{S_k} \vec{a}_k\cdot d\vec{S} \]
	
	\end{enumerate}
	
\subsection{Физическое содержание явления потока}

	Рассмотрим два физических процесса, в которых появляется поток \( \Phi \).
	
\subsubsection{Течение жидкости}

	Пусть течение жидкости задано полем её скоростей \( \vec{v}(x, y, z) \).
	
	Вычислим поток поля \( \vec{v} \) через поверхность \( S \).
	
	Вырежем на поверхности \( S \) малый элемент \( dS \) и построим на нём косой цилиндр длиной \( dl \) с осью на векторе \( \vec{v} \).
	
	Его заполняет объем воды \( dV = dSdh \). По определению, за время \( dt \) через цилиндр пройдет весь \( dV = dSdl\cos\alpha = dSv\cos\alpha dt \).
	
	Тогда расход воды
	\[ \frac{dV}{dt} = vdS\cos\alpha = \vec{v}\cdot d\vec{S} = d\Phi \].
	
	А расход воды через всю поверхность \( S \);
	\[ \left. \frac{dV}{dt}\right|_S = \iint\limits_S \vec{v}\cdot d\vec{S} \frac{\text{м}^3}{\text{с}} \]
	
	Таким образом, поток поля \( \vec{v} \) через поверхность \( S \) -- это расход воды в \( \frac{\text{м}^3}{\text{с}} \) через \( S \).
	
\subsubsection{Распространение тепла}

	В математической теории теплопередачи распространение тепла рассматривается как поток жидкости.
	
	Пусть направление распространения тепла задаётся единичным вектором \( \vec{n} \).
	
%	коммент к рисунку:
%	\( dS_n \) -- элемент площади, перпендикулярный вектору направления распространения тепла

	\begin{definition}
	Векторная величина \( \vec{j} \), показывающая направление распространения тепла (теплопереноса) и численно равная перенесённому количеству тепла \( dQ \) за время \( dt \) через площадку \( dS_n \), перпендикулярную распространению тепла, называется \textbf{плотностью потока тепла} (тепловая мощность, перенесённая через единицу площади).
	
	\begin{equation}
		\vec{j} \equiv \frac{dQ}{dtdS_n}\vec{n}
	\end{equation}
	\end{definition}
	
	\begin{definition}
	Скалярная алгебраическая величина \( dq = jdS_n = jdS\cos\alpha \) называется \textbf{элементарным тепловым потоком}.
	\end{definition}
	
	Тогда скалярная алгебраическая величина \( q = \iint\limits_S \vec{j}\cdot d\vec{S} \) -- это \textbf{поток тепла} через поверхность \( S \) (тепловая мощность, перенесённая через \( S \)).
	
\subsection{Примеры вычисления потока}

	Задача вычисления потока сводится к технике вычисления поверхностного интеграла \( \Phi = \iint\limits_S \vec{a}\cdot d\vec{S} \), но для некоторых видов полей \( \vec{a} \) и поверхностей \( S \) удаётся свести интеграл к умножению.
	
	\begin{example}
	Вычислить поток поля \( \vec{r} \) через плоскую поверхность \( S \), не проходящую через начало координат.
	\end{example}
	
	\begin{solution}
	\( h = r\cos\alpha \) -- расстояние от плоскости до точки \( O \).
	
	Элементарный поток поля \( \vec{r} \) через площадку \( dS \):
	\[ d\Phi = \vec{r}\cdot d\vec{S} = rdS\cos\alpha = hdS \]
	
	Так как \( h = \const \) для всех \( S \), то весь поток:
	\[ \Phi = \iint\limits_S hdS = h\iint\limits_S dS = hS \]
	\end{solution}
	
	\begin{example}
	Вычислить поток поля \( \vec{r} \) через замкнутую поверхность:
	\[ S: x^2 + y^2 = R^2; \ z = 0, \ z = h \]
	\end{example}
	
	\begin{solution}
	В силу аддитивности потока:
	\[ \Phi = \Phi^{\text{нижн}}_{\text{тор}} + \Phi^{\text{верх}}_{\text{тор}} + \Phi_{\text{бок}} \]
	
	Так как на нижнем торце линии поля \( \vec{r} \) всюду перпендикулярны \( d\vec{S}_{\text{нижн}} \), то поток через него равен нулю:
	\[ \Phi^{\text{нижн}}_{\text{тор}} = 0 \]
	
	Поток поля через верхний торец равен \( hS_{\text{тор}} \), в силу предыдущего примера:
	\[ \Phi^{\text{верх}}_{\text{тор}} = hS_{\text{тор}} = \pi R^2h\]
	
	Вычислим поток через боковую поверхность цилиндра:
	\[ \Phi_{\text{бок}} = \iint\limits_{S_{\text{бок}}} r\cos\alpha dS_{\text{бок}} = R\iint\limits_{S_{\text{бок}}} dS_{\text{бок}} = R\cdot 2\pi Rh = 2\pi R^2 h \]
	
	Тогда весь поток:
	\[ \Phi = \pi R^2h + 2\pi R^2h = 3\pi R^2h \]
	\end{solution}
	
	\begin{example}
	Вычислить поток поля \( \vec{r} \) через произвольную замнутую поверхность \( S \), ограничивающую объём \( V \).
	\end{example}
	
	\begin{solution}
	\begin{enumerate}
	\item Пусть точка \( O \) лежит внутри \( S \).
	
	Разобъём поверхность \( S \) на мелкие кусочки \( dS \) и построим узкие пирамиды с общей вершиной в точке \( O \).
	
	\( \vec{n} \) -- единичная нормаль к \( dS \).
	\( dS_n \) -- проекция \( dS \) на плоскость, перпендикулярную \( \vec{r} \).
	\( dV = \frac{1}{3}S_nr = \frac{1}{3}rdS\cos\alpha \) -- объём узкой пирамиды.

	Элементарный поток поля \( \vec{r} \) через площадку \( dS \):
	\[ d\Phi = \vec{r}\cdot d\vec{S} = rdS\cos\alpha = 3dV \]
	
	Тогда весь поток через поверхность \( S \):
	\[ \Phi = 3\oiint\limits_S dS = 3\iiint\limits_V dV = 3V \]
	
	\item Пусть точка \( O \) лежит вне \( S \).
	
	Тогда результат будет такой же:
	\[ \Phi = \oiint\limits_S \vec{r}\cdot d\vec{S} = 3V \]
	 Доказательство в пункте 4.9.
	\end{enumerate}
	\end{solution}
	
\subsection{Дивергенция векторного поля}

	Рассмотрим замкнутую поверхность \( S \), погружённую в поле \( \vec{a} \).
	
	Поток поля \( \vec{a} \) через поверхность \( S \):
	\[ \Phi = \oiint\limits_S \vec{a}\cdot d\vec{S} \]
	-- это величина алгебраическая.
	
	Если \( \Phi > 0 \), то это означает, что из \( S \) выходит линий больше, чем входит в неё, то есть внутри \( S \) есть \textbf{источник} поля, то есть точки, где линии поля \( \vec{a} \) зарождаются.
	
	Если \( \Phi < 0 \), то это означает, что в \( S \) входит линий больше, чем выходит из неё, то есть внутри \( S \) есть отрицательный источник (\textbf{сток}) поля, то есть точки, где линии поля \( \vec{a} \) исчезают.

	Если \( \Phi = 0 \), хотя \( \vec{a} \ne 0 \), то это означает, что из \( S \) выходит линий столько же, сколько входит в неё, то есть внутри \( S \) либо нет ни источников, ни стоков, либо их алгебраическая сумма равна нулю.
	
	Таким образом, поток является некоторой грубой характеристикой характера векторного поля. Более определённой, локальной, его характеристикой является алгебраическая величина, называемая \textit{дивергенцией}.
	
	Пусть задано векторное поле \( \vec{a}(x, y, z) \). Выделим в пространстве линейный объём \( \Delta V \), ограниченный поверхностью \( \Delta S \), в котором есть внутренняя точка \( M \).
	
	Поток поля через \( \Delta S \):
	\[ \Delta\Phi = \oiint\limits_{\Delta S} \vec{a}\cdot d\vec{S} \]
	
	Составим отношение:
	\begin{equation}
		\frac{\Delta\Phi}{\Delta V} = \frac{1}{\Delta V} \oiint\limits_{\Delta S} \vec{a}\cdot d\vec{S} \label{eq4:n1}
	\end{equation}
	
	При \( \Delta V \to 0 \) поток \( \Phi \to 0 \).
	
	\begin{definition}
	Предел (\ref{eq4:n1}) при \( \Delta V \to 0_M \), если он существует, называется \textbf{дивергенцией} поля \( \vec{a} \) в точке \( M \):
	
	\begin{equation}
		\divergence{a} \equiv \lim_{\Delta V \to 0_M} \frac{1}{\Delta V} \oiint\limits_{\Delta S} \vec{a}\cdot d\vec{S} \label{eq4:n2}
	\end{equation}
	\end{definition}
	
	\begin{comment}
	\begin{enumerate}
	\item Дивергенция \( \divergence{a} \) -- инвариант, то есть не зависит от выбора системы координат.
	
	\item Величина \( \divergence{a} \) -- сама образует скалярное поле \( f \), порождаемое векторным полем \( \vec{a} \):
	\[ \divergence{a} = f(x, y, z) \]
	
	\item Если \( \divergence{a} \equiv 0 \) во всей области существования поля \( \vec{a} \), то поле \( \vec{a} \) называется \textbf{соленоидальным}.
	
	\item Если в точке \( M \) дивергенция \( \divergence{a} > 0 \), то в точке \( M \) есть источник, если \( \divergence{a} < 0 \) -- то сток.
	
	\item Если поле \( \vec{a} \) -- соленоидальное, то его линии не имеют начала и конца, они либо замкнуты, либо бесконечно вьются.
	\end{enumerate}
	\end{comment}
	
	Однако, формула (\ref{eq4:n2}) не удобна для вычисления дивергенции.
	Получим формулу для вычисления в декартовых координатах, удобную для вычисления.
	
	Пусть поле \( \vec{a} \) имеет непрерывные частные производные \( \frac{\partial a_x}{\partial x} \), \( \frac{\partial a_y}{\partial y} \) и \( \frac{\partial a_z}{\partial z} \). Выберем в поле \( \vec{a} \) малый объём \( \Delta V = \Delta x\Delta y\Delta z\) в виде кубика с гранями, параллельными осям координат.
	
	% все подписи должны быть НА РИСУНКЕ!
	
	Вычислим поток поля \( \vec{a} \) через все шесть граней кубика.
	
	Поток поля \( \vec{a} \) через заднюю грань \( \Delta S_x^{-} \):
	\[ \Delta\Phi_x^{-} = -a_x(x, y, z)\Delta y\Delta z \]
	“\( -\)” стоит из-за того, что на задней грани \( u_x^{-} \uparrow\downarrow x \).
	
	Поток поля \( \vec{a} \) через переднюю грань \( \Delta S_x^{+} \):
	\[ \Delta\Phi_x^{+} = a_x(x + \Delta x, y, z)\Delta y\Delta z = a_x(x, y, z)\Delta y\Delta z + \frac{\partial a_x}{\partial x}\Delta x\Delta y\Delta z \]
	
	Суммарный поток \( \Delta\Phi_x \) через пару граней:
	\[ \Delta\Phi_x = \Delta\Phi_x^{+} + \Delta\Phi_x^{-} = \frac{\partial a_x}{\partial x}\Delta x\Delta y\Delta z = \frac{\partial a_x}{\partial x}\Delta V \]
	
	Аналогично, поток через две пары других граней:
	\[ \Delta\Phi_y = \frac{\partial a_y}{\partial y}\Delta V \]
	\[ \Delta\Phi_z = \frac{\partial a_z}{\partial z}\Delta V \]
	
	Тогда поток через все шесть граней:
	\[ \Delta\Phi = \Delta\Phi_x + \Delta\Phi_y + \Delta\Phi_z = \left(\frac{\partial a_x}{\partial x} + \frac{\partial a_y}{\partial y} + \frac{\partial a_z}{\partial z}\right)\Delta V \]
	
	Устремляя \( \Delta V \to 0_M \) и учитывая определение (\ref{eq4:n2}), получим:
	\begin{equation}
		\divergence{a} = \lim_{\Delta V \to 0_M} \frac{\Delta\Phi}{\Delta V} = \frac{\partial a_x}{\partial x} + \frac{\partial a_y}{\partial y} + \frac{\partial a_z}{\partial z} \label{eq4:n3}
	\end{equation}
	
	Выражение (\ref{eq4:n3}) удобно записывать в виде скалярного произведения:
	\begin{equation}
		\divergence{a} = \nabla\cdot\vec{a} \label{eq4:n4}
	\end{equation}
	
\subsection{Свойства дивергенции}

	\begin{enumerate}
	\item Если \( \vec{a} \) и \( \vec{b} \) -- два векторных поля, то
	\[ \divergence(\vec{a} + \vec{b}) = \divergence{a} + \divergence{b} \]
	\begin{proof}

	\[ \begin{array}{l}
	\divergence(\vec{a} + \vec{b}) = \frac{\partial (a_x + b_x)}{\partial x} + \frac{\partial (a_y + b_y)}{\partial y} + \frac{\partial (a_z + b_z)}{\partial z} = \\
	= \frac{\partial a_x}{\partial x} + \frac{\partial a_y}{\partial y} + \frac{\partial a_z}{\partial z} + \frac{\partial b_x}{\partial x} + \frac{\partial b_y}{\partial y} + \frac{\partial b_z}{\partial z} = \\
	= \divergence{a} + \divergence{b}
	\end{array} \]
	\end{proof}
	
	\item Пусть \( \vec{a}(x, y, z) \) -- векторное поле, а \( u(x, y, z) \) -- скалярное, тогда
	\begin{equation}
	\divergence(u\vec{a})\, = u\nabla\cdot\vec{a} + \vec{a}\cdot\nabla u = u\divergence{a} + \vec{a}\cdot\gradient{u} \label{eq4:n5}
	\end{equation}
	\begin{proof}
	\[ \begin{array}{l}
	\divergence(u\vec{a})\, = \frac{\partial (ua_x)}{\partial x} + \frac{\partial (ua_y)}{\partial y} + \frac{\partial (ua_z)}{\partial z} = \\
	= a_x\frac{\partial u}{\partial x} + u\frac{\partial a_x}{\partial x} + a_y\frac{\partial u}{\partial y} + u\frac{\partial a_y}{\partial y} + a_z\frac{\partial u}{\partial z} + u\frac{\partial a_z}{\partial z} = \\
	= u\left(\frac{\partial a_x}{\partial x} + \frac{\partial a_y}{\partial y} + \frac{\partial a_z}{\partial z}\right) + a_x\frac{\partial u}{\partial x} + a_y\frac{\partial u}{\partial y} + a_z\frac{\partial u}{\partial z} = \\
	= u\divergence{a} + \vec{a}\cdot\gradient{u}
	\end{array} \]
	\end{proof}
	\end{enumerate}
	
\subsection{Примеры вычисления дивергенции}

	\begin{example}
	Вычислить \( \divergence{r} \).
	\end{example}
	\begin{solution}
	\[ \divergence{r} = \frac{\partial x}{\partial x} + \frac{\partial y}{\partial y} + \frac{\partial z}{\partial z} = 3 \]
	\end{solution}
	
	\begin{example}
	Вычислить \( \divergence{a} \), где \( \vec{a} = r^n\vec{r} \).
	\end{example}
	\begin{solution}
	В силу (\ref{eq4:n5}):
	\[ \divergence(r^n\vec{r})\, = r^n\divergence{r} + \vec{r}\cdot\gradient{r^n} = 3r^n + x\frac{\partial r^n}{\partial x} + y\frac{\partial r^n}{\partial y} + z\frac{\partial r^n}{\partial z} \]
	
	Вычислим \( x\frac{\partial r^n}{\partial x} = x\frac{\partial r^n}{\partial r}\frac{\partial r}{\partial x} = x\cdot nr^{n-1}\cdot\frac{x}{r} = x^2nr^{n-2} \)
	
	Тогда дивергенция:
	\[ \divergence(r^n\vec{r}) = 3r^n + nr^{n-2}(x^2 + y^2 + z^2) = 3r^n + nr^n = (3 + n)r^n \]
	
	Пусть \( n = 0 \). Тогда \( \divergence(r^0\vec{r})\, = 3 \).
	
	Пусть \( n = -1 \). Тогда \( \divergence(r^{-1}\vec{r})\, = \frac{2}{r} \).
	
	Пусть \( n = -2 \). Тогда \( \divergence(r^{-2}\vec{r})\, = \frac{1}{r^2} \).
	
	Пусть \( n = -3 \) -- Кулоновское поле. Тогда \( \divergence(r^{-3}\vec{r})\, = 0 \) -- соленоидальное поле, у него всюду, кроме точки \( O \), дивергенция равна нулю, но в точке \( O \) поле не определено.
	\end{solution}
	
	\begin{example}
	Вычислить дивергенцию поля \( \vec{v} \) скоростей точек твердого тела, вращающегося со скоростью \( \vec{\omega} = \{ 0, 0, \omega \} \).
	\end{example}
	\begin{solution}
	\[ \vec{v} = \divergence(\vec{\omega}\times\vec{r})\, = \begin{bmatrix}
	\vec{e}_x & \vec{e}_y & \vec{e}_z \\
	0 & 0 & \omega \\
	x & y & z
	\end{bmatrix} = \omega\{ -y, x, 0 \} \]
	
	Тогда дивергенция поля \( \vec{v} \):
	\[ \divergence{v} = \frac{\partial (-\omega y)}{\partial x} + \frac{\partial (\omega x)}{\partial y} + \frac{\partial 0}{\partial z} = 0 \]
	
	Это поле -- соленоидально.
	\end{solution}
%	\begin{example}
%	Вычислить дивергенцию поля \( \vec{a} = \frac{\vec{v}\times\vec{r}}{r^3} \) и построить векторные линии этого поля.
%	\end{example}
%	\begin{solution}
%	Распишем векторное произведение:
%	\[ \vec{v}\times\vec{r} = (\vec{\omega}\times\vec{r})\times\vec{r} = -\vec{r}\times(\vec{\omega}\times\vec{r}) = -\vec{\omega}r^2 + \vec{r}(\vec{\omega}\cdot\vec{r}) = \omega z\vec{r} - r^2\vec{omega} \]
%	
%	Тогда дивергенция:
%	\[ \divergence{a} = \divergence(\omega z\vec{r}\, - r^2\vec{omega}) = \divergence\left(\frac{\omega z}{r^3}\vec{r}\right) - \divergence\left(\frac{1}{r}\vec{\omega}\right) \]
%	
%	Посчитаем каждую дивергенцию по отдельности:
%	\[ \begin{array}{l}
%	 \divergence\left(\frac{\omega z}{r^3}\vec{r}\right) = \frac{\omega z}{r^3}\cdot3 + \vec{r}\cdot\{ \omega z\cdot\frac{-3}{r^4}\cdot\frac{x}{r}, \omega z\cdot\frac{-3}{r^4}\cdot\frac{y}{r}, \omega z\cdot\frac{-3}{r^4}\cdot\frac{z}{r} + \frac{\omega}{r^3} \} = \\
%	= 3\frac{\omega z}{r^3} - 3\omega z\cdot\frac{x^2}{r^5} - 3\omega z\cdot\frac{y^2}{r^5} - 3\omega z\cdot\frac{z^2}{r^5} + \frac{\omega}{r^3}z = \frac{\omega}{r^3}z
%	\end{array} \]
%	\[ \begin{array}{c}
%	\divergence\left(\frac{1}{r}\vec{\omega}\right) = \frac{1}{r}\cdot 0 + \vec{\omega}\cdot\{ \frac{-1}{r^2}\cdot\frac{x}{r}, \frac{-1}{r^2}\cdot\frac{y}{r}, \frac{-1}{r^2}\cdot\frac{z}{r} \} = -\frac{\omega}{r^3}z
%	\end{array} \]
%	
%	Вся дивергенция:
%	\[ \divergence{a} = \frac{\omega}{r^3}z + \frac{\omega}{r^3}z = 2\frac{\omega}{r^3}z \]
%	\end{solution}
	
\subsection{Теорема Остроградского}

	Теорема о преобразовании поверхностного интеграла в объёмный является одной из центральных теорем в векторном анализе.
	
	\begin{theorem}
	Пусть поле \( \vec{a} \) имеет непрерывные частные производные \( \frac{\partial a_x}{\partial x} \), \( \frac{\partial a_y}{\partial y} \) и \( \frac{\partial a_z}{\partial z} \). Тогда поток поля \( \vec{a} \) через произвольную \textit{замкнутую} поверхность \( S \) равен объёмному интегралу по объёму \( V \), ограниченному \( S \), от дивергенции поля \( \vec{a} \):
	\begin{equation}
		\oiint\limits_S \vec{a}\cdot d\vec{S} = \iiint\limits_V \divergence{a} dV \label{eq4:nn1}
	\end{equation}
	\end{theorem}
	
	\begin{comment} Определение объёмного интеграла.
	
	Пусть в области \( V \) задано скалярное поле \( u(x, y, z) \). Разобъём объём \( V \) на малые объёмы \( \Delta V_k \), и пусть точка \( M_k(x_k, y_k, z_k) \) лежит в \( \Delta V_k \), а функция \( u(x, y, z) \) в этой точке принимает значение \( u(x, y, z) = u_k(x_k, y_k, z_k) \).
	
	\begin{definition}
	Предельная сумма \( \lim_{N\to\infty \\ \Delta V_k\to 0} \sum\limits_{k = 1}^N u_k \Delta V_k \equiv \iiint\limits_V u(x, y, z)dV \) называется \textbf{объёмным интегралом} функции \( u \) по объёму \( V \).
	\end{definition}
	\end{comment}
	
	\textbf{Доказательство} теоремы Остроградского:
	
	Докажем, что равенство (\ref{eq4:nn1}) может выполнятся с любой степени точности.
	
	Разобьём область \( V \) на малые \( \Delta V_k \), настолько малые, что для них (\ref{eq4:n2}) выполняется с наперёд заданной точностью \( \varepsilon \):
	\begin{equation}
		\left|\frac{1}{V_k}\oiint\limits_S\vec{a}\cdot d\vec{S} - \divergence{a}\right| < \varepsilon \label{eq4:nn3}
	\end{equation}
	
	Домножим (\ref{eq4:nn3}) на \( V_k \) и просуммируем по всем \( V_k \):
	\begin{equation}
		\left|\sum\limits_k \oiint\limits_{S_k} \vec{a}\cdot d\vec{S} - \sum\limits_k \divergence{a}V_k\right| < \varepsilon V \label{eq4:nn4}
	\end{equation}
	
	Рассмотрим первую сумму в левой части (\ref{eq4:nn4}). Её можно разбить на две суммы:
	\[ \sum\limits_k \oiint\limits_{S_k} \vec{a}\cdot d\vec{S} = \underbrace{\sum\limits_{k’}\iint\limits_{S_{k’}} \vec{a}\cdot d\vec{S}}_{\#I} + \underbrace{\sum\limits_{k”}\iint\limits_{S_{k”}} \vec{a}\cdot d\vec{S}}_{\#II} \]
	
	\( \#I \) ведется по внутренним смежным участкам замкнутых поверхностей \( V_k \), а \( \#II \) ведется по участкам, которые являются фрагментами внешней поверхности \( S \). 
	
	В сумме \( \#I \) все смежные участки встречаются попарно и попарно уничтожаются:
	\[ \vec{a}\cdot d\vec{S}_{k’} = \vec{a}\cdot d\vec{S}_{k’ + 1} \]
	
	Следовательно, первая сумма равна нулю:
	\[ \sum\limits_{k’}\iint\limits_{S_{k’}} \vec{a}\cdot d\vec{S} = 0 \]
	
	Вторая сумма дает \( \#II = \oiint_S \vec{a}\cdot d\vec{S} \).
	
	Таким образом:
	\[ \left|\oiint\limits_S \vec{a}\cdot d\vec{S} - \sum\limits_{k = 1}^{N}\divergence{a}V_k\right| < \varepsilon V \]
	
	Но, так как при \( N \to \infty \): \( V_k \to 0 \), то \( \sum\limits_{k = 1}^{N}\divergence{a}V_k \), по определению, становится объемным интегралом \( \iiint\limits_V \divergence{a}dV \):
	
	\[ \left|\oiint\limits_S \vec{a}\cdot d\vec{S} - \iiint\limits_V \divergence{a}dV\right| < \varepsilon V \]
	
	А так как \( \varepsilon \) можно сделать сколь угодно малым, то в итоге получаем равентво (\ref{eq4:nn1}).
	
	\begin{corollary}
	Если поле \( \vec{a} \) -- соленоидально, то есть \( \divergence{a} = 0 \), то из (\ref{eq4:nn1}) следует, что
	\[ \oiint\limits_S \vec{a}\cdot d\vec{S} = \iiint\limits_V \divergence{a}dV = 0 \]
	\end{corollary}
	
	\begin{corollary}
	Пусть имеется изолированный источник поля \( \vec{a} \), то есть \( \divergence{a} = 0 \) во всем пространстве, кроме точки \( O \), в которой находится источник.
	
	Примером такого поля может служить поле \( \vec{a} = \frac{\vec{r}}{r^3} \) -- кулоновское поле.  Дивергенция этого поля всюду, кроме точки \( O \), равна нулю.
	
	Тогда поток поля \( \vec{a} \) через замкнутую поверхность \( S \) не зависит от ее формы.
	
	\begin{proof}
	\begin{enumerate}
	\item Пусть точка \( O \), где находится источник, лежит вне поверхности \( S \). Тогда поток для любой поверхности \( S \):
	\[ \oiint\limits_S \vec{a}\cdot d\vec{S} = \iiint\limits_S \divergence{a} dV = 0 \]
	
	\item Пусть поверхность \( S_1 \) охватывает источник \( q \). Охватим его второй поверхностью \( S_2 \), отличной от \( S_1 \), и покажем, что потоки через эти поверхности являются равными: \( \Phi_1 = \Phi_2 \):
	\[ \begin{array}{l}
		\Phi_1 = \oiint_{S_1} \vec{a}\cdot\vec{n}_1dS \\
		\Phi_1 = \oiint_{S_2} \vec{a}\cdot\vec{n}_2dS
	\end{array} \]
	
	Пусть \( V \) -- это область между \( S_1 \) и \( S_2 \). Для нее \textit{внешними} нормалями являются \( \vec{n}_1’ \) и \( \vec{n}_2 \), где \( \vec{n}_1’ = -\vec{n}_1 \).
	
	Тогда, по теореме Остроградского:
	\[ \oiint\limits_{S_1 + S_2} \vec{a}\cdot d\vec{S} = \iiint\limits_V \divergence{a}dV = 0, \]
	так как в области \( V \): \( \divergence{a} = 0 \).
	
	В силу аддитивности потока получаем:
	\[ \oiint\limits_{S_1 + S_2} \vec{a}\cdot d\vec{S} = \oiint\limits_{S_1} \vec{a}\cdot\vec{n}_1’dS + \oiint\limits_{S_2} \vec{a}\cdot\vec{n}_2dS \]
	
	А так как \( \vec{n}_1’ = -\vec{n}_1 \), то:
	\[ \oiint\limits_{S_1} \vec{a}\cdot\vec{n}_1dS = \oiint\limits_{S_2} \vec{a}\cdot\vec{n}_2dS, \]
	независимо от формы \( S_1 \) и \( S_2 \), что и требовалось доказать.
	\end{enumerate}
	\end{proof}
	\end{corollary}
	
\subsection{Применение теоремы Остроградского}

	Теорема Остроградского является удобным инструментом для вычисления потоков, когда поверхностный интеграл преобразуется в более легкий объемный. Формально, теорема Остроградского применима только к замкнутым поверхностям. Однако, если нужно вычислить интеграл по незамкнутой поверхности, то ее удобно замкнуть нужным фрагментом, вычислить поток через замкнутую поверхность, а затем, в силу аддитивности потока, вычесть поток через замыкающий фрагмент.
	
	\begin{example}
	Вычислить поток поля \( \vec{a} = \vec{r} \) через замкнутую поверхность \( S \), ограничивающую область \( V \).
	\end{example}
	\begin{solution}
	
	По теореме Остроградского:
	\[ \oiint\limits_S \vec{r}\cdot d\vec{S} = \iiint\limits_V \divergence{r}dV = 3V \]
	и не зависит от того, лежит ли источник внутри или вне поверхности \( S \) (см. пункт 4.4.).
	\end{solution}
	
	\begin{example}
	Вычислить поток кулоновского поля \( \vec{a} = \vec{E} = k\frac{\vec{r}}{r^3} \) через замкнутую поверхность \( S \).
	\end{example}
	\begin{solution}
	
	Это поле имеет изолированный источник в точке \( O \), то есть \( \divergence{a} \equiv 0 \) всюду, кроме точки \( O \).
	
	\begin{enumerate}
	\item Пусть поверхность \( S \) не охватывает точку \( O \).
	Тогда, по теореме Остроградского: \( \Phi = \oiint\limits_S \vec{a}\cdot d\vec{S} = \iiint\limits_V \divergence{a}dV = 0 \).

	\item Пусть поверхность \( S \) охватывает точку \( O \).
	
	Охватим точку \( O \) сферой поверхностью \( 4\pi R^2 \), где \( R \) -- радиус сферы.
	
	Тогда, в силу следствия 2 из теоремы Остроградского:
	\[ \oiint\limits_S \vec{E}\cdot d\vec{S} = \oiint\limits_{4\pi R^2}\vec{E}\cdot d\vec{S} \]
	
	 А так как \( \vec{E} = k\frac{\vec{r}}{r^3} \), то:
	 \[ \oiint\limits \vec{E}\cdot d\vec{S} = k\oiint\limits_{4\pi R^2}\frac{\vec{r}\cdot d\vec{S}}{r^3} \]
	 
	 На сфере \( \vec{r} \uparrow\uparrow d\vec{S} \) и \( r = R \), то поток:
	 \[ \Phi = k\oiint\limits_{4\pi R^2} \frac{R}{R^3}dS = k\frac{1}{R^2}\oiint\limits_{4\pi R^2} dS = 4\pi k \]
	 
	 Таким образом, для любой поверхности \( S \):
	 \[ \Phi = 4\pi k = 4\pi \frac{q}{4\pi\varepsilon_0} = \frac{q}{\varepsilon_0} \]
	\end{enumerate}
	\end{solution}
	
	\begin{example}
	Вычислить поток поля \( \vec{a} = \{ xy, 2y, -z \} \) через сферу \( S \): \( x^2 + y^2 + z^2 = R^2 \).
	\end{example}
	\begin{solution}
	
	По теореме Остроградского:
	\[ \oiint\limits_S \vec{a}\cdot d\vec{S} = \iiint\limits_V \divergence{a}dV \]
	
	Дивергенция поля \( \vec{a} \) равна:
	\[ \divergence{a} = \frac{\partial a_x}{\partial x} + \frac{\partial a_y}{\partial y} + \frac{\partial a_z}{\partial z} = y + z - 1 \]
	
	Тогда поток:
	\[ \Phi = \iiint\limits_{\frac{4}{3}\pi R^3} = \iiint\limits_{\frac{4}{3}\pi R^3} ydV + \iiint\limits_{\frac{4}{3}\pi R^3} dV = \frac{4}{3}\pi R^3 + \iiint\limits_{\frac{4}{3}\pi R^3} ydV = \frac{4}{3}\pi R^3 \]
	\end{solution}
