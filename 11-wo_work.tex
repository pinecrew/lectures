\section{Макроэкономическое неравновесие: безработица и инфляция}

Безработный -- это лицо, которое в р/м периоде не имело работы, занималось
активным ее поиском и готово приступить к ее выполнению.

По российскому законодательству, безработный -- трудоспособный гражданин, не
имеющий работы и заработка, зарегистрированный в органах службы занятости в
целях поиска подходящей работы, и готов приступить к ней.

Виды безработицы: скрытая -- формально занятые, но фактически безработные лица,
или лица, желающие работать, но не зарегистрированные в качестве безработных;
зарегистрированная -- незанятое население, занимающееся поиском работы и
официально взятое на учет.

Согласно другой классификации:
\begin{enumerate}
    \item фрикционная -- связана с поиском рабочего места в случае добровольного
    перехода работника с одной работы на другую;
    \item структурная -- вызвана несоответствием структуры спроса на труд и
    структуры имеющейся рабочей силы;
    \item циклическая -- связана с невозможностью найти работу по специальности
    в условиях экономического спада;
    \item естественная -- фрикционная + структурная безработицы (нормальный
    уровень естественной безработицы: 5-6\%);
    \item фактическая -- естественная + циклическая.
\end{enumerate}

Помимо этих выделяют технологическую, сезонную, добровольную, вынужденную
формы безработицы.

Уровень безработицы рассчитывается следующим образом:
\[
    u = U/L,
\]
где \( U \) -- численность безработных, \( L = E + U \) -- численность рабочей
силы, \( E \) -- численность занятых.

Численность рабочей силы есть численность экономически активного населения, но
не является численностью трудоспособного населения.

К категории не включаемых в состав рабочей силы относятся:
\begin{enumerate}
    \item институциональное население -- находится на содержании государственных
    институтов:
    \begin{enumerate}
        \item отбывающие срок заключения,
        \item находящиеся в специализированных медицинских учреждениях,
        \item инвалиды;
    \end{enumerate}
    \item выбывшие из состава рабочей силы:
    \begin{enumerate}
        \item студенты дневного отделения,
        \item пенсионеры (по возрасту или по состоянию здоровья),
        \item домохозяйки,
        \item бродяги,
        \item лица, отчаявшиеся найти работу и прекратившие ее поиски.
    \end{enumerate}
\end{enumerate}

Неравновесие на рынке труда обусловлено:
\begin{enumerate}
    \item законодательное закрепление минимума заработной платы;
    \item использование контрактной системы на рынке труда (профсоюзы,
    индивидуальные трудовые отношения);
    \item незаинтересованность самих фирм в изменении уровня оплаты труда.
\end{enumerate}

Последствия безработицы:
\begin{enumerate}
    \item снижение доходов;
    \item проблемы с психическим здоровьем;
    \item потеря квалификации;
    \item экономические последствия (потеря ВВП);
    \item ухудшение криминогенной ситуации;
    \item ухудшение динамики роста интереса населения к труду;
    \item снижение уровня обеспеченности домохозяйств.
\end{enumerate}

Закон Оукена:
\begin{quotation}
    \emph{
    повышение фактического уровня безработицы на 1\% над величиной естественного
    нормы безработицы вызывает снижение на 2,5--3\% объема ВВП по сравнению с
    потенциально возможным.
    }
\end{quotation}

Расчет потенциального объема ВНП, который м.б. достигнут в условиях полной
занятости (естественного уровня безработицы):
\[
    \emph{ВНП}_\emph{П} = \emph{ВНП}_\emph{ф} + \Delta\emph{ВНП},
\]
где \( \emph{ВНП}_\emph{ф} \) -- фактический ВНП; \( \Delta\emph{ВНП} \) -- абсолютное отставание
фактического ВНП от потенциального.

Определение превышения фактического уровня безработицы над естественным уровнем
(6\%):
\[
    \Delta\emph{У}_\emph{без} = \emph{У}_\emph{без}^\emph{ф} -
    \emph{У}_\emph{без}^\emph{ес}.
\]

Определение относительного отставание фактического ВНП от потенциального по
закону Оукена:
\[
    \Delta B(\%) = \Delta\emph{У}_\emph{без} \times R_0,
\]
где \( R_0 = 2,5 \) -- коэффициент Оукена.

Определение абсолютного (в ден. ед.) отставания фактического ВНП от
потенциального:
\[
    \Delta\emph{ВНП} = \frac{\emph{ВНП}_\emph{ф} \times \Delta B}{100\%}.
\]

\subsection{Инфляция}
Инфляция -- это переполнение сферы обращения денежными знаками сверх реальных
потребностей товарооборота, вызывающее обесценение денег и рост общего уровня
цен.

Инфляция измеряется с помощью статистических показателей -- индексов цен.

Наиболее распространенные индексы цен:
\begin{enumerate}
    \item индекс потребительских цен (ИПЦ);
    \item индекс цен товаров производственного назначения;
    \item дефлятор ВВП.
\end{enumerate}

Расчет индекса цен:
\[
    \emph{Индекс цен} = \frac{\emph{Цена рыночной корзины в данном году}}
    {\emph{Цена рыночной корзины в базовом году}} \times 100\%.
\]

Определение темпа роста цен (темпа инфляции):
\[
    \emph{Темп инфляции} = \frac{\emph{индекс цен текущего года} -
    \emph{индекс цен базового года}}{\emph{индекс цен базового года}} \times
    100\%.
\]

Правило величины 70: позволяет посчитать число лет, на протяжении которых
происходит удвоение уровня цен:
\[
    \emph{Количество лет} = \frac{70}
    {\% \emph{темп ожидаемой ежегодной инфляции}}.
\]

Критерии классификации инфляции.
\begin{enumerate}
    \item по темпам роста цен:
    \begin{enumerate}
        \item умеренная инфляция (ползучая) -- до 10\% в год,
        \item галопирующая инфляция -- до 200\% в год,
        \item гиперинфляция -- 50\% в месяц;
    \end{enumerate}
    \item по степени сбалансированности:
    \begin{enumerate}
        \item сбалансированная -- цены на все товары растут примерно одинаково,
        \item несбалансированная -- цены на некоторые товары растут быстрее,
        чем на другие;
    \end{enumerate}
    \item по ожидаемости и предсказуемости:
    \begin{enumerate}
        \item неожиданная,
        \item ожидаемая;
    \end{enumerate}
    \item по формам:
    \begin{enumerate}
        \item открытая,
        \item скрытая (подавленная).
    \end{enumerate}
\end{enumerate}

Социально-экономические последствия инфляции:
\begin{enumerate}
    \item перераспределение национального дохода в пользу государства и
    монополий;
    \item обесценивание денежных доходов;
    \item обесценивание сбережений населения;
    \item трудности адаптации налоговой системы к инфляционным процессам;
    \item отрицательное влияние на кредит и кредитную систему;
    \item обесценивание инвестиционных средств;
    \item деформации потребительского спроса;
    \item глубокое расстройство денежной системы.
\end{enumerate}

Антиинфляционная политика правительства.
\begin{enumerate}
    \item Шоковая терапия:
    \begin{enumerate}
        \item деноминация -- ,
        \item нулификация -- введение новой денежной системы,
        \item девальвация -- снижение курса валюты,
        \item таргетирование -- регулирование роста денежной массы;
    \end{enumerate}
    \item градуирование -- постепенное уменьшение денежной массы;
    \item адаптационная политика -- политика регулирования цен и доходов.
\end{enumerate}

Способы адаптации к инфляции:
\begin{enumerate}
    \item индексация доходов;
    \item поиск дополнительных источников дохода;
    \item использование контрактной системы, включая специальный пункт об
    инфляционной корректировке заработной платы;
    \item заключение контрактов на выполнение только краткосрочных проектов,
    реализация которых позволяет быстро окупить инвестиции;
    \item изыскание дополнительных внешних источников дохода.
\end{enumerate}

Кривая Филлипса:
увеличение безработицы ведет к замедлению роста заработной платы и цен.
Снижение безработицы сопровождается повышением з. платы и цен.
