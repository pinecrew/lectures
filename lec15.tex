\chapter{Переходные процессы в простейших цепях}

\section{Понятие о переходных процессах}

	\begin{definition}
        \textbf{Переходными} называются процессы перехода от одного
        периодического режима работы в цепи к другому периодическому, чем-либо
        отличающимся от первого: амплитудой, частотой или формой сигнала. Причем
        периодическим является режим постоянного тока или отсутствие тока.
	\end{definition}
	
	Переходные процессы вызываются коммутациями в цепях, то есть включением или
    выключениями рубильников или электронных ключей. Теоретически, переходные
    процессы длятся бесконечно долго, однако реальное время переходных процессов
    \( \tau \sim 10^{-9}\ldots10^{-2} \)с.
	
	Переходные процессы могут быть только в цепях, содержащих как активное
    сопротивление \( R \), так и реактивное \( R \), \( L \). Если цепь содержит
    только активное сопротивление, то переходных процессов не будет, новый режим
    установится мгновенно, если же цепь содержит только реактивное
    сопротивление, то переходных процессов либо нет, либо они длятся бесконечно
    долго.
    
    Анализ переходных процессов основан на решении уравнения Кирхгофа, которое
    является дифференциальным и, вообще говоря, второго порядка. Мы будем
    рассматривать переходные процессы только в цепях, содержащих два или три
    элемента: \( RC \), \( RL \) и \( RLC \).

	Для таких цепей по правилу Кирхгофа
	\[
        u_L + u_R + u_C = u(t), 
    \]
	или, используя функциональные определения элементов  \( R \)
    (\ref{eq13:n1}), \( L \)  (\ref{eq13:n2}) и \( C \) (\ref{eq13:n3})
	\[
        L\frac{\dd i}{\dd t} + iR + \frac{q}{C} = u(t).
    \]
	Дифференцируя по времени, получим:
	\begin{equation}
		\frac{\dd^2i}{\dd t^2} + 2\beta\frac{\dd i}{\dd t} + \omega_0^2i = 
        \frac{1}{L}\cdot\frac{\dd u}{\dd t},
        \label{eq15:1}
	\end{equation}
	где \( \beta = \frac{R}{2L} \), \( \omega_0 = \frac{1}{\sqrt{LC}} \). А так
    как \( i = C\frac{\dd u_C}{\dd t} \), то
	\begin{equation}
		LC\ddot{u} + RC\dot{u} + u = u(t).
        \label{eq15:2}
	\end{equation}
	Общим решением (\ref{eq15:1}) (или (\ref{eq15:2})) является уравнение:
	\[
        i(t) = i_{\textit{общ}}^{\text{однор}}(t) + i_{\textit{частн}}.
    \]
	
	Как известно,
    \[
        i_{\textit{общ}}^{\text{однор}} = e^{-\beta t}(A\sin\omega t +
        B\cos\omega t).
    \]
    При \( t \to \infty \) оно стремится к нулю. Оно зависит от начальных
    условий и не зависит от приложенного напряжения \( u(t) \).
	
	\( i_{\textit{частн}} \) не зависит от начальных условий, а только от
    \( u(t) \). Оно описывает установившийся процесс.
	
	Тогда \( \left.i(t)\right|_{t\to\infty} =  i_{\textit{частн}} \) --
    вынужденный установившийся процесс.
	
\section{Разряд конденсатора через активное сопротивление (цепь \textit{RC})}

	Уравнение Кирхгофа:
	\[ u_R - u_C = 0, \]
	\[ iR - u_C = 0. \]
	А так как для разрядного тока
    \[
        i = -\frac{\dd q}{\dd t} = -C\frac{\dd \phi}{\dd t},
    \]
    то
	\[ RC\dot{u} + u = 0, \]
	или
	\begin{equation}
		\dot{u} + \frac{1}{\tau}u = 0,
        \label{eq15:3}
	\end{equation}
	где \( \tau = RC \) -- постоянная времени цепи \(  RC \) (время релаксации).
	
	Решение (\ref{eq15:3}):
	\[
        u = Ae^{-\frac{t}{\tau}}.
    \]
	Коэффициент \( A \) определяется из начального условия: \( u(-0) = U_0 \)
    -- начальное напряжение на емкости.	А так как напряжение на \( С \) скачком
    измениться не может, так как это противоречило бы уравнению (\ref{eq15:3}),
    где \( u \) -- конечно, то \( u(+0) = u(-0) = U_0 \). Это дает
    \[
        A = U_0,
    \]
	и тогда:
	\begin{equation}
		u(t) = U_0e^{-\frac{t}{\tau}},
        \label{eq15:4}
	\end{equation}
	где \( \tau = RC \). А разрядный ток
	\begin{equation}
		i(t) = C\frac{U_0}{\tau}e^{-\frac{t}{\tau}} =
        \frac{U_0}{R}e^{-\frac{t}{\tau}}.
	\end{equation}
	
	\begin{remark}
	    Реально, в качестве времени переходного процесса берется время
        \( \Delta t_{\textit{перех}} \approx 3\tau \), когда процесс
        заканчивается на \( 1 - e^{-3} \approx 95\% \).
	\end{remark}
	
	\begin{example}
	    Пусть \( R = 1 \)кОм, \( C = 1 \)мкФ, тогда \( \tau = RC =
        10^{-3} \text{с} = 1 \)мс, а время переходного процесса
        \( \Delta t = 3 \)мс.
	\end{example}
	
\section{Включение постоянного напряжения в цепь \textit{RC}}
	Уравнение Кирхгофа:
	\[
        u_R + u_C = U_0,
    \]
	\[
        iR + u_C = U_0.
    \]
	Ток \( i \) -- зарядный ток:
    \[
        i = C\frac{\dd u}{\dd t},
    \]
	\[
        RC\dot{u} + u = U_0,
    \]
	\[
        \dot{u} + \frac{1}{\tau}u = \frac{U_0}{RC}.
    \]
	Решение этого уравнения:
	\[
        u(t) = Ae^{-\frac{t}{\tau}} + u_{\textit{частн}}.
    \]
	
	Частное решение:
	\[
        u_{\textit{частн}} = \frac{u_0}{RC} = U_0,
    \]
	
	Тогда:
	\[
        u(t) = Ae^{-\frac{t}{\tau}} + U_0.
    \]
	
	Коэффициент \( A \) находится из начального условия \( u(+0) = u(-0) = 0 \).
    Это дает \(  A = -U_0 \) и тогда
	\begin{equation}
		u(t) = U_0(1 - e^{-\frac{t}{\tau}}).
	\end{equation}
	За \( \Delta t = 3\tau \) напряжение подойдет к своему установившемус
    значению на \( 95\% \).
	Зарядный ток тогда:
	\[
        i = C\frac{\dd u}{\dd t} = \frac{U_0}{R}e^{-\frac{t}{\tau}}.
    \]
	
\section{Включение постоянного напряжения в цепь \textit{RL}}

	Уравнение Кирхгофа:
	\[
        u_R + u_L = U_0.
    \]
	Применяя функциональные определения элементов:
	\[
        L\frac{\dd i}{\dd t} + iR - U_0 = 0,
    \]
	или
	\[
        \frac{\dd i}{\dd t} + \frac{1}{\tau}i = \frac{U_0}{L},
    \]
	где \( \tau = \frac{L}{R} \) -- постоянная времени цепи \( RL \). Решение
	\[
        i(t) = Ae^{-\frac{t}{\tau}} + i_{\textit{частн}}.
    \]
	Частное решение:
	\[
        i_{\textit{частн}} = \frac{U_0\tau}{L} = \frac{U_0}{R}.
    \]
	Тогда ток:
	\[
        i(t) = Ae^{-\frac{t}{\tau}} + \frac{U_0}{R}.
    \]
	
	Начальное условие: \( i(+0) = i(-0) = 0 \). Тогда коэффициент
    \[
        A = -\frac{U_0}{R}.
    \]
	Тогда ток:
	\begin{equation}
		i(t) = \frac{U_0}{R}(1 - e^{-\frac{t}{\tau}}).
        \label{eq15:5}
	\end{equation}
	
	\begin{remark}
        Если \( R = 0 \), то \( i(t) \):
        \begin{enumerate}
        \item из уравнения Кирхгофа:
            \[
                L\frac{\dd i}{\dd t} = U_0,
            \]
            его решение
            \[
                i = \frac{U_0}{L}t;
            \]
            
        \item из решения (\ref{eq15:5}): раскладывая \( e^{-\frac{t}{\tau}} \)
            в ряд Маклорена:
            \[
                e^{-\frac{t}{\tau}} \approx 1 - \frac{t}{\tau},
            \]
            следовательно:
            \[
                i \sim \frac{U_0t}{R\tau} = \frac{U_0}{L}t. 
            \]
        \end{enumerate}
	\end{remark}
	
\section{Отключение постоянного напряжения от цепи \textit{RL}}
	Пусть в момент времени \( t = 0 \) размыкаетс ключ \( K \). При \( t < 0 \)
    в цепи был ток
    \[
        i = \frac{U_0}{R}.
    \]
    При \( t = 0 \) ток должен быть равным нулю. Однако ток через катушку
    \( L \) скачком изменится не может. При размыкании контактов \( K \),
    сопротивление в них быстро растет, но не скачком. При этом ток \( i \)
    быстро убывает. Следовательно, в катушке наводится ЭДС:
	\[
        \EDS_{\textit{си}} = -L\frac{\dd i}{\dd t}.
    \]
	И чем больше скорость убывания тока \( \frac{\dd i}{\dd t} \), тем больше
    ЭДС \( \EDS_{\textit{си}} \). При очень быстром изменении тока ЭДС
    \( \EDS_{\textit{си}} \to \infty \) и в контактах будет пробой.
	
	\begin{definition}
        Напряжение на элементах цепей, превышающее приложенное к цепи,
        называется \textbf{перенапряжением}.
	\end{definition}
	
	Во избежание искры в контактах при размыкании цепи с индуктивностью в нее
    вводят параллельное сопротивление: \textit{picture}

	Второе уравнение Кирхгофа при \( t > 0 \):
	\[
        i(R_1 + R) + L\frac{\dd i}{\dd t} = 0,
    \]
	\[
        \frac{\dd i}{\dd t} + \frac{1}{\tau}i = 0,
    \]
	где \( \tau = \frac{L}{R_1 + R} \) -- постоянная времени \( (R_1 + R)L \).
	
	Его решение:
	\[
        i(t) = Ae^{-\frac{t}{\tau}}.
    \]
	Из начального условия \( i(0) = \frac{U_0}{R} \) найдем коэффициент
    \[
        A = \frac{U_0}{R}.
    \]
	Тогда ток:
	\begin{equation}
		i(t) = \frac{U_0}{R}e^{-\frac{t}{\tau}}
	\end{equation}
	
	Этот ток идет через сопротивление \( R \), на нем выделяется напряжение:
	\begin{equation}
		u_{R_1} = iR_1 = U_0\frac{R_1}{R}e^{-\frac{t}{\tau}}
	\end{equation}
	
	Если \( R_1 \gg R \), то \( u_{R_1} \gg U_0 \) -- перенапряжение.
	
	\begin{example}
        Пусть \( R_1 = 50 \)кОм (человек), \( R = 10 \)Ом -- сопротивление
        витков катушки, \( U_0 = 1 \)В, \( L = 1 \)Гн.
    \end{example}
    \begin{solution}
        Тогда при размыкании на \( R_1 \) будет напряжение
        \[
            u_{R_1} = U_0\frac{R_1}{R}e^{-\frac{t}{\tau}} = \frac{50000}{10} =
            5 \text{кВ} = 5000 U_0.
        \]
        Время разряда:
        \[
            \tau = \frac{L}{R_1 + R} = \frac{1}{50010} \approx 20 \text{мкс},
        \]
        а энергия, выделяющаяся во время разряда:
        \[
            W = \frac{Li^2}{2} = \frac{L\left(\frac{U_0}{R}\right)^2}{2} =
            \frac{1}{2}(100\cdot10^{-3})^2 = 0,01 \text{Дж}.
        \]
	       
    \end{solution}

	\begin{remark}
        Если \( R_1 = R = 0 \), то уравнение Кирхгофа:
        \[
            L\frac{\dd i}{\dd t} = 0 \Rightarrow i_L = \const \Rightarrow
            \Phi_L = Li_L = \const.
        \]
        Это означает, что короткозанкнутая сверхпроводящая катушка держит ток
        \textit{вечно}.
	\end{remark}
	
\section{Отключение постоянного напряжения от последовательной цепи \textit{RLC}}

	При \( t > 0 \) в контуре \( RLC \) происходят свободные колебания, при
    \( t = 0 \) \( u_C = U_0 \).
	
	Если \( \beta < \omega_0 \), то
    \[
        i(t) = \frac{U_0}{\omega L}e^{-\beta t}\sin\omega t.
    \]
	
	Если \( \beta \geq \omega_0 \), то
    \[
        i(t) = \frac{U_0}{L}te^{-\beta t}.
    \]
	
\section{Включение постоянного напряжения в цепь \textit{RLC}}

	По второму правилу Кирхгофа:
	\[
        u_L + u_R + u_C = U_0 (t > 0),
    \]
	\[
        L\frac{\dd i}{\dd t} + Ri + u_C = U_0.
    \]
	
	Дифференцируя это по \( t \), получим:
	\[
        L\frac{\dd^2 i}{\dd t^2} + R\frac{\dd i}{\dd t} + \frac{1}{C}i = 0.
    \]
	
	При \( \beta < \omega_0 \) решение:
	\[
        i(t) = \left.\frac{U_0}{\omega L}e^{-\beta t}
        \sin\omega t\right|_{t\to\infty} \to 0.
    \]
	Формально, точный вид \( u_C(t) \) можно получить из уравнения Кирхгофа:
	\[
        L\frac{\dd i}{\dd t} + Ri + u_C = U_0.
    \]
	А так как
    \[
        i = C\frac{\dd u}{\dd t},
    \]
    то
	\[
        LC\ddot{u} + RC\dot{u} + u = U_0,
    \]
	\[
        \ddot{u} + 2\beta\dot{u} + \omega_0^2u = \frac{U_0}{LC}.
    \]
	Его решение:
	\[
        u(t) = u^{\text{однор}}_{\textit{общ}} + u_{\textit{частн}}.
    \]
	Видно, что
    \[
        u^{\text{однор}}_{\textit{общ}} = e^{-\beta t}(A\sin\omega t +
        B\cos\omega t) + \frac{U_0}{\omega_0^2LC}.
    \]
	Начальные условия:
	При \( t = +0 \):
	\[
        u_C = 0;\ \dot{u}_C = 0,
    \]
	\[
        u_C(t) = U_0(1 - e^{-\beta t}\cos\omega t).
    \]
	
	% вывод
	\textit{Таким образом, если на последовательную цепь \( RLC \) подавать
    импульсное напряжение, то на конденсаторе будет искажение импульсов:}
	% тут эпюры

\section{Включение синусоидального напряжения в цепь \textit{RLC}}

	Пусть в момент \( t = 0 \) к последовательной цепи \( RLC \) подключают
    синусоидальное напряжение \( u = U\sin\omega t \), где \( \omega \) --
    частота генератора.
	
	Уравнение Кирхгофа при \( t > 0 \):
	\[
        u_L + u_R + u_C = u,
    \]
	\[
        L\frac{\dd i}{\dd t} + iR + u_C = U\sin\omega t.
    \]
	
	Дифференцируя это по времени:
	\[
        L\frac{\dd^2 i}{\dd t^2} + R\frac{\dd i}{\dd t} + \frac{1}{C}i =
        U\omega\cos\omega t.
    \]
	
	Характер переходных процессов в контуре \( RLC \) будет зависеть от
    следующих факторов:
	\begin{itemize}
        \item соотношения между \( \omega \) и \( \omega_0 \);
        \item соотношения между \( \omega_0 \) и \( \beta \);
        \item от фазы включения внешнего напряжения.
	\end{itemize}
	
	Есть три характерных варианта:
	\begin{enumerate}
        \item \( \omega = \omega_0 \); \( \beta \in \Re \).
        \item \( \omega \ne \omega_0 \); \( \beta < \omega_0 \).
        \item \( \omega \ne \omega_0 \); \( \beta \geq \omega_0 \).
	\end{enumerate}
