\chapter{Метод векторных диаграмм}

\section{Амплитудные и фазовые соотношения}

	Установим амплитудные и фазовые соотношения на отдельных элементах \( R \),
    \( L \) и \( C \), а затем на их комбинациях, при синусоидальном внешнем
    воздействии, то есть когда к ним приложено напряжение
	\begin{equation}
		u = U\sin\omega t, \label{eq16:1}
	\end{equation}
	где \( U \) -- амплитуда, \( \omega \) -- частота приложенного напряжения.
	
	\subsection{Элемент R}
	
        По закону Ома для мгновенных значений:
        \begin{equation}
            i = \frac{u}{R} = \frac{U}{R}\sin\omega t = I\sin\omega t,
            \label{eq16:2}
        \end{equation}
        где
        \begin{equation}
            I = \frac{U}{R}
            \label{eq16:3}
        \end{equation}
        -- это амплитуда тока.
        
        Из сравнения (\ref{eq16:1}) и (\ref{eq16:2}) видно, что ток и напряжение
        на активном сопротивлении синфазны, то есть одинаково проходят через
        нули и экстремумы.
        
    \subsection{Элемент C}
        Для него ток и напряжение связаны линейным дифференциальным
        соотношением:
        \begin{equation}
            i = C\frac{\dd u}{\dd t} = CU\omega\cos\omega t =
            \frac{U}{\frac{1}{\omega C}}\sin\left(\omega t +
            \frac{\pi}{2}\right) =  I\sin\left(\omega t + \frac{\pi}{2}\right),
            \label{eq16:4}
        \end{equation}
        где амплитуда тока
        \begin{equation}
            I = \frac{U}{\frac{1}{\omega C}}.
            \label{eq16:5}
        \end{equation}
        
        Из сравнения (\ref{eq16:1}) и (\ref{eq16:4}) видно, что ток через
        ёмкость обгоняет по фазе напряжение на \( \pi/2 \). Из сравнения
        (\ref{eq16:3}) и  (\ref{eq16:5}) видно, что роль сопротивления в
        (\ref{eq16:5}) играет величина
        \[
            X_C(\omega) = \frac{1}{\omega C},
        \]
        называемая ёмкостным сопротивлением.
        
        \begin{example}
            Пусть \( C = 1 \)мкФ, тогда на промышленной частоте \( f = 50 \)Гц
            (\( \omega \approx 300 \)):
            \[
                X_C = \frac{1}{10^{-6}\cdot300} = 3 \text{кОм}.
            \]
            А на частоте \( f = 50 \)кГц:
            \[
                X_C = 3 \text{Ом}.
            \]
        \end{example}
        
    \subsection{Элемент L}
        На нем ток и напряжение связаны линейным дифференцильным соотношением:
        \[
            u = L\frac{\dd i}{\dd t} \Rightarrow i =
            \frac{1}{L}\int u(t)\dd t = -\frac{U}{\omega L}\cos\omega t +
            \underbrace{A}_{=0} = \frac{U}{\omega L}\sin\left(\omega t -
            \frac{\pi}{2}\right).
        \]
        Итак,
        \begin{equation}
            i = I\sin\left(\omega t - \frac{\pi}{2}\right),
            \label{eq16:6}
        \end{equation}
        где
        \begin{equation}
            I = \frac{U}{\omega L} \label{eq16:7}
        \end{equation}
        
        Из сравнения (\ref{eq16:1}) и (\ref{eq16:6}) видно, что ток через
        индуктивность отстает от напряжения по фазе на \( \pi/2 \).
        
        Роль сопротивления каушки играет индуктивное сопротивление
        \[
            X_L = \omega L.
        \]
        
        \begin{remark}
            Емкостное \( X_C = \frac{1}{\omega C} \) и индуктивное
            \( X_L = \omega L \) сопротивления носят общее название
            \textbf{реактивных сопротивлений}, тогда как \( R \) --
            \textbf{активное сопротивление}.
        \end{remark}

\section{Векторные диаграммы}

	Амплитудные и фазовые соотношения между токами и напряжениями на элементах
    \( R \), \( L \) и \( C \) в установившемся синусоидальном режиме удобно
    изображать в виде \textit{векторных диаграмм}.
	
	При таком способе представления каждая синусоидальная величина
    представляется в виде радиального вектора, имеющего начало в точке \( O \),
    длина вектора пропорциональна амплитуде соответствующей синусоидальной
    величины, а угол поворота относительно выбранного \textit{базового}
    направления равен сдвигу фаз, причем положительный угол отсчитывается
    против часовой стрелки. При этом один из таких векторов выбирается в
    качестве \textit{базового} и откладывается вдоль горизонтальной оси.
	
	Рассмотрим сначала векторные диаграммы для отдельных элементов \( R \),
    \( L \) и \( С \). В качестве базового вектора выберем вектор тока \( I \):
	\begin{enumerate}
        \item на элементе \( R \) ток и напряжения синфазны,
        \item на элементе \( L \) ток отстает от напряжения по фазе на
            \( \pi/2 \),
        \item на элементе \( C \) ток опережает напряжение по фазе на
            \( \pi/2 \).
	\end{enumerate}

\section{Векторная диаграмма для последовательной цепи \textit{RLC}}

	Пусть к последовательной цепи \( RLC \) приложено синусоидальное напряжение
    \( u = U\sin\omega t \). Требуется найти ток в цепи
    \( i = I\sin(\omega t + \varphi) \), то есть найти \( I \) и \( \varphi \);
    найти напряжения на элементах \( u_R \), \( u_C \) и \( u_L \).
	
	\begin{solution}
        
        \textbf{Способ 1} -- в синусах и косинусах.
        
        Уравнение Кирхгофа:
        \[
            u_R + u_C + u_L = u(t),
        \]
        \[
            L\frac{\dd i}{\dd t} + iR + u_C = u.
        \]
        Дифференцируя, получим:
        \begin{equation}
            L\frac{\dd^2 i}{\dd t^2} + R\frac{\dd i}{\dd t} + \frac{1}{C}i =
            U\omega\cos\omega t.
            \label{eq16:n1}
        \end{equation}
        
        Его установившееся (частное) решение имеет вид:
        \begin{equation}
            i = I\sin(\omega t + \varphi).
            \label{eq16:n2}
        \end{equation}
        
        Чтобы найти \( I \) и \( \varphi \) нужно (\ref{eq16:n2}) подставить в
        (\ref{eq16:n1}) и выровнять синусы и косинусы слева и справа. После
        этого напряжения на элементах:
        \begin{align*}
            & u_R = i(t)R, \\
            & u_L = L\frac{\dd i}{\dd t}, \\
            & u_C = \frac{1}{C}\int i\dd t.
        \end{align*}
        
        \textbf{Способ 2} -- метод векторных диаграмм.
        
        Изобразим на диаграмме векторы \( U_R \), \( U_L \) и \( U_C \). Так
        как здесь общим для элементов является ток \( i(t) \), то
        \textit{удобно} в качестве базового брать вектор \( I \), а остальные
        векторы откладывать относительно него.
        
        \begin{table}
            \center
            \begin{tabular}[c]{|c|c|c|}\hline
                %------------------------------------------------------------
                Вектор & Длина & Фазовый сдвиг относительно \( I \) \\ \hline
                %------------------------------------------------------------
                \( U_R \) & \( IR \) & \( 0 \) \\
                \( U_L \) & \( I\omega L \) & \( \pi/2 \) \\
                \( U_C \) & \( I\frac{1}{\omega C} \) & \( -\pi/2 \) \\ \hline
                %------------------------------------------------------------
            \end{tabular}
        \end{table}
        Векторная сумма \( U_R \), \( U_L \) и \( U_C \) даст вектор
        результирующего напряжения. По теореме Пифагора:
        \[
            U^2 = (U_L - U_C)^2 + U_R^2,
        \]
        \[
            U^2 = I^2\left[\left(\omega L - \frac{1}{\omega C}\right)^2 +
            R^2\right].
        \]
        
        Отсюда
        \begin{align}
            I = \frac{U}{\sqrt{R^2 + \left(\omega L - 
            \frac{1}{\omega C}\right)^2}},
            \label{eq16:n3} \\
            \tg\varphi = \frac{U_L - U_C}{U_R} =
            \frac{\omega L - \frac{1}{\omega C}}{R}.
            \label{eq16:n4}
        \end{align}
        
        Итак, если \( u = U\sin\omega t \), то
        \( i = I\sin(\omega t - \varphi) \), где амплитуда тока \( I \) и
        фазовый сдвиг \( \varphi \) определяются формулами (\ref{eq16:n3}) и
        (\ref{eq16:n4}).
        
        Далее:
        \begin{align*}
            & u_R = iR, \\
            & u_L = L\frac{\dd i}{\dd t}, \\
            & u_C = \frac{1}{C} \int i(t)dt.
        \end{align*}
    \end{solution}
	
	Формула (\ref{eq16:n3}) выражает закон Ома для последовательной цепи
    \( RLC \). В ней величина
    \[
        Z = \sqrt{R^2 + \left(\omega L - \frac{1}{\omega C}\right)^2}
    \]
     называется \textit{полным} сопротивлением цепи \( RLC \). В нем \( R \)
     -- активное сопротивление, а
     \( X = X_L - X_C = \omega L - \frac{1}{\omega C} \) --
     реактивное сопротивление.
	
	Если на данной частоте \( \omega \):
	\begin{itemize}
        \item \( \omega L > \frac{1}{\omega C} \), то \( \varphi > 0 \) и ток
            \( I \) отстает от напряжения \( U \) (как на рисунке);
        \item \( \omega L < \frac{1}{\omega C} \), то \( \varphi < 0 \) и ток
            \( I \) опережает напряжение \( U \);
        \item \( \omega L = \frac{1}{\omega C} \), то \( X = 0 \) и
            \( \varphi = 0 \), полное сопротивление \( Z \) становится чисто
            активным, а ток и напряжение -- синфазными. Такой режим в цепи
            называется \textbf{резонансом}.
	\end{itemize}
	
	\begin{example}
        Пусть к параллельной цепи \( RC \) (конденсатор с утечкой) приложено
        напряжение \( u = U\sin\omega t \). Определить \( I \), \( I_R \),
        \( I_C \), \( \varphi \) между \( I \) и \( U \).
	\end{example}
	
	\begin{solution}
	
        Так как здесь общим является напряжение, то вектор \( U \) удобно брать
        в качестве базового, а токи \( I \), \( I_R \) и \( I_C \) откладывать
        относительно него.
        
        На \( R \) ток и напряжение синфазны, на конденсаторе ток по фазе
        обгоняет напряжение на \( \pi/2 \). По теореме Пифагора:
        \[
            I^2 = U^2\left(\frac{1}{R^2} + \omega^2C^2\right).
        \]
        Тогда ток:
        \[
            I = U\frac{\sqrt{1 + (\omega RC)^2)}}{R},
        \]
        \[
            I = \frac{U}{\frac{R}{\sqrt{1 + (\omega RC)^2)}}}.
        \]
        Полное сопротивление параллельной цепи \( RC \):
        \[
            Z = \frac{R}{\sqrt{1 + (\omega RC)^2)}}.
        \]
        Фазовый сдвиг тока:
        \[
            \tg\varphi = \frac{U\omega C}{\frac{U}{R}} = \omega RC.
        \]

        Таким образом, если приложенное к цепи \( RC \) напряжение
        \( u = U\sin\omega t \), то ток \( i = I\sin(\omega t + \varphi)\).
        
        Токи в ветвях:
        \begin{align*}
            & I_R = \frac{U}{R}, \\
            & I_C = U\omega C.
        \end{align*} 
	\end{solution}
	
	Угол \( \delta = \frac{\pi}{2} - \varphi \) называется \textbf{углом потери}
    в конденсаторе:
	\[
        \tg\delta = \frac{1}{\omega RC}.
    \]
	
	Чем больше угол \( \delta \), тем хуже конденсатор, так как тем больший ток
    идет через активое сопротивление. У хороших конденсаторов:
	
\section{Мощность, рассеиваемая в цепях \textit{RLC}}

	Пусть к произвольной цепи, содержащей много \( R \), \( L \) и \( C \),
    приложено синусоидальное напряжение:
	\[
        u = U\sin\omega t.
    \]
    Тогда ток будет \( i = I\sin(\omega t + \varphi) \), где \( I \) --
    некоторая амплитуда тока, \( \varphi \) -- фазовый сдвиг тока относительно
    напряжения. По закону Джоуля-Ленца мгновенная мощность, поступающая в цепь:
	\[
        p(t) = u(t)\cdot i(t) = IU\sin\omega t\sin(\omega t + \varphi).
    \]
	
	А так как
    \[
        \sin\alpha\sin\beta = \frac{1}{2}[\cos(\alpha - \beta) -
        \cos(\alpha + \beta)],
    \]
    то:
	\[
        p(t) = \frac{1}{2}IU[\cos\varphi - \cos(2\omega t + \varphi)].
    \]
	
	В одни интервалы времени цепь поглощает мощность, рассеивая её на активных
    и накапливая на реактивных сопротивлениях, в другие -- отдает часть
    мощности, накопленной на реактивных сопротивлениях.
	
	Практический интерес представяет средняя мощность, рассеиваемая в цепи за
    период:
	\[
        P \equiv \midnum{p(t)}_T = \frac{1}{2}IU\cos\varphi -
        \frac{1}{2}IU\midnum{\cos(2\omega t + \varphi)}_T
    \]
	
	А так как \( \midnum{\cos(2\omega t + \varphi)}_T = 0 \)
    \[
        (\midnum{f(t)}_T = \frac{1}{T}\int\limits_0^T f(t)\dd t),
    \]
    то средняя мощность:
	\[
        P \equiv \midnum{p(t)}_T = \frac{1}{2}IU\cos\varphi,
    \]
	где\( \cos\varphi \) называется \textbf{коэффициентом мощности}, а \( I \)
    и \( U \) -- амплитуды тока и напряжения. Практически, вместо амплитудных
    значений удобно использовать \textit{эффективные} значения: 
    \( I_{\textit{эф}} \) и \( U_{\textit{эф}} \).
	
	\begin{definition}
        \textbf{Эффективное значение} тока \( I_{\textit{эф}} \) -- это такой
        постоянный ток, который на активном сопротивлении выделяет ту же
        мощность, что и данный переменный в среднем за период:
        \[
            I_{\textit{эф}}^2R = \frac{1}{T}\int\limits_0^T i^2(t)R\dd t,
        \]
        или
        \begin{equation}
            I_{\textit{эф}} = \sqrt{\frac{1}{T}\int\limits_0^T i^2\dd t} =
            \sqrt{\midnum{i^2}_T}.
            \label{eq16:nn1}
        \end{equation}
	\end{definition}
	
	В частности, если \( i = I\sin(\omega t + \varphi) \), то
	\[
        I_{\textit{эф}} = \sqrt{\frac{I^2}{T}\int\limits_0^T \sin^2(\omega t
        + \varphi)\dd t} = \frac{I}{\sqrt{2}}.
    \]
	
	Таким образом, для синусоидального тока
    \( I_{\textit{эф}} = I/\sqrt{2} \approx 0,7I \). Аналогично,
    \( U_{\textit{эф}} = U/\sqrt{2} \).
	
	Тогда
    \[
        P = \frac{1}{2}IU\cos\varphi =
        I_{\textit{эф}}U_{\textit{эф}}\cos\varphi.
    \]
	
	\begin{remark}
        В сети \( U_{\textit{эф}} = 220 \)В, следовательно,
        \[
            U = \sqrt{2}U_{\textit{эф}} \approx 310\text{В}.
        \]
	\end{remark}
	
	В промышленности коэффициент \( \cos\varphi \) имеет большое значение.
    На предприятиях стараются сбалансировать ёмкостные и индуктивные нагрузки,
    с тем чтобы \( \cos\varphi \) был примерно равен единице.
	
	\begin{example}
        Два завода потребляют одну и ту же мощность \( P \). Однако, у одного
        завода \( \cos\varphi_1 = 0,9 \), а у другого
        \( \cos\varphi_2 = 0,45 \).
        
        Так как \( P_1 = P_2 \), то:
        \[
            I_{\textit{эф}1}U_{\textit{эф}}\cos\varphi_1 = 
            I_{\textit{эф}2}U_{\textit{эф}}\cos\varphi_2.
        \]
        
        Следовательно, \( I_2 = 2I_1 \). Таким образом, потери в линии:
        \( \Delta P_2 = 4\Delta P_1 \), так как \( \Delta P \sim I^2 \).
	\end{example}
