\chapter{Элементы цепей переменного тока}

\section{Условие квазистационарности}

	\begin{definition}
        Ток \( i \) называется \textbf{переменным}, если он меняется со
        временем: \( i = i(t) \).
	\end{definition}
	
	Рассмотрим простейшую цепь переменного тока. Пусть в момент \( t = 0 \)
    замыкается ключ \( K \). Поскольку максимальная скорость распространения
    сигналов \( v_{max} = c = 3\times10^8 \text{м}/\text{с} \), то ток \( i \)
    через резистор \( R \) пойдет лишь через время \( \Delta t = l/c \).
    Следовательно, при \( t < l/c \) ток \( i = 0 \) и второе правило Кирхгофа
    \( \EDS = iR \) неверно. Да и само понятие тока \( i \) при
    \( t < \Delta t \) в линии будет неопределено.
	
	То есть, если процесс в цепи протекает слишком быстро, то несправедливо
    второе правило Кирхгофа. Поэтому в дальнейшем мы будем рассматривать
    процессы, удовлетворяющие условию \textbf{квазистационарности}:
	\begin{equation}
		\tau \gg \Delta t = \frac{l}{c},
        \label{eq13:1}
	\end{equation}
	где \( \tau \) -- характерное время изменения какой-либо электрической
    величины. Это, например, длительность импульса или время роста/спада. Если
    процесс синусоидален, то в качестве \( \tau \) берётся четверть периода:
	\begin{equation}
		\tau = \frac{T}{4} \gg \frac{l}{c}.
        \label{eq13:2}
	\end{equation}
	
	Из (\ref{eq13:2}) видно, что условие квазистационарности связывает частоту
    \( f = \frac{1}{T} \) процесса с длиной цепи \( l \). Так, например, для
    настольных цепей с \( l \sim 1\)м процесс будет квазистационарным для частот
	\[
        f \ll \frac{3\times10^8\text{м}/\text{с}}{4\cdot1\text{м}} \approx
        100 \text{МГц},
    \]
	то есть при \( f \lesssim 10 \)МГц.
	
	На промышленных частотах \( f = 50 \)Гц процесс будет квазистационарным лишь
    в цепях размеров
    \[
        l \ll \frac{c}{4f} = \frac{3\times10^8}{4\cdot50} = 10^6 \text{м}
        = 1000\text{км},
    \]
    то есть при длине линии меньше 100 км. В противном случае процесс надо
    рассматривать как \textit{волновой}.
	
	Для процессов, удовлетворяющих условию квазистационарности (\ref{eq13:1})
    справедливы уравнения Кирхгофа:
	\[ \left\{
	\begin{array}{l}
        \sum i_k(t) = 0, \\
		\sum u_k(t) = \sum\EDS_k(t).
	\end{array} \right.
	\]
	
	\begin{remark}
        В синусоидальных величинах в дальнейшем будем обозначать:
        \begin{itemize}
            \item \( i, u \) -- мгновенные значения \( i(t), u(t) \);
            \item \( I, U \) -- амплитудные значения;
            \item \( I_{\textit{эф}}, U_{\textit{эф}} \) -- эффективные
                значения;
            \item \( \dot{I}, \dot{U} \) -- комплексные значения синусоидальных
                величин.
        \end{itemize}
	\end{remark}
	
\section{Напряжения на элементах цепей переменного тока}
    Под элементами цепей переменного тока имеется в виду:
    \begin{enumerate}
    \item Элемент \( R \) -- активное сопротивление (резистор).
        Напряжение на нём 
        \begin{equation}
            u_R = iR.
            \label{eq13:n1}
        \end{equation}

    \item Элемент \( C \) -- ёмкость (конденсатор).
        Найдем напряжение на нём. По определению
        \[
            C = \frac{q}{u_C}.
        \]
        Следовательно, напряжение на элементе \( C \):
        \begin{equation}
            u_C = \frac{q}{C}.
            \label{eq13:n2}
        \end{equation}

        Продифференцируем (\ref{eq13:n2}) по \( t \) и выразим
        \[
            \frac{\dd q}{\dd t} = i:
        \]
        \begin{equation}
            i = C\frac{\dd u}{\dd t} = C\dot{u}.
            \label{eq13:n2a}
        \end{equation}

    \item Элемент \( L \) -- индуктивность (катушка).
    Найдем напряжение на нём. По закону ЭМИ:
    \[
        \oint\limits_{\Gamma} \vec{E}\cdot\dd \vec{l} =
        -\frac{\dd \Phi}{\dd t} = -L\frac{\dd i}{\dd t}.
    \]

    Однако, интеграл
    \[
        \oint\limits_{\Gamma} \vec{E}\cdot\dd \vec{l}
    \]
    можно разбить на два: интеграл по идеальному проводу катушки
    \[
        \int\limits_{\textit{пров}} E \dd l = 0,
    \]
    и интеграл по воздуху
    \[
        \int\limits_b^a E\dd l = u_{ba}.
    \]
    Тогда
    \[
        \oint\limits_{\Gamma} \vec{E}\cdot\dd \vec{l} =
        \int\limits_{\textit{пров}} E \dd l +
        \int\limits_b^a E\dd l = 0 + u_{ba}.
    \]
    А так как \( u_{ab} = -u_{ba} \), то:
    \begin{equation}
        u_L = u_{ab} = L\frac{\dd i}{\dd t} \label{eq13:n3}
    \end{equation}

    \end{enumerate}
    Выражения (\ref{eq13:n1}), (\ref{eq13:n2}) и (\ref{eq13:n3}) являются 
    \textit{функциональными определениями} элементов переменного тока.

    Рассмотрим последовательную цепь \( RLC \). По второму правилу Кирхгофа:
    \[
        u_R + u_L + u_C = \EDS.
    \]

    Однако, в цепях переменного тока вместо термина ЭДС, приложенного к цепи,
    используют термин \textit{напряжение, приложенное к цепи}:
    \[
        \EDS \to u(t).
    \]
    И тогда второе уравнение Кирхгофа:
    \[
        u_R + u_L + u_C - u = 0,
    \]
    или
    \begin{equation}
        u_R + u_L + u_C = u
    \end{equation}

    Используя функциональные определения элементов (\ref{eq13:n1}),
    (\ref{eq13:n2}) и (\ref{eq13:n3}) получим
    \[
        L\frac{\dd i}{\dd t} + iR + \frac{q}{C} = u(t).
    \]
    Продифференцируем его по времени \( t \):
    \[
        L\frac{\dd ^2i}{\dd t^2} + R\frac{\dd i}{\dd t} + \frac{1}{C}i =
        \frac{\dd u}{\dd t}.
    \]

    Второе уравнение Кирхгофа в цепях переменного тока является уже не
    алгебраическим, а дифференциальным.
