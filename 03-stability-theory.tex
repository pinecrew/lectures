\section{Теория устойчивости}
Положим для простоты, что рассматриваемая нами динамическая система описывается
системой из двух дифференциальных уравнений:
\[
    \left\{
        \begin{array}{l}
            \dot{x}_1 = f_1(x_1, x_2, t),\\
            \dot{x}_2 = f_2(x_1, x_2, t).
        \end{array}
    \right.
\]
Для простоты будем рассматривать автономную динамическую систему -- в ней
отсутствует явная зависимость от времени и соответствующая ей система уравнений
несколько проще:
\[
    \left\{
        \begin{array}{l}
            \dot{x}_1 = f_1(x_1, x_2),\\
            \dot{x}_2 = f_2(x_1, x_2).
        \end{array}
    \right.
\]
Рассмотрим её поведение вблизи точки равновесия. Эта точка определяется из
условия
\[
    \left\{
        \begin{array}{l}
            f_1(\chi_1, \chi_2) = 0,\\
            f_2(\chi_1, \chi_2) = 0.
        \end{array}
    \right.
\]
Разложим правые части в ряд Тейлора в окрестности точки равновесия:
\begin{align*}
    & f_1(\chi_1+\xi_1, \chi_2+\xi_2) = \pder{f_1}{x_1}\xi_1 +
    \pder{f_1}{x_2}\xi_2,\\
    & f_2(\chi_1+\xi_1, \chi_2+\xi_2) = \pder{f_2}{x_1}\xi_1 +
    \pder{f_2}{x_2}\xi_2,\\
\end{align*}
где значения производных берутся в точке равновесия \( (\chi_1, \chi_2) \).
Тогда вблизи точки равновесия система линеаризуется:
\[
    \left\{
        \begin{array}{l}
            \dot{\xi}_1 = F_{11}\xi_1 + F_{12}\xi_2,\\
            \dot{\xi}_2 = F_{21}\xi_1 + F_{22}\xi_2,
        \end{array}
    \right.
    \text{ где }
    F_{ij} = \left.\pder{f_i}{x_j}\right|_{\mathbf{x}=\boldsymbol{\chi}}.
\]
Решение этой линейной системы имеет вид \( e^{\lambda t} \). Подставив его в
уравнение, найдём значение \( \lambda \):
\[
    \left\{
        \begin{array}{l}
            \lambda\xi_1 = F_{11}\xi_1 + F_{12}\xi_2,\\
            \lambda\xi_2 = F_{21}\xi_1 + F_{22}\xi_2,
        \end{array}
    \right.
\]
\[
    \left\{
        \begin{array}{l}
            (F_{11} - \lambda) \xi_1 + F_{12}\xi_2 = 0,\\
            F_{21}\xi_1 + (F_{22} - \lambda) \xi_2 = 0.
        \end{array}
    \right.
\]
Эта линейная однородная система имеет ненулевое решение, только если её
определитель равен нулю:
\[
    \begin{vmatrix}
        F_{11} - \lambda & F_{12}          \\
        F_{21}           & F_{22} - \lambda
    \end{vmatrix}
    = 0,
\]
\[
    (F_{11} - \lambda)(F_{22} - \lambda) - F_{12}F_{21} = 0,
\]
\[
    \lambda^2 - (F_{11} + F_{22}) \lambda + F_{11}F_{22} - F_{12}F_{21} = 0,
\]
\[
    \lambda_{1,2} = \frac{1}{2}\left[ F_{11} + F_{22} \pm
        \sqrt{(F_{11} + F_{22})^2 - 4(F_{11}F_{22} - F_{12}F_{21})} \right].
\]
Немного упростив, окончательно получаем
\[
    \lambda_{1,2} = \frac{F_{11} + F_{22}}{2} \pm
        \sqrt{\left(\frac{F_{11} - F_{22}}{2}\right)^2 + F_{12}F_{21}}.
\]
В зависимости от значений \( \lambda \) возможны 6 вариантов:
\begin{enumerate}
    \item \( \lambda_1 < 0, \lambda_2 < 0 \) -- решение сходится к точке
        равновесия, положение равновесия устойчиво, точка называется устойчивым
        узлом;
    \item \( \Re\lambda_1 < 0, \lambda_2 = \lambda_1^* \) -- решение сходится к
        точке равновесия, совершая затухающие колебания, положение равновесия
        устойчиво, точка называется устойчивым фокусом;
    \item \( \Re\lambda_1 = 0, \lambda_2 = \lambda_1^* \) -- решение совершает
        колебания вокруг точки равновесия, положение равновесия устойчиво, точка
        называется центром;
    \item \( \Re\lambda_1 > 0, \lambda_2 = \lambda_1^* \) -- решение расходится
        от точки равновесия, совершая колебания с возрастающей амплитудой,
        равновесие неустойчиво, точка называется неустойчивым фокусом;
    \item \( \lambda_1 > 0, \lambda_2 > 0 \) -- решение рассходится от точки
        равновесия, равновесие неустойчиво, точка называется неустойчивым узлом;
    \item наконец, если \( \lambda_1 > 0, \lambda_2 < 0 \) или наоборот, то
        существуют как траектории сходящиеся к этой точке, так и расходящиеся от
        неё, такое положение равновесия неустойчиво, а точка называется седлом.
\end{enumerate}

Одним из первых наиболее полно понятие устойчивости исследовал Ляпунов.
Положение равновесия называется устойчивым по Ляпунову если:
\begin{enumerate}
    \item мне стыдно, я не знаю.
\end{enumerate}
