\chapter{Энергия электромагнитного поля}

\section{Вектор Пойнтинга}

	Электромагнитное поле переносит в себе энергию. Ее объемная плотность:
	\[
        w = \simpder{W}{V} = w_E + w_H =
        \frac{\varepsilon\Ezero E^2}{2} + \frac{\mu\mu_0 H^2}{2}.
    \]
	
	А так как в электромагнитной волне, согласно формуле (\ref{eq20.1:10}),
	\[
        \sqrt{\varepsilon\Ezero}E_x(z, t) = \sqrt{\mu\mu_0}H_y(z, t),
    \]
	то:
	\[
        w = \varepsilon\Ezero E^2 =
        \left(\sqrt{\varepsilon\Ezero}E\right) \left(\sqrt{\mu\mu_0}H\right) =
        \frac{EH}{v}.
    \]

	Таким образом,
	\begin{equation}
		w = \frac{EH}{v}.
        \label{eq21:1}
	\end{equation}
	
	Пусть плоская волна бежит вдоль оси \( Oz \) со скоростью \( v \). Построим
    на оси цилиндр длиной \( \dd l = v\dd t \) и сечением \( \dd S \):
	
	где \( \vec{n} \) -- единичная нормаль к торцу цилиндра, показывающая
    направление переноса энергии. В этом цилиндре заключена энергия поля:
	\begin{equation}
		\dd W = w\dd V = wv\dd t\dd S = EH\dd t\dd S.
        \label{eq21:2}
	\end{equation}
	За время \( \dd t \) вся эта энергия, по определению, пройдет через передний
    торец.
	
	\begin{definition}
        Векторная величина \( \vec{\Pi} \), показывающая направление переноса
        энергии и численно равная количеству энергии, переносимому за единицу 
        времени через единичную площадку, перпендикулярную направлению переноса
        энергии, называется \textit{плотностью потока энергии}:
        \[
            \vec{\Pi} = \frac{\dd W}{\dd t\dd S}\cdot\vec{n}.
        \]
	\end{definition}
	
	С учетом (\ref{eq21:2}):
	\[
        \vec{\Pi} = EH\vec{n}.
    \]
	А так как \( \vec{n} \uparrow\uparrow (\vec{E} \times \vec{H}) \), то
	\begin{equation}
		\vec{\Pi} = \vec{E}\times\vec{H}.
        \label{eq21:3}
	\end{equation}
	Вектор \( \vec{\Pi} \), записанный в виде (\ref{eq21:3}), называется
    \textit{вектором Пойнтинга}. Таким образом, если задано поле вектора
    \( \vec{\Pi} \), то через поверхность \( S \) переносится мощность
    (энергия в единицу времени):
	\[
        P = \iint\limits_S \vec{\Pi}\cdot\dd\vec{S}.
    \]
	
\section{Интенсивность синусоидальных волн}

	Вычислим вектор Пойнтинга для плоских синусоидальных волн, бегущих вдоль оси
    \( Oz \):
	\[
        \left\{
        \begin{array}{l}
            E_x = E_0\sin(\omega t - kz), \\
            H_y = H_0\sin(\omega t - kz).
        \end{array}
        \right.
    \]
	Так как \( \vec{E} \perp \vec{H} \), то вектор Пойнтинга:
	\[
        \Pi(z, t) = E_0H_0\sin^2(\omega t - kz).
    \]
	Практический интерес представляет среднее значение вектора Пойнтинга за
    период:
	\[
        \midnum{\vec{\Pi}}_T = E_0H_0\midnum{\sin^2(\omega t -kz)}_T =
        \frac{E_0H_0}{2}.
    \]
	
	\begin{definition}
        Среднее значение вектора Пойнтинга (плотности потока энергии) называется
        \textit{интенсивностью} синусоидальной волны.
	\end{definition}
	Таким образом, интенсивность:
	\[
        I \equiv \midnum{\Pi}_T = \frac{E_0H_0}{2}.
    \]
	В силу справедливости (\ref{eq20.1:10}), получаем:
	\[
        I = \frac{\varepsilon\Ezero E_0^2}{2}v.
    \]
	В вакууме:
    \[
        I = \frac{\Ezero E_0^2}{2}c.
    \]
    \begin{remark}
        Интенсивность солнечного излучения на расстоянии орбиты Земли:
        \[
            I_\textit{с} = 1380 \frac{\text{Вт}}{\text{м}^2}.
        \]
    \end{remark}

	В плоской волне интенсивность постоянна, так как \( E_0 = \const \). В 
    сферической волне интенсивность обратно пропорциональна квадрату расстояния
    до источника, так как \( E_0 \sim r^{-1} \).
	
	\begin{example}
        Интенсивность излучения лазера мощностью \( P = 30 \)Вт и сечением
        \( S = 1\text{мм}^2 \)
        \[
            I = \frac{P}{S} = \frac{30\text{Вт}}{10^{-6}\text{м}^2} =
            3\cdot10^7\frac{\text{Вт}}{\text{м}^2}
        \]
	\end{example}

    \begin{example}
        Доказать, что в стоячей волне \( \midnum{\Pi} = 0 \), то есть стоячая
        волна не переносит энергию.
    \end{example}
	
\section{Закон сохранения энергии электромагнитного поля}

	Закон сохранения энергии в локальной форме:
	\begin{quote}
        Если в некоторой области \( V \) энергия электромагнитного поля
        \( W = W_{E} + W_{H} \) убывает, то это происходит только путем ее
        вытекания за границы области, причем скорость убывания энергии в области
        \( V \) равна потоку вектора \( \vec{\Pi} \) через ее границы \( S \):
        \begin{equation}
            -\partder{W}{t} = \oiint\limits_S \vec{\Pi}\cdot\dd\vec{S}.
            \label{eq21.1:1}
        \end{equation}
	\end{quote}
	
	Так как
    \[
        W = \iiint\limits_V w\cdot\dd V,
    \]
    а, по теореме Остроградского,
    \[
        \oiint\limits_S \vec{\Pi}\cdot\dd\vec{S} =
        \iiint\limits_V \div\vec{\Pi}\dd V,
    \]
    то формула (\ref{eq21.1:1}) в дифференциальном виде:
	\begin{equation}
		-\partder{w}{t} = \div\vec{\Pi}.
        \label{eq21.1:1a}
	\end{equation}
	
	Покажем, что закон сохранения энергии в виде (\ref{eq21.1:1a}) согласуется
    с уравнениями Максвелла:
	\[
        \left\{
        \begin{array}{l}
            \rot\vec{E} = -\mu\mu_0\partder{\vec{H}}{t}, \\
            \rot\vec{H} = \varepsilon\Ezero\partder{\vec{E}}{t}.
        \end{array}
        \right.
    \]
	
	\[
        \div\vec{\Pi} = \div(\vec{E}\times\vec{H}) = \vec{H}\cdot\rot\vec{E} -
        \vec{E}\cdot\rot\vec{H}.
    \]
	
	Записывая роторы \( \rot\vec{E} \) и \( \rot\vec{H} \) из уравнений
    Максвелла, получим:
    \begin{align*}
        & \div\vec{\Pi} = -\mu\mu_0\vec{H}\cdot\partder{\vec{H}}{t} -
            \varepsilon\Ezero\vec{E}\cdot\partder{\vec{E}}{t} =
            -\frac{1}{2} \left(\mu\mu_0\partder{H^2}{t} +
            \varepsilon\Ezero\partder{E^2}{t}\right) = \\
        & = -\partder{}{t} \left(\frac{\varepsilon\Ezero E^2}{2}
            + \frac{\mu\mu_0 H^2}{2}\right) = -\partder{w}{t},
    \end{align*}
	что согласуется с законом сохранения энергии.
	
\section{Обобщенное уравнение энергии электромагнитного поля}

	Закон сохранения энергии (\ref{eq21.1:1}) не вполне точен. В нем не учтены
    возможности потери энергии внутри самой области \( V \) и превращения ее,
    например, в тепло. Обобщенное уравнение энергии электромагнитного поля же
    выглядит следующим образом:
	\begin{equation}
		-\iiint\limits_V \partder{w}{t}\dd V =
        \oiint\limits_S \vec{\Pi}\cdot\dd\vec{S} + P_{V},
        \label{eq20.1:2}
	\end{equation}
	где \( P_{V} \) -- потеря электромагнитной энергии в единицу времени (потеря
    мощности) внутри области \( V \). Она больше нуля, когда энергия
    генерируется внутри области \( V \), и меньше нуля, когда энергия
    поглощается внутри области. Вычислим ее.
	
	На каждый заряд \( q \), движущийся со скоростью \( v \), со стороны
    электромагнитного поля действует сила Лоренца
	\[
        \vec{F}_{q} = q(\vec{E} + \vec{v}\times\vec{B})
    \]
	и передает ему мощность
	\[
        P_{q} = \vec{F}\cdot\vec{v} = q(\vec{E}\cdot\vec{v}).
    \]
	
	Если в единице объема содержится \( n \) зарядов, то в единице объема
    теряется мощность
	\[
        p = nP_{q} = nq(\vec{E}\cdot\vec{v}) = \vec{j}\cdot\vec{E}.
    \]
	Тогда потеря мощности во всей области \( V \)
	\[
        P_{V} = \iiint\limits_V p\dd V =
        \iiint\limits_V (\vec{E}\cdot\vec{j})\dd V
    \]
	и обобщенное уравнение энергии поля принимает вид:
	\begin{equation}
        -\iiint\limits_V \partder{w}{t}\dd V =
        \oiint\limits_S \vec{\Pi}\cdot\dd\vec{S} +
        \iiint\limits_V (\vec{E}\cdot\vec{j})\dd V.
        \label{eq21.1:3}
	\end{equation}
	
	Дифференциальный вид уравнения (\ref{eq21.1:3}):
	\begin{equation}
		-\partder{w}{t} = \div\vec{\Pi} + \vec{E}\cdot\vec{j}ю
        \label{eq21.1:3a}
	\end{equation}

\section{Примеры, иллюстрирующие справедливость уравнения энергии}

    \subsection{Зарядка конденсатора}

        Рассмотрим процесс зарядки плоского конденсатора с круглыми обкладками.
        Поле \( \vec{E}(t) \) будет порождать поле \( \vec{H} \) в соответствии
        с уравнением Максвелла:
        \[
            \oint\limits_C \vec{H}\cdot\dd\vec{l} =
            \Ezero\iint\limits_S \partder{\vec{E}}{t}\cdot\dd\vec{S}.
        \]
        Рассмотрим канал попадания электромагнитной энергии в конденсатор. Если
        идет зарядка конденсатора (поле \( \vec{E} \) увеличивается), то линии
        поля \( \vec{H} \) наматываются правым винтом на линии поля
        \( \vec{E} \). Следовательно, вектор Пойнтинга
        \( \vec{\Pi}~=~\vec{E}\times\vec{H} \) направлен к оси конденсатора.
        
        Таким образом, при зарядке конденсатора энергия втекает в конденсатор
        через его боковой зазор:
        
        Покажем, что скорость ее изменения подчиняется уравнению
        (\ref{eq21.1:1}). Поток вектора \( \vec{\Pi} \) внутрь \( V \):
        \[
            -\iint\limits_{S_\textit{бок}} \vec{\Pi}\cdot\dd\vec{S} = 
            -\iint\limits_{S_\textit{бок}} (\vec{E}\times\vec{H})\cdot\dd\vec{S}
            = \iint\limits_{S_\textit{бок}} EH\dd S = EH\cdot2\pi rl.
        \]
        Из уравнения Максвелла следует, что при однородном поле \( \vec{E} \),
        поле \( \vec{H} \):
        \[
            H = \frac{\Ezero r}{2}\partder{E}{t}.
        \]
        Тогда поток поля \( \vec{\Pi} \):
        \[
            E\Ezero\partder{E}{t}\pi r^2 l =
            \partder{}{t}\left(\frac{\Ezero E^2}{2}V\right) = \partder{W}{t},
        \]
        что совпадает с уравнением (\ref{eq21.1:1}).
        
        \begin{example}
            А как энергия попадает внутрь соленоида?
        \end{example}

    \subsection{Нагрев провода током}
        Исследуем канал попадания электромагнитной энергии внутрь провода
        радиусом \( r \). Рассмотрим элемент провода длины \( l \), по которому
        идет ток \( i \). Напряжение между его концами \( U = El\). Поток
        вектора \( \vec{\Pi} \) внутрь провода:
        \[
            -\iint\limits_{S_\textit{бок}} \vec{\Pi}\cdot\dd\vec{S} =
            \Pi S_\textit{бок} = EH\cdot2\pi rl.
        \]
        Из закона Био-Савара следует, что
        \[
            H = \frac{B}{\mu_0} = \frac{i}{2\pi r}.
        \]
        Тогда поток:
        \[
            EH\cdot2\pi rl = \frac{U}{l}il = iU = P_\textit{Дж}.
        \]
        Однако, энергия, поступающая в провод, не накапливается в нем,
        следовательно, она уходит в тепло:
        \[
            \partder{W}{t} = 0 \Rightarrow
            \oiint\limits_S \vec{\Pi}\cdot\dd\vec{S}
            + \iiint\limits_V (\vec{E}\cdot\vec{j})\dd V = 0.
        \]
        Докажем последнее соотношение:
        \begin{align*}
            & \iiint\limits_V (\vec{E}\cdot\vec{j})\dd V = iU, \\
            & Ej\pi r^2l = iU, \\
            & Eil = iU, \\
            & \mathit{\mathbf{iU = iU}}
        \end{align*}

\section{Давление электромагнитной волны}

	Механизм давления электромагнитной волны состоит в следующем. Компоненты
    волны \( \vec{E} \) возбуждает колебания электронов \( q_{i} \) и придает
    им мгновенные скорости \( v_{i} \), а со стороны синфазного поля
    \( \vec{B} \) на них действует сила Лоренца \( F_{i} \), направленная внутрь
    вещества:
	\[
        F_{i} = q_{i}(\vec{v}_{i}\times\vec{B}).
    \]
	
	Усредненные по времени силы Лоренца, отнесенные к еденице площади, и есть
    давление электромагнитной волны на вещество:
	\[
        p = \frac{\midnum{|\vec{F}_\textit{Л}|}}{S}.
    \]
	
	Можно показать, что давление синусоидальной волны \( p = \frac{I}{c} \), где
    \( c = 3\cdot 10^8 \text{м}/\text{с} \) -- скорость света в
    вакууме, \( I = \frac{\Ezero E_0^2}{2}c \) -- интенсивность:
	\[
        p = \frac{\Ezero E_0^2}{2}.
    \]
	
	Если среда частично или полностью отражает электромагнитные волны, то 
    давление:
	\[
        p = (k + 1)\frac{I}{c},
    \]
	где \( 0 \leq k \leq 1 \) -- коэффициент отражения.
	
	\begin{example}
        Вычислить давление дуча лазера мощностью \( P = 30 \)Вт и сечением
        \( S = 1\text{мм}^2 \).
	\end{example}
	\begin{solution}
        Давление при полном поглощении:
        \[
            p = \frac{I}{c} = \frac{P}{cS} =
            \frac{30}{3\cdot10^8\cdot10^{-6}} = 0,1\text{Па}.
        \]
        Сила давления луча:
        \[
            F = pS = 0,1\cdot10^{-6} = 10^{-7}\text{Н}.
        \]
	\end{solution}
	
	\begin{example}
        Определить давление солнечного излучения на расстоянии орбиты Земли,
        например, на солнечный парус (коэффициент отражения \( k = 1 \))
        площадью \( 1 \text{км}^2 \).
	\end{example}
	\begin{solution}
        Так как на расстоянии \( R_\textit{орб} \) интенсивность солнечного
        излучения \( I \approx 1500 \text{Вт}/\text{м}^2 \), то давление на
        солнечный парус:
        \[
            p = (k + 1)\frac{I}{c} = \frac{2I}{c} =
            \frac{3\cdot10^3}{3\cdot10^8} = 10^{-5}\text{Па}.
        \]
        Тогда сила давления:
        \[
            F = 10\text{Н}.
        \]
	\end{solution}
