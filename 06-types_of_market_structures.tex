\section{Типы рыночных структур}

Тип рынка определяется степенью развития конкуренции. Конкуренция бывает
следующих видов:
\begin{itemize}
    \item ценовая и неценовая (реклама, качество товаров, бренд).
    \item Совершенная и несовершенная.
    Совершенная конкуренция характеризуется:
    \begin{itemize}
        \item наличием большого числа производителей,  каждый из которых
            занимает долю рынка, не превышающую 1\%,
        \item стандартизированным типом продукции,
        \item отсутствуют барьеры входа и выхода из отрасли, отсутствует влияние
            производителей на цену.
    \end{itemize}
    В условиях совершенной конкуренции максимизируют прибыль или минимизируют
    убытки в случае, если выполняется равенство:
    \[
        MR = MC = P = AC.
    \]
    В условиях совершенной конкуренции в долгосрочном периоде прибыль фирмы
    равна нулю.

    Несовершенная конкуренция бывает нескольких видов:
    \begin{enumerate}
        \item монополистическая конкуренция: большое количество фирм,
        производящих дифференцированную продукцию; фирмы могут занимать как
        значительную долю рынка, так и очень малую; условия вступления в отрасль
        достаточно легкие.
        \item олигополия: малое количество производящих, производящих
        дифференцированную или стандартизированную; значительное влияние
        на цену, затруднённый вход в отрасль. Крайним случаем олигополии
        является дуополия.
        \item монополия: один производитель, обладающий уникальным продуктом;
        высокие, практически непреодолимые, барьеры для вступления в отрасль;
        высокое влияние на ценообразование.
        Виды монополии:
        \begin{enumerate}
            \item чистая монополия: производитель предоставляет товар, которого
                нет на рынке, как внутреннем, так и внешнем;
            \item естественная монополия: монополия, поддерживаемая
                государством;
            \item монопсония: ситуация на рынке, когда существуют
                много продавцов и всего один покупатель.
        \end{enumerate}
        
        В условиях монополии имеет место ценовая дискриминация, которая бывает
        двух видов:
        \begin{enumerate}
            \item предоставление разных условий покупки или оплаты услуги в
                зависимости от количества приобретаемого блага;
            \item продажа одинакового товара на разных рынках или в разное время
                по разным ценам.
        \end{enumerate}
        
        Для осуществления ценовой дискриминации необходимо выполнение следующих
        условий:
        \begin{itemize}
            \item продавец должен обладать монопольной властью;
            \item должны быть ограничены возможности перепродажи товаров
            покупателями, получившими их по низкой цене;
            \item необходимо провести сегментацию рынка.
        \end{itemize}
        
        Последствиями применения ценовой дискриминации для монополиста
        являются увеличения прибыли и объемов продаж.
        
        Цели антимонопольного регулирования:
        \begin{enumerate}
            \item контроль за поведением монополий;
            \item недопущение тайного сговора по установлению цен;
            \item запрещение объединения слияния и поглощения крупными фирмами
            мелких фирм;
            \item поощрение создания товаров-заменителей;
            \item поддержка малого бизнеса;
            \item привлечение иностранных инвестиций;
            \item финансирование мероприятий по расширению выпуска дефицитных
            товаров.
        \end{enumerate}
        
        Для оценки степени монополизации рынка используют индекс Херфиндаля-
        Хиршмана:
        \[
            I_{HH} = S_1^2 + S_2^2 + \ldots + S_n^2,
        \]
        где \( S \) -- доля, занимаемая фирмой на рынке. Индекс \( I_{HH} \)
        изменяется от 1 до 10000. Если \( I_{HH} > 1850 \), то отрасль
        считается монополизированной.
    \end{enumerate}
       
    В условиях несовершенной конкуренции фирма максимизирует прибыль или
    минимизирует издержки, если выполняется условие:
    \[
        MC = MR.
    \]
\end{itemize}
