\chapter{Конкретные социологические исследования}
\section{История эмпирической социологии}

  Впервые анализ социальных данных можно обнаружить в 1662~г., когда английский
  статистик Дж.~Граунт проанализировал данные о смертности населения. Так же до
  становления социологии как науки математик А.~Кетле анализировал различные
  данные о населения для выявления закономерностей.
  
  Несомненно проводилось и множество других социальных исследований: в 1930~г.
  при составлении социологической библиографии было выявлено более двух с
  половиной тысяч книг о социальных исследованиях.
  
  \charskip{*}
  
  Появление эмпирического направления социологии связано с работой Э.~Дюркгейма
  1897~года <<Этюд о самоубийстве>>.
  
  В начале XX~века центр эмпирической социологии перемещается в США, где
  проводится очень большое количество социальных исследований. Названия
  некоторых из них: <<Шайка>>, <<Гетто>>, <<Бродяга>>, <<Неприспособленные
  девочки>>.
  
  \charskip{*}
  
  Принципы эмпирической социологии:
  \begin{enumerate}
    \item истинность научных познаний должна устанавливаться лишь на основе
      эмпирических процедур;
    \item все социальные явления должны быть квантифицированы;
    \item субъективные объекты поведения можно исследовать только через
      открытое, наблюдаемое поведение.
  \end{enumerate}
  
  \charskip{*}
  
  Первым в истории социологическим исследованием, использующим монографический
  метод, т.~е. построенном на анализе документов, стало исследование Томаса и
  Знанецкого <<Польский крестьянин в Европе и Америке>> (1918--1920~гг.). Оно
  строилось на анализе примерно 1500~писем польских эмигрантов и их семей.
  
  Во второй половине XX~века социальные исследования, в основном, сочетают в
  себе эмпирические исследования и теорию, т.~е. перестают быть чисто
  эмпирическими и чисто теоретическими.

\section{Организация конкретных социологических исследований}

  Этапы социального исследования:
  \begin{enumerate}
    \item составление программы и плана социального исследования, включающих в
      себя сроки проведения, затраты, список необходимого оборудования и т.~п.
    \item Учебный этап: проведение пробного (пилотажного) исследования, после
      которого уточняется методика, создается доступ к корреспондентам
      (<<тестирующим>>), рассчитывается выборка по первичным данным.
    \item Полевой этап~-- сбор информации;
    \item подготовка данных для обработки;
    \item обработка, анализ и интерпретация данных;
    \item составление аналитического отчета.
  \end{enumerate}

\section{Виды социологических исследований}

  Существует множество классификаций конкретных социологических исследований по
  различным признакам. Рассмотрим четыре классификации.
  \begin{enumerate}
    \item По цели исследования:
      \begin{itemize}
        \item разведывательные (пробное, пилотажное) исследования проводятся для
          уточнения гипотезы, методики исследования;
        \item описательные~-- собирают максимум информации об одном объекте;
        \item аналитические~-- объясняют качества объекта.
      \end{itemize}
    \item По периодичности:
      \begin{itemize}
        \item разовое (\emph{ad hoc});
        \item повторные
          \begin{itemize}
            \item трендовые~-- проводятся с интервалом во времени на аналогичных
              выборках в рамках единой генеральной совокупности (разделение
              исследований во времени);
            \item сравнительные~-- проводятся на аналогичных выборках в разных
              условиях (разделение исследований в пространстве);
            \item панельные~-- исследования через одинаковые промежутки времени
              одних и тех же людей, требующее соблюдения единообразия (панель~--
              группа респондентов (<<тестируемых>>), с которыми есть
              договоренность об участии в исследовании);
          \end{itemize}
        \item лонгитюдные~-- весьма и весьма продолжительные исследования.
      \end{itemize}
    \item По способу формирования массива информации:
      \begin{itemize}
        \item сплошные~-- изучается каждый элемент объекта;
        \item выборочные;
        \item монографические~-- изучается один элемент объекта.
      \end{itemize}
    \item По методам исследования:
      \begin{itemize}
        \item количественные;
        \item качественные;
        \item комплексные.
      \end{itemize}
  \end{enumerate}

\section{Методология, программа социологических исследований}

  Методология социологического исследования представляет собой особую стратегию
  исследования исходя из специфики именно данного изучаемого объекта на основе
  одного из теоретических подходов.
  
  Методологические основы исследования содержатся в программе социального
  исследования, первый раздел которой так и назван~-- методологическим. В него
  входят:
  \begin{enumerate}
    \item описание проблемной ситуации и постановка проблемы.
    
      Проблемная ситуация~-- практическая ситуация, содержащая противоречие
      между теорией и практикой. Проблема~-- отсутствие информации, необходимой
      для разрешения проблемной ситуации.
    \item Объект и предмет исследования.
    
      Объект~-- социальная система, выделенная по фиксированному основанию, в
      элементах и структуре которой заключается противоречие.
      
      Предмет~-- те элементы, связи и отношения внутри социальной системы,
      которые непосредственно подлежат изучению.
    \item Цель и задачи исследования.
    
      Цель ориентирует на конкретный результат, всегда одна. Задачи выполняются
      на пути достижения цели, их может быть множество.
    \item Инструментализация и операционализация понятий~-- переход от
      <<понятий>> к измеряемым <<показателям>>.
    \item Формулировка гипотез.
  \end{enumerate}

  \begin{thebibliography}{9}
    \addcontentsline{toc}{section}{Список литературы}
    \bibitem{1} \href{http://socioline.ru/sites/default/modules/pubdlcnt/%
      pubdlcnt.php?file=http://socioline.ru/_seminar/library/metod/yadov/%
      met_SI.rar}{Ядов~В.~А. Социологическое исследование: методология,
      программа, методы}
    \bibitem{2} \href{http://padabum.com/x.php?id=78126}{Горшков~М.~К.,
      Шереги~Ф.~Э. Прикладная социология методология и методы}
    \bibitem{3} \href{http://www.soc.univ.kiev.ua/LIB/PUB/D/DEVIATKO/%
      deviatko.pdf}{Девятко~И.~Ф. Методы социологического исследования}
    \bibitem{4} \href{http://vk.cc/2iO5Xy}{Виды соц. исследований (по цели и по
      периодичности)}
  \end{thebibliography}
