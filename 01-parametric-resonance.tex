\section{Линейные уравнения с периодическими коэффициентами. Параметрический
    резонанс}
\subsection{Уравнение параметрических колебаний}
Рассмотрим уравнение параметрических колебаний
\[
    \ddot{u} + [\omega^2 + q(t)]u = 0,
\]
где \( q(t) \) имеет период \( 2T \). Решениями этого уравнения является пара
функций \( \Phi(t_0, t, \omega) \) и \( \Phi^*(t_0, t, \omega) \). Их можно
представить в виде фундаментальной системы решений
\[
    \Psi =
        \begin{pmatrix}
            \Phi \\
            \Phi^*
        \end{pmatrix}
    .
\]
Будем рассматривать решения, удовлетворяющие начальному условию
\[
    \Phi(t_0, t_0, \omega) = \Phi^*(t_0, t_0, \omega) = 1.
\]
Из-за периодичности \( q(t) \) решение в момент времени \( t+2T \) должно быть
линейно связано с решением в момент \( t \):
\[
    \Psi(t_0, t+2T, \omega) = M \Psi(t_0, t, \omega), \quad M =
        \begin{pmatrix}
            a & b^* \\
            b & a^*
        \end{pmatrix}
    \text{ -- матрица монодромии. }
\]
Определитель этой матрицы равен единице.

\subsection{Функции Блоха}
Для определения характера решения определим собственные значения матрицы
монодромии:
\[
    \begin{vmatrix}
        a - \lambda & b^*          \\
        b           & a^* - \lambda
    \end{vmatrix}
    = 0,
\]
\[
    (a - \lambda)(a^* - \lambda) - bb^* = 0,
\]
\[
    \lambda^2 - (a + a^*)\lambda - aa^* - bb^* = 0,
\]
\[
    \lambda^2 - 2\Re{a} \lambda - 1 = 0,
\]
\[
    \lambda = \Re{a} \pm \sqrt{(\Re{a})^2 - 1}.
\]
Очевидно, что при \( \abs{\Re{a}} > 1 \) для одного из значений
\( \abs{\lambda} > 1 \), то есть решение неограниченно возрастает. Такое явление
называется параметрическим резонансом.
