\section{Задача Бернулли}
\subsection{Постановка задачи}
    К математической постановке задачи Бернулли приводит множество задач.
    Приведём несколько примеров.
    \begin{enumerate}
        \item Есть урна с белыми и чёрными шарами. Из урны наугад выбирается шар
            и кладётся обратно, после чего шары перемешиваются. Опыт повторяется
            \( n \) раз. Какова вероятность того, что \( m \) из вынутых шаров
            будут белыми?
        \item Какова вероятность того, что при \( n \) бросках монеты \( m \)
            раз выпадет орёл?
        \item В объёме \( V \) находится \( n \) молекул газа. Какова
            вероятность того, что в объёме \( v \in V \) находится
            \( m \) молекул?
        \item Телефонистка дежурит в течение времени \( T \). Какова вероятность
            того, что во время дежурства произойдёт \( m \) вызовов, если
            вероятность вызова в течение промежутка времени \( \tau \)
            равна \( p \)?
        \item Пусть частица совершает одномерные случайные блуждания, то есть
            она прыжками перемещается вправо или влево на расстояние \( a \) с
            вероятностями \( p \) и \( q \) соответственно. Какова вероятность
            того, что за \( n \) шагов произойдёт \( m \) шагов вправо?
    \end{enumerate}

    Задача Бернулли состоит в определении вероятности происхождения ровно
    \( m \) событий в \( n \) опытах.

\subsection{Биномиальное распределение}
    Во всех задачах из списка выше \( P(A) = p \), \( P(\overline{A}) = 1 - p =
    q \). Искомая вероятность равна \( P_n(m) = C_n^m p^m q^{n-m} \). Такой
    закон распределения называется биномиальным.

    Если полная группа событий состоит из более, чем двух, то распределение
    принимает вид
    \[
        P_n(m_1, \ldots, m_k) =
            \frac{n!}{m_1! \cdot\ldots\cdot m_k!}
            p_1^{m_1} \cdot\ldots\cdot p_k^{m_k}.
    \]
    Вероятность того, что событие произойдёт не более \( m \) раз равна
    \[
        P_n(\le m) = \sum_{k=0}^m C_n^k p^k q^{n-k} =
        \frac{B_q(n-m, m+1)}{B(n-m, m+1)}.
    \]
    Наивероятнейшее значение \( m \) лежит в промежутке
    \( [(n+1)p-1, (n+1)p] \).

    Для определения моментов продифференцируем бином Ньютона по \( p \):
    \[
        n(p+q)^{n-1} = \sum_{m=0}^n m C_n^m p^{m-1} q^{n-m}
    \]
    Домножив на \( p \) обе части, получаем
    \[
        np(p+q)^{n-1} = np = \sum_{m=0}^n m C_n^m p^m q^{n-m} = m_1.
    \]
    Повторив процедуру дифференцирования и умножения \( k \) раз, можно получить
    момент \( k \)-го порядка:
    \[
        m_k = \left( p \pder{}{p} \right)^k (p+q)^n.
    \]
    Для определения смешанного момента перед дифференцированием необходимо обе
    части разделить на \( p^{m_1} \), а после домножить. Тогда получаем
    \[
        \mu_k = \left( p^{m_1+1} \pder{}{p} p^{-m_1} \right)^k (p+q)^n.
    \]
    Отсюда
    \[
        \mu_2 = \sigma^2 = npq.
    \]

