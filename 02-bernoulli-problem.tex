\section{Задача Бернулли}
\subsection{Постановка задачи}
    К математической постановке задачи Бернулли приводит множество задач.
    Приведём несколько примеров.
    \begin{enumerate}
        \item Есть урна с белыми и чёрными шарами. Из урны наугад выбирается шар
            и кладётся обратно, после чего шары перемешиваются. Опыт повторяется
            \( n \) раз. Какова вероятность того, что \( m \) из вынутых шаров
            будут белыми?
        \item Какова вероятность того, что при \( n \) бросках монеты \( m \)
            раз выпадет орёл?
        \item В объёме \( V \) находится \( n \) молекул газа. Какова
            вероятность того, что в объёме \( v \in V \) находится
            \( m \) молекул?
        \item Телефонистка дежурит в течение времени \( T \). Какова вероятность
            того, что во время дежурства произойдёт \( m \) вызовов, если
            вероятность вызова в течение промежутка времени \( \tau \)
            равна \( p \)?
        \item Пусть частица совершает одномерные случайные блуждания, то есть
            она прыжками перемещается вправо или влево на расстояние \( a \) с
            вероятностями \( p \) и \( q \) соответственно. Какова вероятность
            того, что за \( n \) шагов произойдёт \( m \) шагов вправо?
    \end{enumerate}

    Задача Бернулли состоит в определении вероятности происхождения ровно
    \( m \) событий в \( n \) опытах.

\subsection{Биномиальное распределение}
    Во всех задачах из списка выше \( P(A) = p \), \( P(\overline{A}) = 1 - p =
    q \). Искомая вероятность равна \( P_n(m) = C_n^m p^m q^{n-m} \). Такой
    закон распределения называется биномиальным.

    Если полная группа событий состоит из более, чем двух, то распределение
    принимает вид
    \[
        P_n(m_1, \ldots, m_k) =
            \frac{n!}{m_1! \cdot\ldots\cdot m_k!}
            p_1^{m_1} \cdot\ldots\cdot p_k^{m_k}.
    \]
    Вероятность того, что событие произойдёт не более \( m \) раз равна
    \[
        P_n(\le m) = \sum_{k=0}^m C_n^k p^k q^{n-k} =
        \frac{B_q(n-m, m+1)}{B(n-m, m+1)}.
    \]
    Наивероятнейшее значение \( m \) лежит в промежутке
    \( [(n+1)p-1, (n+1)p] \).

    Для определения моментов продифференцируем бином Ньютона по \( p \):
    \[
        n(p+q)^{n-1} = \sum_{m=0}^n m C_n^m p^{m-1} q^{n-m}
    \]
    Домножив на \( p \) обе части, получаем
    \[
        np(p+q)^{n-1} = np = \sum_{m=0}^n m C_n^m p^m q^{n-m} = m_1.
    \]
    Повторив процедуру дифференцирования и умножения \( k \) раз, можно получить
    момент \( k \)-го порядка:
    \[
        m_k = \left( p \pder{}{p} \right)^k (p+q)^n.
    \]
    Для определения смешанного момента перед дифференцированием необходимо обе
    части разделить на \( p^{m_1} \), а после домножить. Тогда получаем
    \[
        \mu_k = \left( p^{m_1+1} \pder{}{p} p^{-m_1} \right)^k (p+q)^n.
    \]
    Отсюда
    \[
        \mu_2 = \sigma^2 = npq.
    \]

\subsection{Распределение Пуассона}

    Пусть число испытаний увеличивается, а вероятность \( P(A) \) уменьшается
    так, что среднее число событий остаётся постоянным.

    Воспользуемся биномиальным распределением и устремим \( n \) к
    бесконечности:
    \begin{gather*}
        P_n(m) = \frac{n\cdot\ldots\cdot(n-m+1)}{m!}
        \left(\frac{\average{m}}{n}\right)^m\cdot
        \left(1 - \frac{\average{m}}{n}\right)^{n-m} =\\
        =\frac{\average{m}^m}{m!}\left(1 - \frac{\average{m}}{n}\right)^n
        \left(1 - \frac{\average{m}}{n}\right)^{-m} \cdot 1 \cdot
        \left(1 - \frac{1}{n}\right)\cdot\ldots\cdot
        \left(1 - \frac{m-1}{n}\right) \xrightarrow[n\to\infty]{}\\
        \xrightarrow[n\to\infty]{} \frac{\average{m}^m}{m!}e^{-\average{m}}.
    \end{gather*}
    Полученное распределение называется распределением Пуассона.

    Определим его моменты:
    \[
        m_1 = \sum_{k=0}^\infty k\frac{\average{m}^k}{k!}e^{-\average{m}} =
        \sum_{k=0}^\infty \frac{\average{m}^k+1}{k!}e^{-\average{m}} =
        \average{m},
    \]
    \[
        m_2 = \sum_{k=0}^\infty k^2\frac{\average{m}^k}{k!}e^{-\average{m}} =
        \sum_{k=0}^\infty k(k-1)\frac{\average{m}^k}{k!}e^{-\average{m}} +
        \sum_{k=0}^\infty k\frac{\average{m}^k}{k!}e^{-\average{m}} +
        = \average{m}^2 + \average{m},
    \]
    \[
        \mu_2 = \sigma^2 = m_2 - m_1^2 = \average{m}.
    \]

    Вероятность того, что событие произойдёт не более \( m \) раз равна
    \[
        P(\le m) = \sum_{k=0}^m \frac{\average{m}^k}{k!}e^{-\average{m}} =
        1 - \frac{\Gamma(m+1, \average{m})}{\Gamma(m+1)}.
    \]

\subsection{Распределение Муавра-Лапласа. Распределение Гаусса}
    Если, устремляя в бесконечность \( n \), оставлять постоянной не среднее, а
    вероятность события, то биномиальное распределение переходит в распределение
    Муавра-Лапласа:
    \[
        L(m) = \frac{1}{\sqrt{2\pi}\sigma}
            e^{-\frac{\left(m-\average{m}\right)^2}{2\sigma^2}}.
    \]
    Если устремить в бесконечность и \( m \), принимая \( x = m / n \), то можно
    перейти к непрерывному распределению
    \[
        w(x) = \frac{1}{\sqrt{2\pi}\sigma}
            e^{-\frac{\left(x-\average{x}\right)^2}{2\sigma^2}},
    \]
    которое называется нормальным распределением или распределением Гаусса.
    Для двух величин распределение Гаусса имеет вид:
    \[
        w(x_1, x_2) = \frac{
            \exp\left\{-\frac{1}{2(1-r^2)}\left[
                \frac{(x_1 - \average{x_1})^2}{\sigma_1^2} -
                2r\frac{(x_1 - \average{x_1})(x_2 - \average{x_2})}
                    {\sigma_1\sigma_2} +
                \frac{(x_2 - \average{x_2})^2}{\sigma_2^2}
            \right]\right\}
            }{2\pi\sigma_1\sigma_2\sqrt{1 - r^2}}.
    \]

\subsection{Задачи}
    \begin{enumerate}
        \item По одной и той же цели производится пуск 5 ракет, причём
            вероятность попадания при каждом пуске равна \( p = 0,8 \).
            Построить ряд распределения числа попаданий, многоугольник
            распределения и функцию распределения.

            Число попаданий распределено по биномиальному закону:
            \[
                P(m) = C_n^m p^m q^{n-m}.
            \]
            \begin{table}[h!]
                \center
                \begin{tabular}{|c|c|} \hline
                    Число попаданий & Вероятность \\ \hline
                    0 & \( 0,2^5 = 0,00032 \) \\
                    1 & \( 5 \cdot 0,8 \cdot 0,2^4 = 0,0064 \) \\
                    2 & \( 10 \cdot 0,8^2 \cdot 0,2^3 = 0,0512 \) \\
                    3 & \( 10 \cdot 0,8^3 \cdot 0,2^2 = 0,2048 \) \\
                    4 & \( 5 \cdot 0,8^4 \cdot 0,2 = 0,4096 \) \\
                    5 & \( 0,8^5 = 0,32768 \) \\ \hline
                \end{tabular}
            \end{table}
            \begin{figure}[h]
                \center
                \begin{tikzpicture}[x=1cm,y=5cm]
                    \draw [thick,->] (-1,0) -- (6,0);
                    \draw [thick,->] (0,0) -- (0,1.2);
                    \node[draw,fill,circle,inner sep=1pt] at (0,0.00032) {};
                    \draw (0,0.00032) -- (1,0.0064);
                    \node[draw,fill,circle,inner sep=1pt] at (1,0.0064) {};
                    \draw (1,0.0064) -- (2, 0.0512);
                    \node[draw,fill,circle,inner sep=1pt] at (2,0.0512) {};
                    \draw (2, 0.0512) -- (3, 0.2048);
                    \node[draw,fill,circle,inner sep=1pt] at (3,0.2048) {};
                    \draw (3, 0.2048) -- (4, 0.4096);
                    \node[draw,fill,circle,inner sep=1pt] at (4,0.4096) {};
                    \draw (4, 0.4096) -- (5, 0.32768);
                    \node[draw,fill,circle,inner sep=1pt] at (5,0.32768) {};
                \end{tikzpicture}
                \hfill
                \begin{tikzpicture}[x=1cm,y=5cm]
                    \draw [thick,->] (-1,0) -- (6,0);
                    \draw [thick,->] (0,0) -- (0,1.2);
                    \draw (0,0.00032) -- (1,0.00032);
                    \draw [dashed] (1,0) -- (1,1.1);
                    \draw (1,0.00672) -- (2,0.00672);
                    \draw [dashed] (2,0) -- (2,1.1);
                    \draw (2,0.05792) -- (3,0.05792);
                    \draw [dashed] (3,0) -- (3,1.1);
                    \draw (3,0.26272) -- (4,0.26272);
                    \draw [dashed] (4,0) -- (4,1.1);
                    \draw (4,0.67232) -- (5,0.67232);
                    \draw [dashed] (5,0) -- (5,1.1);
                    \draw (5,1) -- (6,1);
                \end{tikzpicture}
            \end{figure}
        \item Из 10 транзисторов, среди которых 2 бракованных, случайным образом
            выбрано 2 транзистора для проверки. Определить и построить ряд
            распределения числа бракованных транзисторов в выборке.

            Два транзистора из 10 можно выбрать \( C_{10}^2 \) различными
            способами.
            \begin{table}[h!]
                \center
                \begin{tabular}{|c|c|} \hline
                    Число бракованных & Вероятность \\ \hline
                    0 & \( C_8^2 / C_{10}^2 = 28/45 \) \\
                    1 & \( C_8^1 \cdot C_2^1 / C_{10}^2 = 16/45 \) \\
                    2 & \( C_2^2 / C_{10}^2 = 1/45 \) \\ \hline
                \end{tabular}
            \end{table}

        \item Стрельба ведётся по наблюдаемой цели. Вероятность попадания при
            каждом выстреле \( p = 0,5 \). Вычислить математическое ожидание и
            дисперсию числа попаданий в цель при 5 выстрелах.

            Число попаданий в цель подчиняется биномиальному распределению.
            Поэтому \( \average{m} = np = 5 \cdot 0,5 = 2,5 \),
            \( \sigma^2 = npq = 5 \cdot 0,5 \cdot 0,5 = 1,25 \).
        \item Вероятность отыскания малоразмерного объекта в заданном районе в
            каждом вылете равна \( p \). Определить математическое ожидание и
            дисперсию числа произведённых независимых вылетов, которые
            выполняются до первого обнаружения объекта.

            Вероятность обнаружения в \( n \)-том вылете равна \( pq^{n-1} \).
            Поэтому
            \[
                \average{n} = \sum_{n=0}^\infty npq^{n-1} =
                p\cdot\pder{}{q}\sum_{n=0}^\infty q^n =
                p\cdot\pder{}{q}\frac{1}{1-q} = \frac{p}{(1-q)^2} = \frac{1}{p}.
            \]
            Для определения дисперсии определим \( \average{n^2} \):
            \[
                \average{n^2} = \sum_{n=0}^\infty n^2 p q^{n-1} =
                pq\sum_{n=0}^\infty n(n-1) q^{n-2} +
                p\sum_{n=0}^\infty n q^{n-1} = \frac{2pq}{p^3} + \frac{1}{p} =
                \frac{q+1}{p^2},
            \]
            \[
                \sigma^2 = \average{n^2} - \average{n}^2 = \frac{1-p}{p^2}.
            \]
        \item На маяк-ответчик поступает в среднем 15 запросов в час. Число
            запросов распределено по закону Пуассона. Какова вероятность
            того, что за 4 минуты поступит 3 запроса? Хотя бы 1 запрос?

            Среднее число запросов за 4 минуты равно \( \average{m} = 1 \).
            \[
                P(3) = \frac{1^3}{3!}e^{-1} = \frac{1}{6e} = 0,061.
            \]
            \[
                P(\ge 1) = 1 - P(0) = 1 - \frac{1^0}{0!}e^{-1} = 1 - \frac{1}{e}
                =0,63.
            \]
    \end{enumerate}
