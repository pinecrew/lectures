\chapter{Уравнения Максвелла}

\section{Гипотеза Максвелла}
\label{sec19:1}
	Рассмотрим процесс зарядки конденсатора. По теореме о циркуляции, вокруг
    провода будет магнитное поле:
	\begin{equation}
		\oint\limits_C \vec{B}\cdot\dd\vec{l} = \mu_0i.
        \label{eq19:1}
	\end{equation}
	Так как ток внутри конденсатора \( i_C = 0 \), то и поле \( \vec{B} \)
    внутри конденсатора должно быть \( \vec{B} = 0 \). Получается, что поле
    \( \vec{B} \) в вакууме меняется скачком, более того, что не определено
    само понятие обхвата контуром. Однако, опыт показывает, что поле
    \( \vec{B} \ne 0 \) не только напротив тока, но и напротив конденсатора,
    то есть скачка нет.
	
	Максвелл в 1850-е годы исследовал вопрос: чем создается поле \( \vec{B} \)
    в пустоте конденсатора? Он предположил, что поле \( \vec{B} \) создается
    меняющимся полем \( \vec{E}(t) \). Действительно, в плоском конденсаторе:
	\[
        E = \frac{\sigma}{\Ezero} = \frac{q}{\Ezero S}.
    \]
	При зарядном токе \( i \):
	\[
        \frac{\dd E}{\dd t} = \frac{1}{\Ezero S}\frac{\dd q}{\dd t} =
        \frac{i’}{\Ezero S}.
    \]
	
	Отсюда видно, что величина
    \[
        i’ = \varepsilon_0S\frac{\dd E}{\dd t}
    \]
    имеет размерность тока и равна по величине зарядному току \( i \). Она
    называется \textit{током смещения} \( i_{\textit{см}} \). Можно сказать,
    что ток смещения является “продолжением” зарядного тока в конденсаторе в
    том плане, что он тоже создает поле \( \vec{B} \) вокруг себя.
	
	В общем случае неоднородного поля \( \vec{E} \) в неплоском конденсаторе
    (в любом разрыве цепи):
	\begin{equation}
		i_{\textit{см}} =
        \Ezero \iint\limits_S \frac{\partial\vec{E}}{\partial t}\cdot\dd\vec{S},
        \label{eq19:2}
	\end{equation}
	где \( S \) -- поверхность, пронизываемая полем \( \vec{E} \).

	Таким образом, если в правой части (\ref{eq19:1}) вместо \( i \) поставить
    \( i + i_{\textit{см}} \), то оно будет выглядеть следующим образом:
	\[
        \oint\limits_C \vec{B}\cdot\dd\vec{l} =
        \mu_0\left(i + \Ezero\iint\limits_S
        \frac{\partial\vec{E}}{\partial t}\cdot\dd\vec{S}\right).
    \]
	
	Или, так как \( \Ezero\mu_0 = 1/c^2 \):
	\begin{equation}
		\oint\limits_C \vec{B}\cdot\dd\vec{l} = \mu_0i +
        \frac{1}{c^2}\iint\limits_S
        \frac{\partial\vec{E}}{\partial t}\cdot\dd\vec{S}.
        \label{eq19:3}
	\end{equation}
	
	Уравнение (\ref{eq19:3}) дает для поля \( \vec{B} \) всегда верный
    результат, как для с цепей с разрывом, так и без него. При \( i = 0 \)
    получим уравнение, согласующееся с законом ЭМИ (\ref{eq11:3}):
	\begin{equation}
		\oint\limits_C \vec{B}\cdot\dd\vec{l} =
        \frac{1}{c^2}\iint\limits_S
        \frac{\partial\vec{E}}{\partial t}\cdot\dd\vec{S}.
        \label{eq19:4}
	\end{equation}
	
	По теореме Стокса:
	\[
        \oint\limits_C \vec{B}\cdot\dd\vec{l} =
        \iint\limits_S \rot\vec{B}\cdot\dd\vec{S}.
    \]
	В силу произвольности \( S \) и с учетом (\ref{eq19:4}), получим:
	\begin{equation}
		\rot\vec{B} = \frac{1}{c^2}\frac{\partial\vec{E}}{\partial t}.
        \label{eq19:5}
	\end{equation}
	
	Уравнения (\ref{eq19:5}) и (\ref{eq11:3a}) выражают единство и симметрию
    электрического и магнитного полей: меняющееся поле \( \vec{E} \) порождает
    поле \( \vec{B} \) (\ref{eq19:5}), а меняющееся поле \( \vec{B} \) порождает
    поле \( \vec{E} \) (\ref{eq11:3a}).
	
	Если ток \( i \) распределен непрерывно, то есть
    \[
        i = \iint\limits_S \vec{j}\cdot\dd\vec{S},
    \]
    то получим дифференциальный вид уравнения (\ref{eq19:3}):
	\begin{equation}
		\rot\vec{B} = \mu_0\vec{j} +
        \frac{1}{c^2}\frac{\partial \vec{E}}{\partial t}.
        \label{eq19:3a}
	\end{equation}
	
\section{Примеры, подтверждающие гипотезу Максвелла}

    \subsection{Закон сохранения заряда}
        
        Запишем теорему о циркуляции в дифференциальном виде:
        \begin{equation}
            \rot\vec{B} = \mu_0\vec{j}.
            \label{eq19:ex1}
        \end{equation}
        Возьмем дивергенцию от обоих частей:
        \[
            \div\rot\vec{B} = \mu_0\div\vec{j}.
        \]
        А так как \( \div\rot\vec{B} \equiv 0 \), то и \( \div\vec{j} \) должно
        быть равно нулю, однако, в силу закона сохранения заряда, это не так:
        \[
            \div\vec{j} = -\partder{\rho}{t}.
        \]
        Вывод: неверно уравнение (\ref{eq19:ex1}). Если вместо (\ref{eq19:ex1})
        записать уравнение (\ref{eq19:3a}), то:
        \[
            \div\rot\vec{B} = \mu_0\div\vec{j} +
            \frac{1}{c^2}\partder{\div\vec{E}}{t} = \mu_0\div\vec{j} +
            \frac{1}{\Ezero}\partder{\rho}{t} = \mu_0\left(\div{j} +
            \partder{\rho}{t}\right. = 0.
        \]
        Следовательно:
        \[
            \div{j} = -\partder{\rho}{t},
        \]
        что совпадает с законом сохранения заряда.

    \subsection{Изолированный источник зарядов}

        Рассмотрим маленький шарик, который может испускать заряды изотропно во
        все стороны. Так как \( \vec{j} \ne 0 \), то вокруг шарика должно быть
        поле \( \vec{B} \ne 0 \). Однако цилиндрическая симметрия поля
        \( \vec{B} \) несовместима со сферической симметрией источника, то есть
        невозможно провести замкнутую линию поля \( \vec{B} \), не выделив
        какого-либо направления в пространстве. С этой точки зрения поле
        \( \vec{B} \) должно быть равно нулю. Запишем уравнение Максвелла
        (\ref{eq19:3a}) и покажем, что из него следует равенство нулю поля
        \( \vec{B} \).
        
        Так как вокруг шарика поле
        \[
            E = \frac{q(t)}{4\pi\Ezero r^2},
        \]
        то уравнение Максвелла примет вид:
        \[
            \frac{1}{c^2}\partder{E}{t} =
            \frac{1}{c^2}\frac{1}{4\pi\Ezero r^2}\simpder{q}{t}.
        \]
        Но, согласно закону сохранения заряда,
        \[
            \simpder{q}{t} =
            -\iint\limits_{S = 4\pi r^2} \vec{j}\cdot\dd\vec{S} =
            -j\cdot4\pi r^2.
        \]
        Следовательно
        \[
            \frac{1}{c^2}\partder{E}{t} = -\mu_0j,
        \]
        и в правой части уравнения Максвелла (\ref{eq19:3a}) для такого
        источника будет ноль, то есть \( \rot\vec{B} = 0 \). А в силу симметрии
        задачи и само поле \( \vec{B} \) будет равно нулю.

    \subsection{Саморазряд конденсатора}
        Рассмотрим плоский конденсатор с дискообразными обкладками и со
        слабопроводящим заполнением с удельной проводимостью \( \lambda \).
        Пусть он заряжен до некоторого напряжения и предоставлен сам себе.
        Будет ли поле \( B(r) \) не равно нулю, и если да, то найти его.

\section{Система уравнений Максвелла}

	Система из четырех уравнений
	\begin{equation}
        \left\{
        \begin{array}{l}
            \oiint\limits_S \vec{E}\cdot\dd\vec{S} = \frac{q_S}{\Ezero}, \\
            \oint\limits_C \vec{E}\cdot\dd\vec{l} =
            -\oiint\limits_S \partder{\vec{B}}{t}\cdot\dd\vec{S}, \\
            \oiint\limits_S \vec{B}\cdot\dd\vec{S} = 0, \\
            \oint\limits_C \vec{B}\cdot\dd\vec{l} = \mu_0i_C +
            \frac{1}{c^2}\oiint\limits_S \partder{\vec{E}}{t}\cdot\dd\vec{S},
        \end{array}
        \right.
        \label{eq19p3:1}
    \end{equation}
	называется \textbf{системой уравнений Максвелла} в интегральном виде. Ее
    также можно записать и в дифференциальном виде:
	\[
        \left\{
        \begin{array}{l}
            \div\vec{E} = \frac{\rho}{\Ezero}, \\
            \rot\vec{E} = -\partder{\vec{B}}{t}, \\
            \div\vec{B} = 0, \\
            \rot\vec{B} = \mu_0\vec{j} + \frac{1}{c^2}\partder{\vec{E}}{t}.
        \end{array}
        \right.
    \]
    \begin{itemize}
        \item Уравнение (\ref{eq19p3:1}.1) -- теорема Гаусса, которая является
            следствием закона Кулона -- источниками поля \( \vec{E} \) являются
            заряды;
        \item уравнение (\ref{eq19p3:1}.2) -- закон ЭМИ -- поле \( \vec{E} \)
            порождается так же и меняющимся полем \( \vec{B} \);
        \item уравнение (\ref{eq19p3:1}.3) -- источников поля \( \vec{B} \)
            (монополей) не существует;
        \item уравнение (\ref{eq19p3:1}.4) -- закон магнитно-электрической
            индукции -- поле \( \vec{B} \) порождается как токами \( \vec{j} \),
            так и меняющимся полем \( \vec{E} \).
    \end{itemize}
	
	Свойства уравнений (\ref{eq19p3:1}):
	\begin{itemize}
	\item они линейны относительно полей \( \vec{B} \) и \( \vec{E} \);
	\item они согласуются с законом сохранения заряда;
	\item поля \( \vec{B} \) и \( \vec{E} \) в них не совсем симметричны: у поля
        \( \vec{E} \) есть источники, а у поля \( \vec{B} \) их нет;
	\item они предсказывают возможность существования электромагнитных волн --
        электромагнитных возмущений, распространяющихся в пространстве со
        скоростью света \( c \) или \( v = c/\sqrt{\varepsilon\mu} \),
        если волна распространяется не в вакууме. Действительно, изменяющееся
        поле \( \vec{B}(t) \), согласно (\ref{eq19p3:1}.2), порождает поле
        \( \vec{E} \), а изменяющееся поле \( \vec{E} \), согласно
        (\ref{eq19p3:1}.4), порождает поле \( \vec{B} \);
	\item в статическом случае уравнения (\ref{eq19p3:1}) распадаются на две
        подсистемы уравнений электро-
		\[
            \left\{
            \begin{array}{l}
                \div\vec{E} = \frac{\rho}{\Ezero}, \\
                \rot\vec{E} = 0;
            \end{array}
            \right.
        \]
		 и магнитостатики:
		 \[
            \left\{
            \begin{array}{l}
                \div\vec{B} = 0, \\
                \rot\vec{B} = \mu_0\vec{j}.
            \end{array}
            \right.
        \]
	\end{itemize}

\section{Уравнения Максвелла для вещества}

	Система уравнений (\ref{eq19p3:1}), в принципе, описывает электромагнитные
    процессы как в вакууме, так и в веществе. Однако, в ней в уравнениях
    (\ref{eq19p3:1}.1) и (\ref{eq19p3:1}.4) стоят полные токи и полные заряды --
    как свободные, так и связанные. Удобно выделить в уравнениях только
    свободные заряды и токи, которые можно измерять и управлять ими.
	
	Запишем уравнения Максвелла для вещества, в которых стоят только свободные
    заряды и токи.
	
	Уравнение (\ref{eq19p3:1}.1) будет выглядеть так:
	\[
        \oiint\limits_S \vec{D}\cdot\dd\vec{S} = q_S
    \]
	или так:
	\[
        \div\vec{D} = \rho.
    \]

	Получим теперь уравнение, аналогичное (\ref{eq19p3:1}.4), но для свободных
    токов и зарядов. Для этого вновь рассмотрим процедуру раздела
    (\ref{sec19:1}). Так как поле \( \vec{D} \) связано с поверхностной
    плотностью свободных зарядов \( \sigma \): \( D = \sigma \), то в разрыве:
	\[
        \simpder{D}{t} = \simpder{\sigma}{t} = \frac{1}{S}\simpder{q}{t} =
        \frac{1}{S}i.
    \]
	
	Следовательно, свободным током смещения в таком случае будет величина
	\[
        i’ = \simpder{D}{t}S,
    \]
	или, в общем случае неоднородного поля \( \vec{D} \):
	\[
        i’ = \iint\limits_S \partder{\vec{D}}{t}\cdot\dd\vec{S}.
    \]
	И тогда, вместо теоремы о циркуляции вектора \( \vec{H} \)
    \[
        \oint\limits_C \vec{H}\cdot\dd\vec{l} = i,
    \]
    для цепи с разрывом следует записать:
	\[
        \oint\limits_C \vec{H}\cdot\dd\vec{l} = i + i’ =
        i + \iint\limits_S \partder{\vec{D}}{t}\cdot\dd\vec{S},
    \]
	или в дифференциальном виде:
	\[
        \rot\vec{H} = \vec{j} + \partder{\vec{D}}{t}.
    \]
	
	Тогда система (\ref{eq19p3:1}) для вещества будет иметь вид:
	\begin{equation} 
        \left\{
        \begin{array}{l}
            \div\vec{D} = \rho \\
            \rot\vec{E} = -\partder{\vec{B}}{t} \\
            \div\vec{B} = 0 \\
            \rot\vec{H} = \vec{j} + \partder{\vec{D}}{t}
        \end{array}
        \right.
        \label{eq19p3:2}
    \end{equation}
	
	Однако, система (\ref{eq19p3:2}) стала неполной, так как появляются два
    новых поля -- \( \vec{D} \) и \( \vec{H} \), а количество уравнений не
    увеличилось. Чтобы ее дополнить надо дописать связи между полями
    \( \vec{D} \) и \( \vec{E} \), \( \vec{H} \) и \( \vec{B} \). В общем случае
    эта связь очень сложна. Однако, если выполнены следующие условия:
	\begin{itemize}
        \item если поля \( \vec{B} \) и \( \vec{E} \) изменяются не очень
            быстро: период \( T \) таков, что соответствующая длина волны
            \( \lambda = cT \gg r_\textit{межатомн} \sim 1
            \overset{\text{О}}{\text{А}} \approx 0,1\text{нм} \), то есть
            \( \lambda \gtrsim 10 \)нм;
            
        \item среда линейна, то есть \( \vec{B} \sim \vec{H} \) и
            \( \vec{D} \sim \vec{E} \) (не ферромагнетик и не сегнетоэлектрик);
            
        \item среда изотропна,
	\end{itemize}	
	то связь \( \vec{D} \) и \( \vec{E} \), \( \vec{H} \) и \( \vec{B} \) будет
    простой:
	\[
        \vec{D} = \varepsilon\Ezero\vec{E},
    \]
	\[
        \vec{B} = \mu\mu_0\vec{H}.
    \]
	Эти уравнения называют \textit{материальными}.
	
    Уравнения (\ref{eq19p3:2}) принимают особенно простой вид при следующих
    дополнительных ограничениях:
	\begin{itemize}
	\item среда однородна, то есть \( \varepsilon = \const_{x, y, z} \) и
        \( \mu = \const_{x, y, z } \);
	\item среда нейтральна, то есть \( \rho = 0 \);
	\item среда непроводящая, то есть \( \vec{j} = 0 \).
	\end{itemize}
	Тогда уравнения (\ref{eq19p3:2}):
	\begin{equation}
        \left\{
        \begin{array}{l}
            \div\vec{E} = 0, \\
            \rot\vec{E} = -\mu\mu_0\partder{\vec{H}}{t}, \\
            \div\vec{H} = 0, \\
            \rot\vec{H} = \varepsilon\Ezero\partder{\vec{E}}{t}.
        \end{array}
        \right.
        \label{eq19p3:3}
    \end{equation}
