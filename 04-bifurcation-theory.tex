\section{Элементы теории бифуркаций}

Под бифуркациями обычно понимают явления потери устойчивости колебательной
системы, обычно происходящие при изменении каких-либо её параметров.

\subsection{Локальный анализ неограниченного движения}
Движение системы описывается вторым законом Ньютона:
\[
    \ddot{x} = F(x).
\]
Рассмотрим движение вблизи точки \( x_0 \):
\[
    \ddot{x} = F(x_0),\quad \dot{x}^2 - 2F(x_0)x = C.
\]
Движение инфинитное, никаких намёков на бифуркацию.

\subsection{Локальный анализ вблизи точки равновесия}
\[
    \ddot{x} = F(x),
\]
в точке равновесия \( F(x_0) = 0 \), поэтому разложим силу в ряд Тейлора вблизи
точки равновесия:
\[
    \ddot{\xi} = F'(x_0)\xi,\quad \xi = x - x_0.
\]
Интегрируя, получаем
\[
    \dot{\xi}^2 - F'(x_0)\xi^2 = C.
\]
В зависимости от знака производной, получаем финитное (\( F'(x_0) < 0 \)) или
инфинитное (\( F'(x_0) > 0 \)) движение. И снова без намёка на бифуркацию.

\subsection{Бифуркация ``седло-центр''}
Пусть теперь сила зависит от некоторого параметра. Уравнение движения имеет тот
же вид
\[
    \ddot{x} = F(x, a).
\]
Пусть сила в точке равновесия удовлетворяет условию
\[
    F(x_0, a_0) = F'_x(x_0, a_0) = 0.
\]
Тогда раскладывая её в ряд Тейлора, получим
\[
    \ddot{x} = F'_a(x_0, a_0)(a-a_0) + \frac{1}{2}F''_{xx}(x_0, a_0)(x-x_0)^2.
\]
Переходя к скользящим переменным, имеем
\[
    \ddot{\xi} = F'_\alpha(0, 0)\alpha + \frac{1}{2}F''_{\xi\xi}(0,0)\xi^2.
\]
Интегрируя это уравнение, приходим к
\[
    \dot{\xi}^2 - 2F'_\alpha \alpha x - \frac{1}{3}F''_{\xi\xi}\xi^3 = C.
\]
При переходе \( \alpha \) через ноль характер движения меняется. Это и есть
бифуркация. Тут нужен фазовый портрет, профиль потенциальной энергии и другие
пояснения.
