\section{Рынок как саморегулирующаяся система}

Рынок -- это сложная система взаимоотношений производителей и потребителей и
хозяйственных связей, включая непосредственные контакты с участием посредников,
складывающееся в связи с формированием свободных цен, колеблющихся в зависимости
от динамики спроса и предложения.

Рынок функционирует на основе обмена эквивалентами ``товар--деньги--товар''.

\subsection{Функции рынка}
\begin{enumerate}
    \item Стимулирование экономии используемых ресурсов.
    \item Адаптация производства к меняющему спросу.
    \item Санация рынка.
    \item Информационная функция.
\end{enumerate}

\subsection{Структура рыночной системы}
\begin{enumerate}
    \item Покупатели и производители.
    \item Спрос и предложение.
    \item Конкуренция.
    \item Инфраструктура рынка (контролирующие системы, способы доставки, \ldots).
\end{enumerate}

\subsection{Преимущества рынка}
\begin{itemize}
    \item Ориентирован на потребителя.
    \item Является саморегулирующейся системой.
    \item Задаёт ориентиры для капиталовложений и производства.
\end{itemize}

\subsection{Недостатки}
\begin{itemize}
    \item Не обеспечивает удовлетворение общества в общественных благах.
    \item Порождает внешние эффекты -- экстернали.
    \item Не в состоянии обеспечить стратегический прорыв в науке.
    \item Не обеспечивает устранение неравенства в распределении доходов.
    \item Не решает региональных проблем.
    \item Не реализует интересы отдельных стран в сфере международных отношений.
\end{itemize}

\subsection{Классификация рынков}
\begin{itemize}
    \item С точки зрения объекта купли-продажи:
    \begin{enumerate}
        \item рынки товаров и услуг;
        \item рынки труда;
        \item рынки капитала;
        \item рынки технологий;
    \end{enumerate}
    \item По географическому положению:
    \begin{enumerate}
        \item местные;
        \item региональные;
        \item национальные;
        \item мировые.
    \end{enumerate}
    \item В зависимости от развития конкуренции:
    \begin{enumerate}
        \item рынки совершенной конкуренции;
        \item рынки несовершенной конкуренции;
    \end{enumerate}
    \item По характеру продаж:
    \begin{enumerate}
        \item оптовые;
        \item розничные.
    \end{enumerate}
    \item С позиции соответствия правовым нормам:
    \begin{enumerate}
        \item легальные;
        \item нелегальные.
    \end{enumerate}
\end{itemize}

\subsection{Основные понятия}
Спрос -- это экономическая категория, характеризующая объем товаров, которые
потребители желают и могут купить по определённой цене за определённое время.

Закон спроса -- при прочих равных условиях спрос на товары в количественном
отношении изменяется в обратной зависимости от цены.

% график PQ-диаграммы (Price - цена, Quantity - количество, Demand - спрос)

При изменении цены изменяется цена спроса и происходит движение по кривой
спроса. Неценовые факторы производства приводят к изменению объема спроса и
характерному сдвигу кривой.

Примеры неценовых факторов:
\begin{itemize}
    \item доходы покупателей;
    \item их вкусы и предпочтения;
    \item изменения в экономической и политической ситуациях в стране;
    \item наличие или отсутствие товаров-субститутов;
    \item число покупателей.
\end{itemize}

Предложение -- это количество товаров, которое производители готовы продать по
определённой цене в определённые период времени.

Закон предложения -- чем выше при прочих равных условиях цена, тем больше
стремление продавцов товара продавать его на рынке.

% кривая предложения (S - предложение)

Неценовые факторы предложения:
\begin{itemize}
    \item налоги;
    \item уровень развития научно-технического прогресса;
    \item конкуренция;
    \item цены на ресурсы.
\end{itemize}

Равновесной называется цена, при которой размеры спроса соответствуют величине
предложения и отсутствуют дефицит или избыток товаров.

% наложение графиков 1 и 2 с пересечением в т. E. Выше E -- избыток,
% ниже -- дефицит. что происходит, если цена становится

Эластичность спроса по цене -- степень реагирования спроса на изменения цены
товара:
\[
    E^P_D = \frac{\Delta Q}{\Delta P}\cdot100\%.
\]

\begin{itemize}
    \item если \( E^P_D > 1 \), то спрос считается эластичным;
    \item если \( E^P_D = \infty \), то спрос -- абсолютно эластичный;
    \item если \( E^P_D = 1 \), то спрос -- единично эластичный;
    \item если \( 0 < E^P_D < 1 \), то спрос -- не эластичный.
\end{itemize}

Точечная эластичность:
\[
    E_D = -\frac{\Delta Q}{\Delta P}\cdot\frac{P}{Q} =
    \frac{Q_2 - Q_1}{P_2 - P_1}\cdot\frac{P_1 + P_2}{Q_1 + Q_2}.
\]

Перекрёстная эластичность -- эластичность спроса на благо \( a \) относительно
цен на благо \( b \):
\[
    \eps_{ab} = -\frac{\Delta Q_a}{Q_a}\cdot\frac{P_b}{\Delta P_b}.
\]

Эластичность спроса по доходам -- характеризует изменение величины спроса в
зависимости от изменения величины дохода потребителей:
\[
    E_D^I = -\frac{\Delta Q}{Q}\cdot\frac{I}{\Delta I}.
\]

Эластичность предложения по цене -- реакция величины предложения на изменение
цены товара:
\[
    E_S^P = \frac{\Delta Q}{\Delta P}.
\]
