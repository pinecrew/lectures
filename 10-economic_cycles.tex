\section{Цикличность развития экономики и экономические кризисы}

Макроэкономическое неравновесие -- это отклонение ключевых показателей
национальной экономики от нормальных значений.

Экономический цикл -- это постоянное повторение периодов подъема и спада
экономической активности хозяйствующих субъектов, отражающееся через изменение
темпов экономического роста уровня цен и размеров безработицы.

Виды экономических циклов:
\begin{itemize}
    \item по продолжительности:
    \begin{enumerate}
        \item краткосрочные,
        \item среднесрочные,
        \item долгосрочные (длинные волны);
    \end{enumerate}
    \item по сфере действия:
    \begin{enumerate}
        \item промышленные,
        \item аграрные;
    \end{enumerate}
    \item по специфике проявления:
    \begin{enumerate}
        \item нефтяные,
        \item продовольственные,
        \item энергетические,
        \item сырьевые,
        \item экологические,
        \item валютные;
    \end{enumerate}
    \item по формам развертывания:
    \begin{enumerate}
        \item структурные,
        \item отраслевые;
    \end{enumerate}
    \item по пространственному признаку:
    \begin{enumerate}
        \item национальные;
        \item межнациональные.
    \end{enumerate}
\end{itemize}

Виды краткосрочных циклов:
\begin{itemize}
    \item циклы Дж. Китчина (продолжительность 3 года и 4 месяца), причиной
    которых являются колебания мировых запасов золота;
    \item циклы У. Митчелла (продолжительность 3 года и 4 месяца), причина
    которых внутри экономической системы, экономические циклы -- это продукт
    <<денежного хозяйства>>;
    \item современные циклы США (продолжительность 3 года и 4 месяца):
    восстановление экономического равновесия на потребительском рынке.
\end{itemize}

Виды среднесрочных циклов:
\begin{itemize}
    \item промышленные циклы К. Маркса (8-12 лет), причиной является массовое
    обновление производственных фондов;
    \item экономические циклы К. Жугляра (10 лет), причинами являются нарушения
    в кредитно-денежной системе;
    \item строительные циклы С. Кузнеца (15-20 лет), причиной является
    обновление жилых зданий и производственных сооружений.
\end{itemize}

Виды долгосрочных циклов:
\begin{itemize}
    \item циклы Кондратьева (40-60 лет), причины: технический прогресс,
    структурные изменения;
    \item циклы Форрестера (200 лет), причина: смена энергетических ресурсов и
    материалов;
    \item циклы Тоффлера (1000-2000 лет), причина: развитие цивилизаций.
\end{itemize}

Классификация экономических теорий циклов по П. Самуэльсону:
\begin{itemize}
    \item экстернальные теории: причины возникновения циклов -- это внешние
    факторы, такие как войны, революции, различные политические события,
    миграция населения, научные открытия;
    \item интернальные теории: причины возникновения циклов -- это внутренние
    факторы: диспропорции в структуре производства, нарушения в области
    денежного обращения, перенакопление основного капитала.
\end{itemize}

Концепции государственного антициклического регулирования:
\begin{itemize}
    \item неокейнсианский подход: рычагами воздействия являются изменение
    совокупного спроса, увеличение или уменьшение расходов государства,
    изменение налоговой системы, изменение уровня заработной платы;
    \item неоконсервативный (монетаристскиий) подход: рычагами воздействия
    являются изменение кредитно-денежной политики, таргетирование (установление
    верхних и нижних пределов роста) денежной массы, изменение кредитной
    системы, изменение процентной ставки.
\end{itemize}

Важнейшие фазы цикла -- спад и подъем производства.

Характерные черты подъема: увеличение запасов, рост производственных инвестиций,
увеличение спроса на труд, повышение прибылей, расширение спроса на кредит, рост
производства.

Характерные черты спада: ликвидация запасов, спад производственных инвестиций,
падение спроса на труд, резкое падение прибылей, сужение спроса на кредит, спад
производства.

Виды кризисов:
\begin{enumerate}
    \item циклический кризис перепроизводства;
    \item промежуточный кризис;
    \item частичный кризис;
    \item отраслевой
    \item структурный
    \item системный.
\end{enumerate}

Промышленный цикл и его фазы:
по К. Марксу --------- по К. Макконеллу и С. Брю