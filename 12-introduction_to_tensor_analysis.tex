\section{Введение в тензорный анализ.}

\subsection{Тензорное поле.}

	Если анизотропная среда является к тому же и неоднородной, хотя и линейной, то тензор \( T \), описывающий ее свойства, в разных ее точках различен, то есть является функцией координат точки: \( T = T(x, y, z) \). Это означает, что каждая компонента тензора является функцией координат точки: \( a_{ij} = f_{ij}(x, y, z) \).
	
	\begin{definition}
	
	Если в каждой точке \( M(x, y, z) \) некоторой области пространства определен тензор постоянного ранга, то говорят, что в этой области задано \textbf{тензорное} поле \( T = T(x, y, z) \).
	\end{definition}
	
	\begin{definition}
	
	Если во всей области тензор \( T \) является постоянным \( T = \const \), то тензорное поле называют \textbf{однородным}.
	\end{definition}
	
	Изучением свойств тензорных полей, операций интегрирования и дифференцирования над ними занимается тензорный анализ. 

\subsection{Некоторые операции тензорного анализа.}

\subsubsection{Дифференцирование по времени.}

	Операция дифференцирования тензора по времени -- это операция дифференцирования по времени каждой его компоненты.
	
	Пусть \( T = [a_ij] \), тогда \( \der{T}{t} = \left[\der{a_{ij}}{t}\right] \).

\subsubsection{Градиент векторного поля.}

	Производная поля \( \vec{a} \) по направлению: \( \pder{\vec{a}}{l} = (\vec{l}\cdot\nabla)\vec{a} \).
	
	Эту операцию можно представить как тензорную:
	\[ \pder{\vec{a}}{l} = \vec{l}\cdot\nabla\vec{a}, \]
	где \( \nabla\vec{a} \) -- полное произведение символического вектора \( \nabla \) на вектор \( \vec{a} \), результатом которого является тензор второго ранга, называемый \textbf{градиентом векторного поля}.
	\[ \nabla\vec{a} = \left[\pder{a_i}{x_j}\right] =
	\begin{bmatrix}
	\pder{a_1}{x_1} & \pder{a_1}{x_2} & \pder{a_1}{x_3} \\[0.3em]
	\pder{a_2}{x_1} & \pder{a_2}{x_2} & \pder{a_2}{x_3} \\[0.3em]
	\pder{a_3}{x_1} & \pder{a_3}{x_2} & \pder{a_3}{x_3}
	\end{bmatrix}. \]
	
	Он является мерой неоднородности векторного поля.
	
	Тогда производная векторного поля по направлению -- это скалярное произведение единичного вектора \( \vec{l} \) на тензор \( \nabla\vec{a} \):
	\[ \left(\pder{\vec{a}}{\vec{l}}\right)_i = \sum\limits_j l_j \pder{a_i}{x_j}. \]
	
	\begin{remark}
	Сверткой тензора \( \nabla\vec{a} \) является дивергенция поля \( \vec{a} \):
	\[ \sum\left(\nabla\vec{a}\right)_{ii} = \sum\limits_i \pder{a_i}{x_i} = \pder{a_1}{x_1} +  \pder{a_2}{x_2} + \pder{a_3}{x_3} = \divergence\vec{a}. \]
	\end{remark}

\subsubsection{Градиент тензорного поля.}

\subsubsection{Производная тензорного поля по направлению.}

	
