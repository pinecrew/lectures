\subsection{Тепловое излучение}
\subsubsection{Развитие представлений об излучении нагретых тел}
    Классическая теория конца XIX века не могла объяснить излучение и поглощение электромагнитных волн веществами. \ldots надо что-то написать сюда.
\subsubsection{Основные определения}
    Всё излучение делится на тепловое излучение и остальное, называемое люминесценцией.
    
    Тепловое излучение -- излучение за счет внутренней энергии тела, оно есть всегда.
    
    Люминесценция -- излучение за счет других видов энергии:
    \begin{itemize}
        \item хемилюминесценция -- за счет энергии химических реакций;
        \item электролюминесценция -- за счет воздействия электрическим полем;
        \item фотолюминесценция -- за счет внешнего облучения электромагнитными волнами.
    \end{itemize}
    
    Основные свойства теплового излучения:
    \begin{enumerate}
        \item Из всех видов излучения является единственным равновесным (сколько энергии в виде электромагнитных волн излучается с единицы площади в единицу времени, столько же и поглощается). Это обусловлено тем, что интенсивность теплового излучения возрастает с увеличением температуры.
        \item Немонохроматичность излучаемых электромагнитных волн.
        \item Неполяризованность излучаемых электромагнитных волн.
    \end{enumerate}
    
    Количественные характеристики теплового излучения:
    \begin{enumerate}
        \item Энергетическая светимость \( R_T \) (интегральная испускательная способность) -- количество энергии излучаемое телом в виде электромагнитного излучения с единичной площадки за единицу времени:
        \[ R_T = \der{W}{S\ dt}; [R_T] = \left[ \frac{\text{Дж}}{\text{м}^2\cdot\text{с}} \right] = \left[ \frac{\text{Вт}}{\text{м}^2} \right]. \]
        \item Спектральная испускательная способность \( r \) -- количество энергии излучаемое телом в виде электромагнитного излучения с единичной площадки за единицу времени в узкий частотный диапазон \( [\omega;\ \omega + d\omega] \).
        \[ r_\omega = \der{W}{S\ dt\ d\omega} = \der{R}{\omega}; r_\lambda = \der{R}{\lambda}. \]
        \[ r_\omega\ d\omega = r_\lambda\ d\lambda; r_\omega = r_\lambda\der{\lambda}{\omega} = \frac{2\pi c}{\omega^2}r_\lambda. \]
        \item Поглощающая способность -- безразмерная величина, равная отношению поглощенной телом энергии к падающей на это тело энергии в виде электромагнитных волн в узком частотном интервале \( [\omega;\ \omega + d\omega] \).
        \[ a_\omega = \der{W_\textit{полг}}{W_\textit{пад}} = [0, \ldots, 1]. \]
        Если \( a = 1 \) при любой температуре, то такое тело называется абсолютно черным. Если какая-либо величина описывает это тело, то к ней добавляется индексом символ \( ^* \): \( a^*(\omega, T) = 1 \).
        
        Если \( a = 0 \) при любой температуре, то такое тело называется абсолютно белым.
        
        Абсолютно серыми называются тела, которые имеют постоянный коэффициент поглощения во всем частотном интервале: \( a^\textit{сер}(\omega, T) = const < 1 \).
    \end{enumerate}
    
\subsection{Законы теплового излучения}

    Закон Кирхгофа -- отношение испускательной к поглощательной способности не зависит от природы тела и является функцией частоты и температуры, причем эта функция одинакова для всех тел.
    
    \[ \left(\frac{r(\omega, T)}{a(\omega, T)}\right)_1 = \left(\frac{r(\omega, T)}{a(\omega, T)}\right)_2 = \ldots = r^*(\omega, T) = f(\omega, T). \]
    
    Из этого закона есть два следствия:
    \begin{enumerate}
    \item Если на какой-то частоте тело излучает больше, то на этой частоте тело и поглощает больше:
    \[ r_1 > r_2;\, a_1 = a_2\frac{r_1}{r_2} > a_2. \]
    \item Среди всех тел наибольшей испускательной способностью обладает абсолютно черное тело:
    \[ r(\omega, T) = r^*(\omega, T)\cdot a(\omega, T) < r^*(\omega, T). \]
    \end{enumerate}
    
    % график %
    
    \[ f(\omega, T) = r^*(\omega, T) = r^*(\lambda, T)\cdot\frac{2\pi c}{\omega^2} = \frac{2\pi c}{\omega^2} \phi(\lambda, T). \]
    \begin{enumerate}
    \item Площадь под графиком -- энергетическая светимость \( R_T \), с увеличением температуры \( T \) она резко возрастает.
    \item С увеличением температуры длина волны, на которую приходится 
    \end{enumerate}