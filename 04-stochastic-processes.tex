\section{Случайные функции}

Скалярная или векторная функция одной или нескольких переменных называется
случайной, если её значения при произвольной последовательности выбора значений
аргументов являются набором случайных величин. Случайная функция времени
называется случайным процессом, а случайная функция координат (и, быть может,
времени) -- случайным полем.

Случайные процессы делятся на два класса:
\begin{itemize}
    \item с дискретным временем -- случайные последовательности;
    \item с непрерывным временем.
\end{itemize}

Примером случайного процесса может служить броуновское движение. В каждый момент
времени такой процесс характеризуется одномерной плотностью вероятности
\( w_t(x) \). Она даёт представление о процессе лишь в данный момент времени.

Для случайного процесса можно ввести функцию распределения:
\[
    F(x_1, \ldots, x_n; t_1, \ldots, t_n) =
        P(X(t_1) \le x_1, \ldots, X(t_n) \le x_n).
\]

Она имеет два свойства:
\begin{enumerate}
    \item симметричность -- функция распределения инвариантна относительно
        одновременной перестановки \( x_m \) и \( x_k \) и соответствующих
        им моментов времени \( t_m \) и \( t_k \);
    \item согласованность.
\end{enumerate}
