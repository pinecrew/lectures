\lecture{Выборочный метод социологического исследования}
  \section{Основные понятия выборочного социологического исследования}

    \emph{Генеральная совокупность} (или объект исследования)~-- вся группа,
    общность людей, про которую нужно собрать данные.

    \emph{Выборочная совокупность}~-- часть генеральной совокупности, которая
    непосредственно подвергается исследованию.

    \emph{Единица выборки} (отбора, наблюдения)~-- элемент генеральной
    совокупности, который непосредственно подвергается изучению.

    \emph{Контур выборки}~-- все единицы генеральной совокупности, из которых
    формируется выборка.

    В связи с тем, что контур выборки может не совпадать со всей генеральной
    совокупностью, возникает \emph{ошибка контура выборки}~-- степень отклонения
    контура выборки от генеральной совокупности.

    Естественно, существует и \emph{ошибка выборки}~-- разница между полученными
    по выборке и фактическими данными по генеральной совокупности. В связи с
    этим выдвигают основное требование к выборке: выборка должна позволять
    собирать точную и надежную информацию о генеральной совокупности.

    Основное влияние на точность, достоверность и надежность получаемой
    информации оказывают процесс извлечения и объем выборки. Выборка должна быть
    \emph{репрезентативной}. Репрезентативность~-- свойство выборки достаточно
    полно и точно представлять наиболее важные для исследователя и
    коррелирующими с ними признаки генеральной совокупности.

  \section{Расчет объема выборки}

    Для расчета случайной бесповторной выборки используется следующая формула:
    \[
      n = \frac{Z^2 Npq}{\D^2 N + Z^2 pq},
    \]
    где \( N \)~-- генеральная совокупность, \( p \)~-- вероятность обнаружить
    признак, \( q = 1 - p \)~-- вероятность не обнаружить признак, \( Z \)~--
    коэффициент доверия (принимает значения 1,64, 1,96, 2,58), \( \D \)~--
    предельная ошибка репрезентативности (принимает значения 0,01, 0,05, 0,1).
    Вероятности \( p \) и \( q \) берутся из предварительного исследования,
    либо принимаются равными 0,5. Коэффициент доверия и ошибку исследователь
    выбирает сам, в зависимости от желаемой точности исследования.

    \bigskip

    Для примера посчитаем необходимую выборку из жителей Волгограда старше
    14~лет (882449~человека) для наиболее точного определения какого-либо
    признака. Половозрастная структура генеральной выборки дана в
    таблице~\ref{tab_gen_sov}а).

    \begin{table}[h]
      \center
      \caption{Генеральная совокупность}
      \label{tab_gen_sov}

      \parbox{.2\textwidth}{а) в человеках} \hspace{12.5em}
      \parbox{.2\textwidth}{б) в процентах} \\[.1em]
      \begin{tabular}{|C{.2}|*{2}{C{.1}|}} \hline
        \multirow{2}{*}{Возраст} & \multicolumn{2}{c|}{Пол} \\ \cline{2-3}
                                 &      М & Ж \\ \hline
                          14--34 & 167442 & 169728 \\
                          35--54 & 138348 & 160596 \\
                             55+ &  90244 & 156091 \\ \hline
                          Итого: & 396034 & 486415 \\ \hline
      \end{tabular}
      \hspace{1em}
      \begin{tabular}{|C{.2}|*{2}{C{.1}|}} \hline
        \multirow{2}{*}{Возраст} & \multicolumn{2}{c|}{Пол} \\ \cline{2-3}
                                 &       М & Ж       \\ \hline
                          14--34 & 18,97\% & 19,23\% \\
                          35--54 & 15,68\% & 18,20\% \\
                             55+ & 10,23\% & 17,69\% \\ \hline
                          Итого: & 44,88\% & 55,12\% \\ \hline
      \end{tabular}
    \end{table}

    Примем коэффициент доверия равным \( Z = 1,\!96 \), а ошибку
    \( \D = 0,\!05 \). Из условия \( N = 882449 \). Вероятности \( p \) и
    \( q \) примем равными \( 0,\!5 \).

    Тогда объем выборки:
    \[
      n = \frac{1,\!96^2 \cdot 882449 \cdot 0,\!5 \cdot 0,\!5}
        {0,\!05^2 \cdot 882449 + 1,\!96^2 \cdot 0,\!5 \cdot 0,\!5} \approx 384.
    \]

    Зная объем выборки определим из таблицы~\ref{tab_gen_sov}б) структуру
    респондентов по половозрастным критериям. Полученная выборка представлена в
    таблице~\ref{tab_ch_sov}.

    \begin{table}[h]
      \center
      \caption{Выборочная совокупность в человеках}
      \label{tab_ch_sov}
      \begin{tabular}{|C{.2}|*{2}{C{.1}|}} \hline
        \multirow{2}{*}{Возраст} & \multicolumn{2}{c|}{Пол} \\ \cline{2-3}
                                 &   М &   Ж \\ \hline
                          14--34 &  73 &  74 \\
                          35--54 &  60 &  70 \\
                             55+ &  39 &  68 \\ \hline
                          Итого: & 172 & 212 \\ \hline
      \end{tabular}
    \end{table}

    Таким образом, для проведения исследования необходимо изучить 384~человека
    с распределением по полу и возрасту, указанном в таблице~\ref{tab_ch_sov}.

  \section{Методы формирования выборки}

    В основе большинства случаев формирования выборки лежит случайность, которая,
    в данном случае, означает не произвольность выбора, а равенство вероятностей
    быть отобранными для всех объектов.

    Методы отбора:
    \begin{enumerate}
      \item случайные:
        \begin{itemize}
          \item вероятностный~-- способ формирования выборки, при котором каждая
            единица генеральной совокупности имеет равную вероятность быть
            включенной в выборку;
          \item систематический~-- упрощенный вариант вероятностного, в котором
            отбор единиц осуществляется через один и тот же интервал~-- шаг~-- в
            списке (номер первого объекта определяется случайным образом);
          \item районированный (стратифицированный, расслоенный)~-- прежде чем
            осуществить выборку объект разделяют на районы (страты, слои), что
            предполагает предварительный анализ элементов генеральной
            совокупности для выделения подмножеств по сходным признакам;
          \item гнездовой~-- группировка в однородные группы~-- гнезда.
        \end{itemize}
      \item неслучайные (направленные)~-- отборы, в которых не соблюдаются
        законы случайности:
        \begin{itemize}
          \item квотный~-- отбор единиц наблюдения, обладающих определенными
            комбинациями интересующих исследователя признаков;
          \item метод основного массива~-- отбор объектов репрезентации
            осуществляется не из всего множества объектов, входящих в
            генеральную совокупность, а из некоторого, достаточно большого,
            подмножества;
          \item стихийный отбор~-- абсолютно случайный отбор людей:
            \begin{itemize}
              \item прессовый отбор;
              \item отбор себе подобных (метод снежного кома);
              \item отбор первого встречного.
            \end{itemize}
        \end{itemize}
    \end{enumerate}

  \section{Измерительные процедуры в социологических исследованиях. Виды шкал}

  \section{Методы обработки данных}
