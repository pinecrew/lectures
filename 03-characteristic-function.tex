\section{Характеристическая функция}
\subsection{Определение и основные свойства характеристической функции}

Характеристической функцией называется Фурье-образ плотности вероятности:
\[
    \theta(u) = \average{e^{iux}} = \int_{-\infty}^{+\infty} w(x) e^{iux} dx.
\]
\[
    w(x) = \frac{1}{2\pi}\int_{-\infty}^{+\infty} \theta(u) e^{-iux} du.
\]

Основные свойства характеристической функции
\begin{enumerate}
    \item (внезапно!) комплексная;
    \item ограниченная: \( \abs{\theta(u)} \le \theta(0) = 1 \);
    \item \( \theta{u} = \theta^*(-u) \);
    \item если \( w(x) \) чётная, то \( \theta(u) \) вещественная;
    \item если известен Фурье-образ \( \phi(u) \) функции \( F(x) \), то
        \[
            \average{F(x)} = \int_{-\infty}^{+\infty} \phi(u)\theta(u) du.
        \]
    \item \( \theta(u) \) -- производящая функция моментов:
        \[
            \theta(u) = \average{e^{iux}} =
            \sum_{n=0}^\infty \frac{(iu)^n}{n!}\average{x^n} =
            \sum_{n=0}^\infty \frac{i^n m_n}{n!}u^n,\quad
            m_n = \frac{1}{i^n}\left.\frac{d^n\theta}{du^n}\right|_{u=0}.
        \]
        Проделав обратное преобразование, получим представление \( w(x) \) через
        моменты:
        \[
            w(x) = \frac{1}{2\pi}\int_{-\infty}^{+\infty} e^{-iux}
            \sum_{n=0}^\infty \frac{i^n m_n}{n!}u^n du =
            \frac{1}{2\pi} \sum_{n=0}^\infty \frac{m_n}{n!}
            \int_{-\infty}^{+\infty}(iu)^n e^{-iux} du.
        \]
        Значение интегралов найдём при помощи функции Дирака:
        \[
            \int_{-\infty}^{+\infty} \delta(x) e^{iux} dx =
            \left.e^{iux}\right|_{x=0} = 1 \Rightarrow
            \frac{1}{2\pi}\int_{-\infty}^{+\infty} e^{-iux} du = \delta(x).
        \]
        Возьмём производную \( n \)-го порядка по \( x \) от обеих частей:
        \[
             \frac{1}{2\pi}\int_{-\infty}^{+\infty} (-iu)^n e^{-iux} du =
             \frac{d^n}{dx^n}\delta(x).
        \]
        Теперь подставим этот интеграл в выражение для \( w(x) \) и
        окончательно
        \[
            w(x) =  \sum_{n=0}^\infty \frac{(-1)^n m_n}{n!}
                \frac{d^n}{dx^n}\delta(x).
        \]
\end{enumerate}
В случае совместного распределения нескольких величин для определения
характеристической функции производится \( n \)-мерное преобразование Фурье:
\[
    \theta_n(\textbf{u}) = \average{e^{i\textbf{u}\cdot\textbf{x}}} =
    \int_{-\infty}^{+\infty} w(\textbf{x}) e^{i \textbf{u}\cdot\textbf{x}}
    d^n \textbf{x},
\]
где \( \textbf{x} \) и \( \textbf{u} \) -- \( n \)-компонентные вектора.

\subsection{Кумулянты распределения и их связь с моментами}
Разложим в ряд логарифм характеристической функции:
\[
    \ln\theta(u) = \ln\{1 + [\theta(u) - 1]\} = \sum_{k=1}^\infty
    (-1)^{k+1}\frac{[\theta(u) - 1]^k}{k}.
\]
Но характеристическая функция также может быть представлена в виде степенного
ряда:
\[
    \theta(u) = 1 + \sum_{n=1}^\infty \frac{(iu)^n}{n!}.
\]
Подставив этот ряд в первый, получим
\[
    \ln\theta(u) = \sum_{k=1}^\infty \sum_{n=1}^\infty
    (-1)^{k+1}\frac{(iu)^n}{n! \cdot k} =
    \sum_{n=1}^\infty\frac{(iu)^n}{n!}K_n.
\]
\( K_n \) называется кумулянтом (семиинвариантом), а \( \ln\theta(u) \) --
кумулянтной функцией. Кумулянты низших порядков связаны с моментами
соотношениями:
\begin{align*}
    & K_1 = m_1, \\
    & K_2 = \mu_2 = \sigma^2, \\
    & K_3 = \mu_3, \\
    & K_4 = \mu_4 - 3\mu_2, \\
    & K_5 = \mu_5 - 10\mu_2\mu_3, \\
    & K_6 = \mu_6 - 15\mu_2\mu_4 - 10\mu_3^2 + 20\mu_2^3.
\end{align*}
Часто используют нормированные кумулянты \( \kappa_n \), связанные с обычными
соотношением \( K_n = \sigma^n\kappa_n \).

В отличие от моментов, все кумулянты (за исключением первого) инвариантны
относительно изменения начала отсчёта. С их помощью удобно приближать функцию
плотности распределения. Покажем это на примере нормального распределения.

Через моменты её можно представить в виде
\[
    w(x) = \delta(x) - m_1\delta'(x) + \frac{1}{2}m_2\delta''(x).
\]
Характеристическая функция через кумулянты записывается в виде
\[
    \theta(u) = \exp\left( iK_1u - \frac{K_2u^2}{2} \right).
\]
Сделав обратное преобразование Фурье, получим
\[
    w(x) = \frac{1}{\sqrt{2\pi K_2}}\exp\left[ -\frac{(x-K_1)^2}{2K_2} \right].
\]
И без производных от функции Дирака.

Так как для гауссианы все кумулянты порядка выше второго равны нулю, то понятно,
что они показывают отклонение плотности вероятности от гауссовой кривой. Так,
\( \kappa_3 \) показывает отклонение кривой от симметричной формы и называется
коэффициентом асимметрии. Если \( \kappa_3 > 0  \), то спад слева круче, чем
справа. И наоборот. \( \kappa_4 \) характеризует за остроту пика кривой: если он
положителен, то пик острее, чем у гауссовой кривой, а если нет -- то нет.

И что самое характерное, \( \kappa_4 - \kappa_3^2 + 2 \ge 0 \). Предлагаю
читателю доказать это самостоятельно.

\subsection{Центральная предельная теорема}

Под термином ``центральная предельная теорема'' понимается целое семейство
теорем различной степени общности, в которых фигурируют различные ограничения на
суммируемые величины. Для самого общего случая теорему доказал Ляпунов. ЦПТ
может быть сформулирована следующим образом:
\begin{quotation}
    Сумма большого числа статистически независимых слагаемых, каждая из которых
    имеет произвольное распределение вероятностей с конечным средним и
    дисперсией, распределена по нормальному закону.
\end{quotation}

Рассмотрим сумму независимых случайных величин \( Y = \sum_{j=1}^{n} X_j \).
Пусть характеристическая функция величины \( X_j \) имеет вид
\[
    \theta_j(u)=\exp\left[\sum_{m=1}^\infty\frac{(iu)^m}{m!}K_{j,m}\right].
\]
Тогда характеристическая функция их суммы
\[
    \theta(u)=\exp\left[\sum_{j=1}^n\sum_{m=1}^\infty
        \frac{(iu)^m}{m!}K_{j,m}\right] =
        \exp\left[\sum_{m=1}^\infty\frac{(iu)^m}{m!}K_m\right],\quad
    K_m = \sum_{j=1}^n K_{j,m}.
\]
Учтём теперь, что
\[
    K_2 = \sigma^2 = \sum_{j=1}^n \sigma_j^2
\]
и получим выражения для нормированных кумулянтов
\[
    \kappa_m = \frac{\sum\limits_{j=1}^n \kappa_{j,m} \sigma_j^m }
        {\left(\sum\limits_{j=1}^n \sigma_j^2\right)^\frac{m}{2}} =
        \frac{O(n)}{O(n^\frac{m}{2})} \xrightarrow[n\to\infty]{m>2} 0.
\]
Таким образом, при больших \( n \) от нуля отличны только \( \kappa_1 \) и
\( \kappa_2 \), что соответствует нормальному распределению.

