\section{Определение и основные свойства характеристической функции}

Характеристической функцией называется Фурье-образ плотности вероятности:
\[
    \theta(u) = \average{e^{iux}} = \int_{-\infty}^{+\infty} w(x) e^{iux} dx.
\]
\[
    w(x) = \frac{1}{2\pi}\int_{-\infty}^{+\infty} \theta(u) e^{-iux} du.
\]

Основные свойства характеристической функции
\begin{enumerate}
    \item (внезапно!) комплексная;
    \item ограниченная: \( \abs{\theta(u)} \le \theta(0) = 1 \);
    \item \( \theta{u} = \theta^*(-u) \);
    \item если \( w(x) \) чётная, то \( \theta(u) \) вещественная;
    \item если известен Фурье-образ \( \phi(u) \) функции \( F(x) \), то
        \[
            \average{F(x)} = \int_{-\infty}^{+\infty} \phi(u)\theta(u) du.
        \]
    \item \( \theta(u) \) -- производящая функция моментов:
        \[
            \theta(u) = \average{e^{iux}} =
            \sum_{n=0}^\infty \frac{(iu)^n}{n!}\average{x^n} =
            \sum_{n=0}^\infty \frac{i^n m_n}{n!}u^n,\quad
            m_n = \frac{1}{i^n}\left.\frac{d^n\theta}{du^n}\right|_{u=0}.
        \]
        Проделав обратное преобразование, получим представление \( w(x) \) через
        моменты:
        \[
            w(x) = \frac{1}{2\pi}\int_{-\infty}^{+\infty} e^{-iux}
            \sum_{n=0}^\infty \frac{i^n m_n}{n!}u^n du =
            \frac{1}{2\pi} \sum_{n=0}^\infty \frac{m_n}{n!}
            \int_{-\infty}^{+\infty}(iu)^n e^{-iux} du.
        \]
        Значение интегралов найдём при помощи функции Дирака:
        \[
            \int_{-\infty}^{+\infty} \delta(x) e^{iux} dx =
            \left.e^{iux}\right|_{x=0} = 1 \Rightarrow
            \frac{1}{2\pi}\int_{-\infty}^{+\infty} e^{-iux} du = \delta(x).
        \]
        Возьмём производную \( n \)-го порядка по \( x \) от обеих частей:
        \[
             \frac{1}{2\pi}\int_{-\infty}^{+\infty} (-iu)^n e^{-iux} du =
             \frac{d^n}{dx^n}\delta(x).
        \]
        Теперь подставим этот интеграл в выражение для \( w(x) \) и
        окончательно
        \[
            w(x) =  \sum_{n=0}^\infty \frac{(-1)^n m_n}{n!}
                \frac{d^n}{dx^n}\delta(x).
        \]
\end{enumerate}
В случае совместного распределения нескольких величин для определения
характеристической функции производится \( n \)-мерное преобразование Фурье:
\[
    \theta_n(\textbf{u}) = \average{e^{i\textbf{u}\cdot\textbf{x}}} =
    \int_{-\infty}^{+\infty} w(\textbf{x}) e^{i \textbf{u}\cdot\textbf{x}}
    d^n \textbf{x},
\]
где \( \textbf{x} \) и \( \textbf{u} \) -- \( n \)-компонентные вектора.
