\chapter{Трехфазная цепь}
	Трехфазная цепь рассмотрена в \href{http://google.com}{работе Ф321}.
	
\section{Вращающееся магнитное поле. Асинхронный двигатель}
	Пусть имеется вращающийся подковообразный магнит, а в нем помещена
    проводящая рамка со сторонами \( a \) и \( l \). Если магнит вращается, то и
    поле \( \vec{B} \) тоже будет вращаться.Оказывается, что, вслед за полем,
    будет вращаться и рамка, то есть силы Ампера будут поворачивать ее. Покажем
    это.
	
	Магнитный поток через рамку:
	\[
        \Phi = BS\cos\alpha,
    \]
	где \( \alpha \) -- угол между нормалью плоскости рамки \( \vec{n} \) и
    вектором \( \vec{B} \). При \( \alpha = \omega t \):
	\[
        \Phi = BS\cos\omega t.
    \]
	В рамке наводится ЭДС:
	\[
        \EDS = -\simpder{\Phi}{t} = BS\omega\sin\omega t,
    \]
	и идет ток \( i \):
	\[
        i = \frac{\EDS}{R} = \frac{BS\omega}{R}\sin\omega t.
    \]
	Следовательно, на продольные стороны рамки действует пара сил Ампера:
	\[
        \vec{F}_{A} = i(\vec{l}\times\vec{B}).
    \]
	Их максимум, достигаемый при \( \alpha = \pi/2 \):
	\[
        F_{A}^\mathrm{max} = \frac{BS\omega}{R}lB.
    \]
	Следовательно, эта пара сил создает момент, максимум которого:
	\[
        M_{max} = F_{A}^\mathrm{max}\cdot a = \frac{B^2Sla\omega}{R} =
        \frac{B^2S^2\omega}{R}.
    \]
	Этот момент и вращает рамку вслед за полем \( \vec{B} \).
	
	Но оказывается, что для получения вращающегося поля \( \vec{B} \)
    необязательно вращать магнит. Его можно получить в трех парах катушек, на
    которые подается трехфазное напряжение.
	
	Итак, пусть имеются три пары катушек (полюсов), уложенных в пазы
    металлического статора. Если к катушкам приложено трехфазное напряжение
	\[
        \left\{
        \begin{array}{l}
            u_1 = U\sin\omega t, \\
            u_2 = U\sin\left(\omega t - \frac{2}{3}\pi\right), \\
            u_3 = U\sin\left(\omega t + \frac{2}{3}\pi\right),
	    \end{array}
        \right.
    \]
	то между полюсами будут магнитные поля:
	\[
        \left\{
        \begin{array}{l}
            \vec{B}_1 = \vec{B}_{01}\sin\omega t; \\
            \vec{B}_2 = \vec{B}_{02}\sin\left(\omega t - \frac{2}{3}\pi\right); \\
            \vec{B}_3 = \vec{B}_{03}\sin\left(\omega t + \frac{2}{3}\pi\right),
        \end{array}
        \right.
    \]
	причем \( |\vec{B}_{01}| = |\vec{B}_{02}| = |\vec{B}_{03}| = B_0 \). Эти три
    вектора повернуты относительно друг друга в пространстве на \( 120^\circ \).
	
	Покажем, что при сложении трех этих векторов получается вращающееся поле
    \( \vec{B} \) постоянной величины. Для этого сложим по отдельности \( x \)-
    и \( y \)-компоненты векторов \( \vec{B}_{i} \).
	
	Для начала, сделаем это во временной форме:
	
	\( x \)-компонента:
    \[
        B_{x} = B_{1x} + B_{2x} + B_{3x} = 0 -
        \frac{\sqrt{3}}{2}B_0\sin\left(\omega t - \frac{2}{3}\pi\right) +
        \frac{\sqrt{3}}{2}B_0\sin\left(\omega t + \frac{2}{3}\pi\right) =
        \frac{3}{2}B_0\cos\omega t.
	\]

	\( y \)-компонента:
	\[
        B_{y} = B_{1y} + B_{2y} + B_{3y} = -B_0\sin\omega t +
        \frac{1}{2}B_0\sin\left(\omega t - \frac{2}{3}\pi\right) +
        \frac{1}{2}B_0\sin\left(\omega t + \frac{2}{3}\pi\right) =
        -\frac{3}{2}B_0\sin\omega t.
	\]
	
	В комплексной форме:
	
	\( x \)-компонента:
	\begin{align*}
	    & \dot{B}_{x} = \dot{B}_{1x} + \dot{B}_{2x} + \dot{B}_{3x} =
        \dot{B}_0\left( 0 - \frac{\sqrt{3}}{2} e^{-\imone\frac{2}{3}\pi} +
        \frac{\sqrt{3}}{2} e^{-\imone\frac{2}{3}\pi} \right)e^{\imone\omega t} =
        \\
        & = \dot{B}_0\sqrt{3}e^{\imone\omega t}\cdot\left(\imone
        \frac{e^{\imone\frac{2}{3}\pi} - e^{-\imone\frac{2}{3}\pi}}{2\imone}
        \right) = 
        \dot{B}_0\sqrt{3}\imone\sin\left(\frac{2}{3}\pi\right)e^{\imone\omega t}
        = \dot{B}_0\imone\frac{3}{2}e^{\imone\omega t} = 
        \frac{3}{2}\dot{B}_0e^{\imone\left(\omega t + \frac{\pi}{2}\right)}.
	\end{align*}
	
    Переход во временную форму:
	\[
        B_{x}(t) = \mathbf{Im\ }(\dot{B}_{x}) =
        \frac{3}{2}B_0\sin\left(\omega t + \frac{\pi}{2}\right) =
        \frac{3}{2}B_0\cos\omega t. 
    \]
	
	\( y \)-компонента:
	\begin{align*}
        & \dot{B}_{y} = \dot{B}_{1y} + \dot{B}_{2y} + \dot{B}_{3y} =
        \left(-\dot{B}_0 + \frac{1}{2}\dot{B}_0e^{-\imone\frac{2}{3}\pi} +
        \frac{1}{2}\dot{B}_0e^{\imone\frac{2}{3}\pi}\right)e^{\imone\omega t} =
        \\
        & = -\dot{B}_0\left(1 - \frac{e^{\imone\frac{2}{3}\pi} +
        e^{-\imone\frac{2}{3}\pi}}{2}\right)\cdot e^{\imone\omega t} =
        -\dot{B}_0\left(1-\cos\frac{2}{3}\pi\right)e^{\imone\omega t} =
        -\frac{3}{2}\dot{B}_0e^{\imone\omega t}.
	\end{align*}
	Переход во временную форму:
	\[
        B_{y}(t) = \mathbf{Im\ }(\dot{B}_{y}) = -\frac{3}{2}B_0\sin\omega t.
    \]
	
	Таким образом, компоненты результирующего поля \( \vec{B} \) в пространстве
    между полюсами:
	\[
        \left\{
        \begin{array}{l}
            B_{x}(t) = \frac{3}{2}B_0\cos\omega t, \\
            B_{y}(t) = -\frac{3}{2}B_0\sin\omega t.
        \end{array}
        \right.
    \]
	
	Это поле постоянно по величине:
	\[
        |\vec{B}(t)| = \sqrt{B_{x}^2 + B_{y}^2} = \frac{3}{2}B_0 = \const.
    \]
	
	Угол поворота \( \alpha \) вектора \( \vec{B} \) относительно оси \( x \):
	\[
        \tg\alpha = \frac{B_{y}}{B_{x}} = -\tg\omega t \Rightarrow
        \alpha = -\omega t.
    \]
	
	Если теперь в пространстве между полюсами поместить проводящий цилиндр
    (ротор), то он будет вращаться вслед за полем \( \vec{B} \).
	
	На практике, ротор исполняется в виде медного “беличьего колеса”, уложенного
    в пазы железного наборного цилиндра.
    
    Такой двигатель называется “\textbf{асинхронным}” потому, что если ротор
    нагрузить, то он будет оставать от поля \( \vec{B} \), и чем больше
    нагрузка, тем сильнее отставание и тем сильнее вращающий момент.
