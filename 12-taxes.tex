\section{Фискальная политика государства} % налоговая

Фискальная политика (\emph{fiscal policy}) государства предполагает
регулирование правительством совокупного спроса путем стимулирования или
сокращения государственного потребления и инвестиций.

Фискальная политика бывает двух видов:
\begin{enumerate}
    \item недискреционная (политика встроенных стабилизаторов);
    \item дискреционная:
    \begin{itemize}
        \item фискальная экспансия (стимулирующая политика);
        \item фискальная рестрикция (ограниченная политика).
    \end{itemize}
\end{enumerate}

Фискальная экспансия предполагает рост государственных расходов и снижение
налогов.
Фискальная рестрикция -- снижение государственных расходов и увеличение
налогов.

Государственный бюджет -- основной финансовый план государства на год,
объединяющий главные доходы и расходы государства и имеющий силу закона.

Стадии бюджетного процесса в Российской Федерации:
\begin{enumerate}
    \item составление проекта правительством;
    \item рассмотрение проекта Государственной Думой, Советом Федерации и
    Счетной палатой;
    \item утверждение бюджета, принятие закона о бюджете Федеральным Собранием,
    подписание его президентом;
    \item исполнение бюджета органами исполнительной власти с 1 января по 31
    декабря;
    \item составление отчета об исполнении бюджета и его утверждении в течении
    5 месяцев.
\end{enumerate}

Структура консолидированного бюджета РФ:
% списать с картинки

В бюджетную систему РФ включаются также государственные внебюджетные фонды,
которые формируются из денежных средств, имеющих целевое назначение. Наполнение
этих фондов производится посредством сбора Единого Социального налога.
Например: пенсионный фонд России, фонд социального страхования, фонд
обязательного медицинского страхования и другие.

Доходы федерального бюджета РФ:
\begin{itemize}
    \item налоговые:
    \begin{itemize}
        \item НДС,
        \item налог на прибыль,
        \item акцизы,
        \item НДПИ,
        \item ЕСН,
        \item НДФЛ,
        \item прочие налоговые сборы;
    \end{itemize}
    \item неналоговые:
    \begin{itemize}
        \item доходы от имущества, находящегося в государственной и
        муниципальной собственности;
        \item доходы от внешнеэкономической деятельности;
        \item доходы от оказания платных услуг государством и компенсации
        затрат государства;
        \item доходы от продажи материальных и нематериальных ценностей
        государством;
        \item прочие неналоговые доходы (штрафы, пени, ...).
    \end{itemize}
\end{itemize}

Функции налогов:
\begin{enumerate}
    \item фискальная функция;
    \item распределительная (социальная) функция;
    \item регулирующая (экономическая) функция:
    \begin{itemize}
        \item стимулирующая,
        \item дестимулирующая,
        \item воспроизводственная.
    \end{itemize}
\end{enumerate}