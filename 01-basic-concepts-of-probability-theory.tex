\section{Основные понятия теории вероятностей}
\subsection{Предмет статистической радиофизики}

Статистическая радиофизика изучает случайные явления в радиофизике.
Статистическое описание является наиболее подходящим для многих электромагнитных
явлений. Статистическая радиофизика изучает случайные процессы на всех этапах
радиосвязи.

Статистическая радиофизика использует методы радиофизики, статистической физики
и теории вероятностей и работает с информационными характеристиками (дисперсией,
корелляцией, \ldots).

Под сообщением понимаются любые данные или сведения, подлежащие передаче. По
своей природе сообщения делятся на механические, тепловые, электромагнитные и
т.д. Сообщения неэлектрической природы часто преобразуют в электрический сигнал
при помощи преобразователей. При необходимости, электрический сигнал для
передачи преобразуют в радиосигнал.

При преобразовании, передаче, распространении и приёме сигнал подвергается
искажениям. Они обуславливаются:
\begin{enumerate}
    \item внешними и внутренними помехами;
    \item распространением сигнала через турбулентную среду;
    \item техническим несовершенством устройств.
\end{enumerate}
Любые нежелательные возмущения, накладывающиеся на сигнал, называются шумом.

\subsection{Вероятность}

Основное понятие радиофизики -- случайная величина. Случайная величина -- это
форма представления заранее не предсказуемых результатов опыта. Для описания
случайных величин в математике вводится понятие вероятности.

Вероятность события \( A \) есть отношение числа опытов, в которых происходило
это событие, к общему числу опытов при стремлении числа опытов к бесконечности:
\[
    P(A) = \lim_{N\to\infty}\frac{N_A}{N}.
\]

Если событие не может произойти ни в одном опыте, то вероятность такого события
равна нулю, а событие называется невозможным.

Если же событие происходит в каждом опыте, то его вероятность равна единице, а
само событие называется достоверным.

Если два события никогда не происходят вместе, то такие события называются
несовместимыми.

Событие, представляющее собой множество вероятных исходов составляет группу
событий. Группа называется полной, если в результате опыта произойдет одно из
событий этой группы.

Два события, образующие полную группу, называются противоположными, если они
несовместимы.

Если всякий раз, когда происходит событие \( A \), происходит событие \( B \),
то говорят, что \( A \) влечет за собой \( B \): \( A \subset B \). Если
\( A \subset B \) и \( B \subset A \), то \( A = B \) и такие события называют
равносильными или эквивалентными.

Событие \( A \) называется статистически зависимым от события \( B \), если
\( P(A) \) зависит от того, осуществилось ли событие \( B \) или нет. В случае
статистической зависимости можно ввести условную вероятность \( P(A|B) \)
события \( A \) при осуществлении события \( B \). Безусловные вероятности
называются априорными, а условные -- апостериорными.

Произведением событий называется такое событие, которое происходит, когда
одновременно происходят все эти события:
\[
    P(AB) = P(A) \cdot P(B|A) = P(B) \cdot P(A|B).
\]

Суммой событий называется событие, которое происходит тогда и только тогда,
когда происходит хотя бы одно из этих событий:
\[
    P(A+B) = P(A) + P(B) - P(AB).
\]

Если события несовместимы и образуют полную группу, то
\[
    P(\sum A_i) = \sum P(A_i) = 1.
\]
Это условие называется условием нормировки.

Пусть несовместимые события \( B \) и \( C \) входят в полную группу, а \( A \)
может произойти только при наступлении этих двух событий, тогда
\[
    P(A) = P(AB) + P(AC) = P(B) \cdot P(A|B) + P(C) \cdot P(A|C).
\]
Эта формула называется формулой полной вероятности.
\[
    P(B|A) = \frac{P(AB)}{P(A)} = \frac{P(B) \cdot P(A|B)}
    {P(B) \cdot P(A|B) + P(C) \cdot P(A|C)} \text{ -- формула Байеса.}
\]

В технике используются системы, состоящие из нескольких элементов. Надежностью
системы называют вероятность того, что система будет работать без отказа в
течение установленного промежутка времени. Пусть система состоит из двух
элементов, причем надежность первого \( P_1 \), а второго \( P_2 \). Тогда
надежность при последовательном соединении равна \( P_1P_2 \),  а при
параллельном -- \( 1 - (1 - P_1)(1 - P_2) = P_1 + P_2 - P_1P_2 \).

\subsection{Функция распределения случайной величины}

Соотношение, устанавливающее связь между возможным значением случайной величины
и её вероятностью, называется функцией распределения.

Пусть случайная величина \( X \) каким-либо образом распределена на
\( \mathbb{R} \). Тогда \( F(x) = P(X \le x) \) называется (интегральной)
функцией распределения величины \( X \). Эта функция обладает следующими
свойствами:
\begin{itemize}
    \item для дискретной величины \( F(x) = \sum_{X_i \le x}P(x) \);
    \item для непрерывной величины \( F(x) = \int_{-\infty}^x w(\xi)d\xi \);
    \item \( F(-\infty) = 0 \);
    \item \( F(+\infty) = 1 \);
    \item \( F(x) \) -- монотонно неубывающая функция;
    \item \( P(x_1 < X < x_2) = F(x_2) - F(x_1) \);
    \item для непрерывно распределенной величины является гладкой всюду
        непрерывной функцией.
\end{itemize}
В случае непрерывной величины функция
\[
    w(x) = \der{F(x)}{x}
\]
называется плотностью распределения. Она обладает рядом свойств:
\begin{itemize}
    \item \( \int_{-\infty}^{+\infty} w(\xi) d\xi = 1 \);
    \item \( \forall x\ w(x) \le 0 \);
    \item для дискретной величины
        \( w(x) = \sum_i P(X_i) \cdot \delta(x - X_i) \).
\end{itemize}

Для любой функции случайной величины можно ввести понятие среднего значения
\[
    \average{f(x)} = \int_{-\infty}^{+\infty}f(x)w(x)dx,
\]
переходящее для дискретной величины в
\[
    \average{f(x)} = \sum_i P(X_i)f(X_i).
\]

\subsection{Моменты случайной величины}

Момент \( m_n \) \( n \)-го порядка определяется выражением
\[
    m_n(X) = \average{X^n} = \int_{-\infty}^{+\infty} x^n w(x) dx.
\]
\( m_1 \) называется средним значением или математическим ожиданием.

Центральный момент \( \mu_n \) \( n \)-го порядка определяется выражением
\[
    \mu_n(X) = \average{(X - \average{X})^n}.
\]
\( \mu_2 = \sigma^2 \) называется дисперсией, а \( \sigma \) --
среднеквадратичным отклонением.
