\section{Основные понятия теории вероятностей}
\subsection{Предмет статистической радиофизики}

Статистическая радиофизика изучает случайные явления в радиофизике.
Статистическое описание является наиболее подходящим для многих электромагнитных
явлений. Статистическая радиофизика изучает случайные процессы на всех этапах
радиосвязи.

Статистическая радиофизика использует методы радиофизики, статистической физики
и теории вероятностей и работает с информационными характеристиками (дисперсией,
корелляцией, \ldots).

Под сообщением понимаются любые данные или сведения, подлежащие передаче. По
своей природе сообщения делятся на механические, тепловые, электромагнитные и
т.д. Сообщения неэлектрической природы часто преобразуют в электрический сигнал
при помощи преобразователей. При необходимости, электрический сигнал для
передачи преобразуют в радиосигнал.

При преобразовании, передаче, распространении и приёме сигнал подвергается
искажениям. Они обуславливаются:
\begin{enumerate}
    \item внешними и внутренними помехами;
    \item распространением сигнала через турбулентную среду;
    \item техническим несовершенством устройств.
\end{enumerate}
Любые нежелательные возмущения, накладывающиеся на сигнал, называются шумом.

