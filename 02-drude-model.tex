\lecture{Теория электронной проводимости Друде-Лоренца}
\section{Основные положения}

Суть состоит в том, что электроны проводимости в твёрдом теле ведут себя как
газ. Эта теория довольно проста -- в ней нет заморочек с распределением
электронов по скоростям и прочей статистики. Просто решается уравнение движения
для одного электрона, а затем утверждается, что они все ведут себя одинаково.

Теория исходит из двух предположений:
\begin{enumerate}
    \item электроны в твердом теле являются свободными;
    \item со стороны ионов кристаллической решетки на электроны проводимости
    действует сила сопротивления, прямо пропорциональная скорости электронов.
\end{enumerate}

Теперь, вооружившись этой информацией, попробуем количественно описать три
явления: закон Ома, эффект Холла и циклотронный резонанс.

\section{Закон Ома}
    По определению, \( \vec{j} = nq\vec{v} \). Осталось определить скорость.
    Записываем второй закон Ньютона для электрона в однородном электрическом
    поле:
    \[
        \dot{\vec{p}} = q\vec{E} - \frac{\vec{p}}{\tau}.
    \]
    Последнее слагаемое -- это та самая сила сопротивления со стороны ионов
    решетки. Тогда через время порядка \( \tau \) импульс электрона примет
    установившееся значение
    \[
        \vec{p} = q\tau\vec{E},\quad \vec{v} = \frac{q\tau}{m}\vec{E}.
    \]
    Подставляем обратно получаем
    \[
        \vec{j} = \frac{nq^2\tau}{m}\vec{E} = \lambda\vec{E}.
    \]
    Это выражение называется законом Ома в дифференциальной форме.

\section{Эффект Холла}
    Для конкретики, пусть электрическое поле приложено вдоль оси \( y \), а
    магнитное -- вдоль оси \( z \). Тогда в проекциях
    \[
        \left\{
            \begin{array}{l}
                \dot{p}_x = \omega_c p_y - \frac{p_x}{\tau}, \\
                \dot{p}_y = qE - \omega_c p_x - \frac{p_y}{\tau}, \\
                \dot{p}_z = -\frac{p_z}{\tau},
            \end{array}
        \right.
        \quad
        \omega_c = \frac{qB}{m}.
    \]
    Опять же, через время порядка \( \tau \) импульс примет стационарное
    значение. Его проекции можно определить из системы алгебраических уравнений:
    \[
        \left\{
            \begin{array}{l}
                \frac{p_x}{\tau} - \omega_c p_y = 0,  \\
                \omega_c p_x + \frac{p_y}{\tau} = qE, \\
                p_z = 0,
            \end{array}
        \right.
    \]
    решая которую, получаем
    \[
        p_x = \frac{\omega_c\tau}{1+\omega_c^2\tau^2}q\tau E,\quad
        p_y = \frac{1}{1+\omega_c^2\tau^2}q\tau E.
    \]
    Следовательно, ток направлен под углом к электрическому полю и имеет
    проекции
    \[
        j_x = \frac{\omega_c\tau}{1+\omega_c^2\tau^2}\frac{nq^2\tau}{m} E,\quad
        j_y = \frac{1}{1+\omega_c^2\tau^2}\frac{nq^2\tau}{m} E.
    \]
    Так как знак носителя заряда определяет знак циклотронной частоты, то
    направление тока \( j_x \) однозначно определяет знак носителя заряда.

\section{Циклотронный резонанс}
    Рассмотрим падение электромагнитной волны на образец. Конфигурация та же, но
    теперь электрическое поле переменное: \( E = E_0\cos\omega t \). Записываем
    уравнение в проекциях:
    \[
        \left\{
            \begin{array}{l}
                \dot{p}_x = \omega_c p_y - \frac{p_x}{\tau}, \\
                \dot{p}_y = qE\cos\omega t - \omega_c p_x - \frac{p_y}{\tau}, \\
                \dot{p}_z = -\frac{p_z}{\tau},
            \end{array}
        \right.
    \]
    Здесь в установившемся режиме будут наблюдаться вынужденные колебания,
    поэтому для простоты решения введём комплексные амплитуды:
    \[
        \vec{p} = \Re\left(\vec{\hat{p}}e^{i\omega t}\right).
    \]
    Тогда систему уравнений можно переписать в виде
    \[
        \left\{
            \begin{array}{l}
                i\omega\hat{p}_x = \omega_c\hat{p}_y - \frac{\hat{p}_x}{\tau},\\
                i\omega\hat{p}_y = qE - \omega_c\hat{p}_x -
                    \frac{\hat{p}_y}{\tau}.
            \end{array}
        \right.
    \]
    Перепишем эту линейную систему в виде
    \[
         \left\{
            \begin{array}{l}
                (i\omega + \frac{1}{\tau})\hat{p}_x - \omega_c\hat{p}_y = 0,\\
                \omega_c\hat{p}_x + (i\omega + \frac{1}{\tau})\hat{p}_y = qE_0.
            \end{array}
        \right.
    \]
    Её решение имеет вид
    \[
        \hat{p}_x = \frac{\omega_c}{\omega_c^2 + (i\omega + \frac{1}{\tau})^2}
            qE_0, \quad
        \hat{p}_y = \frac{i\omega + \frac{1}{\tau}}
            {\omega_c^2 + (i\omega + \frac{1}{\tau})^2} qE_0.
    \]
    При рассмотрении резонанса нас интересуют не столько токи, сколько
    поглощаемая мощность в единице объёма образца:
    \[
        P = \average{\vec{j}\cdot\vec{E}} =
        \frac{nq}{m}\average{p_xE_0\cos\omega t}.
    \]
    Зная амплитуду \( \hat{p}_x \) нетрудно определить \( p_x \):
    \[
        p_x = \Re (\hat{p}_xe^{i\omega t}) = \Re\hat{p}_x \cos\omega t -
            \Im\hat{p}_x \sin\omega t.
    \]
    Принимая во внимание, что
    \[
        \average{\cos^2\omega t} = \frac{1}{2}, \quad
        \average{\cos\omega t \sin\omega t} = 0,
    \]
    получаем
    \[
        P = \frac{nq^2E^2}{2m}
        \Re\frac{\omega_c}{\omega_c^2 + (i\omega + \frac{1}{\tau})^2} =
        \frac{n\omega_c\tau^2q^2E^2}{2m}
        \frac{1 + (\omega_c^2 - \omega^2)\tau^2}
        {\left[1 + (\omega^2 - \omega_c^2)\tau^2\right]^2 +
        4\omega_c^2\tau^2}.
    \]
