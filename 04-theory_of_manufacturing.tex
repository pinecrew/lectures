\section{Теория производства и затрат}
Индивидуальное воспроизводство -- это непрерывно повторяющийся процесс
производственного соединения факторов производства с целью создания товаров
и получения доходов в рамках экономически обособленной деятельности предприятия.

\subsection{Издержки}
Издержки -- это затраты ресурсов принявшие на рынке денежную форму. Изучение
издержек очень важно для деятельности фирмы.

Издержки бывают:
\begin{itemize}
    \item явные (бухгалтерские, внешние) -- это издержки на покупку чужих
    факторов производства в фактических ценах их приобретения. К ним относят
    материальные затраты, затраты на оплату труда, отчисления на социальные
    нужды, амортизация и пр.;
    \item неявные (экономические, альтернативные, вменённые) -- это стоимость
    благ, которые можно было бы получить при наиболее выгодном использовании
    ресурсов, находящихся в собственности предприятия. 
\end{itemize}

Общая выручка -- это весь доход, который получен от реализации совокупного
объёма продукции.
\[
    TR = \sum_i P_i \cdot Q_i.
\]

Средняя выручка (средний доход) -- это доход от продажи единицы продукции.
\[
    AR = \frac{TR}{Q}.
\]

Предельная выручка -- это приращение общей выручки при увеличении количества
выпускаемой продукции на единицу.
\[
    MR = \lim_{\Delta Q \rightarrow 0} \frac{\Delta TR}{\Delta Q}.
\]

\subsection{Издержки в кратко- и долгосрочный период}
Долгосрочный период -- это  такой промежуток времени, в течение которого фирма
может изменить все факторы производства, используемые для изготовления продукции.

Краткосрочный период -- это отрезок времени, в течение которого фирма не может
изменить общие размеры основного капитала.

\subsubsection{Издержки в краткосрочный период}
В краткосрочном периоде издержки делятся на
\begin{itemize}
    \item постоянные издержки, не зависящие от объёмов производства
        (арендная плата, налоги);
    \item переменные издержки, величина которых изменяется в зависимости от
        изменения объёмов производства (сырье, заработная плата).
    \item общие издержки, являющиеся суммой постоянных и переменных издержек;
    \item средние издержки, пересчитанные на единицу продукции:
    \begin{itemize}
        \item средние постоянные,
        \item средние переменные,
        \item средние общие.
    \end{itemize}
\end{itemize}

Закон убывающей предельной производительности: начиная с некоторого момента
времени каждая добавочная единица переменного фактора производства приносит
меньшее приращение общего объёма продукции, чем предыдущая.

Средние издержки помогают определить эффективные объёмы производства. Возможны
следующие варианты:
\begin{enumerate}
    \item если \( ATC < P \), то фирма получает максимальную прибыль на единицу
        продукции;
    \item если \( ATC = P \), то фирма получает нулевую экономическую прибыль,
        ей безразлично -- оставаться или уйти с рынка;
    \item если \( AVC > P \), имеет место нерентабельное производство -- фирме
        лучше уйти с рынка.
\end{enumerate}

Предельные издержки -- это издержки, связанные с производством дополнительной
единицы продукции:
\[
    MC = \lim_{\Delta Q \rightarrow 0} \frac{\Delta TC}{\Delta Q}.
\]

Кривая \( MC \) пересекает \( ATC \) и \( AVC \) в точках их минимального
значения. Если \( MC < ATC \), то производство эффективно.

\subsubsection{Издержки в долгосрочном периоде}

В долгосрочном периоде все издержки являются переменными. Поэтому используются
только средние издержки. В долгосрочном периоде имеет место эффект экономии
от масштаба, который заключается в том, что с ростом производства снижаются
издержки на единицу продукции.

\subsection{Прибыль}

Прибыль бывает двух видов:
\begin{itemize}
    \item бухгалтерская -- разница между общей выручкой от продажи и денежными
        затратами на приобретение ресурсов, т.е явными издержками;
    \item экономическая -- разница между выручкой и экономическими издержками.
        \[
            T = TR - TC.
        \]
\end{itemize}

\subsection{Вопросы}
\begin{enumerate}
    \item Основные характеристики и виды фирм.
    \item Формы объединения предприятий.
    \item Издержки в краткосрочном периоде.
    \item Основной и оборотный капитал, сущность амортизации.
    \item Издержки в долгосрочном периоде.
    \item Прибыль как элемент издержек.
\end{enumerate}
