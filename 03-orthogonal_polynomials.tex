\section{Ортогональные многочлены.
Элементарные свойства общих ортогональных многочленов}

\subsection{Дифференциальное уравнение для спец. функций.
Уравнения гипергеометрического типа}

Многие задачи теоретической и математической физики приводят к необходимости
решения уравнений вида (\ref{eq3.0.1}):
\begin{equation}
    u'' + \frac{\tilde{\tau}(z)}{\sigma(z)}u' +
    \frac{\tilde{\sigma}(z)}{\sigma^2(z)}u = 0,
    \label{eq3.0.1}
\end{equation}
где \( \sigma(z) \), \( \tilde{\sigma}(z) \) -- полиномы не выше второй степени,
\( \tilde{\tau}(z) \) -- полином не выше первой степени.

Уравнения такого типа возникают при решении уравнения Лапласа и Гельмгольца в
различных криволинейных системах координат методом разделения переменных,
а также при рассмотрении основных задач квантовой механики: 
\begin{enumerate}
    \item движение частицы в сферически симметричном поле;
    \item гармонический осциллятор;
    \item решение уравнений Шредингера, Дирака и Кейна-Комптона для кулоновского
    поля потенциала;
    \item движение частицы в однородных электрическом и магнитным полях.
\end{enumerate}
Кроме того, к уравнению (\ref{eq3.0.1}) приводят также многие задачи
молекулярной, атомной и ядерной физики.

Частными решениями уравнения вида (\ref{eq3.0.1}) являются следующие классы
специальных функций:
\begin{enumerate}
    \item классические ортогональные многочлены (полиномы Якоби, Лагерра и 
    Эрмита);
    \item сферические, цилиндрические и гипергеометрические функции.
\end{enumerate}
Эти функции часто называют специальными функциями математической физики.

В дальнейшем будем предполагать, что функция \( z \) и коэффициенты полиномов
\( \sigma(z) \), \( \tilde{\sigma}(z) \), \( \tilde{\tau}(z) \) могут принимать
любые вещественные или комплексные значения.

С помощью замены \( u = \phi(z)y \) приведем уравнение (\ref{eq3.0.1}) к более
простому виду путем специального выбора функций \( \phi(z) \). Имеем
(\ref{eq3.0.2}):
\begin{equation}
    y'' + \left(2\frac{\phi'}{\phi} + \frac{\tilde{\tau}}{\sigma}\right)y' +
    \left(\frac{\phi''}{\phi} + \frac{\phi'}{\phi}\frac{\tilde{\tau}}{\sigma} +
    \frac{\tilde{\sigma}}{\sigma^2}\right)y = 0.
    \label{eq3.0.2}
\end{equation}

Для того, чтобы (\ref{eq3.0.2}) не было более сложным, чем (\ref{eq3.0.1})
необходимо, чтобы коэффициент при \( y' \) имел вид \( \tau(z)/\sigma(z) \),
где \( \tau(z) \) -- полином не выше первой степени. Тогда для функции
\( \phi(z) \) получим уравнение (\ref{eq3.0.3}):
\begin{equation}
    \frac{\phi'}{\phi} = \frac{\pi(z)}{\sigma(z)},
    \label{eq3.0.3}
\end{equation}
где \( \pi(z) \) -- полином не выше первой степени:
\begin{equation}
    \pi(z) = [\tau(z) - \tilde{\tau}(z)]/2.
    \label{eq3.0.4}
\end{equation}

Так как
\[
    \frac{\phi''}{\phi} = \left(\frac{\phi'}{\phi}\right)' +
    \left(\frac{\phi'}{\phi}\right)^2 = \left(\frac{\pi}{\sigma}\right)' +
    \left(\frac{\pi}{\sigma}\right)^2,
\]
то уравнение (\ref{eq3.0.2}) принимает вид:
\begin{equation}
    y'' + \frac{\tau(z)}{\sigma(z)}y' +
    \frac{\bar{\sigma}(z)}{\sigma^2(z)}y = 0,
    \label{eq3.0.5}
\end{equation}
где
\begin{align}
    & \tau(z) = \tilde{\tau}(z) + 2\pi(z), \label{eq3.0.6} \\
    & \bar{\sigma}(z) = \tilde{\sigma}(z) + \pi^2(z) +
    \pi(z)\left[\tilde{\tau}(z) - \sigma'(z)\right] + \pi'(z)\sigma(z).
    \label{eq3.0.7}
\end{align}

Функции \( \tau(z) \) и \( \bar{\sigma}(z) \) являются полиномами соответственно
первой и второй степени. Поэтому уравнение (\ref{eq3.0.5}) является уравнением
того же типа, что и уравнение (\ref{eq3.0.1}).

Таким образом, был найден класс преобразований, не меняющих тип уравнений -- это
 преобразование уравнения (\ref{eq3.0.1}) с помощью замены \( u = \phi(z)y \),
 где \( \phi(z) \) удовлетворяет уравнению (\ref{eq3.0.3}), в котором
 \( \pi(z) \) -- произвольный полином первой степени.

Воспользуемся произволом в выборе \( \pi(z) \) для того, чтобы среди всех
возможных видов уравнения (\ref{eq3.0.5}) выбрать наиболее простой и удобный для
исследования свойств решений. Выберем коэффициенты полинома \( \pi(z) \) из
условий, чтобы входящий в (\ref{eq3.0.5}) полином \( \bar{\sigma}(z) \) делился
без остатка на \( \sigma(z) \), то есть:
\begin{equation}
    \bar{\sigma}(z) = \lambda\sigma,
    \label{eq3.0.8}
\end{equation}
где \( \lambda \) -- постоянная. В результате (\ref{eq3.0.5}) будет иметь вид
(\ref{eq3.0.9}):
\begin{equation}
    \sigma(z)y'' + \tau(z)y' + \lambda y = 0.
    \label{eq3.0.9}
\end{equation}

Уравнение (\ref{eq3.0.9}) называется уравнением гипергеометрического типа, а его
решения -- функциями гипергеометрического типа. В соответствии с этим уравнение
(\ref{eq3.0.1}) называется обобщенным уравнением гипергеометрического типа,
а если \( \sigma(z) \) является полиномом второй степени, то уравнение
(\ref{eq3.0.1}) является частным случаем уравнения Римана.

Для определения полинома \( \pi(z) \) и постоянной \( \lambda \) перепишем
условие (\ref{eq3.0.8}) в следующем виде:
\begin{equation}
    \pi^2 + (\tilde{\tau} - \sigma')\pi + \tilde{\sigma} - k\sigma = 0,
    \label{eq3.0.10}
\end{equation}
где \( k = \lambda - \pi'(z) \).

Если считать постоянную \( k \) известной, то, в результате решения квадратного
уравнения, для \( \pi(z) \) получим:
\begin{equation}
    \pi(z) = \frac{\sigma' - \tilde{\tau}}{2} \pm \sqrt{\left(\frac{\sigma' - \tilde{\tau}}{2}\right)^2 - \tilde{\sigma} + k\sigma}.
    \label{eq3.0.11}
\end{equation}

Так как \( \pi(z) \) -- полином, то подкоренное выражение должно представляться
в виде квадрата какого-либо полинома. Это возможно только тогда, когда
дискриминант полинома второй, стоящего под корнем, равен нулю.

Получаем, что для постоянной \( k \) записывается квадратное уравнение. После
определения \( k \) находим \( \pi(z) \) по формуле (\ref{eq3.0.11}), а затем
\( \phi(z) \) и \( \tau(z) \), а также \( \lambda \) с помощью (\ref{eq3.0.3}),
(\ref{eq3.0.6}) и (\ref{eq3.0.10}).

Очевидно, что сведение уравнения (\ref{eq3.0.1}) к (\ref{eq3.0.9}) может быть
осуществлено несколькими способами в зависимости от выбора значений постоянной
\( k \) и знаков в формуле (\ref{eq3.0.11}) для \( \pi(z) \).

Рассмотренные преобразования позволяют вместо изучения (\ref{eq3.0.1})
ограничиться изучением более простого уравнения (\ref{eq3.0.9}).

\subsection{Теорема о существовании и критерий ортогональности}

\begin{definition}
    Функция \( h(x) \) называется весовой функцией на конечном интервале
    \( (a, b) \), если на этом интервале она не отрицательна, интегрируема и ее
    интеграл положителен, то есть \( h(x) \ge 0 \) и выполняется следующее
    условие:
    \[
        0 < \int\limits_a^b h(x)\,dx < \infty.
    \]
    
    Если интервал \( (a, b) \) бесконечен, то, кроме этого, должны сходиться
    следующие интегралы, называемые степенными моментами функции \( h(x) \):
    \[
        h_n = \int\limits_a^b x^n h(x)\,dx,
    \]
    где \( n = 0, 1, 2, \ldots \).
\end{definition}

Пусть задана последовательность многочленов:
\begin{equation}
    P_0(x), P_1(x), P_2(x), \ldots, P_n(x),
    \label{eq3.1.3}
\end{equation}
в которой каждый многочлен \( P_k(x) \) имеет степень \( k \).
Если для двух любых многочленов из этой системы выполняется условие
\[
    \int\limits_a^b h(x)P_n(x)P_m(x)\,dx = 0,
\]
где \( m \ne n \), то многочлены (\ref{eq3.1.3}) называются ортогональными с
весовой функцией \( h(x) \) на интервале \( (a, b) \), называемым интервалом
ортогональности. Если \( a \), \( b \) -- конечны, то он называется сегментом
ортогональности \( [a, b] \).

\begin{definition}
    Система ортогональных многочленов (\ref{eq3.1.3}) называется
    ортонормированной, если каждый многочлен имеет положительный старший
    коэффициент и его норма с весом \( h(x) \) равна единице:
    \[
        \| P_n \| = \sqrt{\int\limits_a^b h(x)P_n^2(x)\,dx} = 1.
    \]
\end{definition}

Таким образом, условие ортонормированности системы (\ref{eq3.1.3}) имеет вид
(\ref{eq3.1.4}):
\begin{equation}
    \int\limits_a^b h(x)P_n(x)P_m(x)\,dx = \delta_{mn}.
    \label{eq3.1.4}
\end{equation}

\begin{definition}
    Регулярными точками весовой функции интервала \( (a, b) \) называются такие
    точки, в которых функция \( h(x) \) непрерывна и положительна, а особыми --
    остальные точки \( h(x) \), включая ее нули.
\end{definition}

Рассмотрим два вспомогательных утверждения, которые будут использованы в
дальнейшем.
\begin{lemma}
    Если в системе \( n + 1 \) многочленов
    \[
        F_0(x), F_1(x), \ldots, F_n(x), \ldots,
    \]
    каждый многочлен \( F_k(x) \) имеет степень \( k \), то всякий многочлен
    \( Q_n(x) \) степени \( n \) можно единственным образом представить в виде:
    \[
        Q_n(x) = a_0F_0(x) + a_1F_1(x) + \ldots + a_nF_n(x).
    \]
\end{lemma}

\begin{lemma}
    Если многочлен \( \varPhi_m(x) \) степени \( m \ge 1 \) не отрицателен на
    интервале \( (a, b) \), то выполняется следующее условие
    \[
        \int\limits_a^b h(x)\varPhi_m(x)\,dx > 0,
    \]
    где \( h(x) \) -- весовая функция.
\end{lemma}

Сформируем основную теорему о существовании и единственности системы
ортонормированных многочленов, соответствующую данной весовой функции \( h(x) \).

\begin{theorem}
    Для любой весовой функции \( h(x) \) существует единственная
    последовательность многочленов \( \{ P_n(x) \} \), в которой каждый
    многочлен имеет положительный старший коэффициент и удовлетворяет условию
    ортонормированности (\ref{eq3.1.4}).
    \label{th3.1.1}
\end{theorem}

Установим необходимые и достаточные условия ортогональности полиномов
 \( P_n(x) \), которые будем использовать в дальнейшем.
\begin{theorem}
    Для того, чтобы многочлен \( P_n(x) \) степени \( n \) был ортогонален с
    весом \( h(x) \), необходимо и достаточно, чтобы для всякого многочлена
    \( Q_m(x) \) степени \( m < n \) выполнялось условие:
    \[
        \int\limits_a^b h(x)P_n(x)Q_m(x)\,dx = 0.
    \]
\end{theorem}
\begin{theorem}
    Если сегмент ортогональности симметричен относительно начала координат, а
    весовая функция \( h(x) \) -- четная, то каждый ортогональный многочлен
    \( P_n(x) \) содержит только те степени \( x \), которые имеют с \( n \)
    одинаковую четность, то есть
    \[
        P_n(-x) \equiv (-1)^nP_n(x).
    \]
\end{theorem}

Согласно теореме \ref{th3.1.1} всякая весовая функция однозначно определяет
систему ортонормированных многочленов. Наиболее важное теоретическое и
прикладное значение имеют следующие системы классических ортогональных
многочленов, приведённые в таблице~\ref{table:scop}.
\begin{table}[h]
\center
\caption{Системы классических ортогональных многочленов}
\label{table:scop}
\begin{tabular}{|C{0.2}|C{0.12}|C{0.17}|C{0.25}|} \hline
    Название & Обозначение &
    Сегмент ортогональности & Весовая функция \\ \hline
    Многочлены Чебышева 1-го рода & \( \{ T_n(x) \} \) &
    \( [-1; 1] \) & \[ h(x) = \frac{1}{\sqrt{1-x^2}}. \] \\ \hline
    Многочлены Чебышева 2-го рода & \( \{ U_n(x) \} \) &
    \( [-1; 1] \) & \[ h(x) = \sqrt{1-x^2}. \] \\ \hline
    Многочлены Лежандра & \( \{ P_n(x) \} \) &
    \( [-1; 1] \) & \[ h(x) \equiv 1. \] \\ \hline
    Многочлены Якоби & \( \{ P_n(x; \alpha, \beta) \} \) &
    \( [-1; 1] \) & \[ h(x) = (1-x)^\alpha(1+x)^\beta, \] \( \alpha,\ \beta > -1. \) \\ \hline
    Многочлены Чебышева-Эрмита & \( \{ H_n(x) \} \) &
    \( (-\infty; \infty) \) & \[ h(x) = e^{-x^2}. \] \\ \hline
    Многочлены Чебышева-Лагерра & \( \{ L_n(x; \alpha) \} \) &
    \( (0; \infty) \) & \[ h(x) = x^\alpha e^{-x}, \] \( \alpha > -1. \) \\ \hline
\end{tabular}
\end{table}

Если \( \alpha = \beta \), то многочлены Якоби называются ультрасферическими и
обозначаются \( \{ P_n(x, \alpha) \} \). Наиболее важными изученными частными
случаями многочленов Якоби являются многочлены Чебышева и Лежандра, которые
можно записать в виде:
\[
    T_n(x) = P_n\left(x, -\frac{1}{2}\right),
    \ U_n(x) = P_n\left(x, \frac{1}{2}\right),\ P_n(x) = P_n(x, 0).
\]

\subsection{Алгебраические свойства ортогональных многочленов}

Рассмотрим основные алгебраические свойства ортогональных многочленов, а именно:
трехчленную рекуррентную формулу, формулу Кристоффеля-Дарбу и представление
многочленов через моменты весовой функции.
\begin{theorem}
    Для любых трех соседних ортогональных многочленов справедлива рекуррентная
    формула (\ref{eq3.2.1}):
    \begin{equation}
        \frac{\mu_n}{\mu_{n+1}}P_{n+1}(x) = (x-\alpha_n)P_n(x) -
        \frac{\mu_{n-1}}{\mu_n}P_{n-1}(x).
        \label{eq3.2.1}
    \end{equation}
\end{theorem}

Соотношение (\ref{eq3.2.1}) можно переписать в виде:
\[
    \lambda_nP_{n+1}(x) = (x-\alpha_n)P_n(x) - \lambda_{n-1}P_{n-1}(x),
\]
где \( \lambda = \mu_n/\mu_{n+1} \), \( n = 0, 1, 2, \ldots \)

\begin{lemma}
    Если сегмент \( [a, b] \) конечен, то последовательность
    \( \{ \lambda_n \} \) ограничена, причем
    \[
        0 < \lambda_n \le C = \max\{ |a|, |b| \}.
    \]
\end{lemma}

\begin{theorem}
    Для ортонормированных многочленов имеет место формула Кристоффеля-Дарбу:
    \begin{equation}
        \sum\limits_{k=0}^n P_k(x)P_k(t) =
        \lambda_n\frac{P_{n+1}(x)P_n(t) - P_n(x)P_{n+1}(t)}{x-t}.
        \label{eq3.2.5}
    \end{equation}
\end{theorem}

\begin{theorem}
    Для ортогонального многочлена при \( n \ge 1 \) имеет место представление
    через моменты весовой функции (\ref{eq3.2.6}):
    \begin{equation}
        P_n(x) = \frac{1}{\sqrt{\Delta_{n-1}\Delta_n}}
        \begin{vmatrix}
        h_0 & h_1 & h_2 & \ldots & h_n \\
        h_1 & h_2 & h_3 & \ldots & h_{n+1} \\
        \vdots & \vdots & \vdots & \ddots & \vdots \\
        h_{n-1} & h_n & h_{n+1} & \ldots & h_{2n-1} \\
        1 & x & x^2 & \ldots & x^n
        \end{vmatrix},
        \label{eq3.2.6}
    \end{equation}
    \[
        \text{где } \Delta_n =
        \begin{vmatrix}
            h_0 & h_1 & \ldots & h_n \\
            h_1 & h_2 & \ldots & h_{n+1} \\
            \vdots & \vdots & \ddots & \vdots \\
            h_n & h_{n+1} & \ldots & h_{2n}
        \end{vmatrix}.
    \]
    
    Для старшего коэффициента \( \mu_n \) ортогонального многочлена \( P_n(x) \)
    справедливы формулы:
    \[
        \mu_n = \sqrt{\frac{\Delta_{n-1}}{\Delta_n}};\ 
        \lambda_n = \frac{\mu_n}{\mu_{n+1}} =
        \frac{\sqrt{\Delta_{n-1}\Delta_{n+1}}}{\Delta_n}.
    \]
    
    Считая, что \( \Delta_{-1} = 1 \), можно считать, что эти формулы
    распространяются и на случай \( n = 0 \). В силу равенства
    \( \Delta_0 = h_0 \), вместо (\ref{eq3.2.6}) имеем
    \[
        P_0(x) = \mu_0 = \frac{1}{\sqrt{\Delta_0}}.
    \]
\end{theorem}

\subsection{Ряды Фурье по ортогональным многочленам}
Пусть \( f(x) \) определена на интервале \( (a,b) \) и её квадрат интегрируем
с весом \( h(x) \) по этому интервалу. Множество таких функций будем обозначать
\( L_2[a,b;h(x)] \) или просто \( L_2 \). Всякой функции класса \( L_2 \)
ставится в соотвествие весовая норма
\[
    \| f \| = \sqrt{\int\limits_a^b h(x)f^2(x)\,dx}.
\]

Для каждой функции пространства \( L_2 \) можно определить коэффициенты Фурье
\[
    a_n = \int\limits_a^b h(t)f(t)P_n(t)\,dt
\]
и рассматривать ряд Фурье по ортогональным многочленам. Как и у всяких
ортогональных рядов, частичные суммы такого ряда являются в некотором смысле
наилучшими приближениями \( f(x) \) в метрике пространства \( L_2 \). В самом
деле для произвольного многочлена степени \( n \) имеем:
\[
    Q_n(x) = c_0P_0(x) + c_1P_1(x) + \ldots + c_nP_n(x),
\]
\[
    \| f - Q_n \|^2 = \int\limits_a^b h(x)\left[f(x) - Q_n(x)\right]^2\,dx =
    \| f \|^2 - 2 \sum_{k=0}^n c_ka_k + \sum_{k=0}^n a_k^2 = \| f \|^2 -
    \sum_{k=0}^n a_k^2 + \sum_{k=0}^n (c_k - a_k)^2 \ge \| f \|^2 -
    \sum_{k=0}^n a_k^2.
\]

В частности, для частичных сумм ряда Фурье
\( s_n(x, f) = \sum\limits_{k=0}^n a_k P_k(x) \) получается равенство:
\[
    \| f - s_n \|^2 = \| f \|^2 - \sum_{k=0}^n a_k^2,
\]
из которого
\[
    \| f - s_n \| \le \| f - Q_n \|.
\]
Такиим образом, при любом \( n \) частичная сумма ряда Фурье даёт наилучшее
среднеквадратичное приближение функции \( f(x) \) по сравнению со всеми
многочленами \( Q_n(x) \) степени не выше \(n\).

Поскольку правая часть равенства
\[
    \| f - s_n \|^2 = \| f \|^2 - \sum_{k=0}^n
\]
больше нуля, то сумма квадратов коэффициентов Фурье не превосходит квадрата
нормы функции, а отсюда следует неравенство Бесселя
\[
    \sum_{k=0}^n a_k^2 \le \| f \|^2,
\]
из которого имеем \( \lim\limits_{k\rightarrow\infty} a_k = 0 \).

Если сегмент \( [a,b] \) конечен, а \( f(x) \) на нём непрерывна, то по теореме
Вейерштрасса о приближении непрерывной функции многочленами для каждого наперёд
заданного \( \eps > 0 \) существует \( F_n(x) \), такой, что
\( | f(x) - F_n(x) | < \eps,\ x\in[a,b] \). Из этого факта следует, что для
данной непрерывной \( f(x) \) существует \( \left\{ F_n(x) \right\} \),
сходящаяся равномерно к \( f(x) \) на \( [a,b] \). Но из
\( \| f - s_n \| \le \| f - Q_n \| \) с учётом определения нормы находим:
\[
    \| f - s_n \| \le \| f - Q_n \| = \sqrt{\int\limits_a^b h(x)
    \left[f(x)-Q_n(x)\right]^2\,dx} < \eps\sqrt{\int\limits_a^b h(x)\,dx} =
    \eps\sqrt{h_0}.
\]
Следовательно, если \( f(x) \) непрерывна на сегменте \( [a,b] \), то \( s_n \)
сходится к \( f(x) \) в среднем, то есть в метрике \( L_2 \).

Теперь рассмотрим условие сходимости ряда \( \sum\limits_{k=0}^\infty a_kP_k(x)
\) к функции \( f(x) \)  в отдельной точке \( [a,b] \). Вводя обозначение
\[
    K_n(x,t) = \sum\limits_{k=0}^n P_k(x)P_k(t),
\]
представим частичную сумму ряда в виде
\[
    s_n(x,f) = \int\limits_a^b h(t)f(t)K_n(x,t)\,dt.
\]
Далее, умножая тождество
\[
    \int\limits_a^b h(t)K_n(x,t)\,dt = 1
\]
на \( f(x) \) и вычитая из него почленно предыдущее, получим
\[
    f(x) - s_n(x,f) = \int\limits_a^b h(t)[f(x)-f(t)]K_n(x,t)\,dt.
\]
Заменив при помощи формулы Кристоффеля-Дарбу \( K_n(x,t) \), получим
\[
    f(x) - s_n(x,f) = \lambda_n\int\limits_a^b h(t)
    \left[\frac{f(x)-f(t)}{x-t}\right](P_{n+1}(x)P_n(t)-P_{n+1}(t)P_n(x))\,dt.
\]
Будем считать, что \( x \) зафиксирован на \( [a,b] \). Положим для краткости
\[
    \phi_x(t) = \frac{f(x)-f(t)}{x-t},
\]
а \( \{ \alpha_n \} \) -- коэффициенты разложения в ряд Фурье функции
\( \phi_x(t) \). Тогда получим 
\begin{equation}
    f(x) - s_n(x,f) = \lambda_n(\alpha_nP_{n+1}(x) - \alpha_{n+1}P_n(x)),
    \label{eq:helpful_for_Fourier}
\end{equation}
с помощью которой можно формулировать достаточные условия сходимости рядов Фурье
по ортогональным многочленам в отдельной точке.
\begin{theorem}
    Если сегмент \( [a,b] \) конечен, и вспомогательная функция
    \( \phi_x(t)\in L_2 \) при фиксированном \( x\in[a,b] \), а \( \{P_n(x)\} \)
    ограничена в точке \( x \), то ряд Фурье по \( \{P_n(x)\} \) функции
    \( f(x) \) сходится к ней в данной точке \( x \), то есть
    \[
        f(x) = \sum\limits_{n=0}^\infty a_nP_n(x),\ x\in[a,b].
    \]
\end{theorem}
\begin{proof}
    По лемме 3 об ограниченности последовательности \( \{\lambda_n\} \) на
    конечном сегменте последовательность \( \{\lambda_n\} \) ограничена, а в
    силу условия \( \phi_x(t)\in L_2 \) последовательность \( \{\alpha_n\} \)
    является бесконечно малой. С другой стороны, \( \{P_n{x}\} \) ограничена в
    точке \( x \). Следовательно, в правой части равенства
    (\ref{eq:helpful_for_Fourier}) стоит бесконечно малая последовательность,
    поэтому
    \[
        \lim_{n\rightarrow\infty} f(x) - s_n(x,f) = 0,\text{ или }
        f(x) = \sum_{n=0}^\infty a_nP_n(x).
    \]
    Теорема доказана.
\end{proof}

Таким образом, последнее равенство имеет место в такой точке \( x \), где 
\( \{ P_n(x) \} \) ограничены и выполняется условие
\begin{equation}
    \int\limits_a^b h(t)\left[\frac{f(x)-f(t)}{x-t}\right]^2\,dt < \infty.
    \label{eq:convergence_Fourier_in_L_2}
\end{equation}

В частности, условие (\ref{eq:convergence_Fourier_in_L_2}) выпоняется, если
\( f(t) \) в некоторой \( \eps \)-окрестности точки \( x \) удовлетворяет
условию Липшица порядка \( \alpha = 1 \), то есть
\[
    |f(x) - f(t)| \le M_1 |x - t|.
\]
Интеграл существует как несобственный, если в той же окрестности \( h(t) \)
ограничена, а \( f(t) \) удовлетворяет условию Липшица порядка
\( \alpha = \frac{1}{2} \):
\[
    |f(x) - f(t)| \le M_\frac{1}{2} |x - t|^\alpha, \frac{1}{2} < \alpha < 1.
\]

Остановимся теперь на принципе локализации условий сходимости:
\begin{theorem}
    Если функции \( f(x) \) и \( g(x) \) из пространства \( L_2 \) совпадают в
    \( (x_0 - \delta, x_0 + \delta) \), причём в \( x_0 \) \( \{P_n(x)\} \)
    ограничена, то в этой точке ряды Фурье этих функций по ортогональным
    многочленам сходятся или расходятся одновременно.
\end{theorem}
\begin{proof}
    Для разности частичных сумм этих функций нетрудно получить выражение
    \[
        s_n(x_0, f) - s_n(x_0, g) = s_n(x_0, f-g).
    \]
    Поскольку \( f(x) \) и \() g(x) \) -- функции класса \( L_2 \), то для
    функции \( f-g \) выполняется условие (\ref{eq:convergence_Fourier_in_L_2})
    и её ряд Фурье сходится к ней в точке \( x_0 \). Отсюда получаем
    \[
        \lim_{n\rightarrow\infty}(s_n(x_0, f) - s_n(x_0, g)) = 0 \Rightarrow
        \lim_{n\rightarrow\infty}s_n(x_0, f) =
        \lim_{n\rightarrow\infty}s_n(x_0, g).
    \]
    Следовательно, ряды сходятся или расходятся одновременно. Теорема доказана.
\end{proof}

Все предыдущие результаты были получены при условии, что \( f(x) \) входит в
пространство \( L_2[a,b;h(x)] \), то есть для функции сществует интеграл
\[
    \int\limits_a^b h(x) f^2(x)\,dx.
\]
Но ряды Фурье по ортогональным многочленам \( \{P_n(x)\} \) иногда можно
рассматриватть и в более общем случае.

Обозначим через \( L_1[a,b;h(x)] \) множество функций, для которых существует
интеграл
\[
    \int\limits_a^b h(x) |f(x)|\,dx.
\]
Если сегмент \( [a,b] \) конечен, то все интегралы
\[
    a_n = \int\limits_a^b h(t)f(t)P_n(t)\,dt
\]
сходятся, и данной функции \( f(x) \) из \( L_1 \) можно поставить в
соответствие ряд Фурье по ортогональным многочленам. В случае бесконечного
интервала \( (a,b) \) коэффициенты определяются при условии, что \( h(x)f(x) \)
имеет конечные степенные моменты. При этом, однако, мы не можем утверждать
экстремальное свойство такого ряда и справедливость неравенства Бесселя.