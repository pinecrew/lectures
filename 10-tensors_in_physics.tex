\section{Тензоры в физике.}
\subsection{Тензор инерции.}
	Примером использования тензоров в физике является описание вращения твёрдого тела относительно фиксированной точки. Пусть тело объёма \( V \) вращается с угловой скоростью \( \omega \) вокруг неподвижного центра: \[ \vec{v} = \omega\times\vec{r} \]
	Выделим в теле малый объём \( \,d V \), \( \,d m = \rho \,d V \) -- масса этого объёма. Его импульс \( \,d \vec{p} = \,d m \vec{v} \), а момент импульса \( \,d \vec{L} = \vec{r}\times\,d\vec{p} = \vec{r}\times\vec{v}\,d m = \vec{r}\times(\vec{\omega}\times\vec{r})\rho\,d V \). Раскрыв двойное векторное произведение, получим:
	\[ \,d \vec{L} = [\vec{\omega} r^2 - \vec{r}(\vec{\omega}\cdot\vec{r})]\rho\,d V, \]
	а проинтегрировав по всему объёму тела \( V \), мы найдём момент импульса всего тела относительно точки \( O \):
	\[ \vec{L} = \int\limits_V [\vec{\omega} r^2 - \vec{r}(\vec{\omega}\cdot\vec{r})]\rho\,d V. \]

	Теперь распишем подынтегральное выражение покоординатно: % за одно только это можно любить векторный анализ))
	\[
	\begin{array}{c}
		\vec{\omega} r^2 - \vec{r}(\vec{\omega}\cdot\vec{r}) =
		(\omega_1 \vec{e}_1 + \omega_2 \vec{e}_2 + \omega_3 \vec{e}_3)(x_1^2+x_2^2+x_3^2) - \\
		- (x_1 \vec{e}_1 + x_2 \vec{e}_2 + x_3 \vec{e}_3)(\omega_1 x_1 + \omega_2 x_2 + \omega_3 x_3) = \\
		= (\omega_1 x_1^2 + \omega_1 x_2^2 + \omega_1 x_3^2 - \omega_1 x_1^2 - \omega_2 x_1 x_2 - \omega_3 x_1 x_3 ) \vec{e}_1 + \\
		+ (\omega_2 x_1^2 + \omega_2 x_2^2 + \omega_2 x_3^2 - \omega_1 x_1 x_2 - \omega_2 x_2^2 - \omega_3 x_2 x_3 ) \vec{e}_2 + \\
		+ (\omega_3 x_1^2 + \omega_3 x_2^2 + \omega_3 x_3^2 - \omega_1 x_1 x_3 - \omega_2 x_2 x_3 - \omega_3 x_3^2 ) \vec{e}_3.
	\end{array}
	\]

	Следовательно, компоненты момента импульса будут следующими:

	\[
	\left\{
	\begin{array}{l}
		L_1 = \int\limits_V(\omega_1(x_2^2+x_3^2)+\omega_2(-x_1x_2)+\omega_3(-x_1x_3))\rho\,d V, \\
		L_2 = \int\limits_V(\omega_1(-x_1x_2)+\omega_2(x_1^2+x_3^2)+\omega_3(-x_2x_3))\rho\,d V, \\
		L_3 = \int\limits_V(\omega_1(-x_1x_2)+\omega_2(-x_2x_3)+\omega_3(x_1^2+x_2^2))\rho\,d V.
	\end{array} \right.
	\]

	Отсюда видно, что компоненты момента импульса связаны с угловой соростью линейными соотношениями

	\[
	\left\{
	\begin{array}{l}
		L_1 = I_{11}\omega_1+I_{12}\omega_2+I_{13}\omega_3, \\
		L_2 = I_{21}\omega_1+I_{22}\omega_2+I_{23}\omega_3, \\
		L_3 = I_{31}\omega_1+I_{32}\omega_2+I_{33}\omega_3,
	\end{array} \right.
	\]

	или \( \vec{L} = [I_{ij}] \vec{\omega} \), где 

	\[ [I_{ij}] = 
		\begin{bmatrix}
			I_{11} I_{12} I_{13}, \\
			I_{21} I_{22} I_{23}, \\
			I_{31} I_{32} I_{33}.
		\end{bmatrix}
	 \]

	 \( [ I_{ij} ] \) определяет преобразование, то есть \( [ I_{ij} ] \) определяет симметричный тензор второго ранга. Этот тензор называется тензором инерции твёрдого тела. Его диагональные элементы являются мооментами инерции этого тела относительно базисных осей, а остальные определяют давление на ось при вращении. Каждая компонента этого тензора зависит от геомерии тела, распределения его массы, выбора базиса, но не зависит от \( \vec{\omega} \).

	 Так как тензор инерции является сииметричным, то его можно привести к главным осям инерции. В базисе главных осей \( ( \vec{e}_1, \vec{e}_2, \vec{e}_3 ) \) матрица тензора инерции является диагональной:
\subsection{Тензор электропроводности.}

\subsection{Тензор диэлектрической проницаемости.}

\subsection{Тензор теплопроводности.}

	
