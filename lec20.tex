\chapter{Электромагнитные волны}

\section{Волновое уравнение (волновая функция)}

	Рассмотрим (\ref{eq19p3:3}.2) и (\ref{eq19p3:3}.4). Возьмем ротор от обоих
    частей (\ref{eq19p3:3}.2) и, с учетом (\ref{eq19p3:3}.4), получим:
	\[
        \rot\rot\vec{E} = -\mu\mu_0\partder{}{t}\rot\vec{H} = 
        -\Ezero\mu_0\varepsilon\mu\dder{\vec{E}}{t} = 
        -\frac{1}{c^2}\varepsilon\mu\dder{\vec{E}}{t} = 
        -\frac{1}{v^2}\dder{\vec{E}}{t},
        \]
	где
    \[
        v = \frac{c}{\sqrt{\varepsilon\mu}}.
    \]
	
	\[
        \rot\rot\vec{E} = \nabla\times(\nabla\times\vec{E}) =
        \nabla\div\vec{E} - \nabla^2\vec{E},
    \]
	а так как \( \div\vec{E} = 0 \), а \( \nabla^2 = \Delta \) -- оператор
    Лапласа, то:
	\begin{equation}
		\Delta\vec{E} = \frac{1}{v^2}\dder{\vec{E}}{t}.
        \label{eq20:1}
	\end{equation}
	
	Аналогично, взяв ротор от (\ref{eq19p3:3}.4) и сравнив его с 
    (\ref{eq19p3:3}.2), получим:
	\begin{equation}
		\Delta\vec{H} = \frac{1}{v^2}\dder{\vec{H}}{t}.
        \label{eq20:2}
	\end{equation}
	
	Формула (\ref{eq20:1}), как и (\ref{eq20:2}), содержит в себе три уравнения:
	\[
        \Delta E_i = \frac{1}{v^2}\dder{E_i}{t};
    \]
	\[
        \Delta H_i = \frac{1}{v^2}\dder{H_i}{t};
    \]
	где \( i = 1, 2, 3 \), \( E_i = E_i(x, y, z, t) \).
	
	\begin{definition}
        Дифференциальное уравнение в частных производных
        \begin{equation}
            \Delta\psi = \frac{1}{v^2}\dder{\psi}{t},
            \label{eq20:3}
        \end{equation}
        где \( \psi = \psi(x, y, z, t) \) -- произвольная, дважды
        дифференцируемая по всем переменным, называется \textbf{волновым}, а
        сама функция  \( \psi \) -- \textbf{волновой}.
	\end{definition}
	
	Оно так называется потому, что функция \( \psi \) описывает распространение
    первоначальных возмущения \( \psi(x, y, z, 0) \) в пространстве со скоростью
    \( v \). Покажем это. Для этого рассмотрим одномерный аналог (\ref{eq20:3}):
	\[
        \psi = \psi(x, t).
    \]
	Тогда (\ref{eq20:3}):
	\begin{equation}
		\dder{\psi}{x} = \frac{1}{v^2}\dder{\psi}{t}.
        \label{eq20:4}
	\end{equation}
	Общим решением (\ref{eq20:4}) является функция:
	\begin{equation}
		\psi(x, t) = f_1(x - vt) + f_2(x + vt),
        \label{eq20:5}
	\end{equation}
	где \( f_1 \) и \( f_2 \) -- произвольные, дважды дифференцируемые функции
    единых аргументов \( x - vt \) и \( x + vt \).
	
	\textbf{Проверка:}
    \begin{align*}
        & \partder{f_1}{x} = f'_1\partder{(x-vt)}{x} = f'_1; \\
        & \partder{f_1}{t} = f'_1\partder{(x-vt)}{t} = f'_1\cdot(-v); \\
        & \dder{f_1}{x} = f''_1 & \dder{f_1}{t} = f''_1\cdot v^2.
	\end{align*}
	Подставив это в (\ref{eq20:5}), получим тождество:
	\[
        f''_1 = \frac{1}{v^2}f''_1v^2.
    \]
	
	Покажем, что функция \( f(x - vt) \) описывает распространение начального
    возмущения \( \left.f(x)\right|_{t = 0} \) вправо по оси \( Ox \) со
    скоростью \( v \). Для этого рассмотрим семейство функций \( f(x - a) \),
    где параметр \( a \) принимает поочередно значения \( a = 0, a_1, a_2,
    \ldots \):
    \begin{itemize}
        \item при \( a = 0 \) имеем начальное возмущение \( f(x) \):
        \item при \( a = a_1 \) имеем начальное возмущение \( f(x) \), сдвинутое
            вправо на \( a_1 \) по оси \( Ox \):
        \item при \( a = a_2 \) имеем начальное возмущение \( f(x) \), сдвинутое
            вправо на \( a_2 \) по оси \( Ox \):
    \end{itemize}
	
	Пусть теперь параметр \( a \) меняется по закону \( a = vt \), где \( t = 0,
    t_1, t_2, \ldots \), причём \( t_i = a_i/v \). Таким образом, в момент
    времени \( t_i \) возмужение будет сдвинуто на \( a_i \) вправо. Если
    \( t \) меняется непрерывно, то график функции двигается вправо по оси
    \( Ox \) со скоростью \( v \). Таким образом, функция \( f(x - vt) \)
    описывает распространение со скоростью \( v \) вправо по оси \( Ox \)
    исходного возмущения.
	
	Аналогично, функция \( f(x + vt) \) описывает распространение со скоростью
    \( v \) влево по оси \( Ox \) исходного возмущения. Следовательно, функция
    \( \psi(x, t) = f_1(x - vt) + f_2(x + vt) \) показывает распространение
    исходного возмущения \( \psi(x, 0) \) влево и вправо.
	
	Конкретный вид \( f_1 \) и \( f_2 \) определяется начальными условиями
    возмущения:
	\[
        \left.
        \begin{array}{l}
            \psi(x, 0) = f(x) \\
            \dot{\psi}(x, 0) = g(x)
        \end{array}
        \right\|
        \Rightarrow f_1(x -vt); \ f_2(x + vt)
    \]
	
	В частности, если \( f(x) = A\sin kx \), то по шнуру пойдут волны
	\[
        f(x - vt) = A\sin\omega\left(t \pm \frac{x}{v}\right).
    \]
	
	\begin{conclusion}
        Уравнения (\ref{eq20:1}) и (\ref{eq20:2}) являются волновыми и описывают
        распространение полей \( \vec{E} \) и \( \vec{H} \) в пространстве со
        скоростью \( \vec{v} \), то есть распространение
        \textbf{электромагнитной волны}.
	\end{conclusion}
	
	Однако, для исследования структуры и свойств электромагнитных волн эти
    уравнения весьма сложны. Для этой цели рассмотрим простейшую
    электромагнитную волну -- \textit{плоскую}, бегущую вдоль одного направления
    -- оси \( Oz \).
	
\section{Плоская электромагнитная волна}

	\begin{definition}
        Волна называется плоской, бегущей вдоль оси \( Oz \), если поля
        \( \vec{E} \) и \( \vec{H} \) в ней зависят только от координаты
        \( z \) и времени \( t \):
        \[
            \left\{
            \begin{array}{l}
                \vec{E} = \vec{E}(z, t), \\
                \vec{H} = \vec{H}(z, t).
            \end{array}
            \right.    
        \]
	\end{definition}
	
	Это означает, что частные производные от компонент полей по координатам
    \( x \) и \( y \) равны нулю:
	\[
        \partder{E_i}{x} = \partder{E_i}{y} = \partder{H_i}{x} =
        \partder{H_i}{y} = 0.
    \]
	Волна называется плоской потому, что плоским является ее \textit{фронт}.
    \begin{definition}
        \textit{Фронт} (или волновая поверхность) -- для синусоидальной волны:
        это поверхность в пространстве, на которой фаза волны \( \psi(x, t) \)
        постоянна; скорость распространения фронта в пространстве называется
        \textit{фазовой скоростью}; для несинусоидальной (например, импульс)
        волны: это поверхность в пространстве, до которой в данный момент дошло
        возмущение.
    \end{definition}
	Фронтом плоской волны является плоскость \( (x, y) \). Для исследования
    структуры и свойств плоской волны запишем уравнения Максвелла для
    однородной, изотропной, линейной, нейтральной и непроводящей среды:
	\[
        \left\{
        \begin{array}{l}
            \div\vec{E} = 0, \\
            \rot\vec{E} = -\mu\mu_0\partder{\vec{H}}{t}, \\
            \div\vec{H} = 0, \\
            \rot\vec{H} = \varepsilon\Ezero\partder{\vec{E}}{t},
        \end{array}
        \right.
    \]
	и распишем их покомпонентно:
	\[
        \left\{
        \begin{array}{l}
            \partder{E_x}{x} + \partder{E_y}{y} + \partder{E_z}{z} = 0, \\
            \partder{E_z}{y} - \partder{E_y}{z} = -\mu\mu_0\partder{H_x}{t}, \\
            \partder{E_x}{z} - \partder{E_z}{x} = -\mu\mu_0\partder{H_y}{t}, \\
            \partder{E_y}{x} - \partder{E_x}{y} = -\mu\mu_0\partder{H_z}{t}, \\
            \partder{H_x}{x} + \partder{H_y}{y} + \partder{H_z}{z} = 0, \\
            \partder{H_z}{y} - \partder{H_y}{z} =
                \varepsilon\Ezero\partder{E_x}{t}, \\
            \partder{H_x}{z} - \partder{H_z}{x} =
                \varepsilon\Ezero\partder{E_y}{t}, \\
            \partder{H_y}{x} - \partder{H_x}{y} =
                \varepsilon\Ezero\partder{E_z}{t}.
	    \end{array}
        \right.
    \]
	С учетом того, что
    \[
        \partder{E_i}{x} = \partder{E_i}{y} = \partder{H_i}{x} =
        \partder{H_i}{y} = 0,
    \]
    получим:
	\begin{equation}
        \left\{
        \begin{array}{l}		
            \partder{E_z}{z} = 0, \\
            \partder{E_y}{z} = \mu\mu_0\partder{H_x}{t}, \\
            \partder{E_x}{z} = -\mu\mu_0\partder{H_y}{t}, \\
            \partder{H_z}{t} = 0, \\
            \partder{H_z}{z} = 0, \\
            \partder{H_y}{z} = -\varepsilon\Ezero\partder{E_x}{t}, \\
            \partder{H_x}{z}= \varepsilon\Ezero\partder{E_y}{t}, \\
            \partder{E_z}{t} = 0.
	    \end{array}
        \right.
        \label{eq20.1:system}
    \end{equation}
	
	Из уравнений (\ref{eq20.1:system}.1) и (\ref{eq20.1:system}.8) видно, что
    компонента поля \( E_z \) не зависит ни от координаты \( z \), ни от времени
    \( t \), то есть \( E_z = \const \), ее можно принять равной нулю.

	Аналогично, из уравнений (\ref{eq20.1:system}.4) и (\ref{eq20.1:system}.5)
    видно, что компонента \( H_z = \const = 0 \). Таким образом, продольных
    компонент у плоской волны нет.
	
	\begin{conclusion}
        Плоская электромагнитная волна является поперечной.
	\end{conclusion}

	Остальные уравнения сгруппируем в две независимые подсистемы:
	\begin{equation}
        \left\{
        \begin{array}{l}
            \partder{E_x}{z} = -\mu\mu_0\partder{H_y}{t}, \\
            \partder{H_y}{z} = -\varepsilon\Ezero\partder{E_x}{t},
	    \end{array}
        \right.
        \label{eq20.1:5}
    \end{equation}
	и
	\begin{equation}
        \left\{
        \begin{array}{l}
            \partder{E_y}{z} = \mu\mu_0\partder{H_x}{t}, \\
            \partder{H_x}{z} = \varepsilon\Ezero\partder{E_y}{t}.
	    \end{array}
        \right.
        \label{eq20.1:6}
    \end{equation}

	Рассмотрим какую-либо систему, например (\ref{eq20.1:5}). Из нее видно, что
    компонента \( H_y \) порождает компоненту \( E_x \), то есть если
    \( H_y = 0 \), то и \( E_x = 0 \).
	
	\begin{conclusion}
        В плоской волне поля \( \vec{E} \) и \( \vec{H} \) перпендикулярны:
        \( \vec{E} \perp \vec{H} \).
	\end{conclusion}
	
	Далее, продифференцируем уравнение (\ref{eq20.1:5}.1) по \( z \) и сложим с
    умноженным на \(  -\mu\mu_0 \) уравнением (\ref{eq20.1:5}.2),
    продифференцированным по \( t \):
	\[
        \dder{E_x}{z} = \varepsilon\Ezero\mu\mu_0\dder{E_x}{t},
    \]
	или, так как
    \[
        \Ezero\mu_0 = \frac{1}{c^2}, \text{ а }
        \frac{\sqrt{\varepsilon\mu}}{c} = \frac{1}{v}:
    \]
	\begin{equation}
		\dder{E_x}{z} = \frac{1}{v^2}\dder{E_x}{t}.
        \label{eq20.1:7}
	\end{equation}
	
	Аналогично получим:
	\begin{equation}
		\dder{H_y}{z} = \frac{1}{v^2}\dder{H_y}{t}.
        \label{eq20.1:8}
	\end{equation}
	
	Уравнения (\ref{eq20.1:7}) и (\ref{eq20.1:8}) являются волновыми, вместе они
    описывают бегущую волну для полей \( E_x \) и \( H_y \) вдоль оси \( Oz \).
	Следовательно, поля \( E_x \) и \( H_y \) можно представить в виде функций
    \( f \) и \( g \):
	\[
        \left\{
        \begin{array}{l}
            E_x(z, t) = f(z - vt), \\
            H_y(z, t) = g(z - vt).
        \end{array}
        \right.
    \]
	
	Подставляя это, например, в (\ref{eq20.1:5}.1), получим:
	\[
        f’\cdot1 = -\mu\mu_0g’\cdot(-v),
    \]
	где символ \( ’ \) означает производную по всему аргументу \( (z - vt) \),
    то есть \( f’ = \partder{f}{(z - vt)} \). Отсюда
	\[
        f’ = v\mu\mu_0g’.
    \]
	Интегрируя, получим:
	\begin{equation}
		E_x(z, t) = v\mu\mu_0H_y(z, t),
        \label{eq20.1:9}
	\end{equation}
	откуда
	\begin{equation}
		\sqrt{\varepsilon\Ezero}E_x(z, t) = \sqrt{\mu\mu_0}H_y(z, t).
        \label{eq20.1:10}
	\end{equation}
	
	\begin{conclusion}
        Из уравнения \textnormal{(\ref{eq20.1:10})} видно, что поля
        \( \vec{E} \) и \( \vec{H} \) в плоской волне изменяются
        \textbf{синфазно}.
	\end{conclusion}
	
	\begin{conclusion}
        Тройка векторов \( \vec{v} \), \( \vec{E} \) и \( \vec{H} \) в плоской
        является правой, то есть
        \[
            \vec{v} \uparrow\uparrow (\vec{E}\times\vec{H}).
        \]
        
        Это является внутренним свойством плоской волны.
	\end{conclusion}
	
	\begin{remark}
	    Из уравнения (\ref{eq20.1:9}) следует, что \( E_x = vB_y \), а в вакууме
        \( E_x = cB_y \).
	\end{remark}

\section{Поляризация плоской волны}

	Если мы проведем те же процедуры с системой (\ref{eq20.1:6}), то получим
    такие же уравнения для компонент \( E_y \) и \( H_x \). В общем случае
    векторы \( \vec{E} \) и \( \vec{H} \) могут быть направлены под углами к
    осям \( Ox \) и \( Oy \).
	
	\begin{definition}
        Если векторы \( \vec{E} \) и \( \vec{H} \) в плоской волне качаются в
        фиксированной плоскости, то волна называется \textbf{плоско}, или
        \textbf{линейно поляризованной}, причем в качестве плоскости поляризации
        принимается плоскость качания вектора \( \vec{E} \).
	\end{definition}
	
	Однако, при прохождении через некоторые вещества, называемые
    \textit{оптически активными}, плоскости качания \( \vec{E} \) и
    \( \vec{H} \) поворачиваются так, что описывают винтовые линии. Такая
    поляризация называется \textbf{круговой}.
	
	Если оптически активное вещество является к тому же и анизотропным, то концы
    векторов \( \vec{E} \) и \( \vec{H} \) будут описывать эллиптические
    винтовые линии. Такая поляризация называется \textbf{эллиптической}.
	
\section{Синусоидальные плоские волны}

	Пусть синусоидальная плоская волна бежит вдоль оси \( Oz \). В этом случае
    функции \( f \) и \( g \) имеют конкретный вид:
	\[ 
        \left\{
        \begin{array}{l}
		    E_x(z, t) = E_{x0}\sin\omega\left(t - \frac{z}{v}\right), \\
		    H_y(z, t) = H_{y0}\sin\omega\left(t - \frac{z}{v}\right).
	    \end{array}
        \right.
    \]
	
	\begin{definition}
        Величина
        \[
            k = \frac{\omega}{v}
        \]
        называется \textbf{волновым числом}.
	\end{definition}
	
	Тогда плоские синусоидальные волны записываются в виде:
	\[
        \left\{
        \begin{array}{l}
		    E_x(z, t) = E_{x0}\sin(\omega t - kz), \\
		    H_y(z, t) = H_{y0}\sin(\omega t - kz).
	    \end{array}
        \right.
    \]
	
	Если зафикировать координату \( z \) в точке \( z_0 \), то получим
    синусоидальные колебания с периодом \( T = 2\pi/\omega \):
	\[
        E_x(z_0, t) = E_{x0}\sin(\omega t +  \varphi),
    \]
	где \( \varphi = -kz \).
	
	Если же зафиксировать время \( t = t_0 \), то получим мгновенный профиля
    волны:
	\[
        E_x(z, t_0) = E_{x0}\sin(kz + \psi)
    \]
	с пространственным периодом (длиной волны)
    \[
        \lambda = \frac{2\pi}{k} = \frac{2\pi}{\omega}v = vT.
    \]
	
	Таким образом, длина волны \( \lambda \) -- это расстояние, которое волна
    проходит за период \( T \):
	\[
        E(z, t) = E(z + \lambda, t) = E(z, t + T).
    \]
	
\section{Сферические волны}

	Пусть электромагнитные волна возбуждается точным источником и
    распространяется в однородном изотропном пространстве. Тогда функция
    \( \psi(x, y, z, t) \) будет иметь вид \( \psi(r, t) \), где \( r \) --
    расстояние до источника. Соответствующее волновое уравнение будет выглядеть
    следующим образом:
	\[
        \Delta_r\psi = \frac{1}{v^2}\dder{\psi}{t},
    \]
	где \( \Delta_r \) -- сферическая часть лапласиана:
	\[
        \Delta_r = \frac{1}{r^2}\partder{}{r}\left(r^2\partder{}{r}\right).
    \]
	Тогда
	\[
        \Delta_r\psi = \frac{1}{r^2}\partder{}{r}
        \left(r^2\partder{\psi}{r}\right) = \frac{1}{r}\dder{(r\psi)}{r},
    \]
	и волновое уравнение для сферических волн будет иметь вид:
	\[
        \frac{1}{r}\dder{(r\psi)}{r} = \frac{1}{v^2}\dder{\psi}{t},
    \]
	или, так как \( r \) и \( t \) -- независимые аргументы:
	\[
        \dder{(r\psi)}{r} = \frac{1}{v^2}\dder{(r\psi)}{t}.
    \]
	
	Если обозначить \( \psi^* = r\psi \), то волновое уравнение в каноническом
    виде:
	\[
        \dder{\psi^*}{r} = \frac{1}{v^2}\dder{\psi^*}{t}.
    \]
	Его решение:
	\[
        \psi^*(r, t) = f(t - vt).
    \]
	Тогда
	\[
        \psi(r, t) = \frac{1}{r}f(t - vt).
    \]
	В частности, для синусоидальных волн:
	\[
        E(r, t) = E_0\frac{r_0}{r}\sin(\omega t - kr).
    \]
	
	\begin{remark}
        При \( r \to 0 \) \( E \to \infty \).
	\end{remark}
	
	\begin{remark}
        Сходящегося решения нет, так как оно невозможно из природы процесса.
	\end{remark}
	
	\begin{remark}
        Сферическая волна не является сферической в физическом смысле.
	\end{remark}
	
\section{Стоячие волны}

	Пусть плоская волна падает нормально на металлическую  поверхность (зеркало)
    и без потерь отражается от нее.
	\[
        \left\{
        \begin{array}{l}
            E_\textit{пад}(z, t) = E_0\sin(\omega t + kz), \\
            E_\textit{отр}(z, t) = E_0\sin(\omega t - kz + \alpha),
        \end{array}
        \right.
    \]
	где \( \alpha \) -- фазовый сдвиг при отражении, то есть при \( z = 0 \).
	
	Фазовый сдвиг \( \alpha \) определяется из условий: при \( z = 0 \), то есть
    на поверхности металла
	\[
        E_\textit{рез}(0, t) = E_\textit{пад}(0, t) + E_\textit{отр}(0, t) = 0.
    \]
	Отсюда
    \[
        \sin\omega t + \sin(\omega t + \alpha) = 0 \Rightarrow \alpha = \pi,
    \]
	и тогда
    \[
        \sin(\omega t - kz + \pi) = -\sin(\omega t - kz).
    \]
	Результатом будет интерференция (суперпозиция) падающей и отраженной волн:
	\begin{equation}
		E_\textit{рез} = E_\textit{пад} + E_\textit{отр} =
        E_0\left(\sin(\omega t + kz) - \sin(\omega t - kz)\right) =
        2E_0\sin kz\cos\omega t
        \label{eq20.2:1}
	\end{equation}
	
	Функция (\ref{eq20.2:1}) дает колебания вектора \( \vec{E} \) с частотой
    \( \omega \) и с амплитудой, зависящей от координаты:
	\begin{equation}
		E_m(z) = 2E_0|\sin kz| = 2E_0|\sin2\pi\frac{z}{\lambda}|.
        \label{eq20.2:2}
	\end{equation}
	Функция (\ref{eq20.2:1}) описывает \textbf{стоячие волны}. В них вектор
    \( \vec{E} \) колеблется с частотой \( \omega \) и с амплитудой
    (\ref{eq20.2:2}). Сечения \( z_n \), где амплитуда колебания поля
    \( \vec{E} \) равна 0, называются \textbf{узлами}, а сечения, где амплитуда
    поля \( \vec{E} \) максимальна, -- \textbf{пучностями}.
	
	Из рисунка и из (\ref{eq20.2:2}) видно:
    \begin{align*}
        & z_n^\text{узел} = n\frac{\lambda}{2}, \\
        & z_n^\text{пучность} = \frac{\lambda}{4} + n\frac{\lambda}{2} =
            (2n + 1)\frac{\lambda}{4}.
    \end{align*}
	
	\begin{remark}
        Функция (\ref{eq20.2:1}) не имеет вида \( f(z - vt) \), но она
        удовлетворяет волновому уравнению.
	\end{remark}
	
	\begin{remark}
        Исходя из свойств плоской волны и условий на поверхности металла:
        \( E_\textit{рез} \), легко видеть, что при отражении от металла фаза
        поля \( \vec{H} \) не меняется:
	    \[
            H(z, t) = 2H_0\cos kz\sin\omega t.
        \]
	\end{remark}
