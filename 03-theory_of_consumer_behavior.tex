\section{Теория потребительского поведения}
Потребление -- это процесс использования результатов производства для
удовлетворения определённых потребностей.

Потребность -- это экономический мотив, возникающий из необходимости или желания
потреблять различные объекты богатства. Потребности человека разнообразны, а их
удовлетворение -- основная цель жизни.

Теория потребительского поведения основана на предположении, что потребители
ведут себя рационально.

Принцип рационального потребления потребителя:
\begin{enumerate}
    \item потребитель располагает свои потребности в соответствии со своими
        предпочтениями;
    \item потребитель приобретает такой набор благ, который обеспечивает ему
        максимальное удовлетворение потребностей при ограниченном размере его
        денежного дохода;
    \item цены не зависят от количества благ, покупаемых отдельными
        домохозяйствами.
\end{enumerate}

Полезность -- это способность экономического блага удовлетворять одну или
несколько человеческих потребностей.

Общая полезность -- это полезность, получаемая от всего набора потребляемого
блага.

Предельная полезность (маржиналисты) -- полезность, которая приносится каждой
дополнительной единицей приобретаемого блага.

Существуют 2 теории полезности:
\begin{enumerate}
    \item кардиналистская -- предполагает точное количественное измерение
        полезности;
    \item ординалистская -- предполагает ранжирование товаров каждым
        покупателем.
\end{enumerate}

На поведение потребителя оказывает влияние потребительский спрос, который бывает функциональный и нефункциональный.

Функциональный спрос -- это спрос на товар, обусловленный качествами, присущими
самому товару.

Нефункциональный спрос -- это спрос, обусловленный факторами, не связанными с
самим товаром. Бывает спекулятивным и нерациональным.

Спекулятивный спрос -- это спрос, связанный с высокими инфляционными ожиданиями.
Нерациональный спрос -- это незапланированный спрос, возникший под влиянием
внезапного желания, настроения или прихоти.

На нефункциональный спрос влияют следующие эффекты:
\begin{itemize}
    \item эффект присоединения к большинству;
    \item эффект сноба;
    \item эффект Веблена -- это эффект увеличения потребительского спроса,
        связанный с тем, что товар имеет более высокую цену.
\end{itemize}

Цена блага определяется его предельной полезностью для потребителя. Закон
убывающей предельной полезности лежит в основе определения спроса.
\[
    MU = \frac{\Delta TU}{\Delta Q}.
\]

Функция полезности достигает наибольшего значения в том случае, когда денежный
доход потребителя распределяется таким образом, что каждый последний рубль,
затраченный на приобретение другого блага приносит одинаковую предельную
полезность.

Инструментом ординалистской теории являются кривые безразличия. которые
показывают различные комбинации двух экономических благ, имеющих одинаковую
полезность для потребителя.

Множество кривых безразличия называется картой кривых безразличия.
Свойства кривых:
\begin{enumerate}
    \item имеют нисходящий характер и выпуклы по отношению к началу координат;
    \item никогда не пересекаются;
    \item чем правее и выше расположена кривая, тем больше удовлетворение
        приносят представленные ею комбинации благ.
\end{enumerate}
Участок кривой безразличия, в котором возможна замена одного блага другим
называется зоной замещения.

Предельная норма замещения -- это количество одного товара, которое потребитель
готов обменять на другой товар так, чтобы степень его удовлетворённости от
потребления данного набора благ осталась без изменения.

Бюджетное ограничение -- это комбинации товаров, которые потребитель имеет
возможность купить на располагаемую сумму.

В точке касания кривой безразличия бюджетного ограничения достигается положение
равновесия потребителя. Поведение потребителя объясняется с помощью эффекта
дохода и эффекта замещения. Эффект дохода определяется как изменение реального
дохода потребителя при изменении цен на товары и услуги. Эффект замещения
состоит в том, что снижение цены на товар относительно удорожает другие товары.
Эффекты взаимодействуют и дополняют друг друга, вызывая расширение объёмов
потребления товаров и услуг при снижении цен.
