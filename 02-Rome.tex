\lecture{Культура Древней Рима}
\section{Периодизация древнеримской культуры}

  \begin{enumerate}
    \item 15--5~вв.~до~н.~э. Древнейший период (период заселения Апеннинский
      полуострова)
      \begin{itemize}
        \item Культура \emph{terra marra} (чернозем). Занимались рыболовством,
          жили на побережье.
        \item Этрусская культура. Этрурия~-- 18 городов-государств. Похоже на
          Грецию.
          \emph{Достижения}:
            \begin{itemize}
              \item основа латинского алфавита (однако писали справа-налево);
              \item римские цифры.
            \end{itemize}
          Религиозный и культурный центр~-- Фанум Вальтумне
            (\emph{Fanum Voltumnae}).
      \end{itemize}
    \item 753--509~гг.~до~н.~э. Царский период (период основания Рима).
      \medskip
      753~год, по легенде, является годом основания Рима. Первые поселения на
      реке Тибр датированы 10~веком до нашей эры. Формирование общин. Римом
      управлял царь, которого избирали. Помимо царя был сенат. Сенаторы~--
      потомки наиболее ранних переселенцев на апеннины.
      \medskip
      Период семицарствия. Первый царь~-- Ромул. Начинает складываться
      социальная система: патриции~-- плебеи~-- клиенты.
      \medskip
      Последний царь~-- Тарквиний Гордый, правивший до 510~года до~н.~э.
    \item 509--31~гг.~до~н.~э. Римская республика.
      \medskip
      В 509 году складывается республика. Рим начинает захватывать территорию
      Апеннинского полуострова. Захватывая южную Италию, заселенную греками,
      римляне перенимали греческую культуру. Греческая система образования.
      Спорт как подготовка к армии. Переход от философии к риторике. Языки,
      литература, история, историография выходят на первый план. Римский
      реалистичный портрет. Основы теории реалистичного искусства.
      \medskip
      Пунические войны с Карфагеном, закончившиеся для Рима победой. Источник
      рабства~-- военнопленные. Победа над Карфагеном способствует развитию
      имперского сознания. 200--300 специализированных рабов в домах у
      патрициев.
    \item 31~г.~до~н.~э.--476~г.~н.~э. Римская империя.
      \medskip
      Октавиан Август~-- первый император (принцепс). Был установлен принципат
      Августа. К концу периода устанавливается доминат~-- монархия~-- устранены
      все признаки республики. Кризис. Ослабевает связь с провинцией из-за
      огромных масштабов государства.
      \medskip
      Период наивысшего расцвета культуры, соединение римской и греческой
      культуры. Литература: Овидий, Гораций, Вергилий.
      \medskip
      395~г.~-- раскол на западную и восточную части; \\
      476~г.~-- падение Рима под натиском варваров.
  \end{enumerate}
  
\section{Самобытность}

  \begin{itemize}
    \item \emph{римский патриотизм}~-- представление об особой богоизбранности
      народа;
    \item \emph{римский миф}~-- особая идеологическая политика, основанная на
      идее установления власти над миром;
    \item \emph{культ полководца, императора};
    \item \emph{особые формы искусства}:
      \begin{itemize}
        \item триумфальная арка;
        \item триумфальная колонна;
        \item мосты.
      \end{itemize}
  \end{itemize}
