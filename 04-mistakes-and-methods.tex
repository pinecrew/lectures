\lecture{Ошибки выборочного исследования и методы сбора социологической
  информации}
\section{Ошибки выборочного исследования}

  Традиционно выделяют две компоненты ошибки выборочного исследования:
  систематическую и случайную.

  Причины систематической ошибки:
  \begin{enumerate}
    \item методы исследования (отбор информации, ее обработка, инструментарий);
    \item невыполнение требований случайности при извлечении выборки (ошибка
      контура);
    \item отказы и выбывание респондентов и уменьшение объема выборки;
    \item влияние корреспондента на респондентов.
  \end{enumerate}

  Случайных ошибок избежать невозможно. Причины случайных ошибок зависят от
  специфики выборочного исследования.

  Величина случайной ошибки выборки зависит от следующих факторов:
  \begin{enumerate}
    \item тип выборки (количество ступеней отбора);
    \item способ извлечения выборки на каждой из ступеней отбора;
    \item объем выборки;
    \item характер оцениваемого признака.
  \end{enumerate}

\section{Методы сбора социологической информации}

  Методы:
  \begin{itemize}
    \item \emph{наблюдение}~-- целенаправленное планомерное и фиксируемое
      восприятие изучаемого объекта, его признаков, свойств, факторов
      деятельности, поведения, их повторяемости и типичности.

      Инструментарий в результате исследования~-- бланки, дневники и протоколы
      наблюдения.

      Наблюдение бывает:
      \begin{itemize}
        \item включенное или невключенное: исследователь либо внутри, либо
          вне объекта;
        \item явное или скрытое: объект либо знает, либо не знает о
          наблюдении;
        \item формализованное или неформализованное: у исследователя либо
          есть параметры для наблюдения, либо нет;
        \item систематическое или случайное: наблюдение происходит
          запланированно или нет.
      \end{itemize}

      Основной проблемой наблюдения является субъективность.

    \item \emph{Эксперимент}~-- метод, фиксирующий информацию об изменении
      показателей деятельности в результате воздействия на объект заданных и
      контролируемых факторов.

      В любом социальном эксперименте должны быть две группы: контрольная и
      экспериментальная.

    \item \emph{Анализ документов}. Документ~-- специально созданный автором
      материальный предмет, предназначенный для фиксации, хранения и передачи
      информации.

      Документы делятся
      \begin{itemize}
        \item по характеру источника на официальные и неофициальные;
        \item по способу фиксации на письменные, иконографические и
          фонетические;
        \item по степени персонофикации на личные и безличные;
        \item по источнику на первичные и вторичные.
      \end{itemize}

      Предмет анализа документов~-- это признаки, свойства документов, которые
      могут характеризовать содержание изучаемого явления с точки зрения целей
      и задач исследования.

      Анализ документов делится на традиционный и контент-анализ.

      Традиционный анализ является качественным методом; в свою очередь делится
      на внутренний~-- изучение текста, чувств, посланий, и внешний~-- изучение
      причины и среды создания документа.

      Контент-анализ~-- количественный метод, выявляющий количественные
      статистические характеристики текста, находящий свойства и признаки,
      которые отражали бы его существенные стороны. Чаще всего контент-анализу
      подвергаются СМИ.

    \item \emph{Опрос}~-- устное или письменное обращение к исследуемой
      совокупности людей с вопросами, содержание которых представляет проблему
      исследования.

      Опросы делятся на
      \begin{itemize}
        \item устные (интервью) и письменные (анкетирование);
        \item индивидуальные и групповые;
        \item экспертные и массовые.
      \end{itemize}
  \end{itemize}
