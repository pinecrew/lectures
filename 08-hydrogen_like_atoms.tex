\subsection{Квантование атома водорода}
Рассматривается одноэлектронная задача в гиперболической потенциалььной яме:
\[
    U(r) = -\frac{kZe^2}{r}.
\]
Соответствующее уравнение Шрёдингера имеет вид:
\[
    \Delta\psi + \frac{2m}{\hbar^2}(E-U(r))\psi = 0,
\]
причём задача решается в сферических координатах, т.е 
\[
    \Delta = ...
\]

Следствия из решения этого уравнения:
\begin{enumerate}
    \item уравнение имеет решение при любых положительных энергиях \( E \);
    \item оно также имеет рашения лишь при некоторых значениях \( E < 0 \):
    \[
        E = -\frac{m_e e^4}{2\hbar^2}\frac{Z^2}{n^2};
    \]
    \item оказывается, что волновые функции, решающие уравнение, зависят от трёх
    целочисленных параметров:
    \begin{itemize}
        \item \( n (1, 2, 3, \ldots) \) -- главное квантовое число -- определяет
        энергию электрона в атоме,
        \item \( l (0, 1, \ldots, n-1) \) -- орбитальное квантовое число --
        определяет модуль момента импульса электрона в атоме,
        \item \( m (-l, -l+1, \ldots, l-1, l) \) -- магнитное квантовое число --
        определяет проекцию момента импульса на выделенное напрввление;
    \end{itemize}
       
    \item энергия электрона зависит только от главного квантого числа,
    следовательно, каждому собственному значению \( E_n \) соответствует
    несколько собственных волновых функций \(\psi_{nlm}\), отличающихся числами
    \( l \) и \( m \). Это означает, что атом водорода может иметь одно и то же
    значение энергии, находясь в нескольких различных состояниях. Такие
    состояния называются \emph{вырожденными}, а число состояний с определённым
    значением энергии -- \emph{кратностью вырождения};
\end{enumerate}

Атомная физика пользуется терминологией спектроскопии. Состояния с \( l=0 \)
называются \( s \)-состоянием, дальше \( p,\ d,\ f,\ g,\ h,\ \ldots \). Перед
этими символами указывается главное квантовое число. В квантово механике
доказывается, что испускание и поглощение фотона происходит при переходе
с одного уровня на другой при условии \( \Delta l = \pm1 \). Оно является
следствием закона сохрпнения момента импульса. Построим спектральные уровни
атома водорода [рисунок 3].

Решением уравнения Шрёдингера для атома водорода являются волновые функции,
которые можно представить в виде:
\[
\phi_{nlm} = R_{nl}(r) Y_{lm}(\theta, \phi).
\]
Угловая часть волновой функции является собственной функцией оператора момента
импульса
\[
    \hat{L^2}\psi = L^2\psi.
\]
Также возможно представление
\[
    Y(\theta, \phi) = \Theta_{lm}(\theta)e^{im\phi}.
\]
Волновая функция должна удовлетворять условию нормировки:
\[
    \int \psi\psi^* dV = 1,
\]
откуда 
\[
    \int R^2 r^2\,dr \int YY^*\,d\theta = 1,
\]
\[
    \int\limits_0^\infty R^2 r^2\,dr = 1,\ \int YY^*\,d\theta = 1.
\]

Определим вероятность нахождения электрона на расстоянии \( r \) от ядра.
Для этого найдём вероятность попадания электрона в шаровой слой радиуса \( r \)
толщины \( dr \):
\[
    dW_{r,\theta,\phi} = |\psi|^2\,dV = R^2 r^2 Y Y^* d \Omega,
\]
\[
    \rho(r) = R^2 r^2.
\]
[рисунок 4]
Радиусы орбит по Бору совпадают с наиболее вероятными расстояниями электрона от
ядра. В квантовой теории нельзя говорить о траекториях электронов и для
наглядной иллюстрации положения электрона в аотме вводят понятие электронного
облака: плотность распределения этого электронного облака пропорциональна
плотности верояности нахождения электрона в данном месте пространства.
Главное квантовое число определяет размер облака, орбитальное число определяет
форму электронного облака, а магнитное число определяет ориентацию облака в
пространстве.

По Бору \( L_{min} = \hbar \), а по квантовой теории \( L_{min} = 0 \).

\subsection{Уровни и спектры щелочных металлов}
[рисунок 5]
Приближённо можно считать, что если атом щелочного металла имеет \( Z \)
электронов, то \( Z-1 \) электрон вместе с ядром образуют сравнительно прочный
остов, в поле которого движется внешний электрон, слабо связанный с остовом.
При этом внешний электрон несколько деформирует электронный остов и тем самым
искажает поле, в котором он движется. В первом приближении поле остова можно
рассматривать как суперпозицию поля точечного заряда \( +e \) и поля точечного
диполя, расположенных в центре остова. Чтобы задача была сферически симметричной,
накладывается ограничение: ось диполя всегда направлена к внешнему электрону.
В этом случае поле будет сферически симметричным:
\[
    U = \frac{A}{r^2} + \frac{B}{r},\ A,B < 0.
\]

Решая уравнение Шрёдингера для такой потенциальной ямы приходим к тому, что
энергия электрона в атоме щелочного металла будет зависеть как от главного
числа, так и от орбитального:
\[
    E = -\frac{\hbar R}{(n + \sigma_i)^2},
\]
где \(\sigma_i\) -- Ридберговская поправка, зависящая от орбитального числа
\( l \). Зависимость \( E(l) \) является принципиальным отличием уровней энергии
атомов щелочных металлов от уровней энергии атома водорода. То есть у щелочных
металлов снимается вырождение по квантовому числу l. Схема уровней для щелочного
металла имеет вид: [рисунок 6].

В силу принципа Паули уровень 1s занят двумя электронами, поэтому невозбуждённое
состояние лития -- это 2s. Момент импульса остова щелочного металла равен 0.
Следовательно, момент импульса атома щелочного металла равен моменту импульса
его внешнего электрона, а значит именно его внешний электрон отвечает за
образование спектра. Правило отбора для переходов в атомах щелочных металлов
всё то же: \( \Delta l = \pm1 \).

Ещё одна особенность спектров в щелочных металлов заключается в том, что их
линии являются двойными (дуплетами), то есть линии щелочных металлов образуют
тонкую структуру.

Спектральные линии, состоящие из нескольких компонент, называются мультиплетами
(1-синглет, 2-дуплет, 3-триплет, 4-квартет). Уравнение Шрёдингера не объясняет
мультиплетность. Учёт релятивистских свойств частиц был произведён Дираком.
Решение уравнения Дирака дало объяснение мультиплетности. Из уравнения Дирака
вытекает наличие у электрона собственного механического момента, названного
спином.