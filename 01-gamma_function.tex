\section{Теория гамма-функций}

\subsection{Определение и основное свойство гамма-функций}

\emph{Гамма-функцией} или \emph{эйлеровским интегралом второго рода} называется
функция \( \Gamma(s) \) вида (\ref{eq:gamma_int_def})
\begin{equation}
    \Gamma(s) = \int\lni e^{-x}x^{s-1}\,dx.
    \label{eq:gamma_int_def}
\end{equation}

Это определение пригодно только если интеграл в правой части имеет смысл, то
есть только если \( \Re s > 0 \).

Гамма-функция относится к разряду аналитических функций, то есть имеет
производную, которую можно найти применяя к (\ref{eq:gamma_int_def}) правило
дифференцирования по параметру:
\begin{equation}
    \Gamma'(s) = \int\lni e^{-x}x^{s-1}\ln(x)\,dx.
    \label{eq:gamma_derivation}
\end{equation}

К \( \Gamma \)-функции приводятся многие интегралы; один из которых можно
получить из (\ref{eq:gamma_int_def}) путем замены переменной:
\[ 
    x = nt,\text{ где } n = const, \Rightarrow dx = n\,dt,
    x^{s-1} = n^{s-1}t^{s-1}.
\]
Тогда (\ref{eq:gamma_int_def}) принимает вид:
\[
    \Gamma(s) = \int\lni e^{-nt}n^{s-1}t^{s-1}n\,dt =
    n^s\int\lni e^{-nt}t^{s-1}\,dt.
\]

Разделив последнее выражение на \( n^s \), получим:
\begin{equation}
    \frac{\Gamma(s)}{n^s} = \int\lni e^{-nt}t^{s-1}\,dt.
    \label{eq:gamma_for_applications}
\end{equation}
 
На выражении (\ref{eq:gamma_for_applications}) основаны многие приложения
\( \Gamma \)-функции.

Получим из (\ref{eq:gamma_for_applications})  основное свойство
\( \Gamma \)-функции. Для этого возьмем производную по параметру \( n \) от
обеих частей уравнения. В результате получим выражение
\[
    \frac{s\Gamma(s)}{n^{s+1}} = \int\lni e^{-nt}t^s\,dt.
\]
Считая, что \( n = 1 \), получим:
\[
    s\Gamma(s) = \int\lni e^{-t}t^s\,dt.
\]
Сделаем замену переменной \( t \) на \( x \):
\[
    s\Gamma(s) = \int\lni e^{-x}x^s\,dx = \int\lni e^{-x}x^{(s+1)-1}\,dx.
\]
Сравнивая полученные выражения с (\ref{eq:gamma_int_def}), видим, что правая
часть последнего выражения представляет собой \( \Gamma \)-функцию от аргумента
\( s + 1 \). Это свойство называется основным свойством \( \Gamma \)-функций:
\begin{equation}
    s\Gamma(s) = \Gamma(s+1).
    \label{eq:gamma_main_property}
\end{equation}
Из него следует, что:
\[
    \Gamma(1) = 1;\ \Gamma(2) = 1;\ \Gamma(3) = 1\cdot2; \ 
    \Gamma(4) = 1\cdot2\cdot3;\ \ldots;\ \Gamma(n) = (n - 1)!.
\]
Эти соотношения справедливы для целого, вещественного и положительного аргумента.

Для полуцелого аргумента можно найти значение \( \Gamma \)-функции с помощью
интеграла Эйлера-Пуассона:
\[
    \int\lni e^{-x^2}\,dx = \sqrt{\pi}/2.
\]
Считая \( x = t^{1/2} \) и умножая на 2:
\[
    \int\lni e^{-t}t^{1/2-1}\,dt = \sqrt{\pi}.
\]
Учитывая определение \( \Gamma \)-функции (\ref{eq:gamma_int_def}):
\[
    \Gamma\left(\frac{1}{2}\right) = \sqrt{\pi}.
\]
Используя последнее выражение и (\ref{eq:gamma_main_property}) можно найти
значение \( \Gamma \)-функции для любого полуцелого аргумента.

В общем случае справедливо равенство (\ref{eq:gamma_for_half_integer}):
\begin{equation}
    \Gamma\left(n + \frac{1}{2}\right) =
    \frac{1\cdot3\cdot\ldots\cdot(2n - 1)}{2^n}\sqrt{\pi}.
    \label{eq:gamma_for_half_integer}
\end{equation}

\begin{remark}
Формула (\ref{eq:gamma_main_property}) является функциональным уравнением
\( \Gamma \)-функции. Получим из него еще одно важное следствие: возьмем
натуральный логарифм от обеих частей (\ref{eq:gamma_main_property}) и
продифференцируем полученное выражение:
\[
    \ln{s} + \ln{\Gamma(s)} = \ln{\Gamma(s+1)},
\]
\begin{equation}
    \frac{\Gamma'(s+1)}{\Gamma(s+1)} =
    \frac{\Gamma'(s)}{\Gamma(s)} + \frac{1}{s}.
    \label{eq:gamma_main_property_ln_derivation}
\end{equation}
Придавая \( s \) ряд значений: 1, 2, \ldots, \( n \), найдем значения
(\ref{eq:gamma_main_property_ln_derivation}):
\[
    \frac{\Gamma'(2)}{\Gamma(2)} = \frac{\Gamma'{1}}{\Gamma(1)} + 1;\ 
    \frac{\Gamma'(3)}{\Gamma(3)} = \frac{\Gamma'{2}}{\Gamma(2)} + \frac{1}{2};\ 
    \ldots;\ 
    \frac{\Gamma'(n+1)}{\Gamma(n+1)} =
    \frac{\Gamma'{n}}{\Gamma(n)} + \frac{1}{n}.
\]
Складывая, получим выражение (\ref{eq:gamma_ln_derivation_series}):
\begin{equation}
    \frac{\Gamma'(n)}{\Gamma(n)} =
    \frac{\Gamma'{1}}{\Gamma(1)} + 1 + \frac{1}{2} + \ldots + \frac{1}{n-1}.
    \label{eq:gamma_ln_derivation_series}
\end{equation}
\end{remark}

\subsection{Бета-функция}

Через \( \Gamma \)-функцию выражаются многие определенные интегралы. Один из
простейших называется \( B \)-функцией:
\begin{equation}
    B(p, q) = \int\limits_0^1 x^{p -1} (1-x)^{q-1}\,dx,
    \label{eq:beta_int_def}
\end{equation}
в которой если \( p \), \( q \) -- вещественные,  то \( p > 0 \), а \( q < 0 \);
а если \( p \) и \( q \) -- комплексные, то вещественная часть обоих чисел
больше нуля: \( \Re(p) > 0 \), \( \Re(q) > 0 \).

Выразим \( B \)-функцию через \( \Gamma \)-функции. Для этого рассмотрим две
\( \Gamma \)-функции:
\begin{align*}
    &  \Gamma(p) = \int\lni e^{-x}x^{p-1}\,dx; \\
    &  \Gamma(q) = \int\lni e^{-y}y^{q-1}\,dy.
\end{align*}
Выполним замену переменных: \( x = \zeta^2 \), \( y = \eta^2 \). Тогда:
\begin{align*}
    &  \Gamma(p) = \int\lni e^{-\zeta^2}\zeta^{2p-1}\,dx; \\
    &  \Gamma(q) = \int\lni e^{-\eta^2}\eta^{2q-1}\,dy.
\end{align*}
Перемножим эти равенства:
\[
    \Gamma(p)\Gamma(q) = 4\int\lni e^{-\zeta^2}\zeta^{2p-1}\ d\zeta
    \int\lni e^{-\eta^2}\eta^{2q-1}\ d\eta.
\]
Заменим произведение интегралов на двойной интеграл:
\[
    \Gamma(p)\Gamma(q) = 4\int\lni\int\lni e^{-(\zeta^2 + \eta^2)}
    \zeta^{2p-1}\eta^{2q-1}\ d\zeta\ d\eta.
\]
Введем полярные координаты
\[
    \zeta = r\cos\phi,\ \eta = r\sin\phi,\ \zeta^2 + \eta^2 = r^2,
\]
и, используя их, получим:
\begin{equation}
    \Gamma(p)\Gamma(q) = 4\int\limits_0^{\pi/2}\int\lni e^{-r^2} r^{2p+2q-1}
    \cos^{2p-1}\phi\sin^{2q-1}\phi\,dr\ d\phi.
    \label{eq:gamma_multiplication_to_beta}
\end{equation}
Вычислим сначала интеграл по \( r \)
\[
    \int\lni e^{-r^2}r^{2p+2q-1}\,dr.
\]
Сделаем замену переменной: \( r = u^{1/2} \). Тогда интеграл будет \( \Gamma \)-
функцией от суммы \( p \) и \( q \):
\[
    \frac{1}{2} \int\lni e^{-u}u^{p+q-1}\,du = \frac{1}{2}\Gamma(p + q),
\]
и (\ref{eq:gamma_multiplication_to_beta}) принимает вид:
\[
    \Gamma(p)\Gamma(q) = 2\Gamma(p + q)\int\limits_0^{\pi/2}
    \cos^{2p-1}\phi\sin^{2q-1}\phi\ d\phi.
\]

Обозначим \( \cos\phi = x^{1/2} \) и \( \sin\phi = (1-x)^{1/2} \). Тогда:
\[
    \Gamma(p)\Gamma(q) = 2\Gamma(p + q)\int\limits_0^1 x^{p-1/2} (1-x)^{q-1/2}
    \frac{\frac{1}{2}x^{1/2-1}\,dx}{(1-x)^{1/2}}.
\]

Из этой формулы получим выражение:
\[
    \int\limits_0^1 x^{p-1}(1-x)^{q-1}\,dx =
    \frac{\Gamma(p)\Gamma(q)}{\Gamma(p + q)},
\]
то есть эйлеровский интеграл первого рода или \( B \)-функция выражается через
\( \Gamma \)-функцию или эйлеровский интеграл второго рода по формуле
(\ref{eq:beta_by_gamma}):
\begin{equation}
    B(p, q) = \frac{\Gamma(p)\Gamma(q)}{\Gamma(p + q)}.
    \label{eq:beta_by_gamma}
\end{equation}

\subsection{Значение гамма-функции при любом комплексном значении аргумента}

Значение \( \Gamma \)-функции в плоскости комплексной переменной должно быть
найдено путём аналитического продолжения функции, определённой на вещественной
оси.  Чтобы  получить  это  продолжение,  заметим предварительно, что если
выполнить интегрирование по частям:
\[
    \Gamma(z) = \left.\frac{1}{z}e^{-t}t^z\right|_0^\infty +
    \frac{1}{z}\int\lni e^{-t}t^z\,dt = \frac{1}{z}\int\lni e^{-t}t^z\,dt,
\]
то отсюда следует функциональное соотношение:
\[
    \Gamma(z) = \frac{\Gamma(z+1)}{z} = \frac{\Gamma(z+2)}{z(z+1)} =
    \frac{\Gamma(z+n+1)}{z(z+1)\cdot\ldots\cdot(z+n)},
\]
в предположении, что \( \Re(z) > 0 \).

Рассмотрим теперь функцию комплексной переменной \( f(z) \), которая для любых
значений \( z \), принадлежащих области \( \Re(z) > -(n + 1) \),
\( z \ne 0, -1, -2, \ldots \), определяется при помощи равенства:
\[
    f(z) = \frac{\Gamma(z+n+1)}{z(z+1)\cdot\ldots\cdot(z+n)},
\]
где \( \Gamma(z) \) обозначает \( \Gamma \)-функцию, представляемую интегралом
\( \int\lni e^{-t}t^{z-1}\,dt \).

Функция \( f(z) \) регулярна в рассматриваемой области и ее значение при
\( \Re(z) > 0 \) совпадает с \( \Gamma(z) \). Поэтому \( f(z) \) есть искомое
аналитическое продолжение \( \Gamma(z) \) в область
\( -(n + 1) < \Re(z) \leq 0 \), \( z \ne 0, -1, -2, \ldots \)  и  мы  можем
определить \( \Gamma \)-функцию в этой области посредством формулы:
\begin{equation}
   \Gamma(z) \equiv f(z) = \frac{\Gamma(z+n+1)}{z(z+1)\cdot\ldots\cdot(z+n)}.
   \label{eq:gamma_on_complex_plane}
\end{equation}

Так как число \( n \) может быть выбрано произвольно большим, то
\( \Gamma \)-функция  определена  на  всей  плоскости  комплексной  переменной,
 исключая точки  \( z = 0, -1, -2, \ldots \). Последние  точки  являются
 полюсами  рассматриваемой функции.  Действительно,  в  окрестности  точки
 \( z = -n \) \( \Gamma(z) \) определяется формулой
 (\ref{eq:gamma_on_complex_plane}), откуда следует, что:
\[
    \lim_{z\to -n} \Gamma(z)(z+n) = (-1)^n\frac{\Gamma(1)}{n!} =
    \frac{(-1)^n}{n!}.
\]

Таким образом, \( \Gamma(z) \) есть мероморфная функция комплексной переменной
(определена на всей комплексной плоскости и не имеет в конечной части плоскости
особых точек, отличных от полюсов) с простыми полюсами (изолированная точка
\( z_0 \)  называется полюсом \( f(z) \), если в разложении этой функции в ряд
Лорана в проколотой окрестности точки \( z_0 \)  главная часть содержит конечное
число отличных от нуля членов) в точках \( z = 0, -1, -2, \ldots \).

\subsection{Представление \( \Gamma \)-функции в виде бесконечного произведения}

Эйлер дал представление \( \Gamma(s) \) в виде предела произведения бесконечно
возрастающего числа множителей. Это представление может быть получено исходя из
прежнего определения \( \Gamma \)-функции:
\begin{equation}
    \Gamma(s) = \int\lni e^{-x}x^{s-1}\,dx = 
    \limni\int\limits_0^n e^{-x}x^{s-1}\,dx.
    \label{eq1.4.1}
\end{equation}

С помощью известного равенства
\[
    e^{-x} = \limni\left(1-\frac{x}{n}\right)^n
\]
перепишем (\ref{eq1.4.1}) в виде (\ref{eq1.4.2}):
\begin{equation}
    \Gamma(s) = \limni\int\limits_0^n\left(1-\frac{x}{n}\right)^n
    x^{s-1}\,dx.
    \label{eq1.4.2}
\end{equation}
 
Интегрируя по частям, обозначая \( \left(1-\frac{x}{n}\right)^n = u \),
\( x^{s-1}\,dx = dv \), получим:
\[
    \left.\left(1 - \frac{x}{n}\right)^n\frac{x^s}{s}\right|_0^n +
    \frac{n}{sn}\int\limits_0^n \left(1 - \frac{x}{n}\right)^{n-1}x^s\,dx =
    \frac{n}{sn}\int\limits_0^n \left(1-\frac{x}{n}\right)^{n-1} x^s\,dx.
\]
 \[
    \int\limits_0^n \left(1 - \frac{x}{n}\right)^nx^{s-1}\,dx =
    \frac{n}{sn}\int\limits_0^n \left(1 - \frac{x}{n}\right)^{n-1}x^s\,dx =
    \frac{n(n-1)}{s(s+1)n^2} \int\limits_0^n \left(1 - \frac{x}{n}\right)^{n-2}
    x^{s+1}\,dx,
\]
 
интегрируя \( n \) раз, получим выражение (\ref{eq1.4.3}):
 \begin{equation}
    \frac{1\cdot2\cdot3\cdot\ldots\cdot n}{s(s+1)\cdot\ldots\cdot(s+n-1)}
    \cdot\frac{1}{n^n}\cdot\frac{n^{s+n}}{s+n} =
    \frac{n!n^s}{s(s+1)\cdot\ldots\cdot(s+n)}.
    \label{eq1.4.3}
\end{equation}

Выражение (\ref{eq1.4.3}) является бесконечным произведением Эйлера.

Вейерштрасс придал этому выражению иной вид, исходя из следующего равенства:
\[
    1 + \frac{1}{2} + \frac{1}{3} + \ldots + \frac{1}{n-1} = \ln{n} + c +
    \frac{1}{2n} + \frac{B_1}{2n^2} + (-1)^{k-1}\frac{B_k}{2k}\frac{1}{n^{2k}},
\]
преобразуя которое он получил следующее:
\[
    \limni\left(1+\frac{1}{2}+\frac{1}{3}+\ldots+ \frac{1}{n-1} + \frac{1}{n} -
    \ln{n}\right) = c.
\]

Отсюда можно записать (\ref{eq1.4.4}):
\begin{equation}
    e^{cs} = \limni e^{s + \frac{s}{2} + \frac{s}{3} + \ldots + \frac{s}{n} -
    s\ln{n}} = \limni e^s e^\frac{s}{2} e^\frac{s}{3}\ldots e^\frac{s}{n}n^{-s}.
    \label{eq1.4.4}
\end{equation}

Перемножая (\ref{eq1.4.3}) и (\ref{eq1.4.4}) получим:
\[
    e^{cs}\Gamma(s) = \limni \frac{1}{s}\frac{1\cdot e^s}{s+1}
    \frac{2\cdot e^\frac{s}{2}}{s+2}\cdot\ldots\cdot\frac{ne^\frac{s}{n}}{s+n}.
\]

Тогда:
\[
    e^{cs}\limni s(s+1)e^{-s} \left(1+\frac{s}{2}\right)e^{-s}{2}
    \left(1+\frac{s}{3}\right)e^{-s}{3}\cdot\ldots\cdot
    \left(1+\frac{s}{n}\right)e^{-s}{n}.
\]

Применяя для однотипных множителей символ \( \prod \) из последнего равенства
получаем бесконечное произведение для всех \( \Gamma \)-функций в форме
Вейерштрасса:
\begin{equation}
    \frac{1}{\Gamma(s)} =
    se^{cs}\prod\limits_{k=1}^\infty\left(1+\frac{s}{k}\right)e^{-\frac{s}{k}}.
    \label{eq1.4.5}
\end{equation}

Формулы (\ref{eq1.4.3}) и (\ref{eq1.4.5}) имеют смысл при любых значениях
\( s \) -- как при вещественных, так и комплексных. Они дают для \( \Gamma(s) \)
определенное конечное значение всегда, когда \( s \ne 0,\ -1,\ -2,\ \ldots \).
Если \( s \) равно одному из этих чисел, то \( \Gamma(s) = \infty \).

Формула (\ref{eq1.4.5}) показывает, что функция \( 1/\Gamma(s) \) является
трансцендентной функцией, подобной \( e^x,\ \cos x,\ \sin x \).

Единственные корни этой функции -- \( 0 \), \( -1 \), \( -2 \), \ldots. Функция
\( \Gamma(s) \) корней не имеет.

Взяв логарифмическую производную от формулы (\ref{eq1.4.5}) и изменив знаки на
противоположные, получим (\ref{eq1.4.6}):
\begin{equation}
   \frac{\Gamma'(s)}{\Gamma(s)} = -\frac{1}{s} - c + \sum\limits_{k=1}^\infty
   \left(\frac{1}{k} - \frac{1}{s+k}\right).
   \label{eq1.4.6}
\end{equation}

Считая \( s = 1\):
\[
    \frac{\Gamma'(1)}{\Gamma(1)} = -1 - c + \left(1 - \frac{1}{2}\right) +
    \left(\frac{1}{2} - \frac{1}{3}\right) + \ldots = -c.
\]

\subsection{Второе свойство \( \Gamma \)-функций}\

Из формулы (\ref{eq1.4.4}) получим:
\[
    \frac{\Gamma'(s)}{\Gamma(s)} = -\frac{1}{s} - C + \sum\limits_{n=1}^\infty
    \left(\frac{1}{n} - \frac{1}{n+s}\right),
\]
\[
    \frac{\Gamma'(-s)}{\Gamma(-s)} = \frac{1}{s} - C + \sum\limits_{n=1}^\infty
    \left(\frac{1}{n} - \frac{1}{n-s}\right).
\]
Вычтем первое равенство из второго:
\[
    \frac{\Gamma'(-s)}{\Gamma(-s)} - \frac{\Gamma'(s)}{\Gamma(s)} =
    \frac{2}{s} + \sum\limits_{n=1}^\infty \left(\frac{1}{n+s} -
    \frac{1}{n-s}\right) = 
    \frac{2}{s} + \sum\limits_{n=1}^\infty \frac{2s}{s^2 - n^2}.
\]

Используя разложение \( \ctg x \) на простейшие дроби
\[
    \ctg x = \frac{1}{x} + \sum\limits_{n=1}^\infty \frac{2x}{x^2 - \pi^2n^2},
\]
получаем
\[
    \frac{\Gamma'(-s)}{\Gamma(-s)} - \frac{\Gamma'(s)}{\Gamma(s)} =
    \frac{1}{s} + \pi\ctg\pi s.
\]
Интегрируя это выражение, получим
\[
    -\ln\Gamma(s) - \ln\Gamma(-s) = \ln s + \ln\sin(\pi s) - \ln C.
\]

Используя основное свойство \( \Gamma \)-функции для аргумента \( -s \)
\[
    -s\Gamma(-s) = \Gamma(1-s),
\]
получим
\[
    \Gamma(s)\Gamma(1-s) = \frac{C}{\sin(\pi s)}.
\]
Определим константу \( C \), считая \( s = 1/2 \):
\[
    C = \Gamma^2\left(\frac{1}{2}\right) = \pi.
\]

Таким образом, второе свойство \( \Gamma \)-функций:
\begin{equation}
   \Gamma(s)\Gamma(1-s) = \frac{\pi}{\sin(\pi s)}.
   \label{eq1.5.1}
\end{equation}

\subsection{Третье свойство \( \Gamma \)-функций}

Исходя из формулы (\ref{eq1.6.1})
\begin{equation}
    \Gamma(s) =
    \limni\frac{1\cdot2\cdot3\cdot\ldots\cdot n}{s(s+1)\ldots(s+n)} n^s
    \label{eq1.6.1}
\end{equation}
получим еще одно свойство \( \Gamma \)-функций, заменяя в этой формуле \( s \)
на \( s + 1/2 \):
\begin{equation}
    \Gamma\left(s+\frac{1}{2}\right) =
    \limni\frac{1\cdot2\cdot3\cdot\ldots\cdot n}{
    \left(s + \frac{1}{2}\right) \left(s + \frac{3}{2}\right) \ldots
    \left(s + n + \frac{1}{2}\right)} n^{s+\frac{1}{2}}.
    \label{eq1.6.2}
\end{equation}

Заменяя в (\ref{eq1.6.1}) \( s \) на \( 2s \), \( n \) на \( 2n \), получим
(\ref{eq1.6.3}):
\begin{equation}
    \Gamma(2s) =
    \limni\frac{1\cdot2\cdot\ldots\cdot2n}{2s(2s+1)\ldots(2s+2n)}(2n)^{2s}.
    \label{eq1.6.3}
\end{equation}

Перемножая (\ref{eq1.6.1}) и (\ref{eq1.6.2}) и деля их на (\ref{eq1.6.3}),
получим следующее выражение:
\[
    \frac{\Gamma(s)\Gamma(s+1/2)}{\Gamma(2s)} = \limni\left[
    \frac{(1\cdot2\cdot3\cdot\ldots\cdot n)^2}{1\cdot2\cdot3\cdot\ldots\cdot2n} 
    \frac{\sqrt{n}}{2^{2s}}
    \frac{2s(2s+1)\ldots(2s+2n)}{s(s+1/2)(s+1)\ldots(s+n)(s+n+1/2)}\right].
\]

Умножим обе части на \( 2^{2s} \) и, сокращая, получим:
\[
    \frac{\Gamma(s)\Gamma(s+1/2)}{\Gamma(2s)}2^{2s} = 
    \limni\frac{(1\cdot2\cdot3\cdot\ldots\cdot n)^2 \sqrt{n}}
    {1\cdot2\cdot3\cdot\ldots\cdot2n}\cdot\frac{2^{2n+1}}{s+n+1/2}.
\]

Сделав замену \( \sqrt{n}/(s+n+1/2) = 1/\sqrt{n} \), получим (\ref{eq1.6.4}):
\begin{equation}
    \frac{\Gamma(s)\Gamma(s+1/2)}{\Gamma(2s)} 2^{2s} =
    \limni\frac{(1\cdot2\cdot\ldots\cdot n)^2}{1\cdot2\cdot3\cdot\ldots\cdot2n} 
    \frac{2^{2n+1}}{\sqrt{n}}.
    \label{eq1.6.4}
\end{equation}

Считая \( s = 1/2 \), получим: \( \Gamma(s) = \sqrt{\pi} \),
\( \Gamma(s + 1/2) = \Gamma(1) = 1 \), \( \Gamma(2s) = 1 \). Тогда:
\begin{equation}
    2\sqrt{\pi} = \limni\frac{(1\cdot2\cdot3\cdot\ldots\cdot n)^2}
    {1\cdot2\cdot3\cdot\ldots\cdot2n}\cdot\frac{2^{2n+1}}{\sqrt{n}}.
    \label{eq1.6.5}
\end{equation}

Сравнивания (\ref{eq1.6.4}) и (\ref{eq1.6.5}) находим, что левая часть
(\ref{eq1.6.4}) равна \( 2\sqrt{\pi} \):
\[
    \frac{\Gamma(s)\Gamma(s+1/2)}{\Gamma(2s)}2^{2s} = 2\sqrt{\pi}.
\]

Отсюда получим (\ref{eq1.6.6}):
\begin{equation}
    \Gamma(s)\Gamma(s+1/2) = 2^{1-2s}\sqrt{\pi}\Gamma(2s).
    \label{eq1.6.6}
\end{equation}

Выражение (\ref{eq1.6.6}) определяет третье основное свойство
\( \Gamma \)-функций. Эта формула была получена Лежандром. Она представляет
частный случай формулы, полученной впоследствие Гауссом:
\begin{equation}
    \Gamma(s)\Gamma(s+1/2)\cdot\ldots\cdot\Gamma\left(s + \frac{n-1}{n}\right) =
    n^{1/2-ns}\cdot(2\pi)^{1/2\cdot(n-1)}\Gamma(ns).
    \label{eq1.6.7}
\end{equation}

Формула (\ref{eq1.6.7}) получается аналогично (\ref{eq1.6.6}).

\subsection{Представление логарифма гамма-функции в виде определенного интеграла}

Выражение (\ref{eq1.4.6}) можно записать в виде (\ref{eq1.7.1}):
\begin{equation}
    \frac{\Gamma'(s)}{\Gamma(s)} = -C +
    \sum\limits_{k=1}^\infty\left(\frac{1}{k} - \frac{1}{s+k-1}\right).
    \label{eq1.7.1}
\end{equation}

Формула (\ref{eq1.7.1}) может быть представлена в виде (\ref{eq1.7.2}):
\begin{equation}
    \frac{\Gamma'(s)}{\Gamma(s)} = -C +
    \limni\sum\limits_{k=1}^n \left(\frac{1}{k} - \frac{1}{s+k-1}\right).
    \label{eq1.7.2}
\end{equation}

Величины \( C \), \( \frac{1}{k} \) и \( \frac{1}{s+k-1} \) можно выразить
через определенные интегралы:
\[
    \frac{1}{k} = \int\lni e^{-k\tau}\ d\tau;\ 
    \frac{1}{s+k-1} = \int\lni e^{-(s+k-1)\tau}\ d\tau;\ 
    C =
    \int\lni e^{-\tau}\left(\frac{1}{1-e^{-\tau}}-\frac{1}{\tau}\right)\ d\tau.
\]

Будем считать, что \( \Re(s) > 0 \), тогда интеграл \( \frac{1}{s+k-1} \) будет
сходящимся при всех \( k \), равных положительному целому числу.

Подставляя это в (\ref{eq1.7.2}), получим:
\[
    \frac{\Gamma'(s)}{\Gamma(s)} =
    -\int\lni e^{-\tau}\left(\frac{1}{1-e^{-\tau}} -
    \frac{1}{\tau}\right)\ d\tau + \limni\sum\limits_{k=1}^n \int\lni
    \left[ e^{-k\tau} - e^{-(s+k-1)\tau} \right]\ d\tau.
\]

Меняя местами знаки суммы и интеграла, получим:
\begin{align*}
    & \frac{\Gamma'(s)}{\Gamma(s)} =
    -\int\lni e^{-\tau}\left(\frac{1}{1-e^{-\tau}} -
    \frac{1}{\tau}\right)\ d\tau +
    \limni\int\lni \left[ \frac{e^{-\tau} - e^{-(n+1)\tau}}{1-e^{-\tau}} -
    \frac{e^{-s\tau} - e^{-(n+s)\tau}}{1-e^{-\tau}} \right]\ d\tau = \\
    & = -\int\lni e^{-\tau}\left(\frac{1}{1-e^{-\tau}} -
    \frac{1}{\tau}\right)\ d\tau +
    \int\lni\frac{e^{-\tau} - e^{-s\tau}}{1-e^{-\tau}}\ d\tau +
    \limni\int\limits_0^ne^{-n\tau}
    \frac{e^{-\tau}-e^{-s\tau}}{1-e^{-\tau}}\ d\tau.
\end{align*}

Третье слагаемое стремится к нулю. Приведя подобные в первых двух интегралах,
получим (\ref{eq1.7.3}):
\begin{equation}
    \frac{\Gamma'(s)}{\Gamma(s)} = \int\lni \left(\frac{e^{-\tau}}{\tau} -
    \frac{e^{-s\tau}}{1-e^{-\tau}}\right)\ d\tau.
    \label{eq1.7.3}
\end{equation}

Прибавим к (\ref{eq1.7.3}) почленно два равенства (\ref{eq1.7.4}) и
(\ref{eq1.7.5})\\
\begin{minipage}{0.45\textwidth}
    \centering
    \begin{equation}
        -\ln s = -\int\lni\left(\frac{e^{-\tau}-e^{-s\tau}}{\tau}\right)\ d\tau,
        \label{eq1.7.4}
    \end{equation}        
\end{minipage}
\hfill
\begin{minipage}{0.45\textwidth}
    \centering
    \begin{equation}
        \frac{1}{2s} = \int\lni \frac{e^{-s\tau}}{2}\ d\tau.
        \label{eq1.7.5}
    \end{equation}        
\end{minipage}

Получим (\ref{eq1.7.6}).
\begin{equation}
    \frac{\Gamma'(s)}{\Gamma(s)} - \ln s + \frac{1}{2s} = -\int\lni e^{-s\tau}
    \left(\frac{1}{1-e^{-\tau}} - \frac{1}{\tau} - \frac{1}{2}\right)\ d\tau.
    \label{eq1.7.6}
\end{equation}

Перенесем слагаемое \( -\ln s + 1/(2s) \) в правую часть и проинтегрируем обе
части по \( s \) в пределах \( [0, s] \). Получим (\ref{eq1.7.7}):
\begin{equation}
    \ln \Gamma(s) = s\ln s - s - \frac{1}{2}\ln s + A + \int\lni e^{-s\tau}
    \left(\frac{1}{1-e^{-\tau}} - \frac{1}{\tau} - \frac{1}{2}\right)\ 
    \frac{d\tau}{\tau},
    \label{eq1.7.7}
\end{equation}
\[
    \text{где } A = \int\lni e^{-\tau} \left(\frac{1}{1-e^{-\tau}} -
    \frac{1}{\tau} - \frac{1}{2}\right)\ \frac{d\tau}{\tau} = \ln\sqrt{2\pi}.
\]

\subsection{Формула Стирлинга}

Формула (\ref{eq1.7.7}) дает возможность получить формулу Стирлинга, важную для
вычисления \( \ln\Gamma(s) \) и \( \Gamma(s) \). Формула Стирлинга полезна,
когда \( \Re(s) > 0 \), а \( s \) достаточно велико по абсолютному значению.
Кроме того, данная формула полезна и в остальных случаях, так как соотношение
\( \Gamma(s) = \Gamma(s+1)/s \) позволяет сводить вычисление \( \Gamma(s) \) к
вычислению \( \Gamma(s+n) \), где \( n \) -- целое число, настолько больше, что
можно использовать формулу Стирлинга для \( \ln \Gamma(s+n) \).

Формула Стирлинга имеет вид:
\begin{align*}
    & \ln\Gamma(s) = s\ln s - s - \frac{1}{2}\ln s + \ln\sqrt{2\pi} + 
    \frac{B_1}{1\cdot2\cdot s} - \frac{B_2}{3\cdot4\cdot s^3} + \ldots \\
    & \ldots + (-1)^{n-2}\frac{B_{n-1}}{(2n-3)(2n-2)s^{2n-3}} + (-1)^{n-1}
    \theta\frac{B_n}{(2n-1)2n\xi^{2n-1}}.
\end{align*}

Если \( s > 0 \), то \( \xi = s \). Коэффициенты \( B_1, B_2, \ldots, B_n \) --
некоторые константы. \( \theta \) -- некоторая функция \( \tau \). Получить ее
точное значение невозможно, но можно оценить ее из выражения:
\[
    \left| \int\lni e^{-s\tau}\tau^{2n-2}\theta\ d\tau \right| \le
    \int\lni e^{-\xi\tau}\tau^{2n-2}\ d\tau = \frac{(2n-2)!}{\xi^{2n-1}}.
\]

Формула Стирлинга, в отличие от сходящихся рядов, не дает возможность вычислить
\( \ln\Gamma(s) \) со сколь угодно большой точностью, а только с точностью до
величины последнего члена из числа взятых слагаемых.

\emph{Пример:}
\[
    \ln99! = 100\ln100 - 100 - \frac{1}{2}\ln100 + \ln\sqrt{2\pi} +
    \frac{B_1}{1\cdot2\cdot100} - \frac{B_2}{3\cdot4\cdot100^3} +
    \theta\frac{B_3}{4\cdot5\cdot100^5},
\]
где \( s = 100 \), \( n = 3 \).

Величина последнего слагаемого точно не известна. Но \( 0 < \theta < 1 \),
поэтому последнее слагаемое по абсолютной величине меньше, чем
\[
    \theta\frac{B_3}{4\cdot5\cdot100^5} < \frac{1}{84\cdot10^{11}}.
\]

???

\[
    S_k = \sum\limits_{n=1}^\infty \frac{1}{\pi^{2k}n^{2k}}; B_k =
    \frac{(2k)!}{2^{2k-1}\pi^{2k}}\sum\limits_{n=1}^\infty\frac{1}{n^{2k}}.
\]

Отсюда: \( B_1 = S_1 \), \( B_2 = \ldots \), \( B_3 = \ldots \),
\( B_n/((n+1)(n+2)100^{2n-1}) \), где \( B_n \) -- числа Бернулли.

???

\subsection{Интеграл Римана-Ханкеля. Интеграл Эйлера}

Интеграл Эйлера (\ref{eq:gamma_int_def}) представляет собой \( \Gamma \)-функцию
только при условии \( \Re(s) > 0 \). Риману принадлежит обобщение формулы
Эйлера, дающее интегральное представление для всех значений \( s \).

Для его получения рассмотрим интеграл от функции \( e^{-x}x^{s-1} \), взятый
по контуру \( ABCDEFA \) (рис. \ref{pic1.9.1}). Т.к. функция \( e^{-x}x^{s-1} \)
может иметь особую точку лишь в начале координат, то внутри контура она
регулярна. Поэтому в силу теоремы Коши интеграл равен нулю.
Запишем это математически:
\begin{equation}
    \int\limits_{ABC} e^{-x}x^{s-1}\,dx = \int\limits_{FED} e^{-x}x^{s-1}\,dx -
    \int\limits_{CD} e^{-x}x^{s-1}\,dx - \int\limits_{FA} e^{-x}x^{s-1}\,dx.
    \label{eq1.9.2}
\end{equation}
Интегралы по контурам \( CD \) и \( FA \) будут стремиться к нулю, т. к. длина
пути интегрирования ограничена. Тогда равенство (\ref{eq1.9.2}) принимает вид
\begin{equation}
    \int\limits_{\gamma_1} e^{-x}x^{s-1}\,dx =
    \int\limits_\gamma e^{-x}x^{s-1}\,dx,
    \label{eq1.9.3}
\end{equation}
где \( \gamma_1 \), \( \gamma \) -- означают контуры, имеющие подобно линии
\( АВС \) форму петли, охватывающей начало координат и уходящую концами в
бесконечность вдоль положительного направления оси абсцисс.

Равенство (\ref{eq1.9.3}) показывает, что для любых контуров величина интеграла
\( e^{-x}x^{s-1} \) постоянна.

Предположим, что \( s \) положительное, нецелое число. В качестве контура
\( \gamma_1 \), возьмем петлю, состоящую из двух горизонталей сверху и снизу,
отстоящих от оси абсцисс на величину \( h \). Эти горизонтали соединены дугой
круга радиусом \( \rho \), \( |x| = \rho \), причем \( \rho > h \). Петля
берется так, чтобы она охватывала начало координат. При \( x = re^{i\phi} \),
будем считать, что \( x^{s-1} = r^{s-1}e^{i\phi(s-1)} \). Будем считать, что
\( \phi = \pi \) при \( x < 0 \). Этими данными задается положение центра
окружности относительно начала координат.

Рассмотрим интеграл, взятый по дуге круга; получим, что его абсолютное значение
не превосходит \( 2\pi\rho\rho^{s-1}e^\rho \). Последнее выражение стремится к
нулю при \( \rho \) стремящимся к нулю. Интеграл по верхней горизонтали при
\( h \) стремящимся к нулю, \( \rho \) стремящимся к нулю и отношении
\( h/\rho \) стремящимся к нулю, превращается в интеграл Эйлера, т.е. в
\( \Gamma \)-функцию.

При этом, в интеграле по нижней горизонтали угол \( \phi \) может быть сколь
угодно близким к \( 2\pi \), т.е. \( \phi\to2\pi \) при \( h\to0 \),
\( \rho\to0\) и \(h/\rho \to 0 \). При этом интеграл по нижней горизонтали будет
стремиться к величине
\[
    -e^{2\pi i(s-1)} \int\lni e^{-x}x^{s-1}\,dx.
\]

Сопоставляя записанные выражения, получим равенство (\ref{eq1.9.4})
\begin{equation}
    \left(1-e^{i\pi s}\right)\int\lni e^{-x}x^{s-1}\,dx =
    \int\limits_\gamma e^{-x}x^{s-1}\,dx.
    \label{eq1.9.4}
\end{equation}

Формула (\ref{eq1.9.4}) является предельным случаем формулы (\ref{eq1.9.3}),
т.к. \( h\to0 \), \( \rho\to0\) и \(h/\rho \to 0 \). Контур \( \gamma \) в
правой части может иметь любой вид, форму и размер. Тогда равенство
(\ref{eq1.9.4}) переходит в (\ref{eq1.9.5}):
\begin{equation}
    \Gamma(s) = \frac{1}{1-e^{2\pi is}} \int\limits_\gamma e^{-x}x^{s-1}\,dx.
    \label{eq1.9.5}
\end{equation}

Интеграл в правой части (\ref{eq1.9.5}) представляет собой аналитическую функцию
аргумента \( s \), голоморфную на всей плоскости переменной \( s \).
Следовательно, (\ref{eq1.9.5}) даёт интегральное представление 
\( \Gamma \)-функции в виде частного двух функций, голоморфных на всей
плоскости. Это представление действительно для всех значений \( s \) и даёт 
аналитическое продолжение \( \Gamma \)-функции на всю плоскость. Отметим, что
при \( s \) равном целому положительному числу, правая часть (\ref{eq1.9.5})
принимает неопределённый вид. Эту неопределённость можно раскрыть по правилу
Лопиталя.
