\section{Уравнение Бесселя и его интегралы}

\subsection{Дифференциальные уравнения Бесселя с дробным индексом (функции Бесселя первого рода)}

    Решение многих задач в физике и математике сводятся к решению дифференциального уравнения вида (\ref{eq2.1.1}):
    \begin{equation}
        x^2u'' + xu' + (x^2-n^2)u = 0.
        \label{eq2.1.1}
    \end{equation}
    
    Дифференциальное уравнение (\ref{eq2.1.1}) называется уравнением Бесселя, а его интегралы -- функциями Бесселя.
    
    Название ``функции Бесселя'' дается некоторым функциям, удовлетворяющим уравнению вида (\ref{eq2.1.1}), а так же уравнениям, сводящимся к уравнению (\ref{eq2.1.1}).
    
    Так как уравнение Бесселя является дифференциальным уравнением второго порядка, то для его интегрирования достаточно знать два частных интеграла \( u_1 \) и \( u_2 \) таких, чтобы они были линейно независимы, то есть чтобы их отношение было отлично от постоянной величины. Тогда общее решение может быть выражено в виде (\ref{eq2.1.2}):
    \begin{equation}
        u = C_1u_1 + C_2u_2.
        \label{eq2.1.2}
    \end{equation}
    
    Учитывая, что интеграл уравнения Бесселя не может быть выражен через элементарные функции, будем искать его решение в виде ряда:
    \[ f(x) = f(0) + f'(0)x + \frac{f''(0)}{2!}x^2 + \frac{f'''(0)}{3!}x^3 + \ldots, \]
    где \( f(x) \) -- интеграл уравнения Бесселя, то есть можно записать:
    \begin{equation}
        u = a_0 + a_1x + a_2x^2 + a_3x^3 + \ldots
        \label{eq2.1.3}
    \end{equation}
    
    Ряд (\ref{eq2.1.3}) может выражать далеко не все функции, то есть в теории степенных рядов доказывается, что в области сходимости ряда (\ref{eq2.1.3}) (напр., \( x = 0 \)) сам ряд и все его производные должны иметь конечные значения. В данном случае можно предполагать, что при \( x = 0 \) сама функция или все ее производные  будут обращаться в бесконечность. Это видно из дифференциального уравнения (\ref{eq2.1.1}):
    \[ u''(x) = -\frac{xu'(x) + (x^2-n^2)u(x)}{x^2}. \]
    
    Из последнего выражения видно, что при \( x = 0 \) значение \( u''(0)\to\infty \), при этом \( u'(0) \) и \( u(0) \) имеют конечные значения и \( u(0) = 0 \) и \( n\ne 0\). Тогда решение уравнения (\ref{eq2.1.1}) будем искать в виде ряда (\ref{eq2.1.4}):
    \begin{equation}
        u(x) = a_0x^\alpha + a_1x^{\alpha+1} + a_2x^{\alpha+2} + \ldots
        \label{eq2.1.4}
    \end{equation}
    
    Ряд (\ref{eq2.1.4}) отличается от обычного ряда величиной \( x^\alpha \), и можно
    показать, что в области его сходимости его можно дифференцировать сколь
    угодно раз. Продифференцируем его дважды:
    \begin{align}
        & u'(x) = \alpha a_0x^{\alpha-1} + (\alpha+1)a_1x^\alpha + \ldots \label{eq2.1.5a} \\
        & u''(x) = (\alpha-1)\alpha a_0x^{\alpha-2} + (\alpha+1)\alpha a_1x^{\alpha-1} + \ldots
        \label{eq2.1.5b}
    \end{align}
    
    Умножим равенство (\ref{eq2.1.4}) на \( x^2-n^2 \), \eqref{eq2.1.5.a} на \( x \) и
    \eqref{eq2.1.5b} на \( x^2 \) и сложим получившиеся произведения, сгруппировав
    слагаемые с одинаковой степенью \( x \). Получим:
    \begin{equation}
        \begin{array}{l}
            x^2 u''(x) + xu'(x) + (x^2-n^2)u(x) = [\alpha(\alpha - 1) + \alpha -
            n^2]a_0x^\alpha + [\alpha(\alpha + 1) + (\alpha + 1) - n^2]a_1
            x^{\alpha + 1} + \\ + [(\alpha + 2)(\alpha + 1)a_2 + (\alpha + 1)a_2
            - n^2a_2 + a_0]x^{\alpha + 2} + [(\alpha + 3)(\alpha + 2)a_3 + (\alpha
            + 3)a_3 + a_1]x^{\alpha + 3} + \ldots
        \end{array} \label{eq2.1.6}
    \end{equation}
    
    Если функция \( u \) удовлетворяет уравнению (\ref{eq2.1.1}), то левая часть
    (\ref{eq2.1.6}) должна равняться нулю при любом \( x \). Следовательно, должна
    равняться нулю и правая часть. Для этого достаточно выбрать коэффициенты
    \( a_0 \), \( a_1 \), \( \ldots \), \( a_n \), \( \alpha \) такими, что
    выполняются неравенства (\ref{eq2.1.7}):
    \begin{equation}
        \left\{
        \begin{array}{l}
            \left[\alpha^2 - n^2\right] a_0 = 0, \\
            \left[(\alpha + 1)^2 - n^2\right]a_1 = 0, \\
            \left[(\alpha + 2)^2 - n^2\right]a_2 + a_0 = 0, \\
            \left[(\alpha + 3)^2 - n^2\right]a_3 + a_1 = 0, \\
            \ldots
        \end{array}
        \right.
        \label{eq2.1.7}
    \end{equation}
    
    Будем считать, что \( a_0 \ne 0 \), тогда из (\ref{eq2.1.7}) следует, что
    \( \alpha^2 - n^2 = 0 \), то есть \( \alpha = \pm n \). Выберем \( n > 0 \).
    Тогда \( \alpha = n \), и, начиная со второго выражения, уравнения (\ref{eq2.1.7})
    принимают вид:
    \[
    \begin{array}{l}
        (2n + 1)a_1 = 0, \\
        (4n + 4)a_2 + a_0 = 0, \\
        (6n + 9)a_3 + a_1 = 0, \\
        \ldots
    \end{array}
    \]
    
    Из этих выражений можно найти, что \( a_1 = a_3 = a_5 = \ldots = 0 \), то
    есть все коэффициенты с нечетными индексами равны нулю, а с четными:
    \[
        a_2 = -\frac{a_0}{4\cdot(n+1)};
        \ a_4 = -\frac{a_2}{8\cdot(n+2)} =
        \frac{a_0}{4\cdot8\cdot(n+1)(n+2)}; \ \ldots.
    \]
    
    Подставляя значения коэффициентов \( a_i \) в (\ref{eq2.1.4}) получим (\ref{eq2.1.8}):
    \begin{equation}
        u = a_0x^n\left[1 - \frac{x^2}{4\cdot1\cdot(n+1)} +
        \frac{x^4}{4^2\cdot1\cdot2\cdot(n+1)(n+2)} - \ldots + (-1)^k\frac{n!x^{2k}}
        {4^kk!(n+k)!} + \ldots \right].
        \label{eq2.1.8}
    \end{equation}
    
    Исследуя сходимость данного ряда, находим, что ряд сходится при любых \( x \). Следовательно, функция \( u \) вида (\ref{eq2.1.8}) удовлетворяет уравнению Бесселя при любом значении \( a_0 \). Обычно значение \( a_0 \) выбирают так:
    \[ a_0 = \frac{1}{2^n\Gamma(n + 1)}. \]
    
    Тогда функция (\ref{eq2.1.8}) превращается в функцию Бесселя (\ref{eq2.1.9}):
    \begin{equation}
        J_n(x) = \frac{x^n}{2^n\Gamma(n+1)}\left[1 - \frac{\left(\frac{x}{2}\right)^2}{1!(n+1)} + \frac{\left(\frac{x}{2}\right)^4}{2!(n+1)(n+2)} - \ldots\right].
        \label{eq2.1.9}
    \end{equation}
    
    Раскрыв скобки и воспользовавшись основным свойством гамма-функций, получим (\ref{eq2.1.10}):
    \begin{equation}
        J_n(x) = \frac{\left(\frac{x}{2}\right)^n}{\Gamma(n+1)} - \frac{\left(\frac{x}{2}\right)^{n+2}}{1!\Gamma(n+2)} + \frac{\left(\frac{x}{2}\right)^{n+4}}{2!\Gamma(n+3)} - \ldots
        \label{eq2.1.10}
    \end{equation}
    
    В случае, если \( \alpha = -n \):
    \begin{equation}
        J_{-n}(x) = \frac{\left(\frac{x}{2}\right)^{-n}}{\Gamma(-n+1)} - \frac{\left(\frac{x}{2}\right)^{-n+2}}{1!\Gamma(-n+2)} + \frac{\left(\frac{x}{2}\right)^{-n+4}}{2!\Gamma(-n+3)} - \ldots
        \label{eq2.1.11}
    \end{equation}
    
    Если \( n \) является дробным числом, то \( \Gamma^{-1}(n+1) \ne 0 \) и \( \Gamma^{-1}(-n+1) \ne 0 \), так как \( \Gamma^{-1}(s) = 0 \) только тогда, когда \( s \) равно нулю или целому отрицательному числу. Поэтому разложения (\ref{eq2.1.10}) и (\ref{eq2.1.11}) начинаются с разных степеней \( x \), и, следовательно, функции \( J_n(x) \) и \( J_{-n}(x) \) линейно независимы, то есть одна из них не получается умножением другой на число:
    \[ J_n(x) \ne kJ_{-n}(x). \]
    
    Тогда при дробном \( n \) можно записать общий вид интеграла (\ref{eq2.1.1}):
    \begin{equation}
        u = C_1J_n(x) + C_2J_{-n}(x),
        \label{eq2.1.12}
    \end{equation}
    а формулу (\ref{eq2.1.10}) же можно представить в виде (\ref{eq2.1.13}):
    \begin{equation}
        J_n(x) = \sum\limits_{\nu = 0}^\infty \frac{(-1)^\nu \left(\frac{x}{2}\right)^{n+2\nu}}{\nu!\Gamma(n + \nu + 1)}.
        \label{eq2.1.13}
    \end{equation}
