\subsection{Уравнение Шрёдингера}
Найти уравнение, которому подчиняется волновая функция удалось в 1926 году
Шрёдингеру. Данное уравнение было именно найдено. Оно является новым
фундаментальным законом, который невозможно вывести из прежних представлений
и теорий. Справедливость этого уравнения установлена тем, что все следствия из
него подтверждаются экспериментом. Уравнение Шрёдингера описывает поведение
нерелятивистских микрочастиц. Для релятивистских микрочастиц используется
уравнение Дирака.
\[
    -\frac{\hbar^2}{2m}\Delta\Psi + U(\vec{r}, t)\Psi = i\hbar\pder{\Psi}{t}.
\]
Функция \( U(\vec{r}, t) \) определяется из соотношения. Если она явно не
зависит от времени, то она имеет смысл потенциальной энергии.

Рассмотрим стационарный случай, т.е \( U \) -- потенциальная энергия. Тогда
вероятность обнаружить частицу в данном месте пространства также не должна
зависеть от времени. Поэтому волновая функция может быть представлена в виде:
\[
    \Psi(\vec{r}, t) = \psi(\vec{r}) e^{\frac{iE}{\hbar}t}.
\]
Подставляя в уравнение Шрёдингера, получим уравнение Шрёдингера для стационарных
состояний:
\begin{equation}
    -\frac{\hbar^2}{2m}\Delta\psi(\vec{r}) + U(\vec{r})\psi(\vec{r}) =
    E\psi(\vec{r}).
\end{equation}
Отметим, что квантование в теории Бора вводилось искусственным образом. В теории
Шрёдингера оно возникает автоматически: дело в том, что физический смысл имеют
лишь те решения уравнения Шрёдингера, которые удовлетворяют стандартным
ограничениям, накладываемым на волновую функцию -- она должна быть конечной,
однозначной, непрерывной и гладкой во всём пространстве (даже в тех точках,
в которых потенциальная энергия терпит разрыв). Значения энергии \(E\),
при которых решение удовлетворяет этим условиям, называются собственными
значениями, а волновые функции, являющиеся решениями при данных значениях
энергии -- собственными функциями. Собственные значения энергии могут
образовывать как дискретный, так и непрерывный спектр.

\subsubsection{Свободная частица}
Для свободной частицы
\[
    \Psi = \Psi_0 exp(-i(\omega t - \vec{k}\cdot\vec{r})).
\]
Подставим эту функцию в уравнение Шрёдингера:
\begin{equation}
     E = i\hbar\frac{1}{\Psi}\pder{\Psi}{t},
\end{equation} 
\begin{equation}
     p^2 = -\frac{\hbar^2}{\Psi}\Delta\Psi,
\end{equation}
\begin{equation}
     \frac{p^2}{2m} = E.
\end{equation}
Для свободной частицы уравнение Шрёдингера имеет решение при любых значениях
энергии, т.е свободная частица обладает непрерывным спектром.

\subsubsection{Частица в прямоугольной бесконечно глубокой потенциальной яме}
[рисунок 1]
\begin{equation}
    -\frac{\hbar^2}{2m}\ppder{\psi}{x} + U\psi = E\psi.
\end{equation}
Для области внутри потенциальной ямы получим:
\begin{equation}
    \ppder{\psi}{x} + \frac{2mE}{\hbar^2}\psi = 0.
\end{equation}
\[ \psi(0) = \psi(l) = 0, \]
\begin{equation}
    E_n = \frac{\pi^2\hbar^2}{2ml^2}n^2,\ 
    \psi_n = \sqrt{\frac{2}{l}}\sin(\frac{\pi n}{l}x).
\end{equation}
Построим график распределения вероятности: [рисунок 2]. Видим, что при \(n=2\)
частица не может быть обнаружена в середине потенциальной ямы, хотя согласно
классической теории все положения частицы в яме равновероятны. Также заметим,
что минимальная энергия, которой может обладать частица в такой яме не равна
нулю. Если бы энергия могла бы быть нулевой, то импульс также был бы равен нулю,
а область локализации частицы была бы бесконечной.
\[
    \Delta E_n = E_{n+1} - E_n = \frac{\pi^2\hbar^2}{2ml^2}(2n+1).
\]
Для молекул газа в сосуде ( \( m = 10^{-26}\text{кг}, l = 10^{-1}\text{м} \) )
\( \Delta E_n =  \)
Электрон в пределах атома: \\ \\
\[
    \frac{\Delta E_n}{E_n} = \frac{2n+1}{n^2} \sim \frac{2}{n}
\]
Полученное выражение называется принципом соответствия Бора -- при больших
\( n \) законы квантовой механики переходят в законы классической механики.

\subsubsection{Прохождение частиц через потенциальный барьер}
В квантовой механике имеется вероятность:
\begin{itemize}
    \item отразиться от потенциального барьера при \( E > U_0 \);
    \item имеется вероятность пройти сквозь потенциальный барьер при
    \( E < U_0 \).
\end{itemize}

Рассмотрим случай \( E < U \). Решим уравнение Шрёдингера для областей 1,2 и 3.
Для областей 1 и 3 уравнение Шрёдингера имеет вид
\[
    \dder{\psi}{x} + \frac{2mE}{\hbar^2}\psi = 0,\ \frac{2mE}{\hbar^2} = k^2.
\] 
Его общее решение для области 1 имеет вид
\[
    \psi_1 = A_1 e^{ikx} + B_1 e^{-ikx},
\]
а в области 3
\[
    \psi_3 = A_3 e^{ikx} + B_3 e^{-ikx}.
\]
Решение в виде \( \psi_1 \) можно трактовать как суперпозицию двух бегущих волн:
первое слагаемое представляет собой волну, бегущую в положительном направлении
оси \( Ox \), а второе -- волну бегущую в обратном направлении. Поэтому
\( B_3 = 0 \).

Теперь рассмотрим уравнение Шрёдингера в области 2:
\[
    \dder{\psi}{t} + \frac{2m}{\hbar^2}(E - U_0)\psi = 0.
\]
Его решение имеет вид
\[
    \psi_1 = A_2 e^{\beta x} + B_2 e^{-\beta x}.
\]
Из условия гладкости \( \psi \)-функции получим выражения для связи коэффициентов:
\begin{align*}
    A_1 + B_1 = A_2 + B_2, & A_2 e^{\beta l} + B_2 e^{-\beta l} = A_3 e^{ikl};\\
    ik A_1 - ik B_1 = \beta A_2 - \beta B_2, & \beta A_2 e^{\beta l} - \beta
    B_2 e^{-\beta l} = ik A_3 e^{ikl}. 
\end{align*}

Введём понятие коэффициентов отражения и прохождения барьера:
\[
    R = \frac{|B_1|^2}{|A_1|^2},\ 
    D = \frac{|A_3|^2}{|A_1|^2},\ 
    R + D = 1.
\]

Введём обозначения:
\[
    b_1 = \frac{B_1}{A_1},\ a_2 = \frac{A_2}{A_1},\ b_2 = \frac{B_2}{A_1},\ 
    a_3 = \frac{A_3}{A_1},\ n = \frac{\beta}{k}.
\]

Тогда получим систему уравнений:
\[
    \left\{
    \begin{array}{l}
        1 + b_1 = a_2 + b_2, \\
        i - ib_1 = n a_2 - n b_2, \\
        a_2 e^{\beta l} + b_2 e^{-\beta l} = a_3 e^{ikl}, \\
        na_2 e^{\beta l} - nb_2e^{-\beta l} = ia_3 e^{ikl}.
    \end{array}
    \right.
\]

Решая её, получим:
\[
    \left\{
    \begin{array}{l}
        2i = (n+i)a_2 - (n-i)b_2, \\
        (n-i)a_2 e^{\beta l} - (n+i)b_2 e^{-\beta l} = 0.
    \end{array}
    \right.
\]
Отсюда,
\[
    a_2 = \frac{2i(n+i)e^{-\beta l}}{(n+i)^2e^{-\beta l} - (n-i)^2e^{\beta l}},\  
    b_2 = \frac{2i(n-i)e^{\beta l}}{(n+i)^2e^{-\beta l} - (n-i)^2e^{\beta l}},\ 
    a_3 = \frac{4nie^{-ikl}}{(n+i)^2 e^{-\beta l} - (n-i)^2 e^{\beta l}}.
\]
С учётом \( \beta \gg 1 \), упрощая выражение, получим:
\[
    a_3 = \frac{4nie^{-(ik+\beta)l}}{-(n-i)^2}, \ 
    D = |a_3|^2 = \frac{16n^2}{(n^2 + 1)^2} e^{-2\beta l} \approx e^{-2\beta l}. 
\]

Однако, барьеры бывают не только прямоугольными [рисунок 4].

Явление прохождения частицы через потенциальный барьер называется туннельным
эффектом.Он является следствием соотношения неопределённостей Гейзенберга.
Туннельный эффект может объяснить многие явления, наблюдаемые на практике, такие
как холодная эммиссия электронов из металлов и \( \alpha \)-распад.

Холодная эмиссия: при приложении к металлу электрического поля существенно
возрастает количество электронов. вылетающих из металла. Это легко объясняется 
туннельным эффектом. [рисунок 5]

\( \alpha \)-распад: потенциальный барьер обусловлен кулоновским взаимодействием
ядра, образовавшегося при \( \alpha \)-распаде с \( \alpha \)-частицей.
[рисунок 6]

\subsubsection{Параболическая яма (квантовый механический осциллятор)}
\[
    F = -kx,\ U = \frac{kx^2}{2} = \frac{m\omega^2x^2}{2}.
\]
Рассмотрим одномерный случай. Уравнение Шрёдингера имеет вид:
\[
    \dder{\psi}{x} + \frac{2m}{\hbar^2}(E - \frac{m\omega^2x^2}{2})\psi = 0.
\]
Произведём замену переменных:
\[
    q = x\sqrt{\frac{m\omega}{\hbar}},\ \lambda = \frac{2E}{\hbar\omega}.
\]
Для новых переменных уравнение принимает вид:
\[
    -\dder{\psi}{q} + q^2\psi = \lambda\psi.
\]
Будем искать решение в виде
\[
    \psi = P_n(q)e^{\alpha q^2},
\]
где \( P_n \) -- многочлен \( n \)-ой степени, а \( \alpha \) -- некоторая
постоянная. Подставляя такой вид решения в уравнение, получаем
\[
    \dder{P_n}{q} + 4\alpha q\der{P_n}{q} + 2\alpha P_n + (4\alpha^2 -1)q^2P_n =
    -\lambda P_n.
\]
Так как в правой части равенства многочлен \( n-\text{ой} \) степени, то
равенство возможно лишь при условии \( 4\alpha^2 - 1 = 0 \), откуда
\( \alpha = \pm0,5 \). Условию задачи удовлетворяет \( \alpha = -0,5 \), так как
в этом случае \( \psi \)-функция является ограниченной. Подставив, получим
\[
    \dder{P_n}{q} - 2q\der{P_n}{q} - P_n = -\lambda P_n.
\]
Приравняем коэффициенты при \( q^n \) и получим выражение для допустимых
значений \( E_n \):
\[
    \lambda = 2n+1,\ E_n = \frac{\hbar\omega}{2}\lambda =
    \hbar\omega\left(n+\frac{1}{2}\right).
\]
Само решение уравнения имеет вид:
\[
    \psi_n(q) = N_nP_n(q)e^{-\frac{q^2}{2}},
\]
где \( P_n(q) \) -- полиномы Чебышева-Эрмита. Пронормировав, получим
\[
    N_n = \sqrt{\frac{\alpha^\frac{1}{2}}{\sqrt{\pi}n!2^n}}.
\]
Особенности:
\begin{itemize}
    \item минимальная энергия квантового осциллятора не равна нулю;
    \item энергетические уровни квантового осциллятора эквидистантны;
    \item вероятность излучательного перехода отлична от нуля лишь при переходе
    между соседними уровнями энергии.
\end{itemize}

Неравенство нулю энергии подтверждаетя экспериментом по рассеянию света на
тепловых колебаниях атомов в узлах кристаллической решётки. При понижении
температуры до абсолютного нуля оказывается, что интенсивность рассеянного света
стремится не к нулю, а к некоторому конечному значению. что указывает на то, что
колебания атомов кристаллической решётки не прекращается при абсолютном нуле -- 
нулевые колебания.
[рисунок 8].