\section{Рынки факторов производства}

Основными рынками факторов производства являются:
\begin{enumerate}
    \item капитал,
    \item труд,
    \item земля,
    \item предпринимательские способности,
    \item информация,
    \item время.
\end{enumerate}

Специфика рынков:
\begin{enumerate}
    \item спрос на ресурсы носят вторичный производный характер;
    \item производители выступают в качестве покупателей;
    \item все ресурсы не могут участвовать в производстве обособленно,
        а функционируют в определённых комбинациях, взаимодополняя друг друга.
\end{enumerate}

\subsection{Рынок труда}

Спрос со стороны предпринимателей, предложение со стороны работников.

Закон спроса на труд: чем выше заработная плата, тем ниже спрос.

Реальная заработная плата -- это совокупность материальных благ и услуг,
которые работник может приобрести в данный период на получение, или сумма
денежных средств.

Номинальная заработная плата -- это сумма денежных средств, полученная
работником за свой труд в соответствии с его качеством и количеством.
    
Функции заработной платы:
\begin{enumerate}
    \item стимулирование трудовой активности;
    \item воспроизводство рабочей силы;
    \item создание фонда жизненных благ населению.
\end{enumerate}

Системы оплата труда:
\begin{enumerate}
    \item тарифная -- представляет совокупность норм и нормативов, с помощью
    которых осуществляется дифференциация оплаты труда различных категорий
    работников, исходя из различий в сложности, условий и характера труда.
    
    Основные элементы:
    \begin{itemize}
        \item минимальная ставка работников первого ранга;
        \item тарифные ставки по разрядам рабочих, дифференцированные по
            сложности работ и образующие тарифную сетку;
        \item схемы должностных окладов служащих;
        \item тарифно-квалификационные справочники.
    \end{itemize}
    
    \item Нетарифная система. Она предполагает определение общего фонда
        оплаты труда, а затем его распределение между работниками на основании
        их трудового вклада.
\end{enumerate}

Формы заработной платы:
\begin{enumerate}
    \item повременная -- форма оплаты труда, при которой работник получает
        заработную плату в зависимости от количества отработанного времени и
        уровня квалификации. Она делится на два типа:
    \begin{enumerate}
        \item простая повременная,
        \item повременно-премиальная;
    \end{enumerate}
    
    \item сдельная -- форма оплаты труда, при которой заработная плата
        устанавливается по расценкам за каждую единицу выполненной работы или
        продукции. Она делится на пять типов:
    \begin{enumerate}
        \item прямая сдельная,
        \item сдельно-премиальная,
        \item косвенно-сдельная,
        \item сдельно-прогрессивная,
        \item аккордно-сдельная.
    \end{enumerate}
\end{enumerate}

В состав фонда заработной платы включаются:
\begin{enumerate}
    \item оплата за отработанное время,
    \item оплата за неотработанное время,
    \item единовременные поощрительные выплаты,
    \item выплаты на питание, топливо, жильё, \ldots
\end{enumerate}

Уровень заработной платы зависит от
\begin{itemize}
    \item производительности труда работника,
    \item деятельности профсоюзов,
    \item инвестиций в человеческий капитал.
\end{itemize}

Инвестиции в человеческий капитал -- это любое действие, которое повышает
квалификацию, способности и производительность труда работника.
К ним относят расходы на образование, здравоохранение и мобильность.

\subsection{Рынок земли}

Под землей как ресурсом понимаются все природные богатства, находящиеся в
земле или ее недрах. Особенностью этого фактора производства является его
ограниченность, а в большинстве случаев невозможность воспроизводства.

Спрос на землю делится на сельскохозяйственный и несельскохозяйственный.

Рента -- доход, получаемый собственником ресурса, предложение которого
строго ограничено.

Виды ренты:
\begin{enumerate}
    \item чистая (абсолютная) рента -- это доход, получаемый собственником
        ресурса, характеризующимся абсолютно не эластичным по цене-предложению;
    \item дифференциальная рента -- это доход, полученный собственником более
        производительного ресурса.
\end{enumerate}

Арендная плата -- сумма, выплачиваемая пользователем земля ее владельцу,
включающая ренту, все затраты на капитальные вложения (амортизация) и
процент на вложенный капитал.

Цена земли:
\[
    P_\emph{з} = R/r,
\]
\( R \) -- годовая рента, \( r \) -- процент в банке.
