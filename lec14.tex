\chapter{Свободные колебания в контурах}

	\begin{definition}
        \textbf{Колебания} -- периодический процесс, в котором физическая
        величина \( x(t) \) принимает примерно равные значения через равные
        промежутки времени, называемые \textbf{периодом}:
	    \[ x(t + T) \approx x(t) \]
	\end{definition}
	
	Наиболее подробно рассматриваются следующие виды колебаний:
	\begin{itemize}
        \item синусоидальные незатухающие колебания,
        \item синусоидальные затухающие колебания,
        \item релаксационные колебания.
	\end{itemize}
	
	Если колебания в системе происходят от первоначального запаса энергии
    системы и без поступления энергии извне, то такие колебания называются
    \textbf{свободными}. Если же они поддерживаются периодическим поступлением
    энергии в систему, то такие колебания называются \textbf{вынужденными}.
	
	Свободные колебания реальных макроскопических \textit{всегда} затухающие:
    первоначальный запас энергии в конце превращается в тепло. Однако, сначала
    удобно рассмотреть идеализированную колебательную систему, где потерь
    энергии нет и колебания незатухающие.
	
\section{Колебания в идеальном колебательном контуре}

	Зарядим конденсатор до начального напряжения \( u \) и, в момент
    \( t = 0 \), замкнём ключ \( K \). Конденсатор начнёт разряжаться через
    катушку, по цепи пойдет ток \( i(t) \). Найдём \( i(t) \), \( q(t) \) и
    \( u_C(t) \). Для этого запишем второе уравнение Кирхгофа:
	\[
        \sum \pm u_k = 0 \quad \text{или} \quad u_L - u_C = 0,
    \]
	\begin{equation}
		L\frac{\dd i}{\dd t} - \frac{q}{C} = 0.
        \label{eq14:1}
	\end{equation}
	
	Продифференцируем (\ref{eq14:1}) по времени:
	\[
        L\frac{\dd ^2i}{\dd t^2} + i\frac{1}{C} = 0.
    \]
	Здесь ток \( i = -\frac{\dd q}{\dd t} \) -- \textbf{разрядный} ток.
	Таким образом, мы получили дифференциальное урвнение, описывающее
    колебательный процесс в системе:
	\begin{equation}
		\frac{\dd ^2i}{\dd t^2} + \omega^2i = 0, 
        \label{eq14:2}
	\end{equation}
	где величина
    \[
        \omega = \frac{1}{\sqrt{LC}}
    \] 
     называется собственной частотой колебательного контура.
	
	Общее решение дифференциального уравнения (\ref{eq14:2}) выглядит следующим
    образом:
	\begin{equation}
		i(t) = A\sin\omega t + B\cos\omega t.
        \label{eq14:3}
	\end{equation}
	Коэффициенты \( A \) и \( B \) определяем из начальных условий \( i(t_0) \)
    и 
    \[
        \left. \frac{\dd i}{\dd t}\right|_{t=0}:
    \]
	
	\( i(+0) = i(-0) = 0 \) -- ток \( i \) сразу после замыкания ключа \( K \),
    где \( i(-0) \) -- ток до замыкания ключа. Покажем, что \( i(+0) = i(-0) \).
	
	Действительно, ток \( i \) через \( L \) \textit{скачком измениться не
    может}, так как в уравнении (\ref{eq14:1}) величина \( q \) \textit{всегда
    конечна}. Поэтому \( i(+0) = i(-0) \). Это даёт из (\ref{eq14:3}):
	\[
        0 = A\sin0 + B\cos0 \Rightarrow B = 0.
    \]
	 
	Таким образом,
	\begin{equation}
		i = A\sin\omega t \label{eq14:4}
	\end{equation}
	Из (\ref{eq14:1}):
	\[
        \frac{\dd i}{\dd t} = \frac{q}{LC} = \frac{Cu}{LC} =
        \left.\frac{u}{L}\right|_{t = 0} = \frac{U}{L}.
    \]
	Из (\ref{eq14:4}):
	\[
        \left.\frac{\dd i}{\dd t}\right|_{t = 0} = A\omega\cos0  = \frac{U}{L}
    \]
	Отсюда получаем, что \( A = \frac{U}{\omega L} \). Тогда ток в цепи меняется
    по закону
	\begin{equation}
		i = I\sin\omega t,
        \label{eq14:5}
	\end{equation}
	где
    \[
        I = \frac{U}{\omega L} = U\sqrt{\frac{C}{L}}\text{ -- амплитуда тока}
    \]
	% к рисунку: T = \frac{2\pi}{\omega} = 2\pi\sqrt{CL}
	
	Выражения для \( q(t) \) и \( u_C(t) \) можно получить аналогично из
    (\ref{eq14:1}):
	\[
        L\ddot{q} + \frac{1}{C}q = 0,
    \]
	\begin{equation}
		\ddot{q} + \omega^2q = 0
	\end{equation}
	А так как \( q = Cu_C \), то для \( u_C(t) \) получаем:
	\begin{equation}
		\ddot{u}_C + \omega^2u_C = 0
	\end{equation}
	
	Выражение для \( u(t) \) удобно взять из подстановки (\ref{eq14:5}) в
    (\ref{eq14:1}):
	\begin{equation}
		u_C(t) = U\cos\omega t
	\end{equation}
	
	Преобразования энергии при колебаниях в контуре:
	\[
        \underbrace{W_C}_{\frac{Cu^2}{2}} \to \underbrace{W_L}_{\frac{Li^2}{2}}
        \to W_C \to W_L \to \ldots
    \]
	
	Покажем, что в любой момент времени \( W_C(t) + W_L(t) = \const \):
	\begin{align*}
        & \frac{Cu^2}{2} + \frac{Li^2}{2} = \frac{CU^2}{2}\cos^2\omega t +
        \frac{LI^2}{2}\sin^2\omega t = \frac{CU^2}{2}\cos^2\omega t + \\
        & + \frac{L\cdot\left(U\sqrt{\frac{C}{L}}\right)^2}{2}\sin^2\omega t =
        \frac{CU^2}{2}\cos^2\omega t + \frac{CU^2}{2}\sin^2\omega t =
        \frac{CU^2}{2} = W_0 = \const,
	\end{align*}
	где \( W_0 = W_C \) -- начальный запас энергии в контуре.
	
\section{Свободные колебания в реальном контуре}

	Этот раздел подробно рассмотрен в \href{http://dl.dropbox.com/u/41185505/University/Physics/Electricity/%D0%A4309.pdf}{работе Ф309}. Пусть в момент времени \(  t = 0 \)
    замыкается ключ \( K \). Пойдет разрядный ток
    \[
        i = -\frac{\dd q}{\dd t}.
    \]
	По второму правилу Кирхгофа:
	\begin{equation}
		L\frac{\dd i}{\dd t} + Ri - \frac{q}{C} = 0.
        \label{eq14:n1}
	\end{equation}
	Продифференцируем уравнение (\ref{eq14:n1}) по \( t \):
	\[
        L\frac{\dd^2 i}{\dd t^2} + R\frac{\dd i}{\dd t} + \frac{1}{C}i = 0,
    \]
	\begin{equation}
		\frac{\dd^2 i}{\dd t^2} + 2\beta\frac{\dd i}{\dd t} + \omega_0^2 i = 0,
        \label{eq14:n2}
	\end{equation}
	где \( \beta = \frac{R}{2L} \) -- коэффициент затухания,
    \( \omega_0 = \frac{1}{\sqrt{LC}} \) -- собственная частота контура.
	
	Уравнение (\ref{eq14:n2}) -- уравнение свободных затухающих колебаний в
    каноническом виде.
	
	Произведем в (\ref{eq14:n2}) замену
	\begin{equation}
		i(t) = x(t)e^{-\beta t}
        \label{eq14:n3}
	\end{equation}
	и получим
	\begin{equation}
		\ddot{x} + (\omega_0^2 - \beta^2)x = 0.
        \label{eq14:n4}
	\end{equation}
	
	Так как \( \omega_0^2 - \beta^2 \) может быть больше нуля, равно нулю или
    меньше нуля, то уравнение (\ref{eq14:n4}) имеет три решения:
	\begin{enumerate}
        \item \( \omega_0 > \beta \) -- докритический режим.

            Обозначим \( \sqrt{\omega_0^2 - \beta^2} \) за \( \omega \) --
            частоту свободных колебаний. Тогда решение уравнения (\ref{eq14:n4})
            \[
                x(t) = A\sin\omega t + B\cos\omega t,
            \]
            а решение уравнения (\ref{eq14:n2}) с учетом (\ref{eq14:n3}):
            \begin{equation}
                i(t) = e^{-\beta t}(A\sin\omega t + B\cos\omega t).
                \label{eq14:n5}
            \end{equation}
        
            Коэффициенты \( A \) и \( B \)  определяются из начальных условий:
            \begin{enumerate}
                \item \( i(+0) = i(-0) = 0: \)
                \[
                    0 = e^0(A\sin0 + B\cos0) \Rightarrow B = 0.
                \]
                Таким образом,
                \begin{equation}
                    i(t) = Ae^{-\beta t}\sin\omega t;
                    \label{eq14:n6}
                \end{equation}
                
                \item
                    \[
                        L\left.\frac{\dd i}{\dd t}\right|_{t = 0} =
                        \left.(U_C - Ri)\right|_{t = 0} = U_0,
                    \]
                где \( U_0 \) -- начальное напряжение на конденсаторе.
                \[
                    \left.\frac{\dd i}{\dd t}\right|_{t = 0} = \frac{U_0}{L}.
                \]
                Подставляя в (\ref{eq14:n6}):
                \[
                    \frac{U_0}{L} = \left.A(-\beta e^{-\beta t}\sin\omega t +
                    \omega e^{-\beta t}\cos\omega t)\right|_{t = 0} =
                    A\omega \Rightarrow A = \frac{U_0}{\omega L}.
                \]
                Тогда
                \begin{equation}
                    i(t) = I_0e^{-\beta t}\sin\omega t,
                    \label{eq14:n7}
                \end{equation}
                где
                \[
                    I_0 = \frac{U_0}{\omega L}
                    \text{ -- начальная амплитуда тока}
                \]
            \end{enumerate}
        
            Уравнение (\ref{eq14:n7}), являющееся решением уравнения
            (\ref{eq14:n1}), описывает затухающие колебания, а выражение
            \( I_0e^{-\beta t} = I(t) \) является их амплитудой.
        
            \begin{definition}
                Время \( \tau \), за которое амплитуда \( I \) уменьшается в
                \( e \) раз, называется \textbf{временем релаксации}:
                \[
                    \frac{I(t)}{I(t + \tau)} = 
                    \frac{I_0e^{-\beta t}}{I_0e^{-\beta t}e^{-\beta\tau}} =
                    e^{\beta\tau} = e.
                \]
        
                Следовательно,
                \[
                    \tau = \frac{1}{\beta} = \frac{2L}{R}.
                \]
            \end{definition}
        
            \begin{definition}
                Натуральный логарифм отношения двух соседних амплитуд
                \[
                    \delta = \ln\frac{I_k}{I_{k + 1}}
                \]
                называется \textbf{логарифмическим декрементом затухания}.
            \end{definition}
        
            \[
                \ln\frac{I_k}{I_{k + 1}} =
                \frac{I_0e^{-\beta t}}{I_0e^{-\beta t}e^{-\beta T}} =
                \ln e^{\beta T} = \beta T.
            \]
        
            Таким образом, \( \delta = \beta T \). Выразим его через параметры
            контура:
            \[
                \delta = \frac{R}{2L}\cdot\frac{2\pi}{\omega} =
                \frac{R\pi}{\omega L}.
            \]
            
            Если затухание \textbf{слабое}, то есть если
            \( \beta \ll \omega_0 \), то \( \delta \ll 1 \) и
            \( \omega \approx \omega_0 \).
            
            \begin{remark}
                Если измерены амплитуды \( I_k \) и \( I_{k + N} \), то
                логарифмический декремент затухания:
                \[ \delta = \frac{1}{N}\ln\frac{I_k}{I_{k + N}} \]
            \end{remark}
            \begin{proof}
                \[
                    \ln\frac{I_k}{I_{k + N}} =
                    \frac{I_0e^{-\beta t}}{I_0e^{-\beta t}e^{-\beta NT}} =
                    \ln e^{\beta NT} = \beta NT.
                \]
                А так как логарифмический декремент затухания
                \( \delta = \beta T \), то:
                \[
                    \delta = \frac{1}{N}\ln\frac{I_k}{I_{k + N}}.
                \]
            \end{proof}
            
            Важной характеристикой всякой колебательной системы является
            \textbf{добротность} \( Q \):
            \begin{equation}
                Q = \frac{\pi}{\delta}.
                \label{eq14:17}
            \end{equation}
            
            Для слабозатухающих колебаний:
            \[
                Q = \frac{\pi}{\delta} = \frac{\pi}{\frac{R\pi}{\omega_0 L}} =
                \frac{\omega_0 L}{R} = \frac{1}{R}\sqrt{\frac{L}{C}}.
            \]
            Таким образом,
            \[
                Q = \frac{\pi}{\delta} = \frac{1}{R}\sqrt{\frac{L}{C}}.
            \]
            Для радиоконтуров \( Q \sim 10 \ldots 100 \). Для СВЧ-контуров:
            \( Q \sim 10^4 \ldots 10^5 \).
            
            Другое полезное определение добротности:
            \[
                Q = 2\pi\frac{W}{\Delta W_T},
            \]
            где \( W \) -- энергия контура в данный момент, \( \Delta W_T \) --
            потери энергии за текущий период. Покажем, что это определение
            эквивалентно (\ref{eq14:17}).
            \[
                W(t) = \frac{LI^2(t)}{2} = \frac{LI_0^2}{2}e^{-2\beta t} =
                W_0e^{-2\beta t}
            \]
            
            Тогда убыль энергии за период:
            \[
                -\Delta W = \left. \frac{\partial W}{\partial t}
                \Delta t\right|_{\Delta t = T} = 2\beta W_0e^{-2\beta t}\cdot T.
            \]
            Тогда добротность:
            \[
                Q = 2\pi\frac{W}{\Delta W_T} =
                2\pi\frac{W_0e^{-2\beta t}\cdot\omega_0}{2\beta W_0e^{-2\beta t}
                \cdot 2\pi} = \frac{\omega_0}{2\beta} = 
                \frac{\frac{1}{\sqrt{LC}}}{2\frac{R}{2L}} =
                \frac{1}{R}\sqrt{\frac{L}{C}} = \frac{\pi}{\delta}.
            \]
            
            Так как \( \Delta W_T = PT \), где \( P \) -- мощность потерь
            колебаний, то добротность:
            \[
                Q = \omega_0\frac{W}{P}.
            \]
        
        % -----------------------------------------------------------------------------
        
        \item \( \beta = \omega_0 \) -- критический режим.
            В этом случае уравнение (\ref{eq14:n4}) имеет вид:
            \[ \ddot{x} = 0 \]
            Его решение: \( x = At + B \). А ток тогда:
            \[
                i = xe^{-\beta t} = e^{-\beta t}(At + B).
            \]
            
            Коэффициенты \( A \) и \( B \) определяются из начальных условий:
            \begin{enumerate}
                \item \( i(+0) = i(-0) = 0: \)
                    \[
                        0 = A\cdot0 + B \Rightarrow B = 0.
                    \]
            
                \item
                    \[
                        \left.\frac{\dd i}{\dd t}\right|_0 =
                        \left.(-\beta e^{-\beta t}At + Ae^{-\beta t})\right|_0 =
                        \frac{U_0}{L} \Rightarrow A = \frac{U_0}{\omega L}.
                    \]
                    Тогда ток \( i \):
                    \begin{equation}
                        i = \frac{U_0}{\omega L}te^{-\beta t} \label{eq14:25}
                    \end{equation}
            \end{enumerate}
            
            Процесс (\ref{eq14:25}) является \textit{апериодическим}, а
            сопротивление \( R \), соответствующее варианту
            \( \beta = \omega_0 \), называется \textit{критическим}:
            \[
                \frac{1}{\sqrt{LC}} = \frac{R_{\textit{кр}}}{2L},
            \]
            \begin{equation}
                R_{\textit{кр}} = 2\sqrt{\frac{L}{C}}
            \end{equation}
        
        % -----------------------------------------------------------------------------
        
        \item \( \beta > \omega_0 \) -- закритический режим.
            Тогда решением (\ref{eq14:n4}) будет функция:
            \[
                x = Ae^{-\alpha t} + Be^{\alpha t},
            \]
            где \( \alpha = \sqrt{\beta^2 - \omega_0^2} \).
            Тогда
            \[
                i = xe^{-\beta t} = Ae^{-(\alpha + \beta) t} +
                Be^{(\alpha - \beta) t}.
            \]
    \end{enumerate}
