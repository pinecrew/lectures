\section{Производство экономических благ}
Теория поведения потребителя позволила глубже понять явления, которые лежат на
стороне спроса. Теория фирмы поможет нам глубже разобраться в отношениях,
складывающихся со стороны предложения. Однако прежде чем выявить особенности
рыночной стратегии и тактики фирмы, кратко охарактеризуем общие (универсальные,
не зависящие от типа экономической системы) основы производства благ.

\subsection{Производство с одним переменным фактором}

Под \emph{производством} в современной микроэкономике понимается деятельность
по использованию факторов производства (ресурсов) с целью достижения наилучшего
результата. Если объем использования ресурсов известен, то максимизируется
результат, если известен результат (которого необходимо достичь), то 
минимизируется объем ресурсов. Понятия "затраты", "выпуск", "деятельность
фирмы" трактуются в современной экономической науке довольно широко.

Под \emph{затратами} понимается все, что производитель (фирма) закупает для
использования в целях достижения необходимого результата.

\emph{Выпуском} может быть любое благо (продукция или услуга), изготовленное
фирмой для продажи.

Деятельность фирмы может обозначать как производственную, так и коммерческую
деятельность, например транспортировку, хранение и даже покупку продукции с
целью ее последующей перепродажи. В современном обществе любая фирма
производит, как правило, не одно, а целый ряд экономических благ, однако мы в
целях упрощения будем пренебрегать этим обстоятельством; предполагается, что
производится лишь один товар (или услуга).

Экономическая деятельность фирмы может быть описана производственной функцией:
\[
    Q = f(F_1, F_2, \ldots, F_n),
\]

где \( Q \) — максимальный объем производства при заданных затратах, \( F_1 \)
— количество использованного фактора \( f_1 \), \( F_2 \) — количество
использованного фактора \( f_2 \), \( F_n \) — количество использованного
фактора \( f_n \).

\emph{Закон убывающей предельной производительности.} Предположим сначала, что
\( F_1 \) является переменным фактором, тогда как остальные \( (n - 1) \)
факторов \( (F_2, \ldots, F_n) \) постоянны. Для того чтобы отразить влияние
переменного фактора на производство, вводятся понятия совокупного (общего),
среднего и предельного продукта.

Совокупный продукт — это количество экономического блага, произведённое с
использованием некоторого количества переменного фактора.

Разделив совокупный продукт на израсходованное количество переменного фактора,
можно получить средний продукт:
\[
    AP = \frac{Q}{F_1}.
\]

Предельный продукт (marginal product) обычно определяется как прирост
совокупного продукта, полученный в результате бесконечно малых приращений
количества использованного переменного фактора:
\[
    MP = \frac{DQ}{DF_1}.
\]

Совокупный продукт \( (Q) \) с ростом использования в производстве переменного
фактора \( (F_1) \) будет увеличиваться, однако этот рост имеет определённые
пределы в рамках заданной технологии (рис. 6.1).

При неизменном состоянии техники, например, рост использования труда ограничен.
На первой стадии производства (0А) увеличение затрат труда способствует все
более полному использованию капитала: предельная и общая производительность
труда растут. Это выражается в росте предельного и среднего продукта, при этом
\( MP > AP \) (рис. 6.1б). В точке А' предельный продукт достигает своего
максимума.

Рис. 6.1. Рост переменного фактора: стадии производства

На второй стадии (АБ) величина предельного продукта уменьшается и в точке Б'
становится равной среднему продукту \( (MP = AP) \). Если на первой стадии (0А)
совокупный продукт возрастает медленнее, чем использованное количество
переменного фактора, то на второй стадии (АБ) совокупный продукт растет
быстрее, чем использованное количество переменного фактора (рис. 6.1а).

На третьей стадии производства (БВ) \( MP < AP \), в результате чего совокупный
продукт растет медленнее затрат переменного фактора и, наконец, наступает
четвертая стадия (после точки В), когда \( MP < 0 \). В результате прирост
переменного фактора \( F_1 \) приводит к уменьшению выпуска совокупной продукции
(конечно, при условии, что все единицы переменного фактора качественно
однородны и добавление все новых и новых единиц не ведет к качественному
изменению технологии).

В этом и заключается закон убывающей предельной производительности. Он
утверждает, что с ростом использования какого-либо производственного фактора
(при неизменности остальных) рано или поздно достигается такая точка, в которой
Дополнительное применение переменного фактора ведет к снижению относительно и
далее абсолютного объемов выпуска продукции. Увеличение использования одного из
факторов (при фиксировании остальных) приводит к последовательному снижению
отдачи его применения.

Закон убывающей производительности никогда не был доказан строго теоретически,
он выведен экспериментальным путем (сначала в сельском хозяйстве, а потом и
применительно к другим отраслям производства). Он отражает реально наблюдаемый
факт определённых пропорций между различными факторами. Нарушение их,
выражающееся в чрезмерном росте применения одного из ресурсов, может довольно
быстро исчерпать границы взаимозаменяемости ресурсов и в конечном итоге
приведет к недостаточно эффективному его использованию (если другие факторы
производства остаются неизменными).

Закон убывающей предельной производительности носит не абсолютный, а
относительный характер.
\begin{enumerate}
    \item он применим лишь на краткосрочном отрезке времени, когда хотя бы один
    из факторов производства остается неизменным;
    \item Во-вторых, технический прогресс постоянно раздвигает его границы.
\end{enumerate}

Проиллюстрируем эту мысль на графике.

Максимально возможное использование переменного фактора в технологии I
обозначим через \( X_1 \). Очевидно, что переход к технологии II позволяет
увеличить количество переменного фактора до \( X_2 \) \( (X_2 > X_1) \),
переход к технологии III — до \( X_3 \) и т. д.

Рис. 6.2. Рост совокупного продукта и использование переменного фактора в
процессе перехода к новым технологиям

В заключение скажем несколько слов о взаимосвязи предельного и среднего
продукта. Предельный продукт для какой-либо точки на кривой совокупного выпуск
а равен тангенсу угла наклона касательной к кривой в этой точке. Для точки А
(рис. 6.la)
\[
    \tan\alpha = \frac{AX_1}{X_0X_1}
\]
В точке В совокупный продукт достигает своего максимума, а предельный продукт
равен О \( (MP = 0) \). До точки В \( MP > 0 \), после этой точки \( MP < 0 \).

Потребительское благо не является экономическим, если его предельная полезность
(и соответственно цена) меньше или равна нулю. Поэтому и факторы производства
используются в производстве только тогда, когда их производительность
представляет собой положительную величину.

Спрос на ресурсы является производным от спроса на потребительские блага. Если
обозначить предельный продукт в денежном выражении через MRP (Marginal Revenue
Product), а предельные издержки — через MRC (Marginal Resource Cost), то
правило использования ресурсов может быть выражено равенством:
\[
    MRP = MRC.
\]

Это означает, что для того, чтобы максимизировать прибыль, каждый производитель
(фирма) должен использовать дополнительные (предельные) единицы любого ресурса
до тех пор, пока каждая дополнительная единица ресурса дает прирост совокупного
дохода, превышающий прирост совокупных издержек.

