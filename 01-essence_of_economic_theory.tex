\section{Сущность экономической теории и ее основные категории}

Экономическая наука изучает экономические потребности и способы их
удовлетворения. Предметом экономической теории являются экономические отношения,
которые подразделяются на социально-экономические и организационно-экономические.

Социально-экономические отношения -- это хозяйственные связи между большими
социальными группами, отдельными коллективами и членами общества.

Организационно-экономические отношения возникают благодаря тому, что
общественное производство, распределение, обмен и потребление невозможны без
определённой организации.

Функции экономической теории:
\begin{enumerate}
    \item познавательная;
    \item прогностическая;
    \item практическая.
\end{enumerate}

Методы научного познания экономики:
\begin{enumerate}
    \item всеобщий (метафизика, диалектика);
    \item общенаучный (исторический и логический);
    \item специфический (экономическая психология, экономическая статистика,
    экономическая математика).
\end{enumerate}

Экономические потребности -- недостаток чего-либо необходимого для поддержания
жизнедеятельности и развития личности, фирмы и общества в целом. Средством
удовлетворения потребностей является благо.

Экономические блага бывают:
\begin{enumerate}
    \item материальные, которые, в свою очередь, делятся на
    \begin{enumerate}
        \item продукты производства и
        \item естественные дары природы;
    \end{enumerate}
    \item нематериальные блага, которые делятся на
    \begin{enumerate}
        \item внутренние;
        \item внешние.
    \end{enumerate}
    \item многоразового использования и разового применения;
    \item взаимозаменяемые (субституты) и взаимодополняемые (комплементарные);
    \item единичные и общественные.
\end{enumerate}

Экономические ресурсы -- это элементы, используемые для производства
экономических благ.

К основным экономическим ресурсам относят:  
\begin{enumerate}
    \item труд;
    \item земля;
    \item капитал;
    \item предпринимательские способности;
    \item информация.
\end{enumerate}

Возможности общества по производству экономических благ при полном и эффективном
использовании всех имеющихся ресурсов показывает производственные возможности
предприятия.

% кривая производственных возможностей %

Кривая производственных возможностей показывает альтернативные варианты
производства при полном использовании ресурсов.

Экономические агенты -- субъекты экономических отношений, участвующие в
производстве, распределении, обмене и потреблении экономических благ.

Основными экономическими агентами являются:
\begin{enumerate}
    \item домохозяйства;
    \item фирмы;
    \item государства.
\end{enumerate}

Экономические агенты участвуют в экономическом кругообороте, под которым
понимают движение реальных экономических благ, сопровождающиеся встречным поток
денежных доходов и расходов.

% кругооборот %

Принцип эффективного распределения ресурсов получил название
``Парето-эффективность''. Парето-эффективность -- это такой уровень организации
экономики, при котором общество извлекает максимум полезности из имеющихся
ресурсов и технологий.

Экономическая теория предполагается, что поведение человека должно быть
рациональным. Экономический человек [`homo economicus'] -- это человек, имеющий
неограниченные желания, но располагающий ограниченными ресурсами.

Экономические системы -- это совокупность взаимосвязанных экономических
элементов, образующих определенную целостность -- экономическую структуру
общества.

Существуют две классификации экономических систем: историческая и современная.
По первой классификации существуют следующие виды экономических систем:
\begin{enumerate}
    \item доиндустриальная экономика;
    \item индустриальная экономика;
    \item постиндустриальная экономика.
\end{enumerate}

По второй классификации существуют следующие виды экономических систем:
\begin{enumerate}
    \item рыночная;
    \item командно-административная экономика;
    \item смешанная.
\end{enumerate}
