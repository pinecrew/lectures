\section{Национальная экономика и измерение национального объема производства}

Макроэкономика -- раздел экономической науки, где исследуется функционирование
экономической системы на уровне национальной экономики.

Национальная экономика -- экономическая деятельность хозяйственных субъектов в
масштабе государства, направленная на удовлетворение потребностей нации.

Национальные экономики взаимодействуют следующими способами:
\begin{enumerate}
    \item через торговые операции: экспорт и импорт;
    \item через денежно-кредитные и финансовые операции;
    \item миграция рабочей силы;
    \item обмен научно-технологической информацией.
\end{enumerate}

Государство -- это форма существования и организации общества, представляющее
собой систему общественных институтов и органов управления.

Пирамида целей государственного регулирования:
\begin{enumerate}
    \item высшие цели: формирование благоприятных условий для поддержания
    экономического развития и социальной стабильности общества;
    \item цели первого порядка (магический четырехугольник): обеспечение роста
    ВВП, минимизация безработицы, стабилизация уровня цен, достижение
    бездефицитного платежного баланса;
    \item цели второго порядка: создание законодательных условий для увеличения
    прибыли, стимулирования экономического роста, сглаживания цикличности
    экономики.
\end{enumerate}

Функции государства:
\begin{enumerate}
    \item создание правовой базы;
    \item проведение антимонопольной политики;
    \item создание общественных благ;
    \item регулирование внешних (побочных) эффектов (экстерналий);
    \item создание инфраструктуры экономики;
    \item перераспределение доходов (налоги, субсидии, льготы, \ldots);
    \item макроэкономическая стабилизация экономики;
    \item поддержка малого бизнеса;
    \item регулирование внешнеэкономической деятельности;
    \item поддержка фундаментальной науки;
    \item проведение инновационной политики;
    \item обеспечение экологической безопасности.
\end{enumerate}

Методы государственного регулирования:
\begin{itemize}
    \item прямые (административные) -- запрещение, разрешение, принуждение,
    внедрение норм и стандартов;
    \item косвенные (экономические) -- бюджетно-налоговая, денежно-кредитная
    политики, государственное прогнозирование и программирование,
    внешнеэкономические инструменты.
\end{itemize}

Основные макроэкономические показатели:
\begin{itemize}
    \item ВВП (валовый внутренний продукт) -- это стоимость всей конечной
    продукции, произведенной на территории государства за определенный период
    времени;
    \item ВНП (валовый национальный продукт) -- это стоимость всей конечной
    продукции, произведенной резидентами страны за определенный период времени
\end{itemize}

Основные характеристики ВВП (ВНП):
\begin{enumerate}
    \item это -- денежный показатель;
    \item отражает текущее производство;
    \item не учитывает непроизводительные сделки, к которым относятся:
        \begin{itemize}
            \item операции с ценными бумагами,
            \item трансфертные платежи (выплаты населению),
            \item повторная продажа товаров;
        \end{itemize}
    \item не учитывает результаты теневой экономики;
    \item не учитываются блага, производимые и потребляемые в домашних условиях.
\end{enumerate}

Методы расчета ВНП (ВВП):
\begin{enumerate}
    \item по расходам: \( \emph{ВНП} = C + I_g + G + X_n \), где \( C \) --
    потребительские расходы, исключая покупки жилья, \( I_g = I_n + \emph{Ам} \)
    -- валовые частные внутренние инвестиции -- расходы предприятий на
    приобретение основных фондов и расходы домохозяйств на приобретение жилья,
    \( I_n \) -- чистые инвестиции и \emph{Ам} -- ароматизационные отчисления,
    \( X_n = \emph{Э} - \emph{И} \) -- чистый экспорт -- разница между экспортом
    и импортом, \( G \) -- государственные расходы, за исключением
    трансфертных платежей;
    
    \item по доходам: \( \emph{ВНП} = \emph{ЗП} + \emph{Р} + \emph{П} +
    \emph{Д}_\emph{с} + \emph{П}_\emph{р} + \emph{Ам} + \emph{КН} \), где
    \emph{ЗП} -- заработная плата, \emph{Р} -- рента, \emph{П} -- процентные
    доходы, \( \emph{Д}_\emph{с} \) -- доходы индивидуальных собственников,
    \( \emph{П}_\emph{р} = \emph{Н}_\emph{пр} + \emph{Д}_\emph{в} + \emph{НП} \)
    -- прибыль корпораций, \( \emph{Н}_\emph{пр} \) -- налог на прибыль,
    \( \emph{Д}_\emph{в} \) -- дивиденты, \( \emph{НП} \) --
    нераспределенная прибыль, \emph{Ам} -- доходы с ароматизации, \emph{КН} --
    косвенные налоги;
    
    \item производственный: \( \emph{ВНП} = \emph{В}_\emph{вт} -
    \emph{П}_\emph{п} \), где \( \emph{В}_\emph{вт} \) -- валовый выпуск
    товаров, \( \emph{П}_\emph{п} \) -- промежуточное потребление.
\end{enumerate}
Все три метода равнозначны.

\( \emph{ЧНП} = \emph{ВНП} - \emph{Ам} \) -- чистый национальный продукт.

\( \emph{НД} = \emph{ЧНП} - \emph{КН} \) -- национальный доход.

\( \emph{ЛД} = \emph{НД} - (\emph{СС} + \emph{Н}_\emph{пр} + \emph{НП}) +
\emph{ТП} \) -- личный доход, \emph{СС} -- взносы на социальное страхование,
\emph{ТП} -- трансфертные платежи

\( \emph{РД} = \emph{ЛД} - \emph{Н}_\emph{ин} \) -- располагаемый доход,
\( \emph{Н}_\emph{ин} \) -- индивидуальный налог.

Существуют следующие виды ВНП (ВВП):
\begin{enumerate}
    \item номинальный -- ВНП (ВВП), измеряемый в текущих рыночных ценах;
    \item реальный -- ВНП (ВВП), измеряемый в ценах базисного года;
    \item \( \text{дефлятор ВНП} = \cfrac{\text{ВНП}_\text{ном}}
    {\text{ВНП}_\text{реал}} \) -- это индекс цен, показывающий изменение
    средней цены единицы продукта в текущем году по отношению к базисному году.
    Если дефлятор больше 1, то в обществе имеет место инфляция.
    Номинальный ВНП -- ВНП в ценах базисного года, реальный -- в ценах текущего.
\end{enumerate}