\section{Тензорная алгебра.}

\subsection{Сложение тензоров.}

	Пусть \( A = [a_{ij}] \) и \( B = [b_{ij}] \) -- два тензора. Образуем на основе их компонент числа \( c_{ij} \) по правилу:
	\begin{equation}
	c_{ij} = a_{ij} + b_{ij}. \label{eq11:1}
	\end{equation}
	
	Покажем, что набор чисел \( [c_{ij}] \) также является тензором.
	
	Так как \( [a_{ij}] \) и \( [b_{ij}] \) -- тензоры, то в новом базисе они преобразуются следующим образом:
	\begin{align}
	& a_{kl}{’} = \sum\sum \gamma_{ki}\gamma_{lj}a_{ij}; \nonumber \\
	& b_{kl}{’} = \sum\sum \gamma_{ki}\gamma_{lj}b_{ij}. \nonumber
	\end{align}
	
	Тогда:
	\[ c_{kl}{’} = a_{kl}{’} + b_{kl}{’} = \sum\sum \gamma_{ki}\gamma_{lj}(a_{ij} + b_{ij}) = \sum\sum \gamma_{ki}\gamma_{lj}c_{ij}, \]
	то есть числа \( c_{ij} \) преобразуются как компоненты тензора второго ранга.
	
	\begin{definition}
	Тензор \( C = [c_{ij}] \), компоненты которого образуются по правилу (\ref{eq11:1}), называется суммой двух тензоров: \( C = A + B \).
	\end{definition}
	
	Видно, что складывать можно только тензоры одного ранга.

\subsection{Полное произведение двух тензоров.}

	Пусть \( [a_{ij}] \) и \( [b_{kl}] \) -- два тензора. Образуем на основе их компонент числа \( c_{ijkl} \) по правилу:
	\begin{equation}
		c_{ijkl} = a_{ij}b_{kl}. \label{eq11:2}
	\end{equation}
	
	Покажем, что набор чисел \( [c_{ijkl}] \) также является тензором.
	
	Так как \( [a_{ij}] \) и \( [b_{kl}] \) -- тензоры, то в новом базисе они преобразуются следующим образом:
	\begin{align}
	& a_{pq}{’} = \sum\sum \gamma_{pi}\gamma_{qj}a_{ij}; \nonumber \\
	& b_{rs}{’} = \sum\sum \gamma_{rk}\gamma_{sl}b_{ij}. \nonumber
	\end{align}
	
	Тогда:
	\begin{align}
	c_{pqrs}{’} = a_{pq}{’} + b_{rs}{’} = & \sum\sum\sum\sum \gamma_{pi}\gamma_{qj}\gamma_{rk}\gamma_{sl}(a_{ij}b_{kl}) = \nonumber \\
	= & \sum\sum\sum\sum \gamma_{pi}\gamma_{qj}\gamma_{rk}\gamma_{sl}c_{ijkl}. \nonumber
	\end{align}
	то есть числа \( c_{ijkl} \) преобразуются как компоненты тензора четвертого ранга.
	
	\begin{definition}
	Тензор \( C = [c_{ijkl}] \), компоненты которого образуются по правилу (\ref{eq11:2}), называется полным произведением двух тензоров: \( C = AB \).
	\end{definition}
	
	Видно, что полное произведение не коммутативно, то есть \( AB \ne BA \). Действительно, \( (c_{ijkl} = a_{ij}b_{kl}) \ne (c_{klij} = b_{ij}a_{kl}) \).
	
	Выполнять операцию полного произведения можно с тензорами любого ранга, причем ранг полученного тензора будет суммой рангов исходных: \( \mathrm{rang}C = \mathrm{rang}A + \mathrm{rang}B \).
	
	Полное произведение не обозначается каким-либо символом, следовательно, полное произведение векторов записывается следующим образом:
	\[ C = \vec{a}\vec{b} = \begin{bmatrix}
	a_1b_1 & a_1b_2 & a_1b_3 \\
	a_2b_1 & a_2b_2 & a_2b_3 \\
	a_3b_1 & a_3b_2 & a_3b_3
	\end{bmatrix}. \]
	Поэтому в скалярном произведении векторов надо \textit{ставить} знак умножения, то есть: \( c = \vec{a}\cdot\vec{b} \).
	
	\textbf{Замечание о произведениях двух векторов.}
	\begin{enumerate}
	\item Скалярное произведение: \( \vec{a}\cdot\vec{b} = \vec{b}\cdot\vec{a} \) -- коммутативно. Результат -- скаляр.
	\item Векторное произведение: \( \vec{a}\times\vec{b} = -\vec{b}\times\vec{a} \) -- антикоммутативно. Результат -- вектор.
	\item Полное произведение: \( \vec{a}\vec{b} \ne \vec{b}\vec{a} \) -- некоммутативно. Результат -- тензор второго ранга.
	\end{enumerate}

\subsection{Свертывание тензоров.}

	Свертывание -- это операция суммирования компонент тензора по каким-либо двум индексам одновременно. Ранг тензора при этом понижается на два, поэтому свертывать можно тензоры рангом не ниже второго.
	
	Операцию свертывания можно применять несколько раз. Тогда сверткой тензора четного ранга может быть скаляр, а нечетного -- вектор.
	
	\begin{example}
	
	Сверткой тензора второго ранга \( [a_{ij}] \) является скаляр \( c \):
	\[ c = \sum a_{ii} = a_{11} + a_{22} + a_{33}. \]
	\end{example}
	
	\begin{example}
	
	Сверткой тензора третьего ранга \( [a_{ijk}] \) являются три различных вектора \( \vec{b} \), \( \vec{c} \) и \( \vec{d} \), компоненты которых находятся следующим образом:
	\begin{align}
	& b_k = \sum a_{iik}; \nonumber \\
	& c_k = \sum a_{iki}; \nonumber \\
	& d_k = \sum a_{kii}. \nonumber 
	\end{align}
	
	Покажем, что свертка тензора так же является тензором на примере \( b_k \).
	
	В новом базисе:
	\[ b{'}_r = \sum\limits_p a_{ppr}. \]
	
	Так как \( [a_{ijk}] \) -- тензор, то
	\[ a{'}_{ppr} = \sum\limits_i \sum\limits_j \sum\limits_k \gamma_{pi}\gamma_{pj}\gamma_{rk} a_{ijk}. \]
	
	Тогда
	\[ b{'}_r = \sum\limits_p a_{ppr} = \sum\limits_p \sum\limits_i \sum\limits_j \sum\limits_k \gamma_{pi}\gamma_{pj}\gamma_{rk} a_{ijk} = \sum\limits_i \sum\limits_j \sum\limits_k \left( \sum\limits_p \gamma_{pi}\gamma_{pj} \right) \gamma_{rk} a_{ijk}. \]
	
	А так как матрица \( \Gamma = [\gamma_{ij}] \) -- ортогональная, то
	\[ \sum\limits_p \gamma_{pi}\gamma_{pj} = \delta_{ij}. \]
	
	Следовательно,
	\[ b{'}_r = \sum_i\sum_k \gamma_{rk} a_{iik} = \sum\limits_k \left( \sum\limits_i a_{iik} \right) \gamma_{rk} = \sum\limits_k \gamma_{rk} b_k, \]
	то есть совпадает с определением тензора первого ранга.
	
	Таким образом, результат свертки тензора так же является тензором.
	\end{example}

\subsection{Скалярное произведение двух тензоров.}

	Скалярное произведение двух тензоров -- это операция их полного умножения с последующим свертыванием по индексам, относящимся к разным множителям.
	
	Например, скалярное произведение тензора третьего ранга \( [a_{Ijk}] \) на тензор первого ранга \( [b_i] \) со свертыванием по последним индексам дает тензор второго ранга:
	\[ c_{ij} = \sum\limits_k a_{ijk}b_k. \]
	
	Чаще всего встречаются следующие скалярные произведения:
	\begin{enumerate}
	\item Скалярное произведение двух тензоров первого ранга (векторов). Результатом является скаляр:
		\[ \vec{a}\cdot\vec{b} = \sum\limits_i a_ib_i = c. \]
		
	\item Скалярное произведение тензора второго ранга на тензор первого ранга со свертыванием по последним индексам. Результатом является вектор:
		\[ A\cdot\vec{b} = \sum\limits_j a_{ij}b_j= c_i. \]
		
		Это известная операция умножения матрицы \( [a_{ij}] \) на вектор \( [b_j] \) по правилу ``строка-на-столбец''.
		
	\item Скалярное произведение двух тензоров второго ранга с последующей сверткой по паре разных позиционных индексов. В результате получается новый тензор второго ранга:
		\[ A\cdot B = \sum\limits_j a_{ij}b_{jk} = c_{ik}. \]
		
		Это известная операция умножения матрицы \( [a_{ij}] \) на матрицу \( [b_{jk}] \) по правилу ``строка-на-столбец''.
	\end{enumerate}
