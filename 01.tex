\chapter{Статистическая модель атома Томаса-Ферми}

Рассмотрим простейшую модель атома Ферми.

Вычислим потенциал атома, экранированного своими же электронами. Предполагается,
что в атоме электроны движутся таким образом, что каждому атому можно приписать
собственные кинетическую и потенциальную энергии. Функция распределения имеет
вид:
\[
  f = \frac{1}{\exp[\frac{E - \zeta}{kT}] + 1},
\]
где \( \zeta = U(r) + \frac{p_F^2(r)}{2m} \) -- потенциальная энергия, а
потенциал атома имеет вид \( U(r) = -e\phi(r) \). Под \( p_F(r) \) понимается
максимальный импульс электрона, так называемый импульс Ферми.

Найдем концентрацию электронов в зависимости от \( r \). Из принципа
неопределенности:
\[
  \Delta p\Delta V \geq (2\pi\hbar)^3.
\]

Рассматривая объем фазового пространства, и учитывая, что в одном квантовом
состоянии могут находиться два электрона с противоположными спинами, запишем
число квантовых состояний:
\[
  dN_{r, k} = 2\frac{dV}{(2\pi)^3}4\pi k^2\,dk,
\]

Проинтегрировав это уравнение от \( 0 \) до \( k_F \) получим:
\[
  dN_r = 2\frac{dV}{(2\pi)^3} \frac{4\pi}{3} k_F^3, \quad k_F^3 = 3\pi^2 n(r),
    \quad n(r) = \der{N_r}{V}.
\]

Введем замену \( U - \zeta = U_1 \). Тогда \( \phi(r) \):
\[
  \phi(r) = -\frac{1}{e} U_1(r),
\]
и концентрация электронов
\[
  n(r) = \frac{k_F^3}{2\pi^2} = \frac{1}{3\pi^2} \frac{(2m)^{3 / 2}}{\hbar^3}
    \bigl(-U_1(r)\big)^{3 / 2}.
\]

Подставим \( \phi \) в уравнение Пуассона:
\[
  \Delta\phi = -4\pi\rho, \text{ где } \rho(r) = -en(r),
\]
получим уравнение Томаса-Ферми:
\[
  \frac{1}{r} \dder{}{r}\Bigl(rU_1\Big) = -\frac{4e^2}{3\pi\hbar^3}
    (2m)^{3 / 2}\bigl(-U_1(r)\big)^{3 / 2}.
\]

Введем переменную \( x \):
\[
  x = \frac{r}{ba_0Z^{-1 / 3}},
\]
где константа \( b \) имеет следующий вид:
\[
  b = \frac{(3\pi)^{2 / 3}}{2^{7 / 3}} \quad
  \left(ba_0Z^{-2 / 3} = 0,\!88534a_0Z^{-1 / 3}\right);
\]
и переменную \( \Phi(r) \)
\[
  \Phi(r) = \frac{U_1(r)}{-Ze^2/r},
\]
показывающую во сколько раз потенциальная энергия электрона больше энергии
атома. Тогда производная по \( r \) преобразуется в:
\[
  \der{}{r} = \der{}{x} \left( \frac{1}{ba_0Z^{-1 / 3}} \right).
\]

Преобразуем уравнение Томаса-Ферми, выразив \( U_1 \) из \( \Phi(r), \) и
подставив в рассматриваемое уравнение:
\[
  \dder{}{r}\Phi(r)= -\frac{4e^2(2m)^{3 / 2}}{3\pi\hbar^3} \Phi^{3 / 2}(r)
    \cdot \left( -\frac{Ze^2}{r} \right)^{3 / 2}.
\]

Выразим \( r \) через \( x \) и подставим в рассматриваемое уравнение:
\begin{gather*}
  r = xba_0Z^{-1 / 3}, \\
  \dder{}{r} \Phi(r) = \frac{4e^2(2m)^{3 / 2}}{3\pi\hbar^3} \Phi^{3 / 2}(r)
    \left( \frac{Ze^2}{xba_0} \right)^{1 / 2} =
    \frac{4e^2(2m)^{3 / 2}}{3\pi\hbar^3} \Phi^{3 / 2}(r) Z^{1 / 3}
    \left( \frac{e^2}{xba_0} \right)^{1 / 2}, \\
  \dder{}{r}\Phi(r) = \frac{\Phi^{3 / 2}(r)}{x^{1 / 2}} \frac{4e^2(2m)^{3 / 2}}
    {3\pi\hbar^3} Z^{1 / 3} \left( \frac{e^2}{ba_0} \right)^{1 / 2}, \\
  \left( ba_0Z^{-1 / 3} \right)^{-2} \dder{\Phi(x)}{x} =
    \frac{\Phi^{3 / 2}(x)}{x^{1 / 2}} \frac{2^{7 / 2} e^4 m^{3 / 2}}
    {3\pi\hbar^3} Z^{1 / 3} \left( \frac{e^2}{ba_0} \right)^{1 / 2}, \\ 
  \dder{\Phi(x)}{x} = \frac{\Phi^{3 / 2}(x)}{x^{1 / 2}}.
\end{gather*}
