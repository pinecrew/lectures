\chapter{Атомный потенциал}

  \section{Статистическая модель атома Томаса--Ферми}

  Прежде чем вычислять сечения рассеяния электронов, необходимо знать
  экранированный электронами потенциал ядра~--- атомный потенциал~--- и волновые
  функции электронов. Аналитическими методами теории возмущений эту задачу
  удается решить только для водорода и гелия. Для более сложных атомов
  применяется метод самосогласованного поля Хартри--Фока. Метод Хартри--Фока в
  квантовой механике~--- это приближенный метод решения уравнения Шредингера
  путем сведения многочастичной задачи к одночастичной в предположении, что
  каждая частица движется в некотором усредненном самосогласованном поле,
  создаваемом всеми остальными частицами системы. В квантовой электродинамике
  метод Хартри--Фока для решения уравнения Дирака вместо уравнения Шредингера
  автоматически включает спиновые взаимодействия. При вычислении сечений
  рассеяния это позволяет учесть поляризацию спинов вдоль или против направления
  движения пучка падающих на атом электронов и продвинуться в область энергий на
  порядок меньших, чем в нерелятивистской квантовой механике.
  
  \section{Статистическая модель атома Томаса--Ферми}

  Статистическая модель вещества (модель Томаса--Ферми) составляет основу
  приближенного подхода, который широко применяется для аналитического описания
  свойств вещества на различных его иерархических уровнях. В последние годы
  быстро развивается новый подход в теории конденсированных сред~--- метод
  функционала плотности, близко примыкающий к методу Томаса--Ферми и, в
  сущности, являющийся его развитием.
  
  В методе Томаса--Ферми вычисляется потенциальная энергия электрона в
  сферически симметричном поле ядра, экранированного электронами атома, и затем
  с помощью уравнения Пуассона выражается через плотность электронов в атоме.
  Решение полученного таким способом уравнения дает потенциал атома в форме, не
  зависящей от атомного номера вещества~--- функцию экранирования поля ядра
  атомными электронами. Метод Томаса--Ферми применяется для расчета исходных
  пробных потенциалов в методе самосогласованного поля Хартри--Фока.
  
  Рассмотрим систему электронов, движущихся в объеме \( V_0 \) в поле со
  сферически симметричным потенциалом \( \phi(r) \). Будем считать, что к
  системе можно применять статистику Ферми--Дирака и говорить о кинетической и
  потенциальной энергии каждого отдельного электрона.
  
  Функция распределения имеет вид:
  \begin{equation}
    f = \frac{1}{\exp[(E - \zeta) / (kT)] + 1},
    \label{eq:1.1.1}
  \end{equation}
  где \( \zeta \)~--- химический потенциал, \( E \)~--- энергия электрона.
  Химический потенциал (энергия Ферми), очевидно, выглядит следующим образом:
  \begin{equation}
    \zeta = U(r) + \frac{p_F^2(r)}{2m},
    \label{eq:1.1.2}
  \end{equation}
  где \( U(r) = -e\phi(r) \)~--- потенциальная энергия электрона,
  \( p_F(r) \)~--- максимальный импульс электронов (импульс Ферми).
  
  Получим выражение, связывающее импульс Ферми \( p_F(r) \) с плотностью
  электронов \( n(r) \). Для этой цели вычислим число квантовых состояний
  поступательного движения полностью свободного электрона, которым отвечают
  абсолютные значения импульса в интервале от \( p \) до \( p + dp \). Пусть
  электрон движется в ячейке объемом \( dV \) в отсутствие каких-либо сил.
  Искомое число квантовых состояний равно:
  \begin{equation}
    dN_{r, k} = 2\frac{dV}{(2\pi)^3} 4\pi k^2\,dk,
    \label{eq:1.1.3}
  \end{equation}
  где \( \hbar k = p \), а множитель \( 2 \) учитывает две возможные ориентации
  спина электрона.
  
  Интегрируя выражение \eqref{eq:1.1.3} от \( 0 \) до \( k_F \) и приравнивая
  результат к полному числу электронов \( dN_r \) в объеме \( dV \), получаем:
  \begin{equation}
    dN_r = 2\frac{dV}{(2\pi)^3} \frac{4\pi}{3}k_F^3, \quad k_F^3 = 3\pi^2 n(r),
      \quad n(r) = \der{N_r}{V}.
    \label{eq:1.1.4}
  \end{equation}
  
  Произведем теперь калибровочное преобразование потенциальной энергии
  \( U - \zeta \to U_1 \). Тогда из формул \eqref{eq:1.1.2} и \eqref{eq:1.1.4}
  следует, что
  \begin{equation}
    n(r) = \frac{k_F^3}{3\pi^2} = \frac{1}{3\pi^2} \frac{(2m)^{3 / 2}}{\hbar^3}
      \bigl[-U_1(r)\big]^{3 / 2}.
    \label{eq:1.1.5}
  \end{equation}
  
  \subsection{Уравнение Томаса--Ферми}
  
  Уравнение Пуассона \( \D\phi = 4\pi\rho \) связывает электростатический
  потенциал \( \phi(r) = -U_1(r) / e \) с плотностью заряда
  \( \rho(r) = -en(r) \) и для кинетической и потенциальной энергий электрона в
  атоме приводится к виду
  \begin{equation}
    -\D U_1 = \frac{1}{2}\D p_F^2 = 4\pi e^2 n.
    \label{eq:1.1.6}
  \end{equation}
  
  Подставляя сюда выражение \eqref{eq:1.1.5}, получаем уравнение Томаса--Ферми
  \begin{equation}
    \frac{1}{r}\dder{}{r}\bigl(rU_1\big) = -\frac{4e^2}{3\pi\hbar^3}
      \bigl(2m\big)^{3 / 2} \bigl(-U_1\big)^{3 / 2}.
    \label{eq:1.1.7}
  \end{equation}
  
  Произведем замену переменных
  \[
    x = \frac{r}{ba_0Z^{-1 / 3}}, \quad \Phi(r) = \frac{U_1(r)}{-Ze^2 / r},
  \]
  где
  \[
    b = \frac{(3\pi)^{2 / 3}}{2^{7 / 3}} \quad
      \left(ba_0Z^{-2 / 3} = 0,\!88534a_0Z^{-1 / 3}\right);
  \]
  \( a_0 = \hbar^2 / me^2 = 5,\!28\cdot 10^-9 \)~см~--- первый боровский радиус.
  Это означает, что единицей длины выбрана величина \( ba_0 Z^{-1 / 3} \),
  единицей энергии выбрана потенциальная энергия электрона в поле ядра атома на
  расстоянии \( r \) от него: \( - Ze^2 / r \). Функция \( \Phi(r) \)
  показывает, во сколько раз потенциальная энергия электрона в атоме на
  расстоянии \( r \) от ядра меньше потенциальной энергии электрона на таком же
  расстоянии от <<голого>> ядра и имеет смысл функции экранирования поля ядра
  электронами атома.
  
  Приведем уравнение \eqref{eq:1.1.7} к новым переменным. Выразим \( U_1 \) из
  \( \Phi(r), \) и подставим в рассматриваемое уравнение:
  \[
    \dder{}{r}\Phi(r)= -\frac{4e^2(2m)^{3 / 2}}{3\pi\hbar^3} \Phi^{3 / 2}(r)
      \cdot \left( -\frac{Ze^2}{r} \right)^{3 / 2}.
  \]

  Выразим \( r \) через \( x \), учитывая, что производная по \( r \) равна
  \[
    \der{}{r} = \der{}{x} \left( \frac{1}{ba_0Z^{-1 / 3}} \right),
  \]
  и подставим в рассматриваемое уравнение:
  \begin{gather*}
    r = xba_0Z^{-1 / 3}, \\
    \dder{}{r} \Phi(r) = \frac{4e^2(2m)^{3 / 2}}{3\pi\hbar^3} \Phi^{3 / 2}(r)
      \left( \frac{Ze^2}{xba_0} \right)^{1 / 2} =
      \frac{4e^2(2m)^{3 / 2}}{3\pi\hbar^3} \Phi^{3 / 2}(r) Z^{1 / 3}
      \left( \frac{e^2}{xba_0} \right)^{1 / 2}, \\
    \dder{}{r}\Phi(r) = \frac{\Phi^{3 / 2}(r)}{x^{1 / 2}}
      \frac{4e^2(2m)^{3 / 2}}{3\pi\hbar^3} Z^{1 / 3} \left( \frac{e^2}{ba_0}
      \right)^{1 / 2}, \\
    \left( ba_0Z^{-1 / 3} \right)^{-2} \dder{\Phi(x)}{x} =
      \frac{\Phi^{3 / 2}(x)}{x^{1 / 2}} \frac{2^{7 / 2} e^4 m^{3 / 2}}
      {3\pi\hbar^3} Z^{1 / 3} \left( \frac{e^2}{ba_0} \right)^{1 / 2}, \\ 
    \dder{\Phi(x)}{x} = \frac{\Phi^{3 / 2}(x)}{x^{1 / 2}}.
  \end{gather*}

  Таким образом, в новых переменных уравнение \eqref{eq:1.1.7} принимает вид
  \begin{equation}
    \dder{\Phi}{x} = \frac{\Phi^{3 / 2}}{x^{1 / 2}}
    \label{eq:1.1.8}
  \end{equation}
  и должно решаться с граничными условиями
  \begin{equation}
    \Phi(0) = 1; \qquad \Phi(\infty) = 0.
    \label{eq:1.1.9}
  \end{equation}
  Заметим, что в уравнение Томаса--Ферми не входят какие-либо параметры
  вещества. Функция экранирования \( \Phi(x) \) в модели Томаса--Ферми является
  одинаковой для всех атомов.

  Сделаем некоторые оценки. Численные расчеты показывают, что более половины
  электронов в атоме находятся внутри сферы радиуса \( 1,33a_0Z^{-1 / 3} \).
  Средняя скорость электронов в атоме \( v_{a\nu} \) по порядку величины равна
  скорости электрона в атоме водорода \( v_0 \), умноженной на \( Z^{2 / 3} \),
  поскольку 
  \[ 
    v \sim \sqrt{E} \approx \sqrt{U} \sim \sqrt{\frac{Z}{Z^{-1 / 3}}} =
    Z^{2 / 3},
  \]
  так что \( v_{a\nu} \approx v_0Z^{2 / 3} \).

  Заметим, что \( v_0 = e^2/h = \alpha / c \), где
  \( \alpha \approx 1/137 \)~--- постоянная тонкой структуры; \( h \)~---
  постоянная Планка; \( c \)~--- скорость света.

  Уравнение Томаса--Ферми применимо в области \( 1/Z \lesssim r \lesssim 1 \),
  так как квазиклассическое приближение, которое лежит в основе принятой картины
  движения электронов в поле ядра, когда можно говорить о кинетической и
  потенциальной энергии каждого отдельного электрона, нарушается как при слишком
  малых, так и при больших расстояниях от ядра.

  \section{Решение уравнения Томаса--Ферми}

  Уравнение Томаса--Ферми не имеет аналитического решения и является жестким,
  то~есть его численное решение сильно зависит от малых изменений граничных
  условий. Ниже описан способ численного решения уравнения Томаса--Ферми.
	
  Приведем уравнение \eqref{eq:1.1.8} к системе двух обыкновенных
  дифференциальных уравнений первого порядка
  \begin{equation}
    \der{\Phi}{x} = \xi, \qquad \der{\xi}{x} = \frac{\Phi^{3 / 2}}{\sqrt{x}}.
    \label{eq:1.1.10}
  \end{equation}