\chapter{Культура Древней Греции}
\section{Характерные особенности древнегреческой культуры}

  В качестве одной из определяющих особенностей греческой культуры выступает
  космоцентризм~-- восприятие вселенной как единого целого. По мнению грека,
  космос есть порядок, истина, красота, некий абсолют. Противоположностью
  космоса выступает хаос.
  
  Космос населен богами, определяющих судьбу человека (фаталистическое начало);
  причем боги наделены человеческими качествами, в том числе и пороками
  (достаточно вспомнить похождения Зевса и оставшихся от них полубогов). Эта
  особенность называется антропоморфностью.
  
  Однако, появляются полубоги, изменяющие свою судьбу и судьбу человечества~--
  героическое начало.
  
  Еще одной особенностью древнегреческой культуры является постановка человека в
  центр космоса~-- антропоцентризм.
  
  \charskip{*}
  
  Культура носит пластичный, телесный характер. Большое внимание уделяется
  развитию физической стороны тела. В сочетании с соревновательным духом того
  времени это дает развитию различных игр: Олимпийских, Дельфийских, Пифийских и
  многих других. В их основе лежит агонистика~-- принцип честного и благородного
  противоборства как отдельных личностей, так и социальных групп, и даже
  полисов.
  
  Первые Олимпийские игры датируются летом 776~г. до н.~э. С этого момента
  начинается летоисчисление греков.
  
  Еще одной особенностью древнегреческой культуры является рационализм.

\section{Периодизация древнегреческой культуры}

  \begin{enumerate}
    \item Эгейская эпоха: 20--15~вв. до н.~э.
      \begin{enumerate}
        \item Крито-минойский период: до 15~в. до н.~э.
        
          Первые исследования на Крите начались в 1900~г. А.~Эвансом.
          Обнаруженная культура была морской, ее основой был торговый флот. Она
          имела ряд сходств с древневосточными культурами:
          \begin{itemize}
            \item наличие жрецов и бюрократии;
            \item теократический характер власти;
            \item наличие рабства.
          \end{itemize}
          Однако, не было обнаружено ни одного сакрального сооружения, что резко
          отличает эту культуру от древневосточной. Все обряды проводились во
          дворцах; дворцы были центрами цивилизации, местами хранения всех
          богатств.
          
          Самые известные дворцы: Кносский дворец, дворец в Фесте, Малия,
          Като-Закро.

        \item Крито-микенский период: 15--13~вв. до н.~э.
        
          После природного катаклизма
      \end{enumerate}
  \end{enumerate}

  \begin{thebibliography}{9}
    \addcontentsline{toc}{section}{Список литературы}
    \itemsep -.2em
    \bibitem{1} \href{http://padabum.com/x.php?id=9471}{Древние цивилизации.
      Под~ред. Бонгард-Левина}
    \bibitem{2} \href{http://dwl.alleng.ru/d_ar/hist_vm/hist009_1.zip}{Боннар~А.
      Древнегреческая цивилизация (в 3~частях)}
    \bibitem{3} \href{http://rghost.ru/download/48161358/%
      1c27882309bef2be09ca96fbc9283bea85743dcd/hist_vm040.zip}{Винничук Л. Люди,
      нравы и обычаи Древней Греции и Рима}
    \bibitem{4} \href{http://vk.cc/2j6Z3y}{Куманецкий К. История культуры
      Древней Греции и Рима}
    \bibitem{5} \href{http://rghost.ru/download/48076731/%
      a45295d3ba73f1541079ebf3b1f35a3f9b3e8574/cult013.zip}{Словарь античности}
  \end{thebibliography}
