\subsection{Магнетизм атомов}
Согласно классическим представлениям магнетизм был обусловлен микротоками, циркулирующими внутри частиц вещества (гипотеза Ампера). Природу этих токов классическая теория не могла установить. Она была лишь отчасти установлена с появлением электронных представлений о строении вещества (теория Бора). Считалось, что амперовские микротоки обусловлены движением электронов внутри атома, однако, объяснить законы движения этих электронов классическая физика не смогла. Кроме того, методами классической статистической физики было показано, что в установившемся режиме вещество не может быть намагничено, то есть если намагниченное вещество предоставить самому себе поддерживая его температуру постоянной, то оно самопроизвольно придёт в равновесное состояние, в котором намагниченность отсутствует даже если вещество помещено в магнитное поле. Но это противоречит опытным фактам.
Понимание природы магнетизма пришло только после создания квантовой механики -- магнетиз является квантовым эффектом. Найдём магнитный момент атома.
\[
    \vec{M} = I\cdot\vec{S},\ \vec{S} = \frac{1}{2}\oint\vec{r}\times d\vec{r}.
\]
\[
    \vec{M} = \frac{1}{2m}\oint \vec{r}\times\vec{p} dq.
\]
Из посленей формулы видно, что в ней нет ограничений на геометрию тока в атоме -- осталась лишь система зарядов, каждый из которых характеризуется положением и скоростью движения.
Для одного заряда получим
\[
    \vec{M} = \frac{q}{2m} \vec{r}\times\vec{p},\ \hat{\vec{M}} = \frac{q}{2m}
    \hat{\vec{r}}\times\hat{\vec{p}}.
\]

\[
    \vec{M} = \frac{q}{2m}\vec{L},\ \vec{M} = \Gamma\vec{L}.
\]

\( \Gamma \) -- гиромагнитный момент. Для электрона
\[
    \vec{M} = -\frac{e}{2m}\vec{L}.
\]

Модуль магнитного момента очевидно может быть определён только с одной из проекций.
\[
    M_z = -\frac{e}{2m_e}L_z = -\frac{e\hbar}{2m_e}m.
\]

То есть проекция магнитного момента является дискретной величиной. Величина
\[
    \mu_\text{Б} = \frac{e\hbar}{2m_e} = 0.927\cdot10{-23}
\]
магнетон Бора -- квант магнитного момента.

\subsubsection{Опыты Штерна и Герлаха}
Наличие у атомов магнитных моментов и их квантование было обнаружено в опытах Штерна и Герлаха. В сосуде с высоким вакуумом создавался атомнуй пучок. С помощь диафрагмы пучок был определённым образом ограничен в пространстве. С помощью особой конструкции электромагнита на пути атомного пучка создавалось резко неоднородное магнитное поле.
\[
    \M_x\pder{B_z}{x}vec{f} = (\vec{M}\cdot\nabla)\vec{B},\ f_z = M_x\pder{B_z}{x} + M_y\pder{B_z}{y} + M_z\pder{B_z}{z},\ \average{f} = M_z\pder{B_Z}{z}.
\]
В следствие прецесии первыми двумя слагаемыми можно пренебречь. Заметим, что формула для среднего справедливо и  квантовой механике.Включение сильного магнитного поля вдоль оси z приводит к определённому состоянию атомы с определённым значением \( \M_z \). В ходе экспериментов было установлено, что на фотопластинке наблюдалось как чётное число точек, так и нечётное число.

Были и другие опыты, которые противоречили квантовомехаханическому объяснению магнетизма.

\subsubsection{Опыт Эйнштейна - де Гааза}
1915 г. Под действием магнитного поля соленоида парамагнетик намагничивается, то есть у электронных оболочек атомов возникает магнитный момент. Следовательно, увеличивается их механический момент. Намагничивание происходит засчёт столкновений атомов между собой. При столкновених возникают внутрение, которые не совершают работу, то есть момент импульса не должен изменияться. Следовательно, кристаллическая решётка должна получить момент импульса противоположного знака и стержень должен придти во вращение.
\subsubsection{Опыт Барнета}
Обратный магнито-механический эффект -- раскрученный стержень намагничивается.

Из этих опытов был получен гиромагнитный момент -- он был ровно в 2 раза больше теоретического значения! Также квантовая механика на тот момент не могла объяснить тонкую структуру. В 1925 году Уленбек и Гаудсмит выдвинули гипотезу о спине электрона. Электрон имеет не только момент количества движения и магнитный момент связанные с движением частицы как целого, но и собственный механический момент количества движения. Этот собственный механический момент назвали спином. Спин -- это существенное квантовое понятие, не имеющее классического аналога. Он является одним из фундаментальных свойств частицы наряду с массой и зарядом. 
\[
    L_s = \hbar\sqrt{s(s+1)},\ L_{sz} = m_s h,\ m_s = \pm s.
\]
Для электронов имеем:
\[
    2s + 1 = 2 \rightarrow s = \frac{1}{2}.
\]
В опыте Штерна и Герлаха с атомами водорода пучок разделялся на 2 составляющие. Атомы водорода находились в s состоянии, следовательно орбитальный момент равен 0. Следовательно расщепление пучка обусловлено не орбитальным, а спиновым магнитным моментом. При этом мы считаем, что момент остова пренебрежимо мал по сравнению с моментом валентного электрона.
\[
    L_s = \frac{\sqrt{3}}{2}\hbar, \ L_{sz} = \pm\frac{\hbar}{2}.
\]
С собственным механическим моментом всегда связн собственный магнитный момент. Соотношение было установлено при помощи магнито-механических эффектов: \( M_s = -\frac{e}{m}L_s \). Отсюда
\[
    M_s = -\sqrt{3}\mu_\text{Б},\ M_{sz} = \mu_\text{Б}.
\]
\[
    L_j = \hbar\sqrt{j(j+1)},\ j = |l\pm\frac{1}{2}|.
\]
\[
    L_{jz} = m_j\hbar,\ m_j = m_l + m_s.
\]
Рассмотрим первые два состояния электрона в атоме водорода:
\begin{itemize}
\item l = 0, j = \frac{1}{2}, m_j = \pm\frac{1}{2}
\item l = 1, j_1 = \frac{3}{2} ...
\end{itemize}

Рассмотрим на примере лития как с помощью понятия спина можно объяснить тонкую структуру энергетических уровней щелочных металлов. Тонкая структура обусловлена спин-орбитальным взаимодействием. Само же это взаимодействие обусловлено взаимодействием между спином электрона и зарядом ядра. Природу этого взаимодействия можно наглядно представить на примере атома Бора: электрон, вращающийся по орбите обладает спином и тем самым обладает спиновым магнитным моментом. Электрическое поле ядра оказывает воздействие на спиновый магнитный момент. В этом легко убедиться, если перейти в систему отсчёта движущегося электрона -- в ней электрон покоится, а ядро движется. Следовательно, электрон находится в магнитном поле движущегося ядра и его спиновый магнитный момент взаимодействует с этим полем. Энергия этого взаимодействия равна \( -\vec{M}\cdot\vec{B} = -M_{sz}B_z \). Так как \( M_{sz} \) может принимать лишь два значенния, то возникает расщепление уровня на два подуровня за исключением s состояния.
\[
  ^{2s+1}L_j
\]
В квантовой механике доказывается правило отбора для полного квантового числа j: \( \Delta j = 0, \pm1 \).
Дуплеты обусловлены расщеплением \(p\)-уровней. Так как для различных уровней расщепление различно, следовательно будет различным расщепление дуплетов.
В резкой серии расщепление всех линий одинаково, так как оно обусловлено расщеплением одного и того же уровня 2p.
В диффузной серии наблюдаются триплеты (сложные дуплеты), так как есть 3 возможных перехода. Однако две линии сливаются в одну.

Понятии об уравнении Дирака

Для атомов водорода
\[
    \frac{\Delta E}{E} = \alpha^2,
\]

где \( \alpha \) -- постоянная тонкой структуры.
Релятивистское волновое уравнение должно удовлетворять следующим требованиям:
\begin{itemize}
\item должно быть инвариантным по отношению к преобразованиям Лоренца;
\item должно быть линейным, для выполнения принципа суперпозиций волновых функций;
\item должно быть симметричным относительно пространства и времени;
\item из него должно получаться соотношение \( E^2 = E_0^2 + p^2c^2 \)
\end{itemize}

Для того, чтобы удовлетворить всем этим условием волновая функция должна быть многокомпонентной.
Следствия из уравнения Дирака:
\begin{itemize}
\item электрон должен обладать собственным моментом импульса -- спином;
\item для частиц должны существовать античастицы;
\item была получена точная формула для тонкой структуры \[ E = -\frac{k^2e^4Z^2m_e}{2\hbar^2}\frac{1}{n^2}[1+...] \] 
\end{itemize}

\subsection{Результирующий механический момент многоэлектроннго атома}
\( M_{li} \) -- орбитальный механический момент.
\( M_{si} \) -- собственный механический момент.
\( M_{j} \) -- суммарный механический момент.

\begin{enumerate}
    \item \( LS \)-связь. Орбитальные моменты взаимодействуют между собой сильнее, чем с собственными мехаическими моментами. В результате все механические орбитальные моментвы складываются в суммарный механический момент, а все собственные моменты складываются в суммарный собственный момент. Полный момент атома определяется как их сумма. Данный тип связи осуществляется когда электростатическое взаимодействие между электронами превисходит спин-орбитальное взаимодействие. Такой вид вззаимодействия встречается чаще всего. \( M_j = \hbar\sqrt{J(J+1)},\ M_l = \hbar\sqrt{L(L+1)},\ M_s = \hbar\sqrt{S(S+1)} \).

    В случае двух электронов \( L = l_1+l_2, \ldots, |l_1-l_2| \). Аналогично для остальных.
    Рассмотрим орбитальные механические моменты:
    \[
        M_1 = \hbar\sqrt{l_1(l_1+1)},\ M_2 = \hbar\sqrt{l_2(l_2+1)},
    \]
    \[
        m_{l1} = 0, \pm1, \ldots, \pm l_1,\ m_{l2} = 0, \pm1, \ldots, \pm l_2
    \]
    \[
        \vec{M}_L = \vec{M}_1 + \vec{M}_2.
    \]
    Он может принимать одно из \( (2l_1 + 1)(2l_2 + 1) \) направлений. В соответствии с общими правилами квантовой механики
    \[
        M_L = \hbar\sqrt{L(L+1)},\ M_{Lz} = m_l\hbar,\ m_L = 0, \pm1, \ldots, \pm L.
    \]
    \[
        L_{max} = l_1 + l_2,\ L_{min} = |l_1 + l_2|.
    \]

    Для трёх и более частиц \( L_{max} = \suml_i \). Для нахождения минимального l сначала находится минимальное для двух частиц. Затем каждый из полученных результатов анализируется с третьим и выбирается минимальнное значение. И так далее.
    \item \( jj \) - связь. При данной связи в первую очерередь проявяется спин-орбитальное взаимодействие. Каждая пара моментов отдельной частицы взаимодейчтвуют между собой сильнее, чем пара моментов другой частицы. Поэтому сначала находится полный механический момент отдельной частицы, а затем результирующий механический момент как сумма механических моментов отдельных частиц. Данный вид связи характерен для тяжёлых металлов и встречается достаточно редко.
\end{enumerate}

\subsubsection{Условное обозначение термов многоэлектронных атомов}
\( ^{2S+1}L_J \). Мультиплетность \( 2\min(L, S) + 1 \).

\subsection{Магнитный момент многоэлектронного атома. Векторная модель атома}
Как ранее неоднократно отмечалось, механический и магнитный моменты связаны через гиромагнитное отношение. В случае спиновых моментов это отношение в 2 раза больше.
\( \mu_L \) -- орбитальный магнитный момент. \( \mu_L = - \mu_\text{Б}\sqrt{L(L+1)}. \)
\( \mu_{Lz} = m_L\mu_\text{Б}. \) Для спинового магнитного момента
\( \mu_L = -2\mu_\text{Б}\sqrt{S(S+1)}. \) \( \mu_{Sz} = 2m_S\mu_\text{Б}. \)

\( \mu_J = - \mu_\text{Б}g\sqrt{J(J+1)},\) где \( g \) -- фактор Ланде:
\[
    g = 1 + \frac{J(J+1) + S(S+1) - L(L+1)}{2J(J+1)}.
\]
\( \mu_{Jz} = m_J g \mu_\text{Б}. \)

В ядернй физике все вектора прецессирующие, то есть определена проекция на выделенное направление и длина. В изолированном атоме модуль механического момента должен сохраняться. Найдём полный магнитный момент атома с помощью векторной модели на примере \( LS \)-связи. 

Из-за удвоеного магнетизма спина результирующий магнитный момент начинает прецессировать вокруг направления полного механического момента. Из-за высокой скорости прецессии будет регистрироваться среднее значение магнитного момента.

\[
    \average{\mu_{Jz}} = -1|\mu_L|\cos\alpha - |\mu_J|\cos\beta.
\]
\[
    \cos\alpha = \frac{M_L^2 + M_J^2 - M_S^2}{2M_LM_J},\ \cos\beta = \frac{M_S^2 + M_J^2 - M_L^2}{2M_SM_J}.
\]

\[
    \average{\mu_{Jz}} = -\mu_\text{Б}(1 + \frac{J(J+1) + S(S+1) - L(L+1)}{2J(J+1)}) = -\mu_\text{Б} g \sqrt{J(J+1)} = \mu_J.
\]

Применим векторную модель для установления правил отбора. Рассмотрим процесс излучения фотонов. Применим закон сохранения момента импульса.
\[
    \vec{M}_J = \vec{M}'_J + \vec{S}_{ph},\ S_{ph} = \hbar\sqrt{2}.
\]
Квантово-механический расчёт показывает, что момент импульса фотона определяется его спиновой составляющей. Русским физиком Садовским было установлено, что спиновое число фотона равно 1. Из неравенства треугольника очевидно.
\( J \) -- полуцелое при нечётном числе электронов и целым при чётном числе электронов. Так как в процессе излучения количество электронов не изменяется, то \( \Delta J \in Z \).

\subsection{Эффект Зеемана}
Под эффектом Зеемана понимается расщепление энергетических уровней под действием магнитного поля. Рассмотрим крайние случаи: слабое магнитное поле и сильное магнитное поле. Рассматривать будем \( LS \)-связь. В слабом магнитном поле спин-орбитальное взаимодействие сильнее, чем воздействие магнитного поля. При этом происходят 2 вида прецесии -- прецессия орбитального и спинового моментов вокруг вектора результирующего момента и значительно более медленная прецессия результирующего момента вокруг поля.
В сильном магнитном поле действие поля на каждый из моментов больше их внутреннего взаимодействия. В результате магнитное поле разрывает спин-орбитальное взаимодействие и каждый из моментов прецессирует вокруг линий поля независимо друг от друга.

Рассмотрим случай слабого магнитного поля.
\[
    \Delta L = -\vec{\mu_J}\cdot\vec{B}  = \mu_\text{Б}gm_JB.
\]

Простой эффект Зеемана наблюдается когда фактор Ланде для перехода между двумя уровнями постоянен:
\( g_1 = g_2 \). \( S=0,\ J=L,\ g=1. \) Рассмотрим переход \( ^1P_1 \rightarrow ^1S_0 \). При простом эффекте Зеемана наблюдается триплет. При наблюдении поперёк магнитного поля наблюдаются все три компоненты триплета, а если вдоль поля B, то \( \pi \)-линия пропадает ввиду направления её плоскости поляризации.

Рассмотрим расщепление линий, обладающее тонкой структурой. В этом случае расщепление спектральных линий может быть выражено как \( \Delta \omega = \Delta \omega \frac{p}{q} \).

Рассмотрим эффект Зеемана для натриевого дуплета. Заметим, что первоначальная линия исчезает при включении магнитного поля.

Рассмотрим теперь сильное магнитное поле. В нём связь между моментами разрывается и они независимо друг от друга проецируются на направление внешнего поля. тода дополнительная энергия, приобретаемая атомом в магнитном поле будет состоять из двух составляющих -- спиновой и орбитальной. Такое расщеплеие называется эффектом Пашена-Бака. В результате получатся нормальный Зеемановский триплет.

Таким образом, увеличивая значение магнитного поля сначала наблюдается нормальный эффект Зеемана, затем сложный эффект и, наконец, при очень сильном поле, опять номальный эффект.

