\section{Денежно-кредитная политика государства}

Денежно-кредитная политика является важным направлением деятельности любого
государства. Основная характеристика денежно-кредитной политики -- это
денежная система.

Денежная система -- это организация денежного обращения, сложившаяся
исторически и закрепленная законодательно.

\subsection{Деньги}
Деньги -- это актив, выполняющий три основные функции:
\begin{enumerate}
    \item средство обращения,
    \item мера стоимости,
    \item средство сбережения.
\end{enumerate}

\emph{Средство обращения}. Деньги в данной функции выступают общепринятым
средством для расчетов и платежей. Количество денег, необходимое для выполнения
данной функции:
\[
    M = f(Y, P, V),
\]
где \( Y \) -- физический объем выпуска, \( P \) -- общий уровень цен, \( V \) -- скорость обращения
денег (количество оборотов, совершаемых денежной единицей за год).

\emph{Мера стоимости}. В данной функции деньги соизмеряют стоимость благ и
услуг. Следует различать номинальную стоимость денежной единицы и реальную
(количество благ и услуг, которые можно за нее приобрести). Количество денег,
необходимое для выполнения данной функции:
\[
    M_P = \frac{M_H}{P}.
\]

\emph{Средство сбережения}. Деньги выступают резервом покупательной
способности. Основное преимущество денег -- абсолютная ликвидность. Ликвидность
-- свойство актива без дополнительных затрат быть использованным для расчетов и
платежей при сохранении номинала.

Плюсы и минусы формирования сбережений в денежной форме:
\hspace*{2em}\emph{плюсы}: деньги обладают абсолютной ликвидностью, в любой
момент можно их обменять на необходимый товар,
\hspace*{2em}\emph{минусы}: в условиях инфляции с течением времени деньги
обесцениваются.

Количество денег, используемых в качестве средства сбережения:
\[
    M = f(P, r),
\]
где \( P \) -- уровень инфляции, \( r \) -- доходность по финансовым активам.

Равновесие на денежном рынке % картинка

\subsection{Денежная масса и ее структура}

Денежная масса -- это объем активов реально или потенциально способных
выполнять денежные функции.

Исчисление денежной массы необходимо для оценки взаимосвязи и взаимовлияния
важнейших макроэкономических показателей с целью определения их прогнозных
значений.

Основные подходы к расчету денежной массы:
\begin{enumerate}
    \item система агрегатов (методика ЦБ РФ);
    \item система показателей, рассчитываемых по методологии МВФ.
\end{enumerate}

\begin{table}[h!]
    \center
    \caption{Система агрегатов (методика ЦБ РФ)}
    \begin{tabular}{|c|C{.3}|C{.5}|} \hline
        № & Агрегат & Алгоритм расчета \\ \hline
        1.1 & Денежный агрегат \( M_0 \) & Наличные деньги в обращении вне
        банковской системы \\ \hline
        1.2 & Денежный агрегат \( M_1 \) & \( M_0 \) + депозиты до
        востребования + расчетные, текущие счета \\ \hline
        1.3 & Денежный агрегат \( M_2 \) (денежная~масса) & \( M_1 \) + срочные
        вклады \\ \hline
        1.4 & Денежный агрегат \( M_2^* \) (широкие~деньги) & \( M_2 \) +
        долларовое обращение \\ \hline
        1.5 & Денежный агрегат \( M_3 \) & \( M_2^* \) + депозитные сертификаты
        и облигации государственных займов \\ \hline
    \end{tabular}
\end{table}
Наибольшей ликвидностью обладает агрегат \( M_0 \), наименьшей -- \( M_3 \).

\begin{table}[h!]
    \center
    \caption{Система агрегатов денежной массы по методологии МВФ}
    \begin{tabular}{|C{.2}|C{.6}|} \hline
        Показатель & Алгоритм расчета \\ \hline
        Агрегат ``деньги'' & Деньги вне банков + депозиты до востребования
        \\ \hline
        Агрегат ``квазиденьги'' & Срочные и сберегательные депозиты + депозиты
        в иностранной валюте, учитываемые в балансе ЦБ РФ и коммерческих банков
        \\ \hline
        Агрегат ``широкие деньги'' & Агрегат ``деньги'' + ``квазиденьги''
        \\ \hline
    \end{tabular}
\end{table}

Депозитный сертификат -- это ценная бумага, удостоверяющая сумму вклада,
внесенного в кредитную организацию. В сертификате закрепляются права его
держателя -- на получение по истечении установленного срока суммы вклада и
процентов по нему.

Институциональные субъекты, влияющие на величину денежной массы:
\begin{enumerate}
    \item ЦБ -- задает денежную базу (прямое влияние) и косвенно влияет на
    величину денежной массы через установление нормы обязательного
    резервирования;
    \item КБ (коммерческие банки) -- мультиплицируют попавшие в них резервы и
    определяют депозитную составляющую денежной массы (прямое влияние),
    косвенно влияют на силу мультипликации через установление нормы
    собственного резервирования;
    \item домохозяйства -- прямо влияют на величину активных денег (\(M_1\))
    через изменение предпочтений по форме хранения сбережений. Косвенно влияют
    на силу мультипликации через коэффициент склонности к наличности.
\end{enumerate}

\subsection{Денежно-кредитная политика}
Инструменты политики:
\begin{enumerate}
    \item денежная эмиссия,
    \item резервная политика,
    \item валютная политика,
    \item политика открытого рынка,
    \item учетная политика.
\end{enumerate}

Цели денежно-кредитной политики:
\begin{itemize}
   \item первичные:
   \begin{itemize}
        \item достижение полной занятости,
        \item равновесный и стабильный экономический рост,
        \item внешнеэкономическая стабильность,
        \item стабильность уровня цен.
    \end{itemize}
\end{itemize}

Политика ``дешевых денег'' -- борьба с безработицей и спадом производства.
Политика ``дорогих денег'' -- борьба с инфляцией.

Кредитно-денежная система:
\begin{itemize}
    \item ассоциации, консорциумы, холдинги и другие объединения банков в СКФИ
    \begin{itemize}
        \item банковская система
        \begin{itemize}
            \item ЦБ,
            \item КБ
        \end{itemize}
        \item специализированные кредитно-финансовые институты
        \begin{itemize}
            \item пенсионные фонды, страховые общества, кредитные союзы,
            инвестиционные фонды и другие
        \end{itemize}
    \end{itemize}
\end{itemize}

Операции ЦБ РФ делятся на пассивные (формирование капитала, эмиссия денег,
хранение денежных средств) и активные (учетно-ссудные операции, операции с
банковскими инвестициями, операции с золотом и иностранной валютой).
