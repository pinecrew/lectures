\section{Понятие тензора}
\subsection{Скаляры и векторы как тензоры низших рангов}

	Скалярные величины, например, температура \( T = T(x, y, z) \), давление \( p = p(x, y, z) \), плотность \( \rho = \rho(x, y, z) \), можно рассматривать как тензоры нулевого ранга, так как они характеризуются \( 3^0 = 1 \) числом. При преобразовании координат они не меняются: \( T' = T \), \( p' = p \), \( \rho'=\rho\).
	
	Величины, которые характеризуются не \( 3^0 = 1 \), а \( 3^1 = 3 \) числами, -- это векторы: \( \vec{a} = \{ a_x, a_y, a_z \} \), где скалярные величины \( a_x, a_y, a_z \) являются компонентами вектора.
	Но не любая тройка чисел является вектором. Например,  параметры состояния идеального газа \( (p, V, T) \) -- не образуют вектор.
	Векторы можно считать тензорами первого ранга:
	\[ \begin{array}{rl}
			\vec{F} = \{ F_x, F_y, F_z \},&\text{где } F_x = F_x (x, y, z), \  F_y = F_y (x, y, z), \  F_z = F_z (x, y, z); \\
			\vec{v} = \{ v_x, v_y, v_z \},&\text{где } v_x = v_x (x, y, z), \  v_y = v_y (x, y, z), \  v_z = v_z (x, y, z); \\
			\vec{E} = \{ E_x, E_y, E_z \},&\text{где } E_x = E_x (x, y, z), \  E_y = E_y (x, y, z), \  E_z = E_z (x, y, z).
	\end{array} \]
	
	Однако, часто встречаются величины, для описания которых необходимо не \( 3^0 = 1 \), не \( 3^1 = 3 \), а \( 3^2 = 9 \), \( 3^3 = 27 \) и даже \( 3^4 = 81 \) чисел. Такие величины называются \textbf{тензорными}.
	
	Для общего пояснения понятия \textit{тензор} установим характер изменения ортонормированного базиса при повороте и характер изменения компонент вектора при таких поворотах.
	
	В дальнейшем, под преобразованиями базиса будем считать только \textbf{повороты} вокруг начала координат, а параллельные сдвиги учитывать не будем, так как рассматривать будем только свободные векторы. Сам базис будет всегда считаться правым.
	
\subsection{Преобразования ортонормированного базиса}
	
	Рассмотрим декартову прямоугольную систему координат \( (x, y, z) \), % рис. 1 
	 задаваемую единичными перпендикулярными векторами -- ортами. Базис считается ортонормированным, если:
	\[ \vec{e}_i \cdot \vec{e}_j = \delta_{ij} = \left\{
 	\begin{array}{l l}
    	1, & \text{при } i = j  \\
   	0, & \text{при } i \ne j \\
  	\end{array}, \right. \]
  	где \( i, j = 1, 2, 3 \), а \( \delta_{ij} \) -- символ Кронекера.
  
	Таким образом, \( (\vec{e}_1, \vec{e}_2, \vec{e}_3) \) -- ортонормированный базис.
	
	Рассмотрим вектор \( \vec{x} \), % рис. 2
	 который в заданном ортонормированном базисе может быть разложен по компонентам:
	\[ \vec{x} = x_1 \vec{e}_1 + x_2 \vec{e}_2 + x_3 \vec{e}_3. \]
	
	В общем случае, компонента вектора \( \vec{x} \) в данном базисе выглядит следующим образом:

	\[ x_i = \vec{x}\cdot\vec{e}_i. \]
	
	Отсюда видно, что каждая компонента вектора \( \vec{x} \) зависит не только от самого \( \vec{x} \), но и от выбора базиса. Поэтому в любом другом базисе компоненты  вектора \( \vec{x} \) будут другими.
	
\subsubsection{Матрица преобразования}

	Пусть \( (\vec{e}_1, \vec{e}_2, \vec{e}_3) \) -- старый (исходный) базис, а \( (\vec{e}_1{'}, \vec{e}_2{'}, \vec{e}_3{'}) \) -- новый базис. % пикча 3 %
	
	Условие ортонормированности остается тем же: \( \vec{e}_i{'}\cdot\vec{e}_j{'} = \delta_{ij} \).
	
	Разложим каждый новый орт \( \vec{e}_1{'}, \vec{e}_2{'}, \vec{e}_3{'} \) по старому базису \( (\vec{e}_1, \vec{e}_2, \vec{e}_3) \). Это разложение можно представить в виде трех соотношений:
	\begin{equation} \left\{ \begin{array}{l l}
    		\vec{e}_1{'} = & \gamma_{11}\vec{e}_1 + \gamma_{12}\vec{e}_2 + \gamma_{13}\vec{e}_3  \\
    		\vec{e}_2{'} = & \gamma_{21}\vec{e}_1 + \gamma_{22}\vec{e}_2 + \gamma_{23}\vec{e}_3  \\
    		\vec{e}_3{'} = & \gamma_{31}\vec{e}_1 + \gamma_{32}\vec{e}_2 + \gamma_{33}\vec{e}_3
  	\end{array} \right. \label{eq9:1a} \end{equation}
  	
	Или:
	\begin{equation}
	\vec{e}_i{'} = \sum\limits_{j=1}^3 \gamma_{ij}\vec{e}_j, \label{eq9:1}
	\end{equation}
	 где \( \gamma_{ij} \) -- коэффициент разложения, который, по сути, является \( \gamma_{ij} = \cos\widehat{\vec{e}_i,\; \vec{e}_j{'}} \).
	 
	Они образуют матрицу коэффициентов прямого преобразования:
	\[ \Gamma = [\gamma_{ij}] = \begin{bmatrix}
	\gamma_{11} & \gamma_{12} & \gamma_{13} \\
	\gamma_{21} & \gamma_{22} & \gamma_{23} \\
	\gamma_{31} & \gamma_{32} & \gamma_{33}
	\end{bmatrix}. \]
	
	Из (\ref{eq9:1a}) следует, что \( \gamma_{12} = \vec{e}_1{'}\cdot\vec{e}_2 \). В общем случае:
	\begin{equation}
	\gamma_{ij} = \vec{e}_i{'}\cdot\vec{e}_j. \label{eq9:2} 
	\end{equation}
	
	Аналогично разложим каждый орт старого базиса \( \vec{e}_1, \vec{e}_2, \vec{e}_3 \) по новому базису \( (\vec{e}_1{'}, \vec{e}_2{'}, \vec{e}_3{'}) \).
	
	Получим систему, аналогичную (\ref{eq9:1a}):
	\[ \left\{ \begin{array}{l l}
    		\vec{e}_1 = & \gamma_{11}\vec{e}_1{'} + \gamma_{12}\vec{e}_2{'} + \gamma_{13}\vec{e}_3{'}  \\
    		\vec{e}_2 = & \gamma_{21}\vec{e}_1{'} + \gamma_{22}\vec{e}_2{'} + \gamma_{23}\vec{e}_3{'}  \\
    		\vec{e}_3 = & \gamma_{31}\vec{e}_1{'} + \gamma_{32}\vec{e}_2{'} + \gamma_{33}\vec{e}_3{'}
  	\end{array} \right. \]
  	
	Или, аналогично (\ref{eq9:1}):
	\begin{equation}
	\vec{e}_i = \sum\limits_1^3 \gamma_{ij}'\vec{e}_j{'}, \label{eq9:3}
	\end{equation}
	где, аналогично (\ref{eq9:2}):
	\begin{equation} 
	\gamma_{ij}' = \vec{e}_i\cdot\vec{e}_j{'}. \label{eq9:4}
	\end{equation}	
	
	Коэффициенты разложения \( \gamma_{ij}' \) образуют матрицу обратного преобразования:
	\[ \Gamma' = [\gamma_{ij}'] = \begin{bmatrix}
	\gamma_{11}' & \gamma_{12}' & \gamma_{13}' \\
	\gamma_{21}' & \gamma_{22}' & \gamma_{23}' \\
	\gamma_{31}' & \gamma_{32}' & \gamma_{33}'
	\end{bmatrix} \]
	
\subsubsection{Связь матриц прямого и обратного преобразований}
	
	Докажем, что матрица обратного преобразования \( \Gamma' = \Gamma^T \) -- транспонированная матрица прямого преобразования \( \Gamma \), то есть матрица, у которой \( i \)-строка становится \( i \)-столбцом, то есть матрица, у которой сделаны симметричные перестановки индексов относительно главной диагонали, то есть матрица \( \Gamma' \) такая, что \( (a_{ij})^T = a_{ji} \).
	Итак, докажем, что матрица \( \Gamma' \) такая, что
	\begin{equation}
	\gamma_{ij}' = \gamma_{ji}. \label{eq9:5}
	\end{equation}
	
	\begin{proof}
	
	Рассмотрим (\ref{eq9:2}). Поменяем в нем местами индексы \( i \leftrightarrow j \): \( \gamma_{ji} = \vec{e}_i\cdot\vec{e}_j{'} \), что совпадает с (\ref{eq9:4}).
	Таким образом, \( \gamma_{ji} = \gamma_{ij}' \), то есть \( \Gamma' = \Gamma^T \), что и требовалось доказать.
	\end{proof}
			
\subsubsection{Ортогональные матрицы}

	Установим еще одно свойство матриц \( \Gamma \) и \( \Gamma' \).
	
	Так как новый и старый базисы являются ортонормированными, то есть \( \vec{e}_i \cdot \vec{e}_j = \delta_{ij} \) и \( \vec{e}_i{'}\cdot\vec{e}_j{'} = \delta_{ij} \), то, записав (\ref{eq9:1}) два раза для векторов \( \vec{e}_i{'} \) и \( \vec{e}_j{'} \), получим:
	
	\[ \begin{array}{l l}
	\vec{e}_i{'} = & \sum\limits_k\gamma_{ik}\vec{e}_k; \\
	\vec{e}_j{'} = & \sum\limits_l\gamma_{jl}\vec{e}_l
	\end{array} \]
	
	Тогда:
	\[ \vec{e}_i{'}\cdot\vec{e}_j{'} = \sum\limits_k \sum\limits_l \gamma_{ik}\gamma_{jl} (\vec{e}_k\cdot\vec{e}_l) = \sum\limits_k \sum\limits_l \gamma_{ik}\gamma_{jl}\delta_{kj} = \sum\limits_k \gamma_{ik}\gamma_{jk}. \]
	
	Таким образом:
	\begin{equation}
	\sum\limits_k \gamma_{ik}\gamma_{jk} = \delta_{ij}. \label{eq9:6}
	\end{equation}
	
	В итоге получаем систему из двух уравнений:
	\begin{equation} \left\{ \begin{array}{l l}
    		\gamma_{i1}^2 + \gamma_{i2}^2 + \gamma_{i3}^2 = 1, & \text{ при \( i = j \) }; \\
    		\gamma_{i1}\gamma_{j1} + \gamma_{i2}\gamma_{j2} + \gamma_{i3}\gamma_{j3} = 0, & \text{ при \( i \ne j \) }.
  	\end{array} \right. \label{eq9:6a} \end{equation}
	
	\begin{definition}	
	Матрица \( \Gamma \), у которой элементы обладают свойствами (\ref{eq9:6a}), то есть сумма квадратов элементов любой строки равна единице, а сумма произведения элементов строки на элементы любой другой строки равна нулю, называется \textbf{ортогональной}.
	\end{definition}
	
	Таким образом, всякая ортогональная матрица задает преобразования векторов при переходе из одного базиса в другой.
	
	Аналогично можно получить то, что и матрица \( \Gamma' \) обратного преобразования является ортогональной.
	
	\begin{proof}
	
	В силу (\ref{eq9:3}) можно записать, что:
	\[\begin{array}{r l}
		\vec{e}_i = & \sum \gamma_{ik}'\vec{e}_k{'} \\
		\vec{e}_j = & \sum \gamma_{jl}'\vec{e}_l{'} \\[0.5cm]
		\vec{e}_i\cdot\vec{e}_j = & \delta_{ij} \\ 
		\vec{e}_i\cdot\vec{e}_j = & \sum\limits_k \sum\limits_l \gamma_{ik}'\gamma_{jl}' (\vec{e}_k{'}\cdot\vec{e}_l{'}) \\
		\sum\limits_k \sum\limits_l \gamma_{ik}'\gamma_{jl}' (\vec{e}_k{'}\cdot\vec{e}_l{'}) = & \sum\limits_k \gamma_{ik}'\gamma_{jk}'
	\end{array} \]
	
	\begin{equation}
		\sum\limits_k \gamma_{ik}'\gamma_{jk}' = \delta_{ij} \label{eq9:6b}
	\end{equation}
	\end{proof}
	
	\begin{remark}
	Легко показать, что свойством (\ref{eq9:6a}) обладают и столбцы.
	
	Действительно, переписывая (\ref{eq9:6b}) с учетом (\ref{eq9:5}), получаем:
	\begin{equation}
		\delta_{ij} = \sum\limits_k \gamma_{ik}'\gamma_{jk}' = \sum \gamma_{ki}\gamma_{kj} = \delta_{ij} \label{eq9:7}
	\end{equation}
	\end{remark}
	
\subsubsection{Связь матрицы обратного преобразования с матрицей прямого}

	Покажем, что матрица \( \Gamma' \) обратного преобразования является обратной к матрице \( \Gamma \) прямого преобразования, то есть \( \Gamma' = \Gamma^{(-1)} \), где матрица \( \Gamma^{(-1)} \) такая, что \( \Gamma \Gamma^{(-1)} = E = [ \delta_{ij} ] = \left[ \begin{smallmatrix} 1 & 0 & 0 \\ 0 & 1 & 0 \\ 0 & 0 & 1 \end{smallmatrix} \right] \) -- единичная матрица.
	
	\begin{proof} 
	
	Переписывая (\ref{eq9:7}), с учетом свойства (\ref{eq9:5}), получаем:
	\begin{equation}
		\sum\limits_k \gamma_{ki}\gamma_{kj} = \sum\limits_k \gamma_{ik}'\gamma_{kj}. \nonumber
	\end{equation}
	Однако, \( \sum\limits_k \gamma_{ik}'\gamma_{kj} \) -- это есть элемент произвдения матриц \( ( \Gamma' \Gamma )_{ij} \) по правилу ``строка-на-столбец''.
	
	\begin{comment} Умножение матриц производится следующим образом:
	\[ \begin{array}{rl} \begin{bmatrix}
			a_{11} & a_{12} & a_{13} \\
			a_{21} & a_{22} & a_{23} \\
			a_{31} & a_{32} & a_{33}
	\end{bmatrix}	\cdot \begin{bmatrix}
			b_{11} & b_{12} & b_{13} \\
			b_{21} & b_{22} & b_{23} \\
			b_{31} & b_{32} & b_{33}
	\end{bmatrix} = & \begin{bmatrix}
			c_{11} & c_{12} & c_{13} \\
			c_{21} & c_{22} & c_{23} \\
			c_{31} & c_{32} & c_{33}
	\end{bmatrix} \\[0.5cm]
		c_{ij} = a_{i1}b_{1j} + a_{i2}b_{2j} + a_{i3}b_{3j} = & \sum\limits_{k=1}^3 a_{ik}b_{kj}
	\end{array} \]
	\end{comment}
	
	А так как всевозможные элементы \( i \) и \( j \) образуют матрицу \( [ \delta_{ij} ] \), то \( \Gamma'\Gamma = [ \delta_{ij} ] \Rightarrow \Gamma' = \Gamma^{(-1)} \).
	\end{proof}
	
\subsubsection{Определитель матрицы преобразования}
	
	Покажем, что \( \det \Gamma = 1 \).
	\begin{proof}
	
	\[ \det\Gamma = \begin{vmatrix}
			\gamma_{11} & \gamma_{12} & \gamma_{13} \\
			\gamma_{21} & \gamma_{22} & \gamma_{23} \\
			\gamma_{31} & \gamma_{32} & \gamma_{33}
	\end{vmatrix} \]
	
	Строки этого определителя составлены из коэффициентов разложения  ортов \( \vec{e}_i{'} \) по базису \( (\vec{e}_1, \vec{e}_2, \vec{e}_3) \) (\ref{eq9:1a}), то есть этот определитель равен смешанному произведению \( (\vec{e}_1{'}, \vec{e}_2{'}, \vec{e}_3{'}) \), следовательно он равен объему кубика, построенному на векторах \( \vec{e}_1{'}, \vec{e}_2{'}, \vec{e}_3{'} \), а так как \( |\vec{e}_1{'}| = |\vec{e}_2{'}| = |\vec{e}_3{'}| = 1 \), то \( V_{кубика} = \pm 1\).
	
	``+'' -- если оба базиса правые.
	
	``\( - \)'' -- если один из базисов правый, другой левый.
	
	Далее всегда будем считать, что оба базиса являются правыми.
	
	Таким образом, \( \det\Gamma  = V_{кубика} = 1 \).
	\end{proof}
	
\subsection{Преобразование компонент вектора. Определение тензора первого ранга}

	Разложение вектора \( \vec{x} \) в старом и новом базисах:
	\begin{align}
		\vec{x} = \sum x_i \vec{e}_i  \label{eq9:8} \text{ -- в старом базисе;} \\
		\vec{x} = \sum x_i' \vec{e}_i{'}  \label{eq9:9} \text{ -- в новом базисе}  
	\end{align}
	
	Подставим разложение (\ref{eq9:3}) в (\ref{eq9:8}):
	\[ \vec{x} = \sum\limits_i\sum\limits_j \gamma_{ij}' x_i \vec{e}_j{'}. \]
	
	С учетом (\ref{eq9:5}):
	\[ \vec{x} = \sum\limits_j\sum\limits_i \gamma_{ji} x_i \vec{e}_j{'}. \]
	
	Для удобства поменяем индексы \( i \leftrightarrow j \):
	\[ \vec{x} = \sum\limits_i \left(\sum\limits_i \gamma_{ij} x_j\right) \vec{e}_i{'}. \]
	
	При сравнении последнего уравнения и (\ref{eq9:9}), видно, что новые компоненты вектора \( \vec{x} \) будут следующими:
	\begin{equation}
		x_i' = \sum \gamma_{ij}x_j \label{eq9:10}
	\end{equation}
	
	Аналогично раскладывая \( \vec{e}_i{'} \) из (\ref{eq9:9}), с учетом (\ref{eq9:1}), получим:	
	\begin{equation}
		x_i = \sum \gamma_{ij}' x_j' \label{eq9:11}
	\end{equation}
	
	Из сравнения (\ref{eq9:10}) и (\ref{eq9:1}) видно, что компоненты вектора \( \vec{x} \) преобразуются так же, как и орты \( \vec{e}_i \).
	
	\begin{definition}
	Тройка чисел \( x_1, x_2, x_3 \), занумерованных одним индексом, зависящих от выбора базиса и преобразующихся при переходе от одного базиса к другому по формулам (\ref{eq9:10}) или (\ref{eq9:11}), называется \textbf{тензором первого ранга}. Сами числа называются \textbf{компонентами тензора}. Таким образом, вектор -- частный случай тензора первого ранга.
	\end{definition}
	
	\begin{example}
	Доказать, что коэффициенты \( a_i \) , задающие плоскость, описываемую в старом базисе уравнением \( \sum a_i x_i = 1 \), образуют тензор первого ранга.
	\end{example}
	
	\begin{proof}
	Координаты \( x_i \) точки в старом базисе выражаются  через \( x_i' \) в новом базисе формулой \( \ref{eq9:11} \):

	\( \sum\limits_i a_i \left(\sum\limits_j \gamma_{ij}' x_j'\right) = 1 \) или, с учетом (\ref{eq9:5}):
	
	\( \sum\limits_j\sum\limits_i \gamma_{ji}a_ix_j' = 1 \). Меняя, для удобства, индексы \( i \leftrightarrow j \), получаем:
	
	\( \sum\limits_i\sum\limits_j \gamma_{ij}a_jx_i' = 1 \).
	
	Видно, что внутренняя сумма в этом уравнении образует коэффициенты плоскости в новом базисе:
	\begin{equation}
		a_i' = \sum\limits_j \gamma_{ij}a_j \label{eq9:13}
	\end{equation}
	
	Сравнивая (\ref{eq9:13}) с (\ref{eq9:10}), видно, что коэффициенты \( a_i' \) образуют тензор первого ранга.
	\end{proof}
	
	\begin{example}
	Доказать инвариантность скалярного произведения \( \vec{x}\cdot\vec{y} \), то есть, что \( \sum x_ky_k = \sum x_k' y_k' \).
	\end{example}
	
	\begin{proof}
	
	В соответствии с (\ref{eq9:10}):
	\[ \begin{array}{l}
		x_k' = \sum \gamma_{ki}x_i; \\
		y_k' = \sum \gamma_{ki}y_i.
	\end{array} \]
	
	Тогда, скалярное произведение в новом базисе:
	\[ \sum\limits_k x_k' y_k' = \sum\limits_i\sum\limits_j \left(\sum\limits_k \gamma_{ki}\gamma_{kj}\right)x_iy_j \]
	
	Тогда, с учетом ортогональности матрицы \( \Gamma \) (\ref{eq9:7}), получаем:
	\[ \sum\limits_k x_k' y_k' = \sum\limits_i\sum\limits_j x_iy_j\delta_{ij}. \]
	
	Тогда, расписав символ Кронекера, получаем: \( \sum\limits_k x_k' y_k' = \sum\limits_i x_iy_i \).
	Заменив индекс \( i \rightarrow k \), получим:
	\[ \sum\limits_k x_ky_k = \sum\limits_k x_k' y_k', \]
	что совпадает с исходной формулой.
	\end{proof}
	
\subsection{Определение тензора второго ранга}
\label{sec9.4}

	Рассморим какую-либо центральную поверхность второго порядка, например, однополостный гиперболоид. % пикча
		
	\begin{definition}
	Поверхность второго порядка называется \textbf{центральной}, если выполняется условие:
	
	если точка \( M(x, y, z) \in S \), то и точка \( M'(-x, -y, -z) \in S \).
	\end{definition}
		
	Общее уравнение центральной поверхности второго порядка имеет вид:
	\[ a_1 x^2 + a_2 y^2 + a_3 z^2 + a_4 xy + a_5 yz + a_6 xz = 1 \]
	
	Или:
	\begin{equation}
		\sum\sum a_{ij}x_ix_j = 1 \label{eq9:14}
	\end{equation}
	
	Уравнение (\ref{eq9:14}) -- это уравнение центральной поверхности второго порядка в старом базисе. Запишем его в новом базисе. С учетом \( \sum a_ix_i = 1 \):
	\begin{equation} \left\{	\begin{array}{l}
			x_i = \sum \gamma_{ik}' x_k' \\
			x_j = \sum \gamma_{jl}' x_l'
	\end{array} \right. \label{eq9:13a} \end{equation}
	
	Тогда (\ref{eq9:14}):
	\[ \sum\limits_i\sum\limits_j\sum\limits_k\sum\limits_l a_{ij}\gamma_{ik}'\gamma_{jl}' x_k' x_l' = 1. \]
	
	С учетом (\ref{eq9:5}) получаем:
	
	\[ \sum\limits_k\sum\limits_l \underbrace{\sum\limits_i\sum\limits_j \gamma_{ki}\gamma_{lj}a_{ij}}_{a_{kl}'} x_k' x_l' = 1. \]
	
	Из сравнения последней формулы с (\ref{eq9:14}) видно, что внутренняя двойная сумма образует коэффициенты поверхности в новом базисе:
	\begin{equation}
		a_{kl}' = \sum\limits_i\sum\limits_j \gamma_{ki}\gamma_{lj}a_{ij}. \label{eq9:15}
	\end{equation}
	
	\begin{definition}
	Девятка чисел \( a_{ij} \), занумерованных двумя индексами, то есть составляющих матрицу \( T = [a_{ij}] = \left[ \begin{smallmatrix} a_{11} & a_{12} & a_{13} \\ a_{21} & a_{22} & a_{23} \\ a_{31} & a_{32} & a_{33} \end{smallmatrix} \right] \), зависящих от выбора базиса и преобразующихся при переходах от одного базиса к другому по закону (\ref{eq9:15}), называется \textbf{тензором второго ранга}, а сами числа -- его \textbf{коэффициентами}. Таким образом, любая поверхность второго порядка описывается тензором второго ранга.
	\end{definition}
	
\subsection{Общее определение тензора в ортогональном базисе}

	Итак, тензоры -- это объекты, описываемые наборами чисел, преобразующихся при поворотах базиса по определенным законам. Выпишем уже известные нам законы преобразования:
	\begin{enumerate}
		\item \( a' = a \) -- тензор нулевого ранга \( T = \const \), то есть скаляр, являющийся инвариантом.
		\item \( a_k' = \sum \gamma_{ki}a_i \) -- тензор первого ранга \(T = [x_i] \), в частности -- вектор.
		\item \( a_{kl}' = \sum\sum \gamma_{ki}\gamma_{lj}a_{ij} \) -- тензор второго ранга \( T = [a_{ij}] \), матрица \( 3 \times 3 \).
	\end{enumerate}
	
	Все эти преобразования \textbf{линейны} относительно компоненты \( a \).
	
	Аналогично, то есть как законы преобразования, определяются тензоры более высоких рангов.
	
	Например, тензор третьего ранга \( T = [a_{ijk}] \) -- \( 3^3 = 27 \) чисел, преобразующихся по закону:
	\begin{equation}
		a_{pqr} = \sum\limits_i\sum\limits_j\sum\limits_k \gamma_{pi}\gamma_{qj}\gamma_{rk} a_{ijk}
	\end{equation}
	
	В физике встречаются тензоры максимум четвертого порядка, например, тензоры упругих свойств кристаллов,  \( T = [a_{ijkl}] \) -- \( 3^4 = 81 \) число, преобразующиеся по закону:
	\begin{equation}
		a_{pqrs} = \sum\limits_i\sum\limits_j\sum\limits_k\sum\limits_l \gamma_{pi}\gamma_{qj}\gamma_{rk}\gamma_{sl} a_{ijkl}
	\end{equation}
	
\subsection{Симметрия и антисимметрия тензоров}

	Понятия симметрии и антисимметрии определено для тензоров ранга не ниже второго. Определим его для тензора второго ранга.
	
	\begin{definition}
	Тензор \( T = [a_{ij}] \) называется \textbf{симметричным}, если \( a_{ij} = a_{ji} \). Симметричный тензор имеет не 9, а 6 независимых переменных:
	\[ T_S = \begin{bmatrix}
		\mathbf{a_{11}}	& \mathbf{a_{12}}	& \mathbf{a_{13}} \\
		a_{12}			& \mathbf{a_{22}}	& \mathbf{a_{23}} \\
		a_{13}			& a_{23}			& \mathbf{a_{33}}
	\end{bmatrix} \]
	\end{definition}
	
	\begin{definition}
	Тензор \( T = [a_{ij}] \) называется \textbf{антисимметричным}, если \( a_{ij} = -a_{ji} \). Симметричный тензор имеет не 9, не 6, а 3 независимые переменные:
	\[ T_{AS} = \begin{bmatrix}
		0 		& \mathbf{a_{12}}	& \mathbf{a_{13}} \\
		-a_{12}	& 0				& \mathbf{a_{23}} \\
		-a_{13}	& -a_{23}			& 0
	\end{bmatrix} \]
	\end{definition}
	
	\begin{theorem}
	Свойства симметрии и антисимметрии тензоров являются инвариантными.
	\end{theorem}
	
	\begin{proof}
	
	Пусть в старом базисе тензор симметричен, то есть \( a_{ij} = a_{ji} \).
	
	Тогда, согласно (\ref{eq9:14}), в новом базисе:
	\[ a_{kl}' = \sum\sum \gamma_{ki}\gamma_{lj} a_{ij} = \sum\sum \gamma_{li}\gamma_{kj} a_{ji} = a_{lk}' \]
	
	Таким образом, в новом базисе \( a_{kl}' = a_{lk}' \), то есть тензор симметричен.
	
	Аналогично доказывется свойство антисимметрии. 
	\end{proof}
	
	\begin{theorem}
	Любой тензор \( T \) можно представить в виде суммы симметричного и антисимметричного тензоров, то есть \( T = T_S + T_{AS} \).
	\end{theorem}
	
	\begin{proof}
	
	Пусть тензор \( T \) задан матрицей \( [a_{ij}] \). Тогда любой коэффициент \( a_{ij} \) может быть представлен в виде:
	\[ a_{ij} = \underbrace{\frac{1}{2} (a_{ij} + a_{ji})}_{b_{ij}} + \underbrace{\frac{1}{2} (a_{ij} - a_{ji})}_{c_{ij}} \]
	
	Видно, что \( b_{ij} = b_{ji} \), а \( c_{ij} = -c_{ji} \).
	
	Таким образом, коэффициенты \( b_{ij} \) образуют симметричный тензор \( T_S = [b_{ij}] \), а коэффициенты \( c_{ij} \) образуют антисимметричный тензор \( T_{AS} = [c_{ij}] \). Следовательно, \( T = T_S + T_{AS} \).
	\end{proof}
	
\subsection{Преобразования матрицы тензора при поворотах базиса}

	Ранее был установлен закон преобразования компонент тензора при поворотах базиса.
	
	Установим матричный вид уравнения (\ref{eq9:15}), то есть вид матрицы \( T' \) в новом базисе, если \( T = [a_{ij}] \) -- матрица в старом базисе.
	
	Тогда, используя свойство (\ref{eq9:5}), получаем (\ref{eq9:15}) в виде:
	\[ a_{kl}' = \sum\limits_i \gamma_{ki} \underbrace{\sum\limits_j a_{ij} \gamma_{jl}'}_{( T\cdot\Gamma^{(-1)} )_{il}}. \]
	
	Внутренняя сумма является коэффициентами матрицы \(( T\cdot\Gamma^{(-1)} )_{il} \), полученной умножением \( T \) на \( \Gamma^{(-1)} \) по правилу ``строка-на-столбец''.

	Но, точно так же, учитывая, что \( \Gamma^{(-1)} = \Gamma' \), получаем уравнение:
	\[ a_{kl}' = \sum\limits_i \gamma_{ki} (T\cdot\Gamma')_{il}. \]
	
	В этом уравнении \( \sum\limits_i \gamma_{ki} (T\cdot\Gamma')_{il} \) является элементом произведения матрицы \( \Gamma \) на \( T\cdot\Gamma' \) по правилу ``строка-на-столбец''. Следовательно:
	\[ a_{kl}' = (\Gamma\cdot T\cdot\Gamma')_{kl}. \]
	
	А так как все коэффициенты \( a_{kl}' \) образуют матрицу тензора \( T' \) в новом базисе, то получим:
	\begin{equation}
		T' = \Gamma\cdot T\cdot\Gamma^{(-1)} \label{eq9:n1a}
	\end{equation}
	
	Уравнение (\ref{eq9:n1a}) является матричной формой преобразования (\ref{eq9:15}), где \( \Gamma = [\gamma_{ij}] \) -- закон преобразования базиса: \( \vec{e}_k = \sum \gamma_{ki}\vec{e}_i \).
	
	\begin{example}
	Пусть задана центральная линия второго порядка -- гипербола \( xy = 1 \). Найти матрицу тензора \( T' \) в новом базисе \( (\vec{e}_1{'}, \vec{e}_2{'}) \), повернутом на угол \( \frac{\pi}{4} \) против часовой стрелки. % тройка картинок в примере
	\end{example}
	
	\begin{solution}
	
	Так как общее уравнение центральной линии второго порядка имеет вид:
	\[ \underbrace{a_{11}x^2}_{=0} + \underbrace{a_{12}xy}_{=xy} + \underbrace{a_{21}yx}_{=0} + \underbrace{a_{22}y^2}_{=0} = 1, \]
	то матрица тензора этой линии в старом базисе:
	\[ T = \begin{bmatrix}
			0 & 1 \\
			0 & 0
	\end{bmatrix}. \]
	
	Получим матрицу \( \Gamma \) преобразований базиса. Для этого разложим новые орты по старому базису:
	\[ \left\{ \begin{array}{l}
		\vec{e}_1{'} = \frac{1}{\sqrt{2}}\vec{e}_1 + \frac{1}{\sqrt{2}}\vec{e}_2; \\
		\vec{e}_2{'} = -\frac{1}{\sqrt{2}}\vec{e}_1 + \frac{1}{\sqrt{2}}\vec{e}_2.
	\end{array} \right. \]
	
	Следовательно, матрица преобразований:
	\[ \Gamma = \frac{1}{\sqrt{2}} \begin{bmatrix}
			 1 & 1 \\
			-1 & 1
	\end{bmatrix}. \]
	
	Запишем матрицу обратных преобразований \( \Gamma^{(-1)} = \Gamma^T = \Gamma' \):
	\[ \Gamma' = \frac{1}{\sqrt{2}} \begin{bmatrix}
			1 & -1 \\
			1 & 1
	\end{bmatrix}. \]
	
	Найдем матрицу \( T\cdot\Gamma' \):
	\[ T\cdot\Gamma' = \frac{1}{\sqrt{2}} \begin{bmatrix}
			0 & 1 \\
			0 & 0
	\end{bmatrix} \cdot \begin{bmatrix}
			1 & -1 \\
			1 & 1
	\end{bmatrix}  = \frac{1}{\sqrt{2}} \begin{bmatrix}
			1 & 1 \\
			0 & 0
	\end{bmatrix}. \]
	
	Теперь найдем матрицу \( T' = \Gamma\cdot T \cdot\Gamma' \):
	\[ T' = \Gamma\cdot T \cdot\Gamma' = \frac{1}{2} \begin{bmatrix}
			 1 & 1 \\
			-1 & 1
	\end{bmatrix} \cdot \begin{bmatrix}
			1 & 1 \\
			0 & 0
	\end{bmatrix}  = \frac{1}{2} \begin{bmatrix}
			 1 & 1 \\
			-1 & -1
	\end{bmatrix}. \]
	
	Таким образом, в новом базисе \( (\vec{e}_1{'}, \vec{e}_2{'}) \), уравнение гиперболы имеет вид:
	\[ \sum\sum a_{ij}' x_i' x_k' = 1. \]
	Где коэффициенты \( a_{ij} \):
	\[ \left\{ \begin{array}{l}
			a_{11} = \frac{1}{2} \\
			a_{12} = \frac{1}{2} \\
			a_{21} = -\frac{1}{2} \\
			a_{22} = -\frac{1}{2}
	\end{array} \right. \]
	
	Следовательно, уравнение гиперболы принимает вид:
	\[ x^2 - y^2 = 2. \]
	\end{solution}
	
	\begin{remark}
	Если бы мы вместо матрицы \( T = \left[ \begin{smallmatrix} 0 & 1 \\ 0 & 0 \end{smallmatrix} \right] \) взяли бы симметричную матрицу \( T = \left[ \begin{smallmatrix} 0 & \frac{1}{2} \\ \frac{1}{2} & 0 \end{smallmatrix} \right] \), то уравнение гиперболы в новом базисе не изменилось бы, но матрица \( T' \) была бы симметричной: \( T' = \frac{1}{2}\left[ \begin{smallmatrix} 1 & 0 \\ 0 & -1 \end{smallmatrix} \right] \).
	\end{remark}
	
\subsection{Тензор второго ранга как линейный оператор}

	Важнейшим свойством тензора второго ранга является то, что его можно рассматривать как линейный оператор, а не только как матрицу коэффициентов центральных поверхностей второго порядка.
	
	\begin{definition}
	\textbf{Оператор} -- закон, сопоставляющий каждому вектору \( \vec{x} \) вполне определенный для него вектор \( \vec{y} \):
	\end{definition}
	\begin{equation}
		\vec{y} = T\vec{x}. \label{eq9:nn1}
	\end{equation}
	
	Свойство линейности заключается в выполнении условий:
	\begin{enumerate}
		\item \( T(\vec{x_1} + \vec{x_2}) = T\vec{x_1} + T\vec{x_2} \);
		\item \( T(\alpha\vec{x}) = \alpha T\vec{x} \), где \( \alpha \in \Re \).
	\end{enumerate}
	
	Пусть (\( \vec{e}_1, \vec{e}_2, \vec{e}_3 \)) -- базис. Подействуем оператором \( T \) на \( \vec{e}_1 \). Получим новый вектор. Разложим его по базису (\( \vec{e}_1, \vec{e}_2, \vec{e}_3 \)):
	\begin{equation}
		T\vec{e}_1 = a_{11}\vec{e}_1 + a_{12}\vec{e}_2 + a_{13}\vec{e}_3. \label{eq9:nn2}
	\end{equation}
	Или:
	\begin{equation}
		T\vec{e}_j = \sum a_{ij}\vec{e}_i. \label{eq9:nn3}
	\end{equation}
	
	Рассмотрим (\ref{eq9:nn2}). Домножив его слева на \( \vec{e}_3 \), получим \( \), \( \vec{e}_3T\vec{e}_1 = a_{31} \), так как \( \vec{e}_i\cdot\vec{e}_j = \delta{ij} \). В общем случае:
	\begin{equation}
		a_{ij} = \vec{e}_iT\vec{e}_j. \label{eq9:nn4}
	\end{equation}
	Теперь разложим вектор \( \vec{y} = T\vec{x} \) по базису (\( \vec{e}_1, \vec{e}_2, \vec{e}_3 \)):
	\[ \vec{y} = T\vec{x} = T(x_1\vec{e}_1 + x_2\vec{e}_2 + x_3\vec{e}_3). \]
	В силу свойства линейности оператора \( T \):
	\[ \vec{y} = \sum\limits_j x_j T\vec{e}_j. \]
	В силу (\ref{eq9:nn3}):
	\[ \vec{y} =  \sum\limits_i\underbrace{(\sum\limits_j a_{ij}x_j)}_{y_i}\vec{e}_i. \]
	Внутренняя сумма \( \sum\limits_j a_{ij}x_j \) дает компоненты вектора \( \vec{y} \) в базисе (\( \vec{e}_1, \vec{e}_2, \vec{e}_3 \)):
	\begin{equation}
		y_i = \sum a_{ij}x_j. \label{eq9:nn1a}
	\end{equation}
	Уравнение (\ref{eq9:nn1a}) -- это покоординатная форма соотношения (\ref{eq9:nn1}).
	Таким образом, набор девяти коэффициентов \( [a_{ij}] \) полностью определяет линейный оператор \( T \), то есть позволяет определить компоненты вектора \( \vec{y} \) через компоненты вектора \( \vec{x} \). Матрица \( [a_{ij}] \) называется \textbf{матрицей коэффициентов} линейного оператора \( T \) в данном базисе.

	Покажем, что она является тензором второго ранга. Для этого установим заком преобразования коэффициентов \( a_{ij} \) при переходе к новому базису. 
	
	В новом базисе соотношение (\ref{eq9:nn4}) имеет вид: \( a_{kl}' = \vec{e}_k{'} T\vec{e}_l{'} \). Но, согласно (\ref{eq9:1}), разложение нового базиса по старому имеет вид:
	\[ \left\{ \begin{array}{l}
		\vec{e}_k = \sum \gamma_{ki}\vec{e}_i; \\
		\vec{e}_j = \sum \gamma_{lj}\vec{e}_j.
	\end{array} \right. \]
	Тогда:
	\[ \begin{array}{r}
		a_{kl} = \sum \gamma_{ki}\vec{e}_i T \left(\sum \gamma_{lj}\vec{e}_j\right) = \sum\sum \gamma_{ki} \vec{e}_i T \gamma_{lj}\vec{e}_j = \\
		= \sum\sum \gamma_{ki}\vec{e}_i\gamma_{lj} T\vec{e}_j = \sum\sum \gamma_{ki}\gamma_{lj} a_{ij} \end{array} \]
		
	Последнее и совпадает с определением тензора второго ранга.

	Если координаты двух векторов \( \vec{x} \) и \( \vec{y} \) связаны линейным соотношением (\ref{eq9:nn1}) или (\ref{eq9:nn1a}), то матрица коэффициентов этого соотношения определяет тензор \( T \) в данном базисе.

	
\subsection{Главные оси тензора. Задачи на собственные векторы и собственные значения}
	
	Так как тензор \( T \) можно рассматривать как линейный оператор, то, действуя на какой-либо вектор \( \vec{x} \), он переводит его в некоторый вектор \( \vec{y} \) по закону \( \vec{y} = T\vec{x} \), где \( T \) задается матрицей коэффициентов \( [a_{ij}] \).
	
	Вектор \( \vec{x} \) отличается от вектора \( \vec{y} \) как по длине, так и по направлению. % три пикчерс
	
	\textbf{Задача}. Найти такой вектор, действуя на который линейный оператор \( T \) не поворачивал бы его, а только изменял по длине.
	То есть найти такой \( \vec{x} \), для которого:
	\begin{equation}
		T\vec{x} = \lambda\vec{x} \text{, где } \lambda \in \Re. \label{eq9:99n1}
	\end{equation}
	
	\begin{definition}
	Если для данного тензора  \( T \) существуют вектора, удовлетворяющие условию (\ref{eq9:99n1}), то такие вектора называют \textbf{собственными векторами} тензора \( T \). Направления собственных векторов называются \textbf{главными осями} тензора. Скалярные множители, соответствующие каждому собственному вектору, называются \textbf{собственными значениями}.
	\end{definition}
	
	\begin{comment}
	\textbf{В физике}:
	
	Если кристалл (или другая среда) изотропен, то закон Ома записывется в виде:
	\[ \vec{j} = \lambda\vec{E}, \]
	где \( \lambda \in \Re \) -- проводимость среды. Так как \( \lambda \) -- скаляр, то \( \vec{j} \uparrow\uparrow \vec{E} \). % пикча
	
	Если же кристалл анизотропен, то:
	\[ \vec{j} = \Lambda\vec{E}, \]
	где \( \Lambda = [\lambda_{ij}] \) -- тензор проводимости. Следовательно, \( \vec{j} \uparrow\downarrow \vec{E} \). % tu pikchas
	
	Но существуют такие направления поля \( \vec{E} \), для которых \( \vec{j} \uparrow\uparrow \vec{E} \), то есть появляется задача поиска таких направлений \( \vec{E} \), для которых \( \Lambda\vec{E} = \lambda\vec{E} \).
	\end{comment}
	
	Приступим к решению задачи (\ref{eq9:99n1}) по поиску собственных векторов и значений линейного оператора \( T \).
	
	Для этого запишем (\ref{eq9:99n1}) в матричном виде:
	\[ \begin{bmatrix}
			a_{11} & a_{12} & a_{13} \\
			a_{21} & a_{22} & a_{23} \\
			a_{31} & a_{32} & a_{33}
	\end{bmatrix}\cdot\begin{bmatrix}
			x_1 \\
			x_2 \\
			x_3
	\end{bmatrix} = \lambda \begin{bmatrix}
			x_1 \\
			x_2 \\
			x_3
	\end{bmatrix}. \]
	
	Умножая по правилу ``строка-на-столбец'', получим систему алгебраических уравнений:
	\[ \left\{ \begin{array}{l}
 		a_{11}x_1 + a{12}x_2 + a_{13}x_3 = \lambda x_1; \\
 		a_{21}x_1 + a{22}x_2 + a_{23}x_3 = \lambda x_2; \\
 		a_{31}x_1 + a{32}x_2 + a_{33}x_3 = \lambda x_3.
  	\end{array} \right. \]
	Или:
	\begin{equation} \left\{ \begin{array}{l}
 		(a_{11} - \lambda)x_1 + a{12}x_2 + a_{13}x_3 = 0; \\
 		a_{21}x_1 + (a{22} - \lambda)x_2 + a_{23}x_3 = 0; \\
 		a_{31}x_1 + a{32}x_2 + (a_{33} - \lambda)x_3 = 0.
  	\end{array} \right. \label{eq9:99n2} \end{equation}
	 
	 Система (\ref{eq9:99n2}) -- это система однородных линейных алгебраических уравнений. Известно, что она имеет нетривиальное решение, то есть решение, когда \( x_1, x_2 \) и \( x_3 \) не равны нулю одновременно, если определитель системы равен нулю.
	 \begin{equation} \begin{vmatrix}
		a_{11} - \lambda & a_{12} & a_{13} \\
		a_{21} & a_{22} - \lambda & a_{23} \\
		a_{31} & a_{32} & a_{33} - \lambda
	\end{vmatrix} = 0 \label{eq9:99n3} \end{equation}
	
	Равный нулю определитель (\ref{eq9:99n3}) даст кубическое уравнение относительно допустимых значений \( \lambda \), удовлетворяющих задаче (\ref{eq9:99n1}). Оно называется \textbf{характеристическим уравнением} тензора \( T \).
	
	Из алгебры известно, что кубическое уравнение (\ref{eq9:99n3}) имеет либо три действительных корня, либо один действительный и два комплексно сопряженных.
	
	В последнем случае мы не можем решить задачу (\ref{eq9:99n1}).
	
	Практически важным случаем является условие симметричности тензора \( T \):
	\begin{enumerate}
	\item
		существует три собственных значения тензора \( T \), все они действительные;
	\item
		три соответствующих собственных вектора взаимоперпендикулярны.
	\end{enumerate}
	
	Тогда, после нахождения трех собственных значений \( \lambda_1, \lambda_2 \) и \( \lambda_3 \) из уравнения (\ref{eq9:99n3}) и подставления их поочередно в систему (\ref{eq9:99n2}), найдем три соответствующие собственные вектора. Их нормировка даст базис главных осей (\( \vec{e}_1{'}, \vec{e}_2{'}, \vec{e}_3{'} \)).
	
	\begin{example}
	Пусть в базисе (\( \vec{e}_1, \vec{e}_2 \)) тензор второго ранга \( T \) задан матрицей  \( \bigl[ \begin{smallmatrix} 0 & 2 \\ 2 & 3 \end{smallmatrix} \bigr] \). Найти собственные значения \( \lambda_1, \lambda_2 \) и соответствующие собственные вектора (\( \vec{e}_1{'}, \vec{e}_2{'} \)).
	\end{example}
	
	\begin{solution}
	
	\begin{enumerate}
	\item Записываем уравнение (\ref{eq9:99n3}):
		\[ \det[T - \lambda E] = \begin{vmatrix}
				-\lambda & 2 \\
				2 & 3 - \lambda
		\end{vmatrix} = 0. \]

	\item Решение характеристического уравнения:
		\[ \begin{array}{l}
			\lambda(\lambda - 3) - 4 = \lambda^2 - 3\lambda - 4 = 0;\\
			\lambda_{1, 2} = \frac{3 \pm 5}{2}; \\
			\lambda_1 = -1; \\
			\lambda_2 = 4.
		\end{array} \]

	\item Постановка \( \lambda_1 \) и \( \lambda_2 \) поочередно в систему (\ref{eq9:99n2}):
		\begin{enumerate}
		\item \( \lambda_1 = -1 \):
		
		\[ \left\{ \begin{array}{l}
			x_1 + 2x_2 = 0; \\
			2x_1 + 4x_2 = 0.
		\end{array} \right. \]
		
		Возьмем \( x_{11} = 2, x_{12} = -1 \).
			
		Тогда \( \vec{x}_1{'} = \{ 2; -1 \} \).
			
		После нормировки: \( \vec{e}_1{'} = \frac{1}{\sqrt{5}}\{ 2; -1 \} \).
			
		\item \( \lambda_2 = 4 \):
		\[ \left\{ \begin{array}{l}
				-4x_1 + 2x_2 = 0; \\
				2x_1 - x_2 = 0.
		\end{array} \right. \]
		
		Возьмем \( x_{21} = 1, x_{22} = 2 \).
			
		Тогда \( \vec{x}_2{'} = \{ 1; 2 \} \).
		
		После нормировки: \( \vec{e}_2{'} = \frac{1}{\sqrt{5}}\{ 1; 2 \} \).
		\end{enumerate}
	\end{enumerate}
	
	Видно, что \( \vec{e}_1{'} \cdot \vec{e}_2{'} = 0 \).
	
	Таким образом, тензор \( T \) удлиняет вектор \( \vec{e}_1{'} \) в \( -1 \) раз, а вектор \( \vec{e}_2{'} \) -- \( 4 \) раза:
	\[ \left\{ \begin{array}{l} \begin{bmatrix}
		0 & 2 \\
		2 & 3
	\end{bmatrix} \cdot \begin{bmatrix}
		2 \\ -1
	\end{bmatrix} = -\begin{bmatrix}
		2 \\ -1
	\end{bmatrix}; \\[0.2cm]
	\begin{bmatrix}
		0 & 2 \\
		2 & 3
	\end{bmatrix} \cdot \begin{bmatrix}
		1 \\ 2
	\end{bmatrix} = 4\begin{bmatrix}
		1 \\ 2
	\end{bmatrix}. \end{array} \right. \]
	\end{solution}
	
	\begin{remark}
	Так как согласно (\ref{eq9:1}) преобразования базиса имеют вид:
	\[ \vec{e}_i{'} = \sum \gamma_{ij}\vec{e}_j \]
	
	То матрицы преобразований базиса:
	\[ \begin{array}{l} 
	\Gamma = \frac{1}{\sqrt{5}} \begin{bmatrix}
		1 & 2 \\
		-2 & 1
	\end{bmatrix}	= [\gamma_{ij}]; \\[0.2cm]
	\Gamma{'} = \frac{1}{\sqrt{5}} \begin{bmatrix}
		1 & -2 \\
		2 & 1
	\end{bmatrix} = [\gamma_{ij}{'}]. \end{array} \]
	
	Тогда матрица тензора в базисе главных осей:
	\[ T{'} = \Gamma T \Gamma{'} = \frac{1}{5} \begin{bmatrix}
		20 & 0 \\
		0 & -5
	\end{bmatrix} = \begin{bmatrix}
		4 & 0 \\
		0 & -1
	\end{bmatrix}. \]
	\end{remark}
	
\subsection{Матрица тензора в базисе главных осей}

	Пусть симметричный тензор \( T \) в исходном базисе (\( \vec{e}_1, \vec{e}_2, \vec{e}_3 \)) задан матрицей \( [a_{ij}] \).
	
	Найдем вид матрицы этого тензора \( T' = [a_{ij}{'}] \) в базисе главных осей (\( \vec{e}_1{'}, \vec{e}_2{'}, \vec{e}_3{'} \)).
	
	В базисе главных осей орты имеют такие компоненты:
	\[ \left\{ \begin{array}{l}
		\vec{e}_1{'} = \{ 1, 0, 0 \}; \\
		\vec{e}_2{'} = \{ 0, 1, 0 \}; \\
		\vec{e}_3{'} = \{ 0, 0, 1 \}.
	\end{array} \right. \]
	
	А так как это -- главные оси, то:
	\begin{equation} \left\{	\begin{array}{l}
		T'\vec{e}_1{'} = \lambda\vec{e}_1{'}; \\
		T'\vec{e}_2{'} = \lambda\vec{e}_2{'}; \\
		T'\vec{e}_3{'} = \lambda\vec{e}_3{'}.
	\end{array} \right. \label{eq9:910n1} \end{equation}
	
	Рассмотрим первое уравнение системы:
	\[ \begin{bmatrix}
		a_{11}{'} & a_{12}{'} & a_{13}{'} \\
		a_{21}{'} & a_{22}{'} & a_{23}{'} \\
		a_{31}{'} & a_{32}{'} & a_{33}{'}
	\end{bmatrix}\cdot\begin{bmatrix}
		1 \\ 0 \\ 0
	\end{bmatrix} = \lambda_1 \begin{bmatrix}
		1 \\ 0 \\ 0
	\end{bmatrix} \]
	
	Отсюда видно, что \( \lambda_1 = a_{11}{'}, a_{12}{'} = a_{13}{'} = 0 \).
	
	Далее, аналогично расписывая второе и третее уравнения, получим:
	\[ \lambda_2 = a_{22}{'}, a_{21}{'} = a_{23}{'} = 0 \]
	и
	\[ \lambda_3 = a_{33}{'}, a_{31}{'} = a_{32}{'} = 0. \]
	
	Таким образом, в базисе главных осей матрица тензора \( T \):
	\begin{equation}
		T' = [a_{ij}{'}] =
		\begin{bmatrix}
			\lambda_1 & 0 & 0 \\
			0 & \lambda_2 & 0 \\
			0 & 0 & \lambda_3
		\end{bmatrix} = 
		\begin{bmatrix}
			\lambda_1 \\ \lambda_2 \\ \lambda_3
		\end{bmatrix}E
		= [\lambda_i \delta_{ij}]. \label{eq9:910n2}
	\end{equation}
	
	\begin{definition}
	Матрица вида (\ref{eq9:910n2}) называется \textbf{диагональной}.
	\end{definition}
	
	Однако, возникает задача о единственности базиса главных осей (\( \vec{e}_1{’}, \vec{e}_2{’}, \vec{e}_3{’} \)). Рассмотрим три варианта:
	\begin{enumerate}
	\item Все три корня разные, то есть: \( \lambda_1 \ne \lambda_2 \ne \lambda_3 \).
	
	Тогда, в силу рассмотренной выше процедуры, базис единственнен.
	
	\item \( \lambda_1 = \lambda_2 = \lambda \ne \lambda_3 \).
	
	Тогда любой вектор \( \vec{x} \in (\vec{e}_1{’}, \vec{e}_2{’}) \) является также собственным вектором, то есть \( T\vec{x} = \lambda\vec{x} \).
	
	\begin{proof}
	
	Вектор \( \vec{x} \) здесь можно представить в виде
	\[ \vec{x} = \alpha\vec{e}_1{’} + \beta\vec{e}_2{’}. \]
	
	Тогда:
	\[ T\vec{x} = \alpha T\vec{e}_1{’} + \beta T\vec{e}_2{’} = \alpha\lambda\vec{e}_1{’} + \beta\lambda\vec{e}_2{’} = \lambda(\alpha\vec{e}_1{’} + \beta\vec{e}_2{’}) = \lambda\vec{x}. \]
	\end{proof}
	
	Таким образом, любую пару взаимоперпендикулярных векторов в плоскости (\( \vec{e}_1{’}, \vec{e}_2{’} \)) можно брать в качестве главных осей.

	\item \( \lambda_1 = \lambda_2 = \lambda_3 = \lambda \).
	
	Тогда любой вектор \( \vec{x} \) будет собственным для \( T \).
	
	Доказательство аналогично случаю \( \lambda_1 = \lambda_2 = \lambda \ne \lambda_3 \).
	\end{enumerate}
	
	\begin{theorem}
	Матрица тензора \( T \) в любом базисе будет иметь вид:
	\[ T = \begin{bmatrix}
	\lambda & 0 & 0 \\
	0 & \lambda & 0 \\
	0 & 0 & \lambda
	\end{bmatrix} = [\lambda\delta_{ij}] \text{ -- матрица подобия}, \]
	а сам такой тензор \( T \) называется \textbf{сферическим}.
	\end{theorem}
	
	Для сферического тензора любой вектор является собственным.
	
	Если у сферического тензора \( \lambda = 1 \), то:
	\[ T = \begin{bmatrix}
	1 & 0 & 0 \\
	0 & 1 & 0 \\
	0 & 0 & 1
	\end{bmatrix} = [\delta_{ij}] = E \]
	-- единичная матрица, она дает тождественные преобразования:
	\[ T\vec{x} = E\vec{x} = \vec{x}. \]
	
	\begin{theorem}
	Компоненты сферического тензора инвариантны по отношению к преобразованиям базиса, то есть \( a_{ij}{’} = a{ij} \).
	\end{theorem}
	
	\begin{proof}
	
	По определению тензора второго ранга:
	\[ a_{kl}{’} = \sum\sum \gamma_{ki}\gamma_{lj}a_{ij}. \]
	
	А так как \( a_{ij} = \lambda\delta_{ij} \), то:
	\[ a_{kl}{‘} = \sum\sum \gamma_{ki}\gamma_{lj}\lambda\delta_{ij} = \lambda\delta_{kl}. \]
	
	Таким образом, копоненты сферического тензора инвариантны.
	\end{proof}

\subsection{Тензорный эллипсоид}

	Согласно пункту \ref{sec9.4}, всякая центральная поверхность второго порядка определяется симметричным тензором второго ранга:
	\[ \sum\sum a_{ij} x_i x_j = 1 \]
	
	Оси симметрии этой поверхности совпадают с главными осями тензора. Уравнение этой поверхности в базисе главных осей называется \textbf{каноническим} уравнением и имеет вид:
	\[ \sum \lambda_ix_i{’}^2 = 1 \]
	
	Если все собственные числа положительны (случай, наиболее важный в физике), то этой пооверхностью будет эллипсоид с полуосями \( a_i = \frac{1}{\sqrt{\lambda_i}} \):
	\[ \frac{x_1{’}^2}{a_1^2} + \frac{x_2{’}^2}{a_2^2} + \frac{x_3{’}^2}{a_3^2} = 1. \]
	
	Он называется \textbf{тензорным эллипсоидом}.
	
	Если \( \lambda_1 = \lambda_2 \ne \lambda_3 \), то это -- эллипсоид вращения.
	
	Если \( \lambda_1 = \lambda_2 = \lambda_3 \), то это -- сфера.
	
	Таким образом, тензорной поверхностью сферического тензора является сфера радиуса \( R = \frac{1}{\sqrt{\lambda}} \).
