\chapter{Характеристики колебательных контуров}
    Рассмотрим последовательный и параллельный контуры \( RLC \), к которым
    прикладывается синусоидальное напряжение \( u = U\sin\omega t \).
    
\section{Установившиеся синусоидальные колебания в последовательном контуре}
    \subsection{Амплитудно- и фазо-частотные характеристики (АЧХ и ФЧХ)}
        Этот раздел подробно рассмотрен в \href{http://google.com}{работе Ф310}.
        Импеданс последовательного \( RLC \)-контура:
        \begin{equation}
            Z = R + \imone\left(\omega L - \frac{1}{\omega C}\right).
            \label{eq18:1}
        \end{equation}
        Модуль импеданса:
        \begin{equation}
            |Z| = \sqrt{R^2 + \left(\omega L - \frac{1}{\omega C}\right)^2}.
            \label{eq18:1a}
        \end{equation}
        Амплитуда тока в контуре:
        \begin{equation}
            I = \frac{U}{|Z|} = \frac{U}{\sqrt{R^2 + \left(\omega L -
            \frac{1}{\omega C}\right)^2}}.
            \label{eq18:2}
        \end{equation}
        Фазовый сдвиг:
        \begin{equation}
            \varphi = \mathrm{arctg}\frac{\omega L - \frac{1}{\omega C}}{R}.
            \label{eq18:3}
        \end{equation}
        Таким образом, ток:
        \[
            i = I\sin(\omega t - \varphi),
        \]
        где \( I \) и \( \varphi \) выражаются формулами (\ref{eq18:2}) и
        (\ref{eq18:3}). Из (\ref{eq18:2}) и (\ref{eq18:3}) видно, что амплитуда
        тока \( I \) и фазовый сдвиг \( \varphi \) зависят от частоты:
        \( I = I(\omega) \), \( \varphi = \varphi(\omega) \).
        
        \begin{definition}
            Зависимость \( I(\omega) \), задаваемая формулой (\ref{eq18:2}),
            называется \textit{амплитудно-частотной характеристикой} контура.
        \end{definition}
        
        \begin{definition}
            Зависимость \( \varphi(\omega) \), задаваемая формулой
            (\ref{eq18:3}), называется \textit{фазо-частотной характеристикой}
            контура.
        \end{definition}

        Исследуем вид функции (\ref{eq18:2}). Видно, что при \( \omega = 0 \)
        \( I = 0 \) и при \( \omega \to \infty \) \( I \to 0 \). Максимум
        функции \( I(\omega) \) наблюдается при \( R^2 + \left(\omega L -
        \frac{1}{\omega C}\right) \to \mathrm{min} \). Очевидно, что
        \( I \to \mathrm{max} = U/R = I_m \) при \( \omega L =
        \frac{1}{\omega C} \), то есть при \( \omega = \frac{1}{\sqrt{LC}} =
        \omega_0 \).
        
        % Графики при разных \( Q \) (то есть \( R \)), но при одинаковых
        % \( \omega = \omega_0 \):

        Теперь исследуем поведение функции \( \varphi(\omega) \), согласно
        (\ref{eq18:3}). Крутизна ФЧХ при \( \omega = \omega_0 \)
        \[
            \frac{\dd\varphi}{\dd\omega} \sim Q.
        \]
        
        \begin{proof}
            \[
                \left.\frac{\dd\varphi}{\dd\omega}\right|_{\omega = \omega_0} =
                \frac{\dd\tg\varphi}{\dd\omega} =
                \frac{\dd}{\dd\omega}\left(\frac{\omega L -
                \frac{1}{\omega C}}{R}\right) = 
                \left.\left(\frac{L}{R} + \frac{1}{\omega^2 RC}\right)
                \right|_{\omega = \omega_0} = 2\frac{L}{R} =
                2\frac{L}{R}\frac{\omega_0}{\omega_0} = 2\frac{Q}{\omega_0}.
            \]
        \end{proof}
    %---------------------------------------------------------------------------    
    \subsection{Ширина резонансной кривой}
        \begin{definition}
            \textbf{Шириной резонансной кривой} \( I(\omega) \) называется
            величина \( \Delta\omega = \omega_2 - \omega_1 \), где
            \( \omega_2 \) и \( \omega_1 \) -- граничные частоты, то есть
            частоты, на которых
            \[
                P(\omega_1) = P(\omega_2) = \frac{1}{2}P(\omega_0) = P_{{max}}.
            \]
        \end{definition}
        А так как \( P \sim I^2 \), то
        \[
            I(\omega_1) = I(\omega_2) = \frac{1}{\sqrt{2}}I(\omega_0).
        \]
        
        Найдем \( \Delta\omega \) на основе (\ref{eq18:2}):
        \[
            R^2 + \left(\omega L -\frac{1}{\omega C}\right)^2 = 2R^2,
        \]
        \[
            \omega L - \frac{1}{\omega C} = \pm R,
        \]
        \[
            \frac{\omega^2}{\omega_0^2} - 1 = \pm\omega RC,
        \]
        \[
            \omega^2 \pm \omega\omega_0^2RC - \omega_0^2 = 0.
        \]
        Мы получили уравнение для определения граничных частот \( \omega_1 \) и
        \( \omega_2 \). Его решение:
        \[
            \omega = \frac{\pm\omega_0^2RC \pm
            \sqrt{(\omega_0^2RC)^2 + 4\omega_0^2}}{2}
        \]
        А так как \( \omega_1 \) и \( \omega_2 \) больше нуля, то:
        \[
            \left[
            \begin{array}{l}
                \omega_1 = \frac{1}{2}\left(\sqrt{(\omega_0^2RC)^2 +
                4\omega_0^2} - \omega_0^2RC\right), \\
                \omega_2 = \frac{1}{2}\left(\sqrt{(\omega_0^2RC)^2 +
                4\omega_0^2} + \omega_0^2RC\right),
            \end{array}
            \right.
        \]
        и тогда ширина кривой
        \[
            \Delta\omega = \omega_2 - \omega_1 = \omega_0^2RC =
            \frac{RC\omega_0}{\sqrt{LC}} = \frac{\omega_0}{Q}.
        \]
        
        \begin{remark}
            Это выражение может быть принято за одно из определений добротности:
            \[
                Q = \frac{\omega_0}{\Delta\omega}.
            \]
            % графики
        \end{remark}
        
        \begin{remark}
            Есть задача: определить фазовый сдвиг \( \varphi \) между током
            \( i(t) \) и \( u(t) \) на левой и правой границе.
            \[
                I_{12}U\cos\varphi = \frac{1}{2}I_{{max}}U\cos0.
            \]
             Тогда \( \cos\varphi = 1/\sqrt{2} \), а сам фазовый сдвиг
             \( \varphi = \pi/4 \).
        \end{remark}
    
    \subsection{Кривая напряжений на конденсаторе}

        Амплитуда напряжений на конденсаторе:
        \begin{equation}
            U_C(\omega) = I(\omega)X_C(\omega) = I(\omega)\frac{1}{\omega C} =
            \frac{U}{\omega C \sqrt{R^2 + \left(\omega L -
            \frac{1}{\omega C}\right)^2}}.
            \label{eq18:4}
        \end{equation}
        Исследуем вид (\ref{eq18:4}):
        \begin{itemize}
            \item при \( \omega \to 0 \) \( U_C \to U \);
            \item при \( \omega \to \infty \) \( U_C \to 0 \).
        \end{itemize}
        
        Для определения экстремумов \( U_C(\omega) \) исследуем поведение
        знаменателя:
        \[
            \omega^2C^2\left(R^2 + \left(\omega L - \frac{1}{\omega C}\right)^2
            \right) = f(\omega).
        \]
        \[
            \frac{\dd f}{\dd\omega} = 0 = \frac{\dd}{\dd\omega}
            \left(\omega^2C^2 R^2 + \left(\frac{\omega^2}{\omega_0^2} -
            1\right)^2\right) = 2\omega R^2C^2 + 
            2\left(\frac{\omega^2}{\omega_0^2} -
            1\right)\frac{2\omega}{\omega_0^2} = 0,
        \]
        \[
            \omega^2 R^2C^2 + \frac{2\omega^2}{\omega_0^2} - 2 = 0,
        \]
        \[
            \frac{1}{2}\omega^2R^2C^2 + \frac{\omega^2}{\omega_0^2} - 1 = 0,
        \]
        \[
            \omega = \omega_0\sqrt{1 - \frac{\omega_0^2R^2C^2}{2}} =
            \omega_0\sqrt{1 - \frac{1}{2Q^2}},
        \]
        где
        \[
            \omega_0^2 = \frac{1}{LC},\ Q = \frac{1}{R}\sqrt{\frac{L}{R}}.
        \]
        
        При \( Q \gg 1 \), разложив в ряд Тейлора, получим
        \[
            \omega = \omega_0\left(1 - \frac{1}{4Q^2}\right) = \omega_{{max}}.
        \]
        На этой частоте: \( U_C \to \mathrm{max} \), однако, практически,
        \( \omega_{{max}} \approx \omega_0 \).

    \subsection{Использование резонанса при радиоприеме}

        Пусть в эфире есть много станций с различными несущими частотами.
        Требуется выделить нужную станцию, например, станцию на частоте
        \( \omega_3 \). Настраивая контур на \( \omega_3 \), то есть делая
        \( \omega_0 = \omega_3 \), мы выделим нужную станцию.
        
        Эпюры напряжений (сигналов):

    \subsection{Резонанс}

        По определению, резонанс в контуре наступает, когда ток \( i(t) \) и
        напряжение \( u(t) \) синфазны, то есть при \( \varphi = 0 \). Из этого
        определения и предыдущих пунктов следуют такие особенности резонанса:
        \begin{itemize}
            \item Резонанс в последовательном контуре наступает при
                \( \omega = \omega_0 \);
            \item Ток при резонансе максимален:
                \[
                    \left.I\right|_{\omega = \omega_0} \to \mathrm{max} =
                    \frac{U}{R};
                \]
            \item Импеданс \( Z \) становится чисто активным;
            \item Полное сопротивление при резонансе минимально:
                \[
                    |Z| \to \mathrm{min} = R;
                \]
            \item Напряжения на емкости \( C \) и индуктивности \( L \) при
                резонансе противофазны и в \( Q \) раз превышают приложенное
                напряжение, а напряжение на активном сопротивлении \( R \)
                равно приложенному.
        
                \begin{proof}
                    Напряжение на индуктивности:
                    \[
                        \dot{U}_L = \dot{I}\imone\omega L =
                        \frac{\dot{U}}{R}\imone\omega L = 
                        \imone\dot{U}\frac{1}{R}\sqrt{\frac{L}{C}} =
                        \imone Q\dot{U}.
                    \]
                    Напряжение на емкости:
                    \[
                        \dot{U}_C =
                        \dot{I}\left(-\imone\frac{1}{\omega C}\right) =
                        -\imone\dot{U}\frac{1}{R}\sqrt{\frac{L}{C}} =
                        -\imone Q\dot{U}.
                    \]
                    Напряжение на активном сопротивлении:
                    \[
                        \dot{U}_R = \dot{I}R = \frac{\dot{U}}{R}R = \dot{U}.
                    \]
                \end{proof}
        
                Поэтому в последовательном контуре резонанс называют
                \textbf{резонансом напряжений}.
        \end{itemize}

\section{Установившиеся колебания в параллельном контуре}
    Будем считать, что сопротивление \( R \) невелико, то есть на частоте
    \( \omega_0 \) \( R \ll \omega_0L \). Исследуем зависимость процессов в
    контуре от частоты приложенного напряжения \( \omega \).

    Импеданс:
    \[
        Z =  \frac{Z_1Z_2}{Z_1 + Z_2} =
        \frac{\frac{1}{\imone\omega C}(R + \imone\omega L)}{R +
        j\left(\omega L - \frac{1}{\omega C}\right)} \approx
        \frac{\frac{L}{C}}{R + j\left(\omega L - \frac{1}{\omega C}\right)}.
    \]
    
    Полное сопротивление:
    \[
        |Z| = \frac{L}{C}{\sqrt{R^2 + j\left(\omega L -
        \frac{1}{\omega C}\right)^2}}.
    \]
    
    Амплитуда тока:
    \[
        I = \frac{U}{|Z|}.
    \]
    
    Видно, что максимальное полное сопротивление равно
    \[
        |Z|_{{max}} = \frac{L}{RC} =
        R\left(\frac{1}{R}\sqrt{\frac{L}{C}}\right)^2 = Q^2R,
    \]
    следовательно, минимальный ток
    \[
        I_{{min}} = \frac{U}{Q^2R}.
    \]
    
    Из вышеизложенного можно сделать следующие выводы:
    \begin{itemize}
        \item резонанс наступает при \( \omega = \omega_0 \);
        \item полное сопротивление максимально при резонансе:
            \[
                |Z| \to \mathrm{max} = Q^2R,
            \]
            при \( R \to 0 \) \( |Z| \to \infty \);
        \item ток минимален при резонансе:
            \[
                I \to \mathrm{min} = \frac{U}{Q^2R};
            \]
        \item токи в ветвях одинаковы, противофазны и в \( Q \) раз больше
            подходящего:
            \[
                I_L = I_C = QI;
            \]
        
            \begin{proof}
                Ток на емкости:
                \[
                    \dot{I}_C = \frac{\dot{U}}{Z_C} =
                    \imone\omega C\dot{U} =
                    \imone\sqrt{\frac{C}{L}}\dot{U}.
                \]
                
                Ток на индуктивности:
                \[
                    \dot{I}_L = \frac{\dot{U}}{Z_L} =
                    -\imone\frac{\dot{U}}{\omega L} =
                    -\imone\sqrt{\frac{C}{L}}\dot{U}.
                \]
                
                Амплитуды токов:
                \[
                    |\dot{I}_C| = |\dot{I}_L| = \sqrt{\frac{C}{L}}U =
                    \frac{R}{R}\sqrt{\frac{C}{L}}U = \frac{U}{QR} = QI.
                \]
            \end{proof}
    \end{itemize}
    
\section{Поведение элементов \textit{R}, \textit{L} и \textit{C} на СВЧ}

    При частотах \( f \gtrsim 300 \)МГц (\( \lambda = \frac{c}{f} = 1 \)м) для
    настольных цепей перестает выполнятся условие квазистацинарности,
    следовательно, к таким цепям нельзя применять правила Кирхгофа.
    
    При частотах \( f \gtrsim 10^8 \)Гц элементы \( R \), \( L \) и \( C \)
    начинают вести себя как колебательные контуры, то есть у активного
    сопротивления появляются свойства емкости и индуктивности, у индуктивности
    -- емкости и активного сопротивления, у конденсатора -- индуктивности.
    
    \subsection{Элемент R}
    
        В активном сопротивлении появляется поле \( B \ne 0 \), следовательно,
        на СВЧ у активного сопротивления появляются индуктивные свойства:
        \[
            L_R = \frac{q}{\mu_0 i^2}\iiint\limits_V B\dd V \ne 0,
        \]
        \[
            X_L = \omega L_R.
        \]
        
        Далее, так как \( U_R = iR \ne 0 \), то между торцами \( R \) есть поле
        \[
            E = \frac{U}{l} \ne 0.
        \]
        Следовательно, у активного сопротивления так же проявляются и емкостные
        свойства:
        \[
            X_C = \frac{1}{\omega C}
        \]
        
        Таким образом, общая эквивалентная схема активного сопротивления на СВЧ:
        
        А резонансная частота:
        \[
            \omega_{\textit{рез}} = \frac{1}{\sqrt{L_RC_R}} \Rightarrow 
            f \sim 10^{10} \text{Гц}.
        \]
    
    \subsection{Элемент L}
    
        На СВЧ начинают проявлятся межвитковые емкости индуктивности \( C_L \),
        кроме того, из-за скин-эффекта, начинает расти активное сопротивление
        витков \( R_L \).
    
        Таким образом, общая эквивалентная схема индуктивности на СВЧ:
    
    \subsection{Кусок провода}
        На СВЧ увеличивается его активное сопротивление. Так как есть ток, то и
        поле \( B \ne 0 \), следовательно, индуктивность \( L \ne 0 \). А так
        как \( R \ne 0 \) и \( U \ne 0 \), то и \( E \ne 0 \), следовательно,
        емкость \( C \ne 0 \).
         
        Провод ведет себя как колебательный контур (излучающая антенна):
    
    \subsection{Элемент C}
    
        Согласно уравнению Максвелла (\ref{eq19:4}), переменное поле
        \( \vec{E}(t) \) вызывает появление магнитного поля \( \vec{B}(t) \):
        \[
            \oint\limits_C \vec{B}\cdot\dd\vec{l} =
            \frac{1}{c^2}\iint\limits_S \frac{\partial \vec{E}}{\partial t}
            \cdot\dd\vec{S}.
        \]
        
        Следовательно, у емкости проявляются индуктивные свойства. Но, согласно
        закону ЭМИ (\ref{eq11:3}), переменное магнитное поле вызывает вихревое
        электрическое поле:
        \[
            \oint\limits_C \vec{E}_{\textit{вихр}}\cdot\dd\vec{l} =
            -\iint\limits_S \frac{\partial \vec{B}}{\partial t}\cdot\dd\vec{S}.
        \]
        Оно, складываясь с полем \( \vec{E} \), даст результирующее поле 
        \( \vec{E}_{\textit{рез}} \) в конденсаторе. При увеличении \( \omega \)
        на краях конденсатора результирующее поле стремится к нулю
        \( \vec{E}_{\textit{рез}} \to 0 \).
