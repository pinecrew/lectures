\chapter{Метод комплексных амплитуд (МКА)}

    \begin{definition}
        \textbf{МКА} -- это метод расчета синусоидальных процессов в цепях
        \( RLC \). Он является алгебраической формой и обоснованием метода
        векторных диаграмм. Он позволяет описывать \textit{установившиеся}
        синусоидальные процессы системой не дифференциальных, а алгебраических
        уравнений первого порядка для комплексов токов и напряжений.
    \end{definition}
    
\section{Идея метода}

    Идея метода заключается в том, что всякой синусоидальной величине
    \( x(t) = A\sin\psi \) можно однозначно сопоставить комплексное число:
    \begin{equation}
        z = x + \imone y,
        \label{eq17:1}
    \end{equation}
    где \( \imone = \sqrt{-1} \) -- мнимая единица. Из известной формулы Эйлера:
    \begin{equation}
        e^{\imone\alpha} = \cos\alpha + \imone\sin\alpha
        \label{eq17:2}
    \end{equation}
    следует, что (\ref{eq17:1}) можно представить в виде:
    \begin{equation}
        z = Ae^{\imone\psi}.
        \label{eq17:3}
    \end{equation}
    Подставим (\ref{eq17:2}) в (\ref{eq17:3}) и сравним с (\ref{eq17:1}).
    Получим:
    \begin{align*}
        & x = A\cos\psi, \\
        & y = A\sin\psi,
    \end{align*}
    где \( A = \sqrt{x^2 + y^2} \) -- модуль комплексного числа, 
    \( \psi = \mathrm{arctg}(y/x) \) -- аргумент комплексного числа
    \( z \).
    
    Пусть аргумент \( \psi \) изменяется по закону
    \( \psi = \omega t + \varphi \). А так как любое комплексное число \( z \)
    можно представить в виде вектора, то каждой синусоидальной величине
    \( x(t) = A\cos(\omega t + \varphi) \) можно взаимно однозначно сопоставить
    комплексное число 
    \begin{equation}
        z = Ae^{\imone(\omega t + \varphi)}.
        \label{eq17:4}
    \end{equation}
    Тогда \( x \) будет действительной частью комплексного числа \( z \):
    \( x = \mathbf{Re\ }z \), а \( y \) -- мнимой: \( y = \mathbf{Im\ }z \).
    
    Поскольку в установившемся режиме частота \( \omega \) токов и напряжений
    на всех элементах одинакова и равна частоте задающего генератора, то есть
    она известна, то комплексное число (\ref{eq17:4}) удобно записывать в виде
    произведения:
    \[
        z = \left(Ae^{\imone\varphi}\right)e^{\imone\omega t}.
    \]
    
    \begin{definition}
        Комплексная величина \( \dot{A} = Ae^{\imone\varphi} \) называется
        \textit{комплексной амплитудой} или \textit{комплексом} синусоидальной
        величины \( x(t) \).
    \end{definition}
    
    Таким образом, если заданы ток \( i(t) \) или напряжение \( u(t) \) в
    синусоидальном виде, то есть
    \[
        \begin{array}{l}
            i(t) = I\sin(\omega t + \varphi_i) \\
            u(t) = U\sin(\omega t + \varphi_u),
        \end{array}
    \]
    где \( \varphi_i \) и \( \varphi_u \) -- начальные фазы тока и напряжения
    соответственно, то им можно взаимно однозначно сопоставить комплексные
    числа
    \begin{equation}
        \left\{
            \begin{array}{l}
                i(t) \leftrightarrow \dot{I}e^{\imone\omega t}, \\
                u(t) \leftrightarrow \dot{U}e^{\imone\omega t},
            \end{array}
        \right.
        \label{eq17:5}
    \end{equation}
    где \( \dot{I} = Ie^{\imone\varphi_i} \),
    \( \dot{U} = Ue^{\imone\varphi_u} \) -- комплексы тока и напряжения.
    
    Комплексный множитель \( e^{\imone\omega t} \) во всех линейных операциях 
    (закон Ома, правила Кирхгофа) всегда \textit{сокращается}.
    
\section{Комплексные сопротивления (импедансы)}

    Рассмотрим работу МКА сначала на отдельных элементах \( R \), \( L \) и
    \( C \).
    
    \subsection{Элемент R}
        
        Пусть \( u = U\sin\omega t \), тогда \( i = \frac{u}{R} \). Делаем
        прямые замены (\ref{eq17:5}):
        \[
            \left\{
                \begin{array}{l}
                    i(t) \to \dot{I}e^{\imone\omega t}, \\
                    u(t) \to \dot{U}e^{\imone\omega t}.
                \end{array}
            \right.
        \]
        
        Тогда закон Ома будет в комплексном виде:
        \begin{align}
            \dot{U}e^{\imone\omega t} = R\dot{I}e^{\imone\omega t}, \nonumber \\
            \dot{U} = R\dot{I}. \label{eq17:6}
        \end{align}
        
        Это означает, что на комплексной плоскости вектора \( \dot{U} \) и
        \( \dot{I} \) совпадают по направлению, а отличаются только длиной. Так
        что если \( i = I\sin\omega t \), то \( u = U\sin\omega t \), где
        \( U = IR \).
    
    \subsection{Элемент L}
    
        Для него, по определению:
        \[
            u = L\frac{\dd i}{\dd t}.
        \]
        
        Делая прямые замены (\ref{eq17:5}), получаем:
        \begin{equation}
            \dot{U} = \imone\omega L\dot{I}.
             \label{eq17:7}
        \end{equation}
        
        Покажем, что умножение любого комплексного числа на мнимую единицу
        \( \imone \) дает на комплексной плоскости поворот соответветствующего
        вектора на \( \pi/2 \).
        
        Пусть \( \dot{A} = Ae^{\imone\psi} \) -- некоторая комплексная
        амплитуда. Тогда вектор \( \dot{B} = \dot{A}e^{\imone\alpha} =
        Ae^{\imone(\psi + \alpha)} \) изобразится вектором, повернутым на угол
        \( +\alpha \) относительно исходного \( \dot{A} \).
        
        При \( \alpha = \pi/2 \):
        \[
            e^{\imone\frac{\pi}{2}} = \cos\frac{\pi}{2} +
            \imone\sin\frac{\pi}{2} = \imone.
        \]
        
        Таким образом, множитель \( \imone \) является оператором поворота
        вектора на комплексной плоскости на \( +\pi/2 \), а множитель
        \( -\imone \) -- на \( -\pi/2 \).
        
        Итак, согласно (\ref{eq17:7}), вектор \( \dot{U} \) -- это вектор
        \( \dot{I} \), удлиненный в \( \omega L \) раз и повернутый на
        \( \pi/2 \). Это означает, что напряжение на индуктивности обгоняет по
        фазе ток на \( \pi/2 \).
        
        Следовательно, если \( i = I\sin\omega t \), то
        \( u = U\sin\left(\omega t + \pi/2\right) \), где \( U = I \omega L \).
        
    \subsection{Элемент C}
    
        Для него, по определению:
        \[
            i = C\frac{\dd u}{\dd t}.
        \]
        Делая прямые замены (\ref{eq17:5}), получаем:
        \begin{equation}
            \dot{U} = \frac{1}{\imone\omega C}\dot{I} =
            -\frac{\imone}{\omega C}\dot{I}
            \label{eq17:8}
        \end{equation}
        Итак, согласно (\ref{eq17:8}), вектор \( \dot{U} \) -- это вектор
        \( \dot{I} \), укороченный в \( \omega C \) раз и повернутый на
        \( -\pi/2 \). Это означает, что напряжение на емкости отстает по фазе от
        тока на \( \pi/2 \). Следовательно, если \( i = I\sin\omega t \), то
        \( u = U\sin\left(\omega t - \pi/2\right) \), где
        \( U = \frac{I}{\omega C} \).
    
    Итак, для всех элементов \( R \), \( L \) и \( C \) комплексы токов и
    напряжений связаны \textit{линейными} соотношениями (\ref{eq17:6}),
    (\ref{eq17:7}) и (\ref{eq17:8}), которые могут быть записаны в общщем виде:
    \begin{equation}
        \dot{U} = Z\dot{I},
        \label{eq17:9}
    \end{equation}
    где комлексный коэффициент пропорциональности \( Z \) называется
    \textit{импедансом (комплексным сопротивлением)} данного элемента:
    \begin{align*}
        & Z = R \text{ -- импеданс активного сопротивления}, \\
        & Z = \imone\omega L \text{ -- импеданс индуктивности}, \\
        & Z = -\frac{\imone}{\omega C} \text{ -- импеданс емкости}.
    \end{align*}
    
    Уравнение (\ref{eq17:9}) -- закон Ома в комплексном виде. Взяв модули в
     (\ref{eq17:9}) от обеих частей получаем амплитудное соотношение
    \[
        U = |Z|I.
    \]
    
\section{Уравнения Кирхгофа в комплексном виде}

    Запишем уравнения Кирхгофа во временном виде для мгновенных значений
    синусоидальных величин:
    \begin{equation}
        \left\{
        \begin{array}{ll}
            \sum i_k(t) = 0 & \text{ -- для узлов}, \\
            \sum u_k(t) = 0 & \text{ -- для контуров}.
        \end{array}
        \right.
        \label{eq17:10}
    \end{equation}
    
    Делая в (\ref{eq17:10}) прямые замены (\ref{eq17:5}), получаем:
    \begin{equation}
        \left\{
        \begin{array}{l}
            \sum \dot{I}_k = 0, \\
            \sum \dot{U}_k = 0.
        \end{array}
        \right.
        \label{eq17:10a}
    \end{equation}
    
    Уравнения (\ref{eq17:10a}) и являются уравнениями Кирхгофа в комплексном
    виде. Решая их, получаем \( \dot{I}_k \) и \( \dot{U}_k \), затем, делая
    обратные подстановки (\ref{eq17:5}), получаем \( i_k \) и \( u_k \).
    
\section{Соединения импедансов}
    \subsection{Последовательное соединение}
        Для этой цепочки временные соотноошения выглядят следующим образом:
        \[
            \left\{
            \begin{array}{l}
                u(t) = u_1 + u_2, \\
                i(t) = i_1 = i_2.
            \end{array}
            \right.
        \]

        После прямых преобразований (\ref{eq17:5}):
        \[
            \dot{U} = \dot{U}_1 + \dot{U}_2.
        \]
        
        По закону Ома (\ref{eq17:9}):
        \[
            \dot{I}Z = \dot{I}Z_1 + \dot{I}Z_2,
        \]
        \[
            Z = Z_1 + Z_2.
        \]
        При последовательном соединении импедансы складываются. Причем полное
        сопротивление \( |Z| = |Z_1 + Z_2| \ne |Z_1| + |Z_2| \).
    
    \subsection{Параллельное соединение}
        
        Для этой цепочки временные соотноошения выглядят следующим образом:
        \[
            \left\{
            \begin{array}{l}
                u(t) = u_1 = u_2 \\
                i(t) = i_1 + i_2
            \end{array}
            \right.
        \]

        После прямых преобразований (\ref{eq17:5}):
        \[
            \dot{I} = \dot{I}_1 + \dot{I}_2.
        \]
        
        По закону Ома (\ref{eq17:9}):
        \[
            \frac{\dot{U}}{Z} = \frac{\dot{U}}{Z_1} + \frac{\dot{U}}{Z_2},
        \]
        \[
            Z = \frac{Z_1Z_2}{Z_1 + Z_2}.
        \]
    
\section{Импеданс двухполюсника}

    \begin{definition}
        \textbf{Линейным пассивным двухполюсником} называется цепь произвольно
        соединенных элементов \( R_k \), \( L_k \) и \( C_k \) (но не
        генераторов), имеющая два вывода, к которым может быть приложено внешнее
        напряжение.
    \end{definition}
    
    Пусть к двухполюснику приложено напряжение с комплексом \( \dot{U} \). Тогда
    ток будет иметь комплекс \( \dot{I} \). Так как все элементы цепи линейны,
    а также в силу линейности закона Ома и уравнений Кирхгофа, можно записать,
    что комплексы напряжения \( \dot{U} \) и тока \( \dot{I} \) связаны линейным
    соотношением:
    \begin{equation}
        \dot{U} = Z\dot{I}.
        \label{eq17:n1}
    \end{equation}
    
    Комплексный коэффициент \( Z \) в формуле (\ref{eq17:n1}) называется
    \textbf{импедансом двухполюсника}. Формула (\ref{eq17:n1}) выражает закон
    Ома в комплексном виде для линейного двухполюсника. Импеданс любого
    линейного двухполюсника может быть представлен в виде:
    \begin{equation}
        Z = r + \imone x,
        \label{eq17:n2}
    \end{equation}
    где \( r = \mathbf{Re\ }Z \) -- активное сопротивление двухполюсника, а
    \( x = \mathbf{Im\ }Z \) -- реактивное сопротивление. Импеданс на
    комплексной плоскости отображается вектором:
    \[
        \left\{
        \begin{array}{l}
            r = |Z|\cos\varphi \\
            x = |Z|\sin\varphi
        \end{array}
        \right.
    \]
    Активное сопротивление  \( r \) всегда больше либо равно нулю
    (\( r \geq 0 \)), а реактивное сопротивление может \( x \) принимать любые
    дейсвительные значения. Таким образом, любой двухполюсник можно представить
    в виде
    \[
        \left\{
        \begin{array}{l}
            r = |Z|\cos\varphi \\
            x = |Z|\sin\varphi
        \end{array}
        \right.
    \]
    % эквивалентная схема двухполюсника

    Экспоненциальная форма записи импеданса (\ref{eq17:n2}) выглядит следующим
    образом:
    \[
        Z = |Z|e^{\imone\varphi},
    \]
    где \( |Z| = \sqrt{r^2 + x^2} \) -- \textit{полное сопротивление
    двухполюсника} (модуль импеданса), а \( \varphi = \mathrm{arctg}(x/r) \) --
    аргумент импеданса.
    
    Тогда закон Ома может быть представлен в виде:
    \begin{equation}
        \dot{U} = \underbrace{|Z|e^{\imone\varphi}}_{Z}\dot{I}.
         \label{eq17:n2a}
    \end{equation}
    \begin{itemize}
        \item Если \( x > 0 \) и, следовательно,
            \( \varphi = \mathrm{arctg}(x/r) > 0 \), то напряжение на этом
            двухполюснике \textit{обгоняет} по фазе ток, и говорят, что
            сопротивление носит \textit{индуктивный характер};
        \item если \( x < 0 \) и, следовательно, \( \varphi < 0 \), то
            напряжение на этом двухполюснике \textit{отстает} по фазе от тока,
            и говорят, что сопротивление носит \textit{ёмкостный характер};
        \item если \( x = 0 \) и, следовательно, \( \varphi = 0 \), то
            напряжение на этом двухполюснике \textit{синфазно} току, и
            сопротивление становится чисто активным: \( |Z| = r \).
            Все процессы, происходящие в таком режиме, называются
            \textit{резонансными}.
    \end{itemize}
    
    
    
    \begin{remark}
        Если (\ref{eq17:n2}) взять по модулю, то получим закон Ома для линейного
        двухполюсника в модулях:
        \[
            U = |Z|I.
        \]
        В нем отражена информация только об амплитудных соотношениях.
    \end{remark}
    
\section{Мощность, рассеиваемая в двухполюснике}

    Если через двухполюсник задан именно ток \( \dot{I} \), то рассеиваемая в
    нем мощность:
    \[
        P = \frac{1}{2}IU\cos\varphi = \frac{1}{2}I^2|Z|\cos\varphi =
        \frac{1}{2}I^2r,
    \]
    где \( r \) -- активное сопротивление двухполюсника.
    
    Если же задано приложенное к двухполюснику напряжение  \( \dot{U} \), то
    рассеиваемая мощность:
    \[
        P = \frac{1}{2}IU\cos\varphi = \frac{1}{2}\frac{U^2}{|Z|}\cos\varphi =
        \frac{1}{2}\frac{U^2}{|Z|^2}|Z|\cos\varphi =
        \frac{1}{2}\frac{U^2}{|Z|^2}r.
    \]
    
    \begin{remark}
        Обычно задаются не амплитудные, а эффективные значения напряжения и
        тока. Тогда рассеиваемая мощность:
        \[
            P = I_{\textit{эф}}^2r = \frac{U_{\textit{эф}}^2}{|Z|^2}r.
        \]
    \end{remark}
    
    \begin{example}
        Имеется катушка индуктивностью \( L \) с внутренним сопротивлением
        \( R \). К ней приложено напряжение с комплексом \( \dot{U} \).
        Требуется определить ток \( I \), фазовый сдвиг \( \varphi \), активное
         \( r \), реактивное \( x \) и полное \( |Z| \) сопротивления, а также
          мощность \( P \), рассеиваемую в цепи.
    \end{example}

    \begin{solution}
        Импеданс (последовательное соединение \( R \) и \( \imone\omega L \)):
        \[
            Z = R + \imone\omega L.
        \]
        Активное сопротивление:
        \[
            r = \mathbf{Re\ }Z = R.
        \]        
        Реактивное сопротивление:
        \[
            x = \mathbf{Im\ }Z = \omega L.
        \]
        Полное сопротивление:
        \[
            |Z| = \sqrt{r^2 + x^2} = \sqrt{R^2 + (\omega L)^2}.
        \]
        Ток:
        \[
            I = \frac{U}{|Z|} = \frac{U}{\sqrt{R^2 + (\omega L)^2}}.
        \]
        Фазовый сдвиг:
        \[
            \varphi = \mathrm{arctg}\frac{x}{r} =
            \mathrm{arctg}(\omega L/R) > 0.
        \]
        Мощность:
        \[
            P = \frac{1}{2}I^2r = \frac{1}{2}I^2R.
        \]
    \end{solution}
    
    \begin{example}
        Имеется конденсатор емкостью \( C \) с утечкой \( R \). К нему приложено
        напряжение с комплексом \( \dot{U} \). Требуется определить ток \( I \),
        фазовый сдвиг \( \varphi \), активное \( r \), реактивное \( x \) и
        полное \( |Z| \) сопротивления, мощность \( P \), рассеиваемую в цепи.
    \end{example}

    \begin{solution}
        Импеданс (параллельное соединение \( R \) и 
        \( \frac{1}{\imone\omega C} \)):
        \[
            Z =
            \frac{R\cdot\frac{1}{\imone\omega C}}{R+\frac{1}{\imone\omega C}} =
            \frac{R}{1 + \imone\omega RC} =
            \frac{R - \imone\omega R^2C}{1 + (\omega RC)^2}.
        \]
        
        Активное сопротивление:
        \[
            r = \mathbf{Re\ }Z = \frac{R}{1 + (\omega RC)^2}.
        \]
        Реактивное сопротивление:
        \[
            x = \mathbf{Im\ }Z = -\frac{\omega R^2C}{1 + (\omega RC)^2}.
        \]
        Полное сопротивление:
        \[
            |Z|=\left|\frac{R}{1 + \imone\omega RC}\right| =
            \frac{R}{\sqrt{1 + (\omega RC)^2}}.
        \]
        Ток:
        \[
            I = \frac{U}{|Z|} = \frac{U}{\frac{R}{\sqrt{1+(\omega RC)^2}}} =
            \frac{U}{R}\sqrt{1 + (\omega RC)^2}.
        \]
        Фазовый сдвиг:
        \[
            \varphi = \mathrm{arctg}\frac{x}{r} =
            \mathrm{arctg}(-\omega RC) < 0.
        \]
        Мощность:
        \[
            P = \frac{1}{2}\frac{U^2}{|Z|^2}r = \frac{1}{2}\frac{U^2R}{R^2} =
            \frac{U^2}{R}.
        \]
    \end{solution}