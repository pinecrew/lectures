\lecture{Кинетическое уравнение Больцмана}
\section{Вывод уравнения}

С классической статистической точки зрения для описания электронного газа в
металле нужно ввести функцию распределения \( f \). Нормируем её удобным для
нас способом:
\[
    \int f(\vec{r},\vec{p},t)\,d^3p\,d^3r = N,
\]
где \( N \) -- полное число электронов в рассматриваемой системе. Тогда
плотность тока
\[
    \vec{j}(\vec{r},t) = \frac{q}{m} \int \vec{p} f(\vec{r},\vec{p},t) d^3p.
\]
Осталось лишь определить функцию распределения. Для этого воспользуемся теоремой
Лиувилля:
\[
    \der{f}{t} = 0 \Rightarrow
    \pder{f}{t} + \pder{f}{\vec{p}}\der{\vec{p}}{t} +
    \pder{f}{\vec{r}}\der{\vec{r}}{t} = 0.
\]
Или, если вспомнить второй закон Ньютона и определение скорости
\[
    \pder{f}{t} + \vec{F}\cdot\pder{f}{\vec{p}} + \vec{v}\cdot\pder{f}{\vec{r}}
    = 0.
\]
В случае нерегулярной в пространстве системы в правой части нуля не будет:
\[
    \pder{f}{t} + \vec{F}\cdot\pder{f}{\vec{p}} + \vec{v}\cdot\pder{f}{\vec{r}}
    = I(f).
\]
Это уравнение называется кинетическим уравнением Больцмана. Для того, чтобы его
решить, нужно определиться с конкретным видом функции \( I(f) \).

Пусть \( w(\vec{p},\vec{p}') \) -- плотность вероятности перехода электрона из
состояния с импульсом \( \vec{p} \) в состояние с импульсом \( \vec{p}' \) при
столкновении. Тогда
\[
    I(f) =
        \int [w(\vec{p}',\vec{p})f(\vec{p}')-w(\vec{p},\vec{p}')f(\vec{p})]d^3p'
\]
есть изменение количества электронов в состоянии с импульсом \( \vec{p} \). Эта
функция называется интегральной функцией столкновений. Самый простой вид этой
функции, учитывающий релаксацию и удовлетворяющий соотношению
\( \int I(f) d^3p = 0 \)
\[
    I(f) = -\frac{f - f_0}{\tau},\text{ где } f_0
    \text{ -- равновесная функция распределения.}
\]

При описании закона Ома, эффекта Холла и циклотронного резонанса будем считать
систему достаточно однородной в пространстве, поэтому пренебрежём слагаемым с
градиентом функции распределения и немного упростим уравнение:
\[
    \pder{f}{t} + \vec{F}\cdot\pder{f}{\vec{p}} = -\frac{f-f_0}{\tau}.
\]
Его решение имеет вид
\[
    f = \frac{1}{\tau}\int_{-\infty}^t dt' e^\frac{t'-t}{\tau} f_0(\vec{p}'),
\]
где \( \vec{p}'(t) \) -- решение уравнения
\[
    \der{p'}{t'} = F, \quad \vec{p}'(t) = \vec{p}.
\]

\section{Закон Ома}
Плотность тока в данном случае будет интегральной величиной:
\[
    \vec{j} = q\vec{v}dn = \frac{q}{m}\int\vec{p} f d^3p.
\]
Теперь нам нужно определить функцию распределения. Так как электроны
рассматриваются как газ, то в качестве равновесного логично взять распределение
Максвелла:
\[
    f_0(\vec{p}) = \frac{n}{(2mkT)^\frac{3}{2}}e^{-\frac{p^2}{2mkT}}.
\]
Теперь нужно определиться с \( \vec{p}' \). Пишем второй закон Ньютона:
\[
    \der{\vec{p}'}{t'} = q\vec{E}, \quad \vec{p}'(t) = \vec{p}.
\]
Для удобство введём переменную \( \xi = t' - t \) и дифференцирование по ней
будем отмечать точкой. Тогда
\[
    \dot{\vec{p}'} = q\vec{E}, \quad \vec{p}'(0) = \vec{p},
\]
а решение имеет вид
\[
    \vec{p'} = \vec{p} + q\vec{E}\xi.
\]
Отсюда
\[
    f_0(\vec{p}') = \frac{n}{(2mkT)^\frac{3}{2}}e^{-\frac{{p'}^2}{2mkT}} =
    \frac{n}{(2mkT)^\frac{3}{2}}e^{-\frac{(\vec{p} + q\vec{E}\xi)^2 }{2mkT}}.
\]
Искомая функция распределения определяется выражением
\[
    f = \frac{1}{\tau}\int_{-\infty}^0 d\xi e^\frac{\xi}{\tau}
    \frac{n}{(2mkT)^\frac{3}{2}}e^{-\frac{(\vec{p} + q\vec{E}\xi)^2 }{2mkT}}.
\]
Так как потом нам всё равно придётся её интегрировать, то не будем это делать
прямо сейчас, а подставим её в выражение для плотности тока и поменяем порядок
интегрирования:
\begin{gather*}
    \vec{j} = q\vec{v}dn = \frac{q}{m}\int d^3p \vec{p}
    \frac{1}{\tau}\int_{-\infty}^0 d\xi\, e^\frac{\xi}{\tau}
    \frac{n}{(2mkT)^\frac{3}{2}}e^{-\frac{(\vec{p} + q\vec{E}\xi)^2 }{2mkT}}
    = \\ =
    \frac{qn}{m\tau}
    \int_{-\infty}^0 d\xi\, e^\frac{\xi}{\tau}\int d^3p\, \vec{p}
    \frac{1}{(2mkT)^\frac{3}{2}}e^{-\frac{(\vec{p} + q\vec{E}\xi)^2 }{2mkT}}
\end{gather*}
Внутренний интеграл -- это математическое ожидание от распределения Гаусса.
Поэтому
\[
    \vec{j} = \frac{qn}{m\tau}
    \int_{-\infty}^0 d\xi\, e^\frac{\xi}{\tau} (-q\vec{E}\xi) =
    -\frac{nq^2}{m\tau} \vec{E} \int_{-\infty}^0 \xi e^\frac{\xi}{\tau}d\xi.
\]
Интегрируя в последний раз, приходим к ответу:
\[
    \vec{j} = \frac{nq^2\tau}{m}\vec{E}.
\]
\section{Эффект Холла}

Начнём сразу с решения уравнений для штрихованного импульса:
\[
    \left\{
    \begin{array}{l}
        \dot{p}'_x = \omega_c p'_y, \\
        \dot{p}'_y = qE - \omega_c p'_x, \\
        \dot{p}'_z = 0.
    \end{array}
    \right.
\]
Очевидно, \( p'_z = p_z \). Для решения первых двух уравнений введём величину
\( \zeta = p'_x + ip'_y \) и сгруппируем 2 этих уравнения в одно:
\[
    \dot{\zeta} + i\omega_c\zeta = iqE,
\]
откуда
\[
    \zeta = \frac{qE}{\omega_c} + Ae^{i\xi},
\]
где постоянная \( A \) определяется из начальных условий:
\[
    \zeta(0) = p_x + ip_y = \frac{qE}{\omega_c} + A,
\]
\[
    \Re A = p_x - \frac{qE}{\omega_c},\quad
    \Im A = p_y.
\]
Возвращаясь от комплексов к реальным величинам:
\[
    p'_x = \frac{qE}{\omega_c} +
        \left(p_x - \frac{qE}{\omega_c}\right)\cos\omega_c\xi +
        p_y\sin\omega_c\xi,\quad
    p'_y = p_y\cos\omega_c\xi -
        \left(p_x - \frac{qE}{\omega_c}\right)\cos\omega_c\xi.
\]
Теперь нам нужна величина \( p'^2 \) для подстановки в экспоненту:
\begin{align*}
    p'^2 & = {p'_x}^2 + {p'_y}^2 + {p'_z}^2 =\\
         & = \frac{q^2E^2}{\omega_c^2} +
        \left(p_x - \frac{qE}{\omega_c}\right)^2 + p_y^2 + p_z^2 +
        2\frac{qE}{\omega_c}\left[
            \left(p_x - \frac{qE}{\omega_c}\right)\cos\omega_c\xi +
            p_y\sin\omega_c\xi
        \right]=\\
        & = \frac{q^2E^2}{\omega_c^2}\cos^2\omega_c\xi
        + 2\frac{qE}{\omega_c}\left(p_x - \frac{qE}{\omega_c}\right)
        \cos\omega_c\xi +
        \left(p_x - \frac{qE}{\omega_c}\right)^2 +\\
        & + \frac{q^2E^2}{\omega_c^2}\sin^2\omega_c\xi
        + 2\frac{qE}{\omega_c}p_y\sin\omega_c\xi
        p_y^2 + p_z^2 =\\
        & = \left[p_x - \frac{qE}{\omega_c}(1-\cos\omega_c\xi)\right]^2 +
        \left(p_y + \frac{qE}{\omega_c}\sin\omega_c\xi\right)^2 + p_z^2.
\end{align*}
А теперь самое время искать проекции плотности тока:
\begin{gather*}
    j_x = \frac{qn}{m\tau} \int_{-\infty}^0 d\xi\, e^\frac{\xi}{\tau}
    \frac{1}{(2mkT)^\frac{3}{2}} \cdot \\
    \cdot \int d^3p\, p_x
    \exp\left(-\frac{1}{2mkT}\left\{
        \left[p_x - \frac{qE}{\omega_c}(1-\cos\omega_c\xi)\right]^2 +
        \left(p_y + \frac{qE}{\omega_c}\sin\omega_c\xi\right)^2 + p_z^2
    \right\}\right).
\end{gather*}
Второй интеграл -- это математическое ожидание \( p_x \) в трёхмерном
распределении Гаусса, поэтому
\[
    j_x = \frac{qn}{m\tau} \int_{-\infty}^0 d\xi\, e^\frac{\xi}{\tau}
    \frac{qE}{\omega_c}(1-\cos\omega_c\xi) =
    \frac{nq^2E}{m\tau}
    \int_{-\infty}^0e^\frac{\xi}{\tau}(1-\cos\omega_c\xi)d\xi.
\]
Взяв этот интеграл, получим
\[
    j_x = \frac{\omega_c\tau}{1+\omega_c^2\tau^2}\frac{nq^2\tau}{m}E.
\]
Совершенно аналогично можно придти к
\[
    j_y = -\frac{nq^2E}{m\tau}
    \int_{-\infty}^0e^\frac{\xi}{\tau}\sin\omega_c\xi d\xi =
    \frac{1}{1+\omega_c^2\tau^2}\frac{nq^2\tau}{m}E.
\]

\section{Циклотронный резонанс}
В завершение рассмотрим циклотронный резонанс. Записываем уравнение в проекциях:
\[
    \left\{
        \begin{array}{l}
            \dot{p}'_x = \omega_c p'_y, \\
            \dot{p}'_y = qE\cos\omega t' - \omega_c p'_x, \\
            \dot{p}'_z = 0.
        \end{array}
    \right.
\]
Снова, очевидно, \( p'_z = p_z \). Вводя комплексную величину
\( \zeta = p'_x + ip'_y \), получаем уравнение
\[
    \dot\zeta + i\omega_c\zeta = iqE\cos\omega t'.
\]
Его частное решение имеет вид \( Ce^{i\omega t'} + De^{-i\omega t'} \), где
\[
    (\omega + \omega_c)Ce^{i\omega t'} + (-\omega + \omega_c)De^{-i\omega t'}=
    \frac{qE}{2}(e^{i\omega t'} + e^{-i\omega t'}),
    \quad C = \frac{qE}{2(\omega + \omega_c)},
    \quad D = \frac{qE}{2(\omega_c - \omega)}.
\]
Решение имеет вид 
\[
    \zeta = Ae^{-i\omega_c \xi} +
    \frac{qE}{\omega_c^2 - \omega^2}[\omega_c\cos\omega(\xi+t) -
    i\omega\sin\omega(\xi+t)],
\]
где постоянная \( A \) определяется из начальных условий:
\[
    \zeta(0) = p_x + ip_y = A + 
    \frac{qE}{\omega_c^2 - \omega^2}[\omega_c\cos\omega t -
    i\omega\sin\omega t], \quad
    A =\left(p_x - \frac{qE}{\omega_c^2 - \omega^2}\omega_c\cos\omega t\right) +
    i\left(p_y + \frac{qE}{\omega_c^2 - \omega^2}\omega\sin\omega t\right).
\]
Теперь можно выписать зависимости проекций штрихованного импульса от \( xi \):
\begin{align*}
    & p'_x =
        \left(p_x - \frac{qE}{\omega_c^2 - \omega^2}\omega_c\cos\omega t\right)
        \cos\omega_c\xi
        +
        \left(p_y + \frac{qE}{\omega_c^2 - \omega^2}\omega\sin\omega t\right)
        \sin\omega_c\xi
        +
        \frac{qE}{\omega_c^2 - \omega^2}\omega_c\cos\omega(\xi+t), \\
    & p'_y =
        -\left(p_x - \frac{qE}{\omega_c^2 - \omega^2}\omega_c\cos\omega t\right)
        \sin\omega_c\xi
        +
        \left(p_y + \frac{qE}{\omega_c^2 - \omega^2}\omega\sin\omega t\right)
        \cos\omega_c\xi
        -
        \frac{qE}{\omega_c^2 - \omega^2}\omega\sin\omega(\xi+t),\\
    & p'_z = p_z.
\end{align*}
Для упрощения дальнейших выкладок обозначим
\[
    a = \frac{qE}{\omega_c^2 - \omega^2}\omega_c,\quad
    b = \frac{qE}{\omega_c^2 - \omega^2}\omega.
\]
Теперь выразим величину \( {p'}^2 \):
\begin{align*}
    {p'}^2 & = \left[(p_x - a\cos\omega t)\cos\omega_c\xi +
               (p_y + b\sin\omega t)\sin\omega_c\xi +
               a\cos\omega(\xi+t)\right]^2 + \\
           & + \left[-(p_x - a\cos\omega t)\sin\omega_c\xi +
               (p_y + b\sin\omega t)\cos\omega_c\xi -
               b\sin\omega(\xi+t)\right]^2 + p_z^2 =\\
           & = (p_x - a\cos\omega t)^2 + (p_y + b\sin\omega t)^2 + p_z^2 +
                a^2\cos^2\omega(\xi+t) + b^2\sin^2\omega(\xi+t) +\\
           & + 2a\cos\omega(\xi+t)\cdot[(p_x - a\cos\omega t)\cos\omega_c\xi +
               (p_y + b\sin\omega t)\sin\omega_c\xi] -\\
           & - 2b\sin\omega(\xi+t)\cdot[-(p_x - a\cos\omega t)\sin\omega_c\xi +
               (p_y + b\sin\omega t)\cos\omega_c\xi]
\end{align*}
Обратим пока внимание только на две последние строки и перегруппируем их:
\begin{align*}
    & 2a\cos\omega(\xi+t)\cdot[(p_x - a\cos\omega t)\cos\omega_c\xi +
       (p_y + b\sin\omega t)\sin\omega_c\xi] -\\
    & - 2b\sin\omega(\xi+t)\cdot[-(p_x - a\cos\omega t)\sin\omega_c\xi +
       (p_y + b\sin\omega t)\cos\omega_c\xi] = \\
    & = 2[a\cos\omega(\xi+t)\cos\omega_c\xi + b\sin\omega(\xi+t)\sin\omega_c\xi]
       (p_x - a\cos\omega t) + \\
    & + 2[a\cos\omega(\xi+t)\sin\omega_c\xi - b\sin\omega(\xi+t)\cos\omega_c\xi]
       (p_y + b\sin\omega t).
\end{align*}
Для выделения полных квадратов упростим выражение
\begin{gather*}
    [a\cos\omega(\xi+t)\cos\omega_c\xi + b\sin\omega(\xi+t)\sin\omega_c\xi]^2 +
    [a\cos\omega(\xi+t)\sin\omega_c\xi - b\sin\omega(\xi+t)\cos\omega_c\xi]^2=\\
    =a^2\cos^2\omega(\xi+t)\cos^2\omega_c\xi +
    b^2\sin^2\omega(\xi+t)\sin^2\omega_c\xi + \\ +
    a^2\cos^2\omega(\xi+t)\sin^2\omega_c\xi +
    b^2\sin^2\omega(\xi+t)\cos^2\omega_c\xi =\\=
    a^2\cos^2\omega(\xi+t) + b^2\sin^2\omega(\xi+t).
\end{gather*}
Оно совпадает со слагаемым в выражении \( {p'}^2 \), поэтому
\begin{align*}
    {p'}^2 = & (p_x - a\cos\omega t + a\cos\omega(\xi+t)\cos\omega_c\xi
    + b\sin\omega(\xi+t)\sin\omega_c\xi)^2 + \\
    & + (p_y + b\sin\omega t + a\cos\omega(\xi+t)\sin\omega_c\xi
    - b\sin\omega(\xi+t)\cos\omega_c\xi)^2 + p_z^2.
\end{align*}
Наша задача -- определить среднюю поглощаемую мощность:
\[
    P = \average{\vec{j}\cdot\vec{E}} =
    \int_0^\frac{2\pi}{\omega} j_x E_0 \cos\omega t dt.
\]
При этом \( j_x \) определяется выражением
\begin{gather*}
    j_x = \frac{qn}{m\tau} \int_{-\infty}^0 d\xi\, e^\frac{\xi}{\tau}
    \frac{1}{(2mkT)^\frac{3}{2}} \cdot \\
    \cdot \int d^3p\, p_x
    \exp\left(-\frac{(p_x - \average{p_x})^2 + (p_y - \average{p_y})^2
    + p_z^2}{2mkT}\right),
\end{gather*}
где
\begin{align*}
    & \average{p_x} = a\cos\omega t - a\cos\omega(\xi+t)\cos\omega_c\xi
    - b\sin\omega(\xi+t)\sin\omega_c\xi, \\
    & \average{p_y} = -b\sin\omega t - a\cos\omega(\xi+t)\sin\omega_c\xi
    + b\sin\omega(\xi+t)\cos\omega_c\xi.
\end{align*}
Взяв последний интеграл, получаем
\[
    j_x = \frac{qn}{m\tau} \int_{-\infty}^0 d\xi\,
    e^\frac{\xi}{\tau}\average{p_x}.
\]
Не будем интегрировать сразу, а подставим это выражение в интеграл мощности и
поменяем порядок интегрирования
\[
    P = \frac{qnE_0}{m\tau} \int_{-\infty}^0 d\xi\,
    e^\frac{\xi}{\tau}\int_0^\frac{2\pi}{\omega} \average{p_x}\cos\omega t dt.
\]
Берём последний интеграл:
\begin{gather*}
    \int_0^\frac{2\pi}{\omega} \average{p_x}\cos\omega t dt = \\ =
    \int_0^\frac{2\pi}{\omega} [a\cos^2\omega t -
    a\cos\omega_c\xi\cos\omega(\xi+t)\cos\omega t
    - b\sin\omega_c\xi\sin\omega(\xi+t)\cos\omega t] dt
\end{gather*}
Для краткости, проинтегрируем слагаемые по отдельности:
\begin{align*}
    & \int_0^\frac{2\pi}{\omega}\cos^2\omega t = \frac{1}{2}, \quad
     \int_0^\frac{2\pi}{\omega}\cos\omega(\xi+t)\cos\omega t dt =
        \int_0^\frac{2\pi}{\omega}\cos\omega\xi\cos^2\omega t dt -
        \int_0^\frac{2\pi}{\omega}\sin\omega\xi\sin\omega t\cos\omega t dt =
        \frac{1}{2}\cos\omega\xi,\\
    & \int_0^\frac{2\pi}{\omega}\sin\omega(\xi+t)\cos\omega t dt =
        \int_0^\frac{2\pi}{\omega}\sin\omega\xi\cos^2\omega t dt +
        \int_0^\frac{2\pi}{\omega}\cos\omega\xi\sin\omega t\cos\omega t dt =
        \frac{1}{2}\sin\omega\xi.
\end{align*}
Подставляя полученные значения, получаем
\[
    P = \frac{qnE_0}{2m\tau} \int_{-\infty}^0 d\xi\,
    e^\frac{\xi}{\tau}\left( a - a\cos\omega_c\xi\cos\omega\xi -
    b\sin\omega_c\xi\sin\omega\xi \right).
\]
Теперь преобразуем произведения тригонометрических функций:
\[
    \cos\omega_c\xi\cos\omega\xi =
    \frac{\cos(\omega_c + \omega)\xi + \cos(\omega_c - \omega)\xi}{2},\quad
    \sin\omega_c\xi\sin\omega\xi =
    \frac{\cos(\omega_c - \omega)\xi - \cos(\omega_c + \omega)\xi}{2}.
\]
\[
    P = \frac{qnE_0}{2m\tau} \int_{-\infty}^0 d\xi\,
    e^\frac{\xi}{\tau}\left[ a - \frac{a - b}{2}\cos(\omega_c + \omega)\xi -
    \frac{a + b}{2}\cos(\omega_c - \omega)\xi \right].
\]
Теперь нужно взять интегралы вида
\[
    \int_{-\infty}^0 e^\frac{\xi}{\tau}\cos\alpha\xi d\xi = 
    \Re\int_{-\infty}^0 e^\frac{\xi}{\tau}e^{i\alpha\xi} d\xi =
    \Re\frac{1}{\frac{1}{\tau} + i\alpha} =
    \Re\frac{\tau(1-i\alpha\tau)}{1+\alpha^2\tau^2}=
    \frac{\tau}{1+\alpha^2\tau^2}.
\]
\[
    P = \frac{qnE_0}{2m\tau} \left[ a\tau -
        \frac{a - b}{2}\frac{\tau}{1+(\omega_c+\omega)^2\tau^2}-
        \frac{a + b}{2}\frac{\tau}{1+(\omega_c-\omega)^2\tau^2}\right].
\]
Теперь подставим \(a\) и \(b\) и попробуем упростить полученное выражение:
\[
    P = \frac{nq^2E_0^2}{4m(\omega_c^2 - \omega^2)} \left[ 2\omega_c -
        \frac{\omega_c - \omega}{1+(\omega_c+\omega)^2\tau^2}-
        \frac{\omega_c + \omega}{1+(\omega_c-\omega)^2\tau^2}\right].
\]
Приводим выражение в скобках к общему знаменателю:
\[
    P = \frac{nq^2E_0^2}{4m(\omega_c^2 - \omega^2)}
        \frac{
        2\omega_c[1+(\omega_c+\omega)^2\tau^2][1+(\omega_c-\omega)^2\tau^2] -
        (\omega_c - \omega)[1 + (\omega_c - \omega)^2\tau^2] -
        (\omega_c + \omega)[1 + (\omega_c + \omega)^2\tau^2]}
        {[1+(\omega_c+\omega)^2\tau^2][1+(\omega_c-\omega)^2\tau^2]}.
\]
Теперь раскрываем скобки в числителе и избавляемся от всего лишнего:
\[
    P = \frac{nq^2E_0^2}{4m(\omega_c^2 - \omega^2)}
        \frac{
        2\omega_c(\omega_c^2-\omega^2)^2\tau^4 +
         (\omega_c + \omega)(\omega_c - \omega)^2\tau^2 +
         (\omega_c - \omega)(\omega_c + \omega)^2\tau^2}
        {[1+(\omega_c+\omega)^2\tau^2][1+(\omega_c-\omega)^2\tau^2]}.
\]
Сокращаем числитель второй дроби и знаменатель первой на разность квадратов
частот:
\[
    P = \frac{nq^2E_0^2\tau^2}{4m}
        \frac{
        2\omega_c(\omega_c^2-\omega^2)\tau^2 +
         (\omega_c - \omega) +
         (\omega_c + \omega)}
        {[1+(\omega_c+\omega)^2\tau^2][1+(\omega_c-\omega)^2\tau^2]}.
\]
Доупростив числитель и немного преобразовав знаменатель, получаем окончательный
ответ:
\[
    P = \frac{n\omega_c\tau^2q^2E^2}{2m}
    \frac{1 + (\omega_c^2 - \omega^2)\tau^2}
    {\left[1 + (\omega^2 - \omega_c^2)\tau^2\right]^2 +
    4\omega_c^2\tau^2}.
\]
\section{Нетрудно заметить, что\ldots}
\ldots все три результата были получены ранее, в одноэлектронной теории. Быть
может, это можно заметить из самого уравнения
\[
    \pder{f}{t} + \vec{F}\cdot\pder{f}{\vec{p}} = -\frac{f-f_0}{\tau}?
\]
Попробуем получить уравнение движения <<средней>> частицы:
\[
    \average{\der{\vec{p}}{t}} = \der{\average{\vec{p}}}{t} =
    \der{}{t}\int\vec{p} f d^3p.
\]
Теперь втащим дифференцирование под интеграл и учтём, что импульс явно от
времени не зависит:
\[
    \der{}{t}\int\vec{p} f d^3p = \int\pder{(\vec{p} f)}{t} d^3p =
    \int\vec{p}\pder{f}{t} d^3p.
\]
Теперь вытащим из уравнения частную производную по времени и подставим в
интеграл:
\begin{gather*}
    \int\vec{p}\pder{f}{t} d^3p = -\int\vec{p}
    \left(\vec{F}\cdot\pder{f}{\vec{p}} + \frac{f-f_0}{\tau}\right)d^3p = \\ =
    -\int\vec{p}\left(\vec{F}\cdot\pder{f}{\vec{p}}\right)d^3p
    -\frac{1}{\tau}\int\vec{p}fd^3p + \frac{1}{\tau}\int\vec{p}f_0d^3p.
\end{gather*}
Остановимся на двух последних слагаемых. Первое из них равно
\( -\average{\vec{p}}/\tau \), второе же равно нулю, так как в равновесии
движения нет. Уравнение <<средней>> частицы принимает вид:
\[
    \der{\average{\vec{p}}}{t} =
    -\int\vec{p}\left(\vec{F}\cdot\pder{f}{\vec{p}}\right)d^3p
    -\frac{\average{\vec{p}}}{\tau}.
\]
Попробуем преобразовать интеграл. Дальше будем рассматривать только такие силу
\( \vec{F} \), для которой
\[
    \pder{F_x}{p_x} + \pder{F_y}{p_y} + \pder{F_z}{p_z} = 0.
\]
Для такой <<хорошей>> силы\footnote{между прочим, сила Лоренца -- <<хорошая>>
сила} интеграл можно преобразовать к виду
\[
    \int\vec{p}\left(\vec{F}\cdot\pder{f}{\vec{p}}\right)d^3p =
    \int\vec{p}\sum_{\alpha\in\{x,y,z\}}\pder{fF_\alpha}{p_\alpha}d^3p.
\]
В свою очередь,
\[
    \vec{p} = \sum_{\beta\in\{x,y,z\}}\vec{e}_\beta p_\beta,
\]
поэтому интеграл примет вид
\[
    \int\sum_{\beta\in\{x,y,z\}}\vec{e}_\beta p_\beta
    \sum_{\alpha\in\{x,y,z\}}\pder{fF_\alpha}{p_\alpha}d^3p.
\]
Меняя суммы и интеграл местами, получаем
\[
    \sum_{\alpha,\beta}\vec{e}_\beta \int p_\beta \pder{fF_\alpha}{p_\alpha}
    dp_xdp_ydp_z.
\]
Теперь рассмотрим 2 интеграла:
\begin{enumerate}
    \item
            \[
                \int p_x \pder{fF_x}{p_x} dp_xdp_ydp_z =
                \int_{-\infty}^{+\infty} dp_y
                \int_{-\infty}^{+\infty} dp_z
                \int_{-\infty}^{+\infty} p_x \pder{fF_x}{p_x} dp_x.
            \]
            Берём интеграл по \( p_x \) по частям, учитываем, что на
            бесконечности функция распределения обращается в ноль и получаем
            \[
                \int p_x \pder{fF_x}{p_x} dp_xdp_ydp_z = -\average{F_x}.
            \]
    \item
            \[
                \int p_y \pder{fF_x}{p_x} dp_xdp_ydp_z =
                \int_{-\infty}^{+\infty} p_y dp_y
                \int_{-\infty}^{+\infty} dp_z
                \int_{-\infty}^{+\infty} \pder{fF_x}{p_x} dp_x.
            \]
            Интеграл по \( p_x \) равен нулю, из-за того, что на
            бесконечности функция распределения обращается в ноль:
            \[
                \int p_x \pder{fF_x}{p_x} dp_xdp_ydp_z = 0.
            \]
\end{enumerate}
Этих двух интегралов достаточно, чтобы показать, что
\[
    \int p_\beta \pder{fF_\alpha}{p_\alpha} dp_xdp_ydp_z =
    -\average{F_\alpha} \delta_{\alpha\beta}.
\]
Тогда сумма сильно упрощается
\[
    \sum_{\alpha,\beta}\vec{e}_\beta \int p_\beta \pder{fF_\alpha}{p_\alpha}
    dp_xdp_ydp_z =
    - \sum_{\alpha,\beta}\vec{e}_\beta \average{F_\alpha} \delta_{\alpha\beta} =
    - \sum_{\alpha}\vec{e}_\alpha \average{F_\alpha} = -\average{\vec{F}}.
\]
Возвращаясь к уравнению движения, получаем:
\[
    \der{\average{\vec{p}}}{t} = \average{\vec{F}}
    -\frac{\average{\vec{p}}}{\tau}.
\]
Знакомое уравнение?
