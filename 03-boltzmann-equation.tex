\section{Кинетическое уравнение Больцмана}
\subsection{Вывод уравнения}

С классической статистической точки зрения для описания электронного газа в
металле нужно ввести функцию распределения \( f \). Отнормируем её удобным для
нас способом:
\[
    \int f(\vec{r},\vec{p},t)\,d^3p\,d^3r = N,
\]
где \( N \) -- полное число электронов в рассматриваемой системе. Тогда
плотность тока
\[
    \vec{j}(\vec{r},t) = \frac{q}{m} \int \vec{p} f(\vec{r},\vec{p},t) d^3p.
\]
Осталось лишь определить функцию распределения. Для этого воспользуемся теоремой
Лиувилля:
\[
    \der{f}{t} = 0 \Rightarrow
    \pder{f}{t} + \pder{f}{\vec{p}}\der{\vec{p}}{t} +
    \pder{f}{\vec{r}}\der{\vec{r}}{t} = 0.
\]
Или, если вспомнить второй закон Ньютона и определение скорости
\[
    \pder{f}{t} + \vec{F}\cdot\pder{f}{\vec{p}} + \vec{v}\cdot\pder{f}{\vec{r}}
    = 0.
\]
В случае нерегулярной в пространстве системы в правой части нуля не будет:
\[
    \pder{f}{t} + \vec{F}\cdot\pder{f}{\vec{p}} + \vec{v}\cdot\pder{f}{\vec{r}}
    = I(f).
\]
Это уравнение называется кинетическим уравнением Больцмана. Для того, чтобы его
решить, нужно определиться с конкретным видом функции \( I(f) \).

Пусть \( w(\vec{p},\vec{p}') \) -- плотность вероятности перехода электрона из
состояния с импульсом \( \vec{p} \) в состояние с импульсом \( \vec{p}' \) при
столкновении. Тогда
\[
    I(f) =
        \int [w(\vec{p}',\vec{p})f(\vec{p}')-w(\vec{p},\vec{p}')f(\vec{p})]d^3p'
\]
есть изменение количества электронов в состоянии с импульсом \( \vec{p} \). Эта
функция называется интегральной функцией столкновений. Самый простой вид этой
функции, учитывающий релаксацию и удовлетворяющий соотношению
\( \int I(f) d^3p = 0 \)
\[
    I(f) = -\frac{f - f_0}{\tau},\text{ где } f_0
    \text{ -- равновесная функция распределения.}
\]

При описании закона Ома, эффекта Холла и циклотронного резонанса будем считать
ситсему достаточно однородной в пространстве, поэтому пренебрежём слагаемым с
градиентом функции распределения и немного упростим уравнение:
\[
    \pder{f}{t} + \vec{F}\cdot\pder{f}{\vec{p}} = -\frac{f-f_0}{\tau}.
\]
Его решение имеет вид
\[
    f = \frac{1}{\tau}\int_{-\infty}^t dt' e^\frac{t'-t}{\tau} f(\vec{p}'),
\]
где \( \vec{p}'(t) \) -- решение уравнения
\[
    \der{p'}{t'} = F, \quad \vec{p}'(t) = \vec{p}.
\]

\subsection{Закон Ома}
\subsection{Эффект Холла}
\subsection{Циклотронный резонанс}
\subsection{Нетрудно заметить, что\ldots}
\ldots все три результата были получены ранее, в одноэлектронной теории. Быть
может, это можно заметить из самого уравнения
\[
    \pder{f}{t} + \vec{F}\cdot\pder{f}{\vec{p}} = -\frac{f-f_0}{\tau}?
\]
Попробуем получить уравнение движения <<средней>> частицы:
\[
    \average{\der{\vec{p}}{t}} = \der{\average{\vec{p}}}{t} =
    \der{}{t}\int\vec{p} f d^3p.
\]
Теперь втащим дифференцирование под интеграл и учтём, что импульс явно от
времени не зависит:
\[
    \der{}{t}\int\vec{p} f d^3p = \int\pder{(\vec{p} f)}{t} d^3p =
    \int\vec{p}\pder{f}{t} d^3p.
\]
Теперь вытащим из уравнения частную производную по времени и подставим в
интеграл:
\begin{gather*}
    \int\vec{p}\pder{f}{t} d^3p = -\int\vec{p}
    \left(\vec{F}\cdot\pder{f}{\vec{p}} + \frac{f-f_0}{\tau}\right)d^3p = \\ =
    -\int\vec{p}\left(\vec{F}\cdot\pder{f}{\vec{p}}\right)d^3p
    -\frac{1}{\tau}\int\vec{p}fd^3p + \frac{1}{\tau}\int\vec{p}f_0d^3p.
\end{gather*}
Остановимся на двух последних слагаемых. Первое из них равно
\( -\average{\vec{p}}/\tau \), второе же равно нулю, так как в равновесии
движения нет. Уравнение <<средней>> частицы принимает вид:
\[
    \der{\average{\vec{p}}}{t} =
    -\int\vec{p}\left(\vec{F}\cdot\pder{f}{\vec{p}}\right)d^3p
    -\frac{\average{\vec{p}}}{\tau}.
\]
Попробуем преобразовать интеграл. Дальше будем рассматривать только такие силу
\( \vec{F} \), для которой
\[
    \pder{F_x}{p_x} + \pder{F_y}{p_y} + \pder{F_z}{p_z} = 0.
\]
Для такой <<хорошей>> силы\footnote{между прочим, сила Лоренца -- <<хорошая>>
сила} интеграл можно преобразовать к виду
\[
    \int\vec{p}\left(\vec{F}\cdot\pder{f}{\vec{p}}\right)d^3p =
    \int\vec{p}\sum_{\alpha\in\{x,y,z\}}\pder{fF_\alpha}{p_\alpha}d^3p.
\]
В свою очередь,
\[
    \vec{p} = \sum_{\beta\in\{x,y,z\}}\vec{e}_\beta p_\beta,
\]
поэтому интеграл примет вид
\[
    \int\sum_{\beta\in\{x,y,z\}}\vec{e}_\beta p_\beta
    \sum_{\alpha\in\{x,y,z\}}\pder{fF_\alpha}{p_\alpha}d^3p.
\]
Меняя суммы и интеграл местами, получаем
\[
    \sum_{\alpha,\beta}\vec{e}_\beta \int p_\beta \pder{fF_\alpha}{p_\alpha}
    dp_xdp_ydp_z.
\]
Теперь рассмотрим 2 интеграла:
\begin{enumerate}
    \item
            \[
                \int p_x \pder{fF_x}{p_x} dp_xdp_ydp_z =
                \int_{-\infty}^{+\infty} dp_y
                \int_{-\infty}^{+\infty} dp_z
                \int_{-\infty}^{+\infty} p_x \pder{fF_x}{p_x} dp_x.
            \]
            Берём интеграл по \( p_x \) по частям, учитываем, что на
            бесконечности функция распределения обращается в ноль и получаем
            \[
                \int p_x \pder{fF_x}{p_x} dp_xdp_ydp_z = -\average{F_x}.
            \]
    \item
            \[
                \int p_y \pder{fF_x}{p_x} dp_xdp_ydp_z =
                \int_{-\infty}^{+\infty} p_y dp_y
                \int_{-\infty}^{+\infty} dp_z
                \int_{-\infty}^{+\infty} \pder{fF_x}{p_x} dp_x.
            \]
            Интеграл по \( p_x \) равен нулю, из-за того, что на
            бесконечности функция распределения обращается в ноль:
            \[
                \int p_x \pder{fF_x}{p_x} dp_xdp_ydp_z = 0.
            \]
\end{enumerate}
Этих двух интегралов достаточно, чтобы показать, что
\[
    \int p_\beta \pder{fF_\alpha}{p_\alpha} dp_xdp_ydp_z =
    -\average{F_\alpha} \delta_{\alpha\beta}.
\]
Тогда сумма сильно упрощается
\[
    \sum_{\alpha,\beta}\vec{e}_\beta \int p_\beta \pder{fF_\alpha}{p_\alpha}
    dp_xdp_ydp_z =
    - \sum_{\alpha,\beta}\vec{e}_\beta \average{F_\alpha} \delta_{\alpha\beta} =
    - \sum_{\alpha}\vec{e}_\alpha \average{F_\alpha} = -\average{\vec{F}}.
\]
Возвращаясь к уравнению движения, получаем:
\[
    \der{\average{\vec{p}}}{t} = \average{\vec{F}}
    -\frac{\average{\vec{p}}}{\tau}.
\]
Знакомое уравнение?
