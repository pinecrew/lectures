\section{Кинетическое уравнение Больцмана}
\subsection{Вывод уравнения}

С классической статистической точки зрения для описания электронного газа в
металле нужно ввести функцию распределения \( f \). Отнормируем её удобным для
нас способом:
\[
    \int f(\vec{r},\vec{p},t)\,d^3p\,d^3r = N,
\]
где \( N \) -- полное число электронов в рассматриваемой системе. Тогда
плотность тока
\[
    \vec{j}(\vec{r},t) = \frac{q}{m} \int \vec{p} f(\vec{r},\vec{p},t) d^3p.
\]
Осталось лишь определить функцию распределения. Для этого воспользуемся теоремой
Лиувилля:
\[
    \der{f}{t} = 0 \Rightarrow
    \pder{f}{t} + \pder{f}{\vec{p}}\der{\vec{p}}{t} +
    \pder{f}{\vec{r}}\der{\vec{r}}{t} = 0.
\]
Или, если вспомнить второй закон Ньютона и определение скорости
\[
    \pder{f}{t} + \pder{f}{\vec{p}}\vec{F} + \pder{f}{\vec{r}}\vec{v} = 0.
\]
В случае нерегулярной в пространстве системы в правой части нуля не будет:
\[
    \pder{f}{t} + \pder{f}{\vec{p}}\vec{F} + \pder{f}{\vec{r}}\vec{v} = I(f).
\]
Это уравнение называется кинетическим уравнением Больцмана. Для того, чтобы его
решить, нужно определиться с конкретным видом функции \( I(f) \).

Пусть \( w(\vec{p},\vec{p}') \) -- плотность вероятности перехода электрона из
состояния с импульсом \( \vec{p} \) в состояние с импульсом \( \vec{p}' \) при
столкновении. Тогда
\[
    I(f) =
        \int [w(\vec{p}',\vec{p})f(\vec{p}')-w(\vec{p},\vec{p}')f(\vec{p})]d^3p'
\]
есть изменение количества электронов в состоянии с импульсом \( \vec{p} \). Эта
функция называется интегральной функцией столкновений. Самый простой вид этой
функции, учитывающий релаксацию и удовлетворяющий соотношению
\( \int I(f) d^3p = 0 \)
\[
    I(f) = -\frac{f - f_0}{\tau},\text{ где } f_0
    \text{ -- равновесная функция распределения.}
\]
