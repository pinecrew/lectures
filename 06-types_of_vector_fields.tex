\section{Типы векторных полей}

	Наиболее общей классификацией векторных полей, важной и в физике, является следующая:
	\begin{enumerate}
	\item
		потенциальные поля;
	\item
		соленоидальные поля;
	\item
		лапласовы поля.
	\end{enumerate}

\subsection{Потенциальное поле. Скалярный потенциал}

\subsubsection{Определение потенциального поля}

	Возьмем какое-либо непрерывно дифференцируемое скалярное поле \( \varphi \) и образуем векторное:
	\begin{equation}
		\vec{a} = -\nabla\varphi \label{eq6:1}
	\end{equation}
	
	Образованное таким образом векторное поле \( \vec{a} \) будет потенциальным по определению.
	
	\begin{definition}
	Векторное поле \( \vec{a} \) называется \textbf{потенциальным}, если существует такая однозначная скалярная функция \( \varphi \) такая, что поле \( \vec{a} \) может быть представлено в виде (\ref{eq6:1}), то есть если оно является чьим-то градиентом.
	
	Сама функция \( \varphi \), удовлетворяющая условию (\ref{eq6:1}), называется \textbf{потенциалом} поля \( \vec{a} \).
	\end{definition}
	
	\begin{comment}
	\begin{enumerate}
	
	\item Знак “\( - \)” взят для удобства в физике. Он связан с реальными физическими процессами, например, тепло течет в сторону антиградиента: \( \vec{j} = -k\nabla T \).
	
	\item Если вместо \( \varphi \) подставить в определение \( \varphi* = \varphi + C \), где \( C = \const \), то \( \vec{a} = -\nabla\varphi* = -\nabla\varphi \), так как \( \nabla C = 0 \). Это означает, что потенциал \( \varphi \) определен с точностью до константы.
	
	\item Всякая непрерывно дифференцируемая скалярная функция \( \varphi \) имеет свое потенциальное поле (\ref{eq6:1}), однако не всякое векторное поле \( \vec{a} \) имеет такую функцию, для которой было бы справедливо уравнение (\ref{eq6:1}).
	
	\item Потенциальное поле является простейшим векторным полем, так как все его компоненты определяются одной скалярной функцией, тогда как в общем случае компоненты являются независимыми функциями.
	\end{enumerate}
	\end{comment}
	
\subsubsection{Свойства потенциального поля}

	Если поле \( \vec{a} \) -- потенциально, то оно обладает следующими свойствами:
	\begin{enumerate}
	\item Работа потенциального поля от точки 1 до точки 2 не зависит от формы пути, а определяется только координатами.
	
	\begin{proof}
	\[ \begin{array}{r}
	A = \int\limits_1^2 \vec{a}\cdot\,d\vec{l} = -\int\limits_1^2 \nabla\varphi\,d\vec{l} = -\int\limits_1^2 \left(\pder{\varphi}{x}\,d x + \pder{\varphi}{y}\,d y + \pder{\varphi}{z}\,d z\right) = \\
	= -\int\limits_1^2 \,d\varphi = \varphi(1)-\varphi(2) \end{array} \]
	-- не зависит от формы пути.
	
	Итак,
	\begin{equation}
		\int\limits_1^2 \vec{a}\cdot\,d\vec{l} = \varphi(1) - \varphi(2) \label{eq6:1.1}
	\end{equation}
	\end{proof}
	
	\item Работа потенциального поля по замкнутому контуру равна 0.
	
	\begin{proof}
	
	Возьмем контур \( C \), выберем две точки -- 1 и 2.
	
	В силу (\ref{eq6:1.1}): \( A_{12}^1 = A_{12}^2 \), но \( A_{12}^1 = -A_{12}^2 \).

	Следовательно, \( \underbrace{A_{12}^1 + A_{21}^2}_{\oint\limits_C \vec{a}\cdot\,d\vec{l}} = 0 \).
	
	Итак,
	\begin{equation}
		\oint\limits_C \vec{a}\cdot\,d\vec{l} = 0. \label{eq6:1.2}
	\end{equation}
	\end{proof}
	
	\begin{remark}
	Легко доказать, что из второго свойства следует первое, для этого стоит лишь провести доказательство второго в обратном направлении; следовательно эти свойства эквивалентны.
	
	Они настолько важны, что принимаются за определение потенциального поля.
	\end{remark}
	
	\item Всякое потенциальное поле -- безвихревое, то есть его ротор равен нулю.
	\begin{proof}
	
	\[ \begin{array}{r} \rotor(\nabla\varphi) = \begin{vmatrix}
	\vec{e}_x & \vec{e}_y & \vec{e}_z \\
	\pder{}{x} & \pder{}{y} & \pder{}{z} \\
	\pder{\varphi}{x} & \pder{\varphi}{y} & \pder{\varphi}{z}
	\end{vmatrix} = \vec{e}_x\left(\pcder{\varphi}{y}{z} - \pcder{\varphi}{z}{y}\right) + \\
	+  \vec{e}_y\left(\pcder{\varphi}{z}{x} - \pcder{\varphi}{x}{z}\right) +  \vec{e}_z\left(\pcder{\varphi}{x}{y} - \pcder{\varphi}{y}{x}\right) = \{ 0, 0, 0 \} = \vec{0}.   \end{array} \]
	\end{proof}
	
	Таким образом, имеем второе замечательное тождество векторного анализа:
	\[ \rotor\gradient{\varphi} = \nabla\times\nabla\varphi \equiv 0. \]
	\end{enumerate}
	
	Рассмотренные три свойства потенциальных полей являются его необходимыми условиями, то есть если поле потенциально, то оно обладает всеми этими свойствами.
	
	Расмотрим теперь достаточное условие.
	
\subsubsection{Критерии потенциальности полей}

	\begin{enumerate}
	\item Если работа поля \( \vec{a} \) не зависит от формы пути (\ref{eq6:1.1}) или же если цируляция по любому контуру равна нулю (\ref{eq6:1.2}), то поле \( \vec{a} \) -- потенциально.
	
	\begin{proof}
	
	Так как \( (\ref{eq6:1.2}) \sim (\ref{eq6:1.1}) \).
	
	Возьмем на кривой \( l \) две близкие точки 1 и 2 так, что для них криволинейный интеграл преобразуется в обычное скалярное умножение:
	\[ \int\limits_1^2 \vec{a}\cdot\,d\vec{l} = \vec{a}\cdot\,d\vec{l}. \]
	
	Но для близких точек \( \varphi(2) = \varphi(1) + \,d\varphi \), следовательно получаем:
	\[ \vec{a}\cdot\,d\vec{l} = \varphi(1) - \varphi(2) = -\,d\varphi = -\left(\pder{\varphi}{x}\,d x + \pder{\varphi}{y}\,d y + \pder{\varphi}{z}\,d z \right) = -\nabla\varphi\cdot\,d\vec{l}. \]
	
	Следовательно: \( \vec{a} = -\nabla\varphi \), а это означает, что поле \( \vec{a} \) -- потенциально.
	\end{proof}

	\item Если поле \( \vec{a} \) является безвихревым в односвязной области, то оно потенциально.
	
	\begin{proof}
	
	Запишем теорему Стокса:
	\[ \oint\limits_C \vec{a}\cdot\,d\vec{l} = \iint\limits_S \rotor{a}\cdot\,d\vec{S}. \]
	
	Но, так как \( \rotor{a} = 0 \), то и циркуляция равна нулю, следовательно, поле \( \vec{a} \) -- потенциально.
	\end{proof}
	
	\begin{remark}
	Если ротор поля \( \vec{a} = 0 \), но область определения неодносвязна, то теорема Стокса не работает, и поле \( \vec{a} \) может быть как потенциальным, так и не потенциальным.
	\end{remark}
	\end{enumerate}

\subsubsection{Вычисление потенциала}

	Согласно свойству (\ref{eq6:1.1}), разность потенциалов:
	\[ \int\limits_1^2 \vec{a}\cdot\,d\vec{l} = \varphi(1) - \varphi(2). \]
	
	Для того, чтобы выразить потенциал \( \varphi \) поля \( \vec{a} \) как однозначную функцию точки, то есть \( \varphi = \varphi(x, y, z) \), делают так:
	\begin{enumerate}
	\item точку 2 полагают базовой точкой \( M_0(x_0, y_0, z_0) \);
	\item потенциал в точке \( M_0 \) принимают за 0;
	\item точку 1 берут в качестве переменной \( M(x, y, z) \), в которой и вычисляют потенциал.
	\end{enumerate}
	
	Тогда по любому пути:
	\begin{equation}
		\varphi(x, y, z) = \int\limits_M^{M_0} \vec{a}\cdot\,d\vec{l}. \label{eq6:2.1}
	\end{equation}
	
	Обычно точку \( M_0 \) берут либо в нуле, либо на бесконечности.
	
	Пусть установлено, что поле \( \vec{a} \) потенциально.
	
	Тогда для вычисления потенциала можно использовать следующие три метода, выбор из которых производится на основе структуры конкретного поля:
	
	\textbf{Способ 1:} покоординатный проход.
	
	Так как \( \int\limits_M^0 \vec{a}\cdot\,d\vec{l} \) не зависит от пути интегрирования, то путь выбираем по отрезкам, параллельным координатным осям:
	\[ \int\limits_M^0 \vec{a}\cdot\,d\vec{l} = \int\limits_x^0 a_x(x, y, z)\,d x + \int\limits_y^0 a_y(0, y, z)\,d y + \int\limits_z^0 a_z(0, 0, z)\,d z = \varphi(x, y, z). \]
	
	\begin{example}
	Определить потенциал поля \( \vec{a} = \{ yz, xz, xy \} \) относительно начала координат.
	\end{example}
	\begin{solution}
	
	Поле потенциально, так как его ротор равен нулю и область его определения односвязна.
	
	Его потенциал:
	\[ \varphi(x, y, z) = \int\limits_x^0 yz\,d x + \int\limits_y^0 0z\,d y + \int\limits_z^0 0\,d z = -xyz. \]
	\end{solution}

	\textbf{Способ 2:} составление полного дифференциала.
	
	Если из выражения под интегралом \( a_x\,d x + a_y\,d y + a_z\,d z \) удасться составить полный дифференциал, то есть привести его к виду \( \,d\varphi \), то это \( \varphi \) и будет потенциалом.
	
	\begin{example}
	Определить потенциал поля \( \vec{a} = \{ yz, xz, xy \} \) относительно начала координат.
	\end{example}
	\begin{solution}
	
	\[ \varphi = \int\limits_M^0 yz\,d x + xz\,d y + xy\,d z = \int\limits_{(x, y, z)}^0 \,d(xyz) = -xyz. \]
	\end{solution}
	
	\textbf{Способ 3:} если поле центрально-симметричное (\( \vec{a} = f(r)\vec{r} \)).

	Такие поля всегда являются потенциальными, так как область их определения односвязна и ротор равен нулю.
	
	Следовательно, для вычисления его потенциала путь может быть любым. Для удобства берем его вдоль \( \vec{r} \), то есть \( \,d\vec{l} = \,d\vec{r} \).
	
	Тогда: \( \int\limits_M^\infty \vec{a}\cdot\,d\vec{l} = \int\limits_r^\infty f(r)r\,d r \).
	
	\begin{example}
	Вычислить потенциал квазиупругого (гуковского) поля: \( \vec{a} = -k\vec{r} \) относительно начала координат.
	\end{example}
	\begin{solution}
	
	Здесь \( f(r) = -k \).
	
	Следовательно, потенциал:
	\[ \varphi = \int\limits_r^0 -kr\,d r = \frac{kr^2}{2}. \]
	\end{solution}
	
	\begin{example}
	Вычислить потенциал кулоновского поля: \( \vec{a} = \frac{\vec{r}}{r^3} \) относительно бесконечности.
	\end{example}
	\begin{solution}
	
	Здесь \( f(r) = \frac{1}{r^3} \).
	
	Тогда потенциал:
	\[ \varphi = \int\limits_r^\infty \frac{r}{r^3}\,d r = \int\limits_r^\infty \frac{\,d r}{r^2} = \frac{1}{r}. \]
	\end{solution}

\subsection{Соленоидальное поле. Векторный потенциал}

	\begin{definition}
	Поле \( \vec{a} \) является \textbf{соленоидальным}, если во всей области его определения его дивергенция равна нулю:
	\begin{equation}
		\divergence{a} \equiv 0. \label{eq6.2:1}
	\end{equation}
	\end{definition}
	
	Из (\ref{eq6.2:1}) следует, что поле \( \vec{a} \) -- ротор какого-либо поля:
	\begin{equation}
		\vec{a} = \rotor{A}. \label{eq6.2:2}
	\end{equation}
	
	\begin{definition}
	Векторное поле \( \vec{A}(x, y, z) \), удовлетворяющее условию (\ref{eq6.2:2}), называется \textbf{векторным потенциалом} поля \( \vec{a} \).
	\end{definition}
	
	Так как \( \rotor{A} = \rotor(\vec{A} + \nabla\varphi) \) (в силу второго замечательного тождества векторного анализа), то (\ref{eq6.2:2}) удовлетворяет так же любой вектор \( \vec{A}* = \vec{A} + \nabla\varphi \). Это означает, что векторный потенциал определен с точностью до градиента произвольной постоянной функции.
	
	Вычисление векторного потенциала связано с решением системы трех дифференциальных уравнений в частных производных относительно компонент \( A_x \), \( A_y \) и \( A_z \):
	\[ \left\{ \begin{array}{l}
		\pder{A_x}{y} - \pder{A_y}{x} = a_z; \\
		\pder{A_y}{z} - \pder{A_z}{y} = a_x; \\
		\pder{A_z}{x} - \pder{A_x}{z} = a_y.
	\end{array} \right. \]
	
	Потенциальными и соленоидальными полями не исчерпываются все векторные поля. Есть такие, которые являются ни потенциальми, ни соленоидальными, например: \( \vec{a} = \vec{r} + \omega\{ -y, x, z\} \).

	Однако, существует теорема, называемая \textbf{основной теоремой векторного анализа}, в которой утверждается следующее:
	\begin{theorem}
	Всякое непрерывно дифференцируемое векторное поле \( \vec{a}(x, y, z) \), обращающееся вместе со своими ротором и дивергенцией на бесконечности в ноль, может быть единственным образом представлено в виде суммы потенциального \( \vec{P} \) и соленоидального \( \vec{S} \) полей:
	\[ \vec{a} = \vec{P} + \vec{S}. \]
	\end{theorem}
	
	Однако, существуют и такие поля, которые являются одновременно и потенциальными, и соленоидальными. Такие поля называют \textit{лапласовыми}. Именно такими являются большинство физических полей.

\subsection{Лапласово поле. Гармонические функции}

	\begin{definition}
	Векторное поле \( \vec{a}(x, y, z) \) называется \textbf{лапласовым}, если \( \rotor{a} = 0 \) и \( \divergence{a} = 0 \), то есть это одновременно соленоидальное и потенциальное поле.
	\end{definition}
	
	Из того, что ротор лапласова поля равен нулю, следует, что поле является чьим-то градиентом, так как \( \rotor\gradient{A} = 0 \).
	
	Подставляя это во второе условие, получаем для скалярного потенциала лапласова поля:
	\[ \divergence\gradient\varphi = 0 \]
	или
	\begin{equation}
		\Delta\varphi = 0, \label{eq6.3:1}
	\end{equation}
	где \( \Delta = \nabla^2 \) -- \textbf{лапласиан} или \textbf{оператор Лапласа}:
	\[ \Delta = \ppder{}{x} + \ppder{}{y} + \ppder{}{z}. \]
	
	Уравнение (\ref{eq6.3:1}) называется \textbf{уравнением Лапласа}.
	
	\begin{definition}
	Всякая скалярная функция \( \varphi \), дважды непрерывно дифференцируемая и удовлетворяющая уравнению Лапласа, называется \textbf{гармонической}.
	\end{definition}
	
	Примеры:
	\begin{enumerate}
	\item \( \varphi = \const \);
	\item \( \varphi = ax + by + cz \);
	\item \( \varphi = xyz \);
	\item \( \varphi = \ln\rho \), где \( \rho = \sqrt{x^2 + y^2} \) -- расстояние от точки до оси \( Oz \);
	\item \( \varphi = \frac{1}{r} \), где \( r = \sqrt{x^2 + y^2 + z^2} \) -- расстояние от точки до начала координат.
	\end{enumerate}
